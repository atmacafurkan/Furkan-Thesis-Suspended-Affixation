\subsection{\citet{broadwell2008turkish}} \label{sec:broadwell}
Broadwell provides a representation for SA using the tools of Lexical Functional Grammar (LFG henceforth). In this approach, the two identical phrases form a new phrase in conjunction that has the same structural properties of its parts. After this point, the suspended affixes are added. The phonological exponent of the right edge conjunct and the suspended affixes are \textit{coinstantiated} as one word. Figure \ref{fig:lexicalshare} illustrates the structural representation for the SA of {\Pl-\Poss} in (\ref{tebrikler3}). Broadwell claims that this way of representation for SA saves us from (i) interpreting affixes that can be suspended as clitics, (ii) positing conjunction in the lexicon, and (iii) having special annotation for the rightmost conjunct.

\begin{exe}
    \ex \label{tebrikler3} SA of {\Pl-\Poss}\\*
    \gll tebrik ve teşekkür-ler-im \\ 
    congrats and thanks-{\Pl}-{\Poss}.{\Fsg} \\ 
    \glt `My congratulations and thanks'
\end{exe}
    
\begin{figure}[hbt!]
    \centering
\begin{tikzpicture}
    \Tree[.PossP
            [.PlurP 
                [.NP 
                    [.NP\\tebrik ]
                    [.Conj\\ve ]
                    [.\node(NP2){NP}; ] ]
                [.\node(plr){Plur}; ] ]
            [.\node(PS){Poss}; ]
]
\node[below right= 1em of NP2, draw](co){teşekkür-ler-im};
\draw[thick, ->] (NP2) -- (co);
\draw[thick, ->] (plr) -- (co);
\draw[thick, ->] (PS) -- (co);
\end{tikzpicture}
    \caption{Lexical sharing analysis of {\Pl} and {\Poss} in SA}
    \label{fig:lexicalshare}
\end{figure}

An important point which Broadwell makes is that Turkish is relatively productive in SA, but it also makes distinctions that cannot be addressed with a purely lexical approach. It might be posited that SA is only permitted with affixes that can attach to conjoined phrases. This analysis however does not explain why the suspension of {\Poss} is ungrammatical in a string of {\Pl-\Poss} and does not explain how to categorize suffixes that can have conjoined bases, missing the morphological word requirement of SA in the verbal domain.
