\chapter{\MakeUppercase{experiment 2: self paced items}} \label{selfpaceditems}
\singlespacing
\section{Sentences}
\begin{myparindent}{0pt}
1\_Eski mektupların satırlarında çok koşarmışsın veya gülermişsin diye çok bahsin geçiyor. \\
2\_Eğlencelerin odağı olan partiler düzenlermişsin veya tertiplermişsin diye duydum ben başkalarından. \\
3\_Önceden Sezen Aksu'nun şarkılarını dinlermişim veya ezberlermişim ama şimdi hiçbirini hatırlamıyorum. \\
4\_Ünlü dizi Süper Babayı izlermişiz veya severmişiz çünkü bu ailemizin eğlencesiymiş. \\
5\_Ağacı kırar diye ona kızarmışız veya bağırırmışız ki ağaca hiç tırmanmasın. \\
6\_Bu dersin ödevlerini zamanında yapsaymışım veya gönderseymişim hoca tam puan verecekmiş. \\
7\_Yazıcıdan çıkan belgeyi ceketime koysaymışım veya korusaymışım hiç de ıslanmayacakmış aslında. \\
8\_Böreğin piştiği küçük fırını izleseymişiz veya gözetleseymişiz şimdi yanık börek yemezdik. \\
9\_Genç yaşlarda düzenli şekilde beslenseymişim veya yaşasaymışım daha uzun ömür sürermişim. \\
10\_Arsadan elde edilen madeni işleseymişiz veya satsaymışız çok fazla para kazanırmışız. \\
11\_Hocanın her söylediğini dikkatlice dinleyecekmişiz veya özetleyecekmişiz çünkü bunlar sınavda gerekliymiş. \\
12\_Arabayı yağmurdan korumak için boyatacakmışsın veya kaplatacakmışsın ki araba su geçirmesin. \\
13\_Siparişi verilen bu aletleri monteleyecekmişsin veya taşıyacakmışsın sahipleri gelip görmeden önce. \\
14\_Güvece atılacak bu sarımsakları dilimleyecekmişim veya ezecekmişim ki sadece tadı kalsın. \\
15\_Doktorun tavsiyesine göre gözlerimi dinlendirecekmişim veya ovalayacakmışım ki kan dolaşımı hızlansın. \\
16\_Hurdacıya gelen teknolojik aletleri kurcalamalıymışız veya incelemeliymişiz ancak bunları yapmakta geciktik \\
17\_Bu Hindistan cevizlerini henüz kırmamalıymışım veya yememeliymişim bu yüzden biraz bekledim. \\
18\_Üzerinde toz biriken masayı yıkamalıymışım veya silmeliymişim diye mırıldandım kendi kendime. \\
19\_Bu makinenin içindeki dişlileri sökmeliymişiz veya çıkarmalıymışız ki nasıl çalıştığını öğrenelim. \\
20\_Tasarruf yapmak için fırını kullanmamalıymışız veya açmamalıymışız çünkü fırın çok yakıyormuş. \\
21\_Dışarı çıkmak gerekirse diye hazırlanıyormuşuz veya bekliyormuşuz ama dışarda yağmur yağıyor. \\
22\_Kuşların senelik göç güzergahını izliyormuşuz veya belirliyormuşuz ki çevre düzenlemelerine uyalım. \\
23\_Resmi törende yürüyen askerlere bakıyormuşuz veya şaşırıyormuşuz çünkü çok düzenli yürüyorlardı. \\
24\_Sayfaları pörsümüş anı defterine yaşıyormuşum veya eğleniyormuşum diye usulca not düşüyorum. \\
101\_Kitap almaya giderken karşıma çıkan adama garip bir şekilde bakarak geçtim. \\
102\_Buralara gelerek kendini tehlikeye atmak için çok gençsin desem bana inanırdın. \\
103\_Her günü iple çekerek yaşamak sanırım mutlu olmanın en büyük sırrı. \\
104\_Akşama rus salatası yapmak için pazardan havuç ve uzun turşu aldım. \\
105\_Dolapta birkaç gündür bekleyen zeytinleri bir an evvel sofraya koyalım bence. \\
106\_Kırmızı ışıkta geçen aracın plakası kameralardan çok net bir şekilde gözüküyordu. \\
107\_Ekonomiyi takip ederken dikkat edilmesi gereken en önemli husus değişkenlerin çokluğudur. \\
108\_Everest dağını tırmanmak gerçekten de yetenek ve cesaret gerektiren bir iştir. \\
109\_Solunum yollarında açığa çıkan iltihaptan dolayı yoğun bakım ünitesinde günlerdir bekletiliyor. \\
110\_Karışık ızgara menüde yer alan en pahalı yemeklerden sadece göze çarpanıydı. \\
111\_Matbaadan çıkan yeni basım kitaplar yayınevinin istediği kalitede olmadığından geri gönderildi. \\
112\_Bilgisayar çağında yaşadığımız için bunu hayatımızdaki vazgeçilmezler listesine artık eklememiz gerek. \\
113\_Çok yürümekten ayaklarına kara sular inen coğrafya öğretmeni mola işareti verdi. \\
114\_Bodrum katındaki duvarları nem alan yurtta tadilat çalışmaları günlerdir devam ediyor. \\
115\_Çatı yalıtımının sağladığı enerji tasarrufu yalıtım yaptırmayı maliyet açısından ekonomik yapıyor. \\
116\_Ülkesini savunurken şiddete tanık olan askerler akıl sağlıklarını korumakta güçlük çekiyor. \\
117\_Resimlerinde gökyüzünü hiçbir zaman maviye boyamayan ressam dünyaca ünlü bir sanatkar. \\
118\_Mantar panoya asılacak hatırlatmaları takip ederseniz hangi gün ne yapacağınızı bilirsiniz. \\
119\_Ellerini sıcak sudan soğuk suya sokmayan insanlar risk almak nedir bilmiyorlar. \\
120\_Kitaptaki karakterleri anlayabilmek için satır aralarını çok dikkatli okumak bile yetmiyor. \\
121\_Geri dönüşüm için biriktirilen plastik ve menşei ürünler yeterince iyi saklanmıyor. \\
122\_Elektrik üretiminde sürdürülebilir kaynaklara geçmek kadar tüketimde verimi arttırmak da önemlidir. \\
123\_Çöp kutularını devirerek temizlikçileri sinir eden kedi sonunda sokağı terk etmiş. \\
124\_Fabrikalarda alınacak yeni güvenlik önlemleri resmi gazetede yayımlanarak bugün yürürlüğe girdi. \\
125\_Maden işçilerinin greve gitmesi kömür üretiminde ciddi bir düşüşe neden oldu. \\
126\_Güney Amerika ülkelerindeki yüksek suç oranı ekonomi geliştikçe giderek azalmaya başladı. \\
127\_Kaliforniyada evsiz insanlar sayısı gün geçtikçe artmasına rağmen görevliler önlem almıyor. \\
128\_Uçağa binerken adım attığınız yere dikkat edin çünkü orada boşluk var. \\
129\_Kitaplara düşkün olduğundan ne zaman kitapçının önünden geçsek mutlaka içeri girer. \\
130\_Yanıp sönen ışıkları takip ederseniz yolun sonunda çalışmanın olduğu alandan çıkarsınız. \\
131\_Tadilat parasını toplamak için bir araya gelen köylüler kendi aralarında anlaşamadı. \\
132\_Anotasyon işlemi için gerekli olan kotayı tamamlamak için durmaksızın çalışmak gerekiyor. \\
133\_Yoğun bakıma alınan trafik kazası kurbanı bütün müdahalelere rağmen hayatta kalamadı. \\
134\_Üniversite sınavında yüksek puan almak kadar doğru tercih yapmak da önemlidir. \\
135\_Doğu yakasını kırıp geçiren kasırga arkasında çok büyük mali hasar bıraktı. \\
136\_Sızma zeytinyağını ve sirkeyi ince ağızlı şişe kullanarak salataya yavaşça ekleyiniz. \\
137\_Dünyanın dört bir yanını dolaşan Barış Manço benim en favori şarkıcım. \\
138\_Düğün yeri olarak seçilen salonun bahçesi ve ışıkları herkes tarafından beğenildi. \\
139\_Cam kavanozları salatalık turşusuyla doldurup soğuk bir yere kaldıralım ki bozulmasınlar. \\
140\_Kemençe çalmaya çalışmak gitar çalmaktan çok da farklı bir yetenek gerektirmiyor. \\
141\_Bozuk duş başlığını değiştirirken ayağı birden kayıp küvetin kenarına kafasını çarptı. \\
142\_Kulpları düşen dolabın kapağını açmak için parmağını araya sokup geriye çek. \\
143\_Uçurumun kenarından denize doğru baktığın zaman karşında gördüğün manzara çok güzel. \\
144\_Çocuk sandalyenin ucuna oturup ileri geri sallanırken annesi öteki odadan bağırdı. \\
145\_Ellerine kırmızı saç boyası bulaştığından musluk başlarını ve kapıları dirseğiyle açtı. \\
146\_Pazar yerini geçtikten sonra karşında gördüğün bankamatiğin hemen yanında seni bekliyorum. \\
147\_Anlaşmanın tarafları birkaç farklı hususta ortak bir kanıya varmak için buluştu. \\
148\_Dibi delinmiş şişeyle su taşımaya çalışan çocuk eve geldiğinde şişe bomboştu. \\
\end{myparindent}
\section{Questions}

\begin{myparindent}{0pt}
1\_correct\_Bahsin eski mektuplarda geçiyor. \\
2\_correct\_Partiler eğlencenin odağıydı. \\
3\_correct\_Şarkılar Sezen Aksu'nundu. \\
4\_correct\_Ünlü dizi Süper Baba idi. \\
5\_correct\_Ağacı kırarım diye kızıyorlar. \\
6\_correct\_Dersin ödevi zamanında yapılacak. \\
7\_correct\_Belge yazıcıdan çıkmış. \\
8\_correct\_Fırın küçükmüş. \\
9\_correct\_Genç yaşta düzenli yaşanmalıymış. \\
10\_correct\_Maden arsadan elde edilmiş. \\
11\_correct\_Hocayı dikkatlice dinleyecekmişim. \\
12\_correct\_Arabayı yağmurdan koruyacakmışım. \\
13\_incorrect\_Bisikleti monteleyecekmişsin. \\
14\_incorrect\_Güvece biber atacakmışım. \\
15\_incorrect\_Tavsiyeyi komşum vermiş. \\
16\_incorrect\_Hurdacıya bozuk aletler gelmiş. \\
17\_incorrect\_Fındıkları yememeliymişim. \\
18\_incorrect\_Pencerenin üzerinde toz birikmiş. \\
19\_incorrect\_Kabloları sökmeliymişim. \\
20\_incorrect\_Ocağı kullanmamalıymışım. \\
21\_incorrect\_Dışarda kar yağıyor. \\
22\_incorrect\_Kuşların senelik beslenmesini izliyormuşuz. \\
23\_incorrect\_Öğrencilere bakıyormuşuz. \\
24\_incorrect\_Günlüğe not düşüyormuşum. \\
101\_correct\_Garip bir şekilde adama baktım. \\
102\_correct\_Çok gençsin desem inanırdın. \\
103\_correct\_Mutlu olmanın bir sırrı var. \\
104\_correct\_Rus salatası için turşu aldım. \\
105\_correct\_Dolapta bekleyen zeytinler var. \\
106\_correct\_Araç kırmızı ışıkta geçmiş. \\
107\_correct\_Değişken çokluğu önemlidir. \\
108\_correct\_Dağ tırmanmak cesaret gerektirir. \\
109\_correct\_Solunum yollarında iltihap var. \\
110\_correct\_Karışık ızgara en pahalı ürün. \\
111\_correct\_Yeni basım kitaplar kaliteli olmamış. \\
112\_correct\_Bilgisayar çağında yaşıyoruz. \\
113\_correct\_Coğrafya öğretmeni mola verdi. \\
114\_correct\_Yurtta tadilat çalışması var. \\
115\_correct\_Çatı yalıtımı ekonomik. \\
116\_correct\_Askerler şiddete tanık oluyor. \\
117\_correct\_Ressam gökyüzünü maviye boyamıyor. \\
118\_correct\_Hatırlatmalar mantar panoya asılacak. \\
119\_correct\_İnsanlar risk almak nedir bilmiyor. \\
120\_correct\_Satır araları dikkatli okunmalı. \\
121\_correct\_Geri dönüşüm plastiği iyi saklanmıyor. \\
122\_correct\_Enerji verimini arttırmak önemlidir. \\
123\_correct\_Kedi çöp tenekelerini deviriyor. \\
124\_correct\_Güvenlik önlemleri yürürlüğe girdi. \\
125\_incorrect\_Altın üretiminde düşüş oldu. \\
126\_incorrect\_Suç oranı artıyor. \\
127\_incorrect\_Evsiz sayısı azalıyor. \\
128\_incorrect\_Otobüse binerken dikkat edin. \\
129\_incorrect\_Kişi teknolojiye düşkün. \\
130\_incorrect\_Yolda kaza var. \\
131\_incorrect\_Köylüler tadilat yaptı. \\
132\_incorrect\_Anotasyon kotası yok. \\
133\_incorrect\_Kazazede hayatta. \\
134\_incorrect\_Yüksek puan almak zordur. \\
135\_incorrect\_Doğu yakasında kuraklık olmuş. \\
136\_incorrect\_Zeytinyağı dolmaya eklenecek. \\
137\_incorrect\_Zeki Müren favori şarkıcım. \\
138\_incorrect\_Evin bahçesi var. \\
139\_incorrect\_Kavanozlar domatesle dolu. \\
140\_incorrect\_Ut çalmak gitara benzer. \\
141\_incorrect\_Su vanası bozuk. \\
142\_incorrect\_Dolabın kapağı düşmüş. \\
143\_incorrect\_Karşıdaki ev çok güzel. \\
144\_incorrect\_Çocuğun babası bağırdı. \\
145\_incorrect\_Ellerine duvar boyası bulaşmış. \\
146\_incorrect\_Berberin yanında bekliyorum. \\
147\_incorrect\_Karşılaşmanın tarafları buluştu. \\
148\_incorrect\_Leğenin dibi delikmiş. \\
\end{myparindent}