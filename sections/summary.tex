\section{Summary}

As a summary of the literature presented in this chapter, I provide the following observations about SA:

\begin{itemize}
    \item It is a rightward-bound process in the underlying morpheme order: Examples provided in \citet{kabak2007turkish}, \citet{pounder2006broken}, and \citet{guseva2017postsyntactic} show this for Turkish, German, and Mari.
    
    \item It is found both in inflectional and derivational paradigms: Examples provided in \citet{akkucs2016suspended}, and \citet{yoon2017lexical} show this for Turkish and Korean.

    \item It takes place after vocabulary insertion and before phonological readjustments: Examples provided in  \citet{pounder2006broken}, \citet{guseva2017postsyntactic}, and \citet{erschler2018suspended} show this for German, Mari and Ossetic.
\end{itemize}

These are the observations that seem to be consistent in all the articles. However, not all the articles align in the structural analysis of SA. The dominant account for Turkish seems to be structural sharing in nature \citep{orgun1995flat,kornfilt1996some,broadwell2008turkish,kornfilt2012revisiting}. This account is in line with \citet{ackema2004beyond,kunduraci2016morphology}, and \citet{bruening2018word} since an output of syntax can become an input for morphology and word formation in such form of language derivation. The accounts provided for other languages like Serbian, Mari, and Ossetic are all ellipsis analyses \citep{despic2017suspended,guseva2017postsyntactic,erschler2018suspended}. The summary of the literature for Turkish SA presents the following points to be addressed for any further study. It is the aim of this thesis to scrutinize these issues and contribute to the literature in an orderly and comprehensive manner.

\begin{itemize}
    \item Is SA of derivational suffixes possible in Turkish? If so how, if not why?
    \item What empirical studies can be used to determine the processing cost of SA?
    \item How does SA interact with sentence processing?
\end{itemize}

