\subsection{Serbian}

According to \cite{despic2017suspended}, Serbian does not have SA, but a certain second-place clitic shows some similarities to affixes in Serbian. This clitic in turn can take place in SA-like ellipsis. SA in Turkish verbal domain has a relation to the clitic copula \textit{-i/ y/ $\emptyset$} (\ref{verbalSA}) and the discussion of Serbian provides some insights for it.

\begin{exe}
    \ex \label{verbalSA}
    \begin{xlist}
        \ex \label{verbalSA1}
        \gll Ev-e gel-ec\'{e}k ve uyu-yac\'{a}k-tı-m \\ 
        house-{\Dat} come-{\Fut} {\And} sleep-{\Fut}={\Cop}.{\Pst}-{\Fsg} \\
        
        \ex \label{verbalSA2}
        \gll Ev-e gel-ec\'{e}k ve uyu-yac\'{a}k i-di-m. \\ 
        house-{\Dat} come-{\Fut} {\And} sleep-{\Fut} ={\Cop}-{\Pst}-{\Fsg} \\
        \glt `I was going to come home and sleep'
    \end{xlist}
\end{exe}

(\ref{verbalSA1}) shows an SA of {\Pst} and {\Agr} morphemes, but a closer look reveals what is suspended is a copular form together with tense and agreement markers. This copular which is a clitic can have an overt phonological form \textit{i} which allows for SA (\ref{verbalSA2}). The overtness of the clitic is not enforced, and it is even ungrammatical in some instances (\ref{nompredSA}). The existence of the clitic is inferred from the stress\footnote{stress is indicated by an accent on the vowel}. In Turkish the stress falls on the phonological word and a clitic changes the stress pattern.

\begin{exe}
    \ex \label{nompredSA}
    \begin{xlist}
        \ex \gll hast\'{a} ve yorg\'{u}n-um \\
        sick {\And} tired-{\Fsg}[{\Prs}] \\
        \glt `I am sick and tired'
        
        \ex \gll *hast\'{a} ve yorg\'{u}n i-yim. \\ 
        sick {\And} tired {\Cop}[{\Prs}]-{\Fsg} \\
    \end{xlist}
\end{exe}

The instance where SA-like process takes place involves the infinitival marker \textit{-ti} and second-place future clitic \textit{\'{c}e} in Serbian. The bare bones explanation for second-place clitics is in a clause they occupy the linearly second-place. If they are cliticized to the phonological word they are attached to, then the word can occupy the first place in the clause.

I want to draw a similarity between the infinitival marker \textit{-ti} in Serbian and the infinitival marker \textit{-mAK} in Turkish. Verbs are not free forms in Serbian, just like verbs are not morphological words in Turkish. There is no need for an infinitival marker when the verb is inflected, and the inflection is performed on to the left of \textit{-ti} or \textit{-mAK}.

In Serbian, some phonological processes are not triggered by clitics. (\ref{serbiance}) shows an example for the assimilation of [s] to [\textesh]. This is  triggered by the diminutive suffix \textit{\'{c}e} but not by the second-place future clitic \textit{\'{c}e}. The suffixes both have the same phonological environment.

\begin{exe}
    \ex \label{serbiance}
    \begin{xlist}
        \ex \gll Pa\u{s}-\'{c}e \\ 
        dog-{\Dim} \\
        \glt `small dog'
        
        \ex \label{serbianfutce} \gll Vas \'{c}e videti \\ 
        you.{\Pl}.{\Acc} ={\Aux}.{\Tsg}.{\Fut} see.{\Inf} \\
        \glt `S/he will see you.'\\*
        \hfill Adapted from \cite{despic2017suspended}
    \end{xlist}
\end{exe}

The second-place future clitic \textit{\'{c}e} in (\ref{serbianfutce}) is used as a free-standing word. It does not cause phonological changes like the diminutive suffix \textit{\'{c}e}. (\ref{serbiance2}) shows an example of the second-place future clitic \textit{\'{c}e} causing phonological change. This time, however, it is adjoined to the word instead of being in its free form.

\begin{exe}
    \ex \label{serbiance2}
    \begin{xlist}
        \ex \gll *Jes=\'{c}e\u{s} \\ 
        eat={\Aux}.{\Ssg}.{\Fut} \\
    
        \ex \gll Je\u{s}=\'{c}e\u{s} \\ 
        eat={\Aux}.{\Ssg}.{\Fut} \\
        \glt `You will eat.'\\*
        \hfill Adapted from \cite{despic2017suspended}
    \end{xlist}
\end{exe}


The observation in (\ref{serbiance2}) may place the clitic as a suitable candidate for SA. (\ref{serbianSAlike}) shows an elision of the second-place future clitic \textit{\'{c}e}, from the first conjunct. In (\ref{serbianSAlike}) what is left after the elision is not a phonological string of what comes before the clitic, but an infinitival form.

\begin{exe}
    \ex \label{serbianSAlike} Elision of \textit{\'{c}e} `{\Fut}'
    \begin{xlist}
        \ex \gll Oti\'{c}i \'{c}e i pogleda=\'{c}e novi film. \\ 
        go.{\Inf} {\Aux}.{\Tsg}.{\Fut} {\And} see={\Aux}.{\Tsg}.{\Fut} new.{\Acc} film.{\Acc} \\
        
        \ex \gll *Oti\'{c}i \'{c}e i pogleda novi film. \\ 
        go.{\Inf} {\Aux}.{\Tsg}.{\Fut} {\And} see new.{\Acc} film.{\Acc} \\
        
        \ex \gll Oti\'{c}i \'{c}e i pogledati novi film. \\ 
        go.{\Inf} {\Aux}.{\Tsg}.{\Fut} {\And} see.{\Inf} new.{\Acc} film.{\Acc} \\
        \glt `He will go and see the new movie'\\*
        \hfill Adapted from \cite{despic2017suspended}
    \end{xlist}
\end{exe}

Despi\'{c} goes into an in-depth analysis to refute an idea of structural sharing of the future clitic. He provides the following example (\ref{serbianconjuncts}). There can be two different subjects in (\ref{serbianconjuncts}), so there is no VP-level conjunction. Despi\'c further examines TP level adverbs in conjunctions, refuting a \textit{v}P level conjunction too.

\begin{exe}
    \ex \label{serbianconjuncts}
    \gll Polufinalni program  \'{c}e otvoriti Juentus i {Real Madrid,} a zatvoriti ga Barselone i Bajern \\ semi\_final program {\Aux}.{\Tsg}.{\Fut} open.{\Inf} J {\And} {R M} {\And} close.{\Inf} {\Tsg} B {\And} B \\
    \glt `Juventus and Real Madrid will open the semi-final program, and Barcelona and Bayern will close it.' \\*
    \hfill Adapted from \cite{despic2017suspended}
\end{exe}

I reiterate an example from Despi\'c about the elision of the second-place future clitic \textit{\'{c}e} in (\ref{serbiannomatch}). This example shows that it is possible to delete the second-place future clitic \textit{\'ce} in Serbian under mismatching $\varphi$-features. 

\begin{exe}
    \ex \label{serbiannomatch}
    \begin{xlist}
        \ex \gll Ti \'{c}es do\'{c}i a ja (\'{c}u) oti\'{c}i \\ 
        {\Ssg} {\Aux}.{\Ssg}.{\Fut} arrive.{\Inf} {\And} {\Fsg} ({\Aux}.{\Fsg}.{\Fut}) leave.{\Inf} \\
        \glt `You will come, and I will leave.'\\*
        \hfill Adapted from \cite{despic2017suspended}
    \end{xlist}
\end{exe}

This is a direct contradiction to all the suspendable affixes in Turkish verbal domain which have clitic properties. The suspendable agreement \textit{-Iz} `{\Fpl}' belongs to the m-paradigm and has clitic properties. The unsuspendable agreement \textit{-k} `{\Fpl}' belongs to the k-paradigm and does not have clitic properties. (\ref{kiz}) illustrates both points.

\begin{exe}
    \ex \label{kiz} 
    \begin{xlist}
        \ex \gll Ev-e gid-ec\'{e}k ve dinlen-ec\'{e}ğ-iz \\ 
        house-{\Dat} go-{\Fut} {\And} rest-{\Fut}-{\Fpl} \\
        \glt `We will go home and rest'
        
        \ex \gll *Ev-e git-t\'i ve dinlen-d\'i-k \\ 
        house-{\Dat} go-{\Pst} {\And} rest-{\Pst}-{\Fpl} \\
        \glt Intended `We went home and rested'
    \end{xlist}
\end{exe}

The m-paradigm agreement markers cannot be suspended under mismatching $\varphi$-features unlike the Serbian second-place future clitic \textit{\'ce}. I give an example in (\ref{mismatchm}) where suspension of {\Ssg} is not permitted if the target of the SA is {\Fsg}.

\begin{exe}
    \ex \label{mismatchm}
    \gll *Sen ev-e gid-ecek ve ben dinlen-eceğ-im \\ 
    {\Ssg} house-{\Dat} go-{\Fut} {\And} {\Fsg} rest-{\Fut}-{\Fsg} \\
    \glt Intended `You will go home, and we will rest.'
\end{exe}

As a summary, the Serbian second-place future clitic shows affix like properties, but it undergoes an ellipsis process where mismatches in $\varphi$-features can be overlooked. As a contrast, some agreement markers in Turkish show clitic like properties yet they cannot undergo SA when there is a mismatch in $\varphi$-features.

















