\section{Interim summary of the literature}

The literature of SA for Turkish provides some valuable observations, that make it easier to navigate the problems and workings of SA in Turkish. They feature useful data, approaches like LFG, syntactic movements like RNR, and comprehensive coverage of morphological constraints in SA. In the following paragraphs, I summarise the points made in the literature about SA, and in what ways it can be improved. Then I put a finger on unaddressed issues.

\cite{orgun1995flat} puts forward an anomalous behaviour in the suspension of {\Poss} in a string of {\Pl-\Poss}. Orgun's solution is to hierarchically align the two for handling the problem of inseparable suspension of {\Poss}.

The observations of \cite{kabak2007turkish} indicate that the morphological size of what is left after suspension is crucial for a successful SA. Bare verbs are not considered as morphological words even though they are phonological words and get stress under negation \textit{-mA}. The observation of morphological word constraint in SA is quite important since some similar phenomenon of a backwards process in, for example, German only requires the remnant after suspension to be a phonological word \citep{smith2000word, pounder2006broken,kenesei2007semiwords}. Kabak's paper shows that SA might be possible with some derivational suffixes, yet he strongly suggests that the base for the derivational suffix is a compound like noun that uses a conjoiner for its parts.

\cite{broadwell2008turkish} entertains a different mode of operation for the analysis of SA. Rather than the suspended suffix originating in both conjuncts, the conjoined phrase is only merged with a single projection of the `suspended' suffix. Later, as a tool of LFG, the rightward elements coinstantiate as a single word of multiple exponents, appearing as though only the second conjunct has the suffix whereas structurally it is shared and the two conjuncts are at the same level of representation.

According to \cite{kornfilt1996some, kornfilt2012revisiting}, SA is a syntactic operation of RNR, and suspendable suffixes are projections in syntax. This defines a line for the capability of SA for derivational and inflectional suffixes. Her analysis does not explain why SA of {\Case} is not ambiguous, but the SA of {\Pl} or {\Poss} is. The importance of Kornfilt's proposal is the observation of the productivity of SA in the inflectional paradigm. This places an analysis of SA more in the structural side that should have access to syntactic inputs.

\cite{kharytonava2011morphology,kharytonava2012word,kharytonava2012taming} deviate from all the others in dealing with SA because they deal with a peculiar SA observed in noun compounds. The preference studies that Kharytonava have carried out suggest that partial SA conditions are preferred more than complete morpheme deletions. Unfortunately, the reporting of the studies are not very clear. Only percentages in terms of participant preferences are provided. Furthermore, some arbitrary schemes for grouping subjects by choice frequency are used to draw inferences from the responses being interpreted as grammatical or not. 

\cite{akkucs2016suspended} points to instances of derivational SA and argues that they need an explanation contra Kabak's view of natural coordination \citep{walchli2005co}. He argues for a revised understanding of what Lexical Integrity Hypothesis in the sense of either \cite{ackema2004beyond} or \cite{lieber2006lexical} in explaining instances of derivational SA. Akkuş's paper is the only paper that argues for a structural interpretation of SA in derivational suffixes. 

As a conclusion, the current literature for Turkish SA provides possible solutions and analyses for SA. The literature does not have a good standing when it comes to answering in what level of language derivation SA takes place. It is not pinpointed well enough to argue for or against any analysis, be it ellipsis or structural sharing. There is no exposure of different conjoiners and what they bring to SA. An analysis of should take the SA environment into consideration for a better understanding of the constraints that govern SA.