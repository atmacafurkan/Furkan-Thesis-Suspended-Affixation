\subsection{Mari}

Mari is an Eastern Uralic language that has a rather interesting set of data when it comes to SA. \cite{guseva2017postsyntactic} (GW henceforth) provide some examples and analysis for SA in Mari. In (\ref{mariSA}) I give examples of SA from Mari. Previous observations of SA have shown that it is a rightward-bound process, but the examples in (\ref{mariSA}) show SA that is not rightward-bound.

\begin{exe}
    \ex \label{mariSA}
    \begin{xlist}
        \ex SA of {\Iness}\\*
        \gll Üder mej-en u\u{s}e-m den tej-en süm-e\u{s}te-t. \\ 
        girl {\Fsg}-{\Gen} mind-{\Poss}.{\Fsg} {\And} {\Ssg}-{\Gen} heart-{\Iness}-{\Poss}.{\Ssg} \\
        \glt `The girl is in my mind and in your heart'
        
        \ex SA of {\Ill}\\*
        \gll Pjötr kart-em mej-en perde\u{z}-em den omsa-\u{s}ke-\u{z}e pi\u{z}ekta \\ 
        Peter map-{\Acc} {\Fsg}-{\Gen} door-{\Poss}.{\Fsg} {\And} wall-{\Ill}-{\Poss}.{\Tsg} pin.{\Tsg}.{\Prs}\\
        \glt `Peter pins maps to my door and his wall'
   
        \ex SA of {\Pl}-{\Iness}\\*
        \gll A-vlak tud-en sad-\u{s}e den memn-an pasu-vlak-e\u{s}te-na mod-et \\ 
        child-{\Pl} {\Tsg}-{\Gen} garden-{\Poss}.{\Tsg} {\And} {\Fpl}-{\Gen} field-{\Pl}-{\Iness}-{\Poss}.{\Fpl} play-{\Tpl}.{\Prs} \\
        \glt `The children are playing in his garden and in our fields'\\*
        \hfill Adapted from \cite{guseva2017postsyntactic}
    \end{xlist}     
\end{exe}

For a clear illustration of the SA examples in (\ref{mariSA}) I give the abstract representation of SA for each one of the examples in (\ref{maritemplate}).

\begin{exe}
    \ex \label{maritemplate}
        \begin{xlist}
        \ex N1-[{\Iness}]-{\Poss} conjoiner N2-{\Iness}-{\Poss}
        \ex N1-[{\Ill}]-{\Poss} conjoiner N2-{\Ill}-{\Poss}
        \ex N1-[{\Pl}-{\Iness}]-{\Poss} conjoiner N2-{\Pl}-{\Iness}-{\Poss}
    \end{xlist}
\end{exe}

This peculiar SA should not be taken as an evidence against its rightward-bound nature. In Mari, the order of the morphemes in the nominal domain show a relatively free order. The morphemes in question are {\Pl}, {\Poss}, Structural and Local cases ({\Scase} and {\Lcase} in glosses respectively). Table \ref{tab:mariorder} shows some possible orders of these morphemes. There is an optional positioning for the {\Poss} marker. The {\Poss} either occupies the left or the right edge of the morphemes, where the right edge can only build up to the {\Scase}. It is a barrier that {\Poss} cannot alternate to the right of.

\begin{table}[hbt!]
    \caption{Mari Nominal Domain Morpheme Order}
    \centering
    \begin{tabular}{|ll|}
    \hline 
        {\Pl} \textgreater {\Poss} & \textit{pasu-vlak-na}  \\
        {\Poss} \textgreater {\Pl} & \textit{pasu-na-vlak} \\ \hline
        {\Pl} \textgreater {\Lcase} & \textit{pasu-vlak-e\u{s}te} \\
        {\Pl} \textgreater {\Scase} & \textit{pasu-vlak-em} \\ \hline
        {\Lcase} \textgreater {\Poss} & \textit{pasu-\u{s}te-na} \\
        {\Poss} \textgreater {\Scase} & \textit{pasu-na-m} \\ \hline
        {\Pl} \textgreater {\Lcase} \textgreater {\Poss} & \textit{pasu-vlak-e\u{s}te-na} \\
        {\Poss} \textgreater {\Pl} \textgreater {\Lcase} & \textit{?pasu-na-vlak-e\u{s}te} \\ \hline
        {\Pl} \textgreater {\Poss} \textgreater {\Scase} & \textit{pasu-vlak-na-m} \\
        {\Poss} \textgreater {\Pl} \textgreater {\Scase} & \textit{pasu-na-vlak-em} \\ \hline 
        \multicolumn{2}{|l|}{\textit{pasu} `garden', \textit{-vlak} {\Pl}, \textit{-na} {\Poss}.{\Fpl}, \textit{-(e)\u{s}te} {\Iness}, \textit{-(e)m} {\Acc}} \\
        \hline 
    \end{tabular}
    \label{tab:mariorder} \\
    ${}$ \hfill Adapted from \cite{guseva2017postsyntactic}
\end{table}

In Mari, there are two linearizations of {\Poss}, {\Pl} and {\Lcase}. SA with the surface orderings of {\Lcase}-{\Poss} and {\Pl}-{\Lcase}-{\Poss} goes against the rightward-bound constraint, but this observation overlooks the other possible orders of {\Poss}-{\Lcase} and {\Poss}-{\Pl}-{\Lcase}. This ambiguous ordering of morphemes is the clue to understanding in what level of derivation SA takes place. This is the point that GW show with an example, adapted here as (\ref{mariSA3}).

\begin{exe}
 \ex \label{mariSA3}
    \begin{xlist}
        \ex \label{mariSA3a}
        \gll 
        \textit{Pörjeng} \textit{memnam} \textit{da} \textit{nunem} \textit{u\u{z}-e\u{s}} \\ man.{\Nom} us.{\Acc} {\And} them.{\Acc} see-{\Tsg}.{\Prs} \\
        
        \ex \label{mariSA3c}
        \gll 
        \textit{*Pörjeng} \textit{me} \textit{da} \textit{nunem} \textit{u\u{z}-e\u{s}} \\ man.{\Nom} us.{\Acc} {\And} them.{\Acc} see-{\Tsg}.{\Prs} \\
        
        \ex \label{mariSA3b} 
        \gll
        \textit{Pörjeng} \textit{memna} \textit{da} \textit{nunem} \textit{u\u{z}-e\u{s}} \\ man.{\Nom} us {\And} them.{\Acc} see-{\Tsg}.{\Prs} \\
        \glt `The man sees us and them'\\*
        \hfill Adapted from \cite{guseva2017postsyntactic}
    \end{xlist}
\end{exe}

The {\Fpl} pronoun is \textit{me} in Mari, and the stem for {\Acc} changes from \textit{me} to \textit{memna}. SA is not possible with \textit{me}, but it is possible with the plural stem \textit{memna}. A similar base or stem change in Turkish also happens when {\Fsg} and {\Ssg} pronouns are used with {\Dat} (\textit{ben \textgreater bana, sen \textgreater sana}). Turkish does not allow SA in such instances (\ref{mariturkish}), with or without base or stem change.

\begin{exe}
    \ex \label{mariturkish}
    \begin{xlist}
        \ex SA with unchanged base\\*
        \gll *Ben ve san-a kitab-ı bul-du. \\ 
        {\Fsg} {\And}  {\Ssg}-{\Dat} book-{\Acc} buy-{\Pst}[{\Tsg}] \\
        \glt ${}$
        
        \ex SA with base change\\*
        \gll *Ban ve san-a kitab-ı bul-du. \\ 
        {\Fsg} {\And}  {\Ssg}-{\Dat} book-{\Acc} buy-{\Pst}[{\Tsg}] \\
        \glt ${}$
        
        \ex No SA\\*
        \gll Ban-a ve san-a kitab-ı bul-du. \\ 
        {\Fsg}-{\Dat} {\And}  {\Ssg}-{\Dat} book-{\Acc} buy-{\Pst}[{\Tsg}] \\
        \glt `S/he bought the book for me and you'
    \end{xlist}
\end{exe}

GW go on to analyze SA in Mari with proposed projections for {\Poss}, {\Pl}, and {\Case} as NumP, DP, and KP. Following \cite{merchant2015ineffable} they propose an underlying order like (\ref{mariordering}). Onto this order a process of D-lowering takes place and the new ordering looks like (\ref{mariorderinglowered}). It is at the order of (\ref{mariorderinglowered}) that SA marks morphemes for zero exponance (shown with a subscript 0) as in (\ref{mariSAmark}). Later a D-metathesis is performed and the ordering for vocabulary insertion looks like (\ref{mariSAform}). This is how the suffix orderings in (\ref{mariSA}) are achieved, an example is partly repeated here.
\begin{exe}
    \ex \begin{xlist}
    \ex \label{mariordering}
    [[[ NP ] Num ]$_{NumP}$D]$_{DP}$ K ]$_{KP}$ \hfill Underlying Order
    
     \ex \label{mariorderinglowered}
    [[[ NP ] D Num ]$_{NumP}$ t$_D$ ]$_{DP}$ K ]$_{KP}$ \hfill D-Lowering
    
    \ex \label{mariSAmark}
    [[[ NP ] D Num$_0$ ]$_{NumP}$ t$_D$ ]$_{DP}$ K$_0$]$_{KP}$ \hfill SA marking
    
    \ex \label{mariSAform}
    [[[ NP ] D K$_0$ Num$_0$ ]$_{NumP}$ t$_D$ ]$_{DP}$ t$_K$ ]$_{KP}$ \hfill D-Metathesis
    \end{xlist}
     ${}$ \hfill Adapted from \cite{guseva2017postsyntactic}

    \exi{(\ref{mariSA}')} SA of {\Pl-\Iness}\\*
    \gll tud-en sad-\u{s}e den memn-an pasu-vlak-e\u{s}te-na \\ 
    {\Tsg}-{\Gen} garden-{\Poss}.{\Tsg} {\And} {\Fpl}-{\Gen} field-{\Pl}-{\Iness}-{\Poss}.{\Fpl}\\ 
    \glt `\ldots in his garden and in our fields'
\end{exe}

There are important observations to be made in \cite{guseva2017postsyntactic}. First, the examples in (\ref{mariSA}) show that SA is not performed at the surface form. This observation is vital to distinguish SA from Backward Ellipsis in Turkish because Backward Ellipsis takes the surface form into account. Second, (\ref{mariSA3}) shows that SA does not operate morphemes on a derivational level before their phonological representations are in place, yet (\ref{mariturkish}) show that even taking those representations into account does not result in a successful SA in Turkish. 