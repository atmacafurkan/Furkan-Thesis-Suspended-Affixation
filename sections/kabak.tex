\subsection{\cite{kabak2007turkish}}
Among the papers discussing SA in Turkish, Kabak's paper seems to be the most extensive in terms of providing how SA can take shape in both verbal and nominal domains. The paper provides the conditions in which we can expect SA. The analysis of Kabak relies on the definition of a morphological word. Kabak claims that any inflectional morpheme can be suspended as long as the remainder is a morphological word. Kabak proposes the following:
\begin{itemize}
    \item Terminal suffix: \textit{A suffix that is allowed to appear at the end of a word, where further affixation is not obligatory.}
\end{itemize}
He claims that only terminal suffixes can be suspended. He posits that bare verbs are not morphological words in Turkish. He provides Table (\ref{tab:terminalmorphemes}) for Verbal terminal suffixes.
\begin{table}[hbt!]
\caption{Verbal Terminal Morphemes}
    \centering
    \begin{tabular}{|ll|}
    \hline 
                                    {(i) Agreement markers} &  \\ \hline
         \multirow{5}{20em}{(ii) Aspect/ Modality markers}  & {\Aor} \textit{-(I)r/(A)r} \\ 
                                                            & {\Prog} \textit{-Iyor} \\
                                                            & {\Fut} \textit{-(y)AcAK} \\
                                                            & {\Evi} \textit{-mIş} \\
                                                            & {\Nec} \textit{-mAlI} \\ \hline
        \multirow{2}{20em}{(iii) Converb markers}           & \textit{-(y)IncA} \\
                                                            & \textit{-(y)Ip} \\
    \hline                                                         
    \end{tabular}
    \label{tab:terminalmorphemes}
\begin{flushright}
    Adapted from \cite{kabak2007turkish}
\end{flushright}
\end{table}

If any suspension attempt is made with these morphemes, it is only permitted under the condition that what is left is a morphological word. Kabak classifies such clitics \textit{=mI}, and \textit{=DA} as non-terminal but also recognizes their ability to end an expression in Turkish (\ref{clitics}).

\begin{exe}
    \ex \label{clitics}
    \begin{xlist}
    \begin{multicols}{2}
        \ex
        \gll
        \textit{koş-tu-n} \textit{mu?} \\ run-{{\Pst}}-{\Second}{\Sg} =Q \\
        \glt `Did you run?'
        
        \ex 
        \gll
        \textit{ağla-mış-sın} \textit{da} \\ cry-{\Evi}-{\Second}{\Sg} =TOO \\ 
        \glt `It looks like you have cried also.'
    \end{multicols}
    \end{xlist}
    \hfill Adapted from \cite{kabak2007turkish}
\end{exe}
Kabak argues against \cite{kornfilt1996some}'s formulation for SA (\ref{kornfiltsa}) with two examples. According to \cite{kornfilt1996some}'s analysis only the copular forms and further inflectional morphemes can be suspended.
\begin{exe}
    \ex \label{kornfiltsa}
    [V$_{Participle}$ conjunction V$_{Participle}$] + V$_{Copula}$ + Inflectional Morphemes
\end{exe}

First, some forms that can be complements of copula are not participles and do not always give way to grammatical instances of SA. Although \cite{kornfilt1996some} does not define \textit{-DI} as a participle, it is still able to be a complement to a copular \textit{i}. Hence an expected SA in (\ref{kornfiltpart}) where the suspension is applied to the copula and not the participle.

\begin{exe}
    \ex \label{kornfiltpart}
    \begin{xlist}
        \ex
        \gll
        \textit{*O} \textit{yaz} \textit{Finike-ye} \textit{git-ti} \textit{ve} \textit{deniz-e} \textit{gir-di-y-di-k.} \\ that summer Finike-{\Dat} go-{{\Pst}} {\And} sea-{\Dat} enter-{{\Pst}}-{\Cop}-{{\Pst}}-{\First}.{\Pl} \\
        
        \ex 
        \gll 
        \textit{*Ev-im-iz-i} \textit{sat-sa} \textit{ve} \textit{bir} \textit{dükkan} \textit{al-sa-y-dı-k.} \\ house-1-{\Pl}-{\Acc}. sell-{\Cond}. {\And} a shop buy-{\Cond}.-{\Cop}-{{\Pst}}-{\First}.{\Pl} \\
    \end{xlist}
\hfill Adapted from \cite{kabak2007turkish}
\end{exe}

Second, at least one suffix deemed participle, namely the necessitative marker \textit{-mAlI}, does not behave like a participle that can modify NPs unlike other participles like \textit{-mIş} and \textit{-(y)AcAK} (\ref{kabakmali}). It should be noted however, contra Kabak, not all participle forms can modify nouns for example \textit{-Iyor}, and despite a lack of modifying capability, \textit{-mAlI} acts as predicted by \cite{kornfilt1996some}.
\begin{exe}
    \ex \label{kabakmali}
    \gll 
    \textit{*çalış-malı} \textit{adam} \\ work-{\Nec}. man \\
\end{exe}
% Kornfilt deems any inflection that is a complement of copula as participle, and Kabak points to this generalization being faulty in accurately predicting instances of SA in verbal domain.

Another point Kabak provides, contradicting Orgun, is the suspension of possessive when coming after a plural marker (\ref{plurposs}).

\begin{exe}
    \ex \label{plurposs}
    \gll
    \textit{Asker-ler} \textit{ve} \textit{komutan-lar-ım-ız.} \\ soldier-{\Pl} {\And} commander-{\Pl}-{\Poss}.{\First}-{\Pl} \\
    \glt `Our soldiers and commanders'
    \hfill Adapted from \cite{kabak2007turkish}
\end{exe}

Kabak provides an approach that is rather interesting. He makes an observation from \cite{good2005morphosyntax}, in the spirit of \cite{erdal2000clitics}, about the agreement paradigms in Turkish. In his citing, Kabak says that z-paradigm of agreement markers contain cliticized forms of words, and k-paradigm of agreement has lexical suffixes, thereby explaining the ungrammaticalities in (\ref{kornfiltpart}). Kabak also realizes the shortcomings of this approach and notes there are constructions in which k-paradigm SA is applicable and other conditions where z-paradigm SA is not applicable. As a last summary of Kabak's observations, he gives the following points in verbal domain SA:
\begin{exe}
\sn \begin{xlisti}
    \ex the ability of a verbal morpheme to terminate a word is related to its ability to stand without an agreement marker
    \ex SA is only applicable if what is left after suspension is a morphological word, and both the conjuncts end with terminal morphemes
    \ex Conjuncts with cliticlike endings are interpreted as 3$^{rd}$ person singular, causing agreement mismatches in SA
    \ex Nonfinal conjunct's terminal suffix must be overt
    \end{xlisti}
    \hfill Adapted from \cite{kabak2007turkish}
\end{exe}

Kabak recognizes that in SA what is relevant is actually the size of what is left after SA. Not on the lower end of the size, but also at the higher end. However the `cliticlike' condition on his third point is not clear-cut, and can be extended to other suffixes which have 3$^{rd}$ person singular suffixes which allow SA, which seemingly can end a word without copula. \textit{-mIş}, \textit{-(y)AcAK}, and \textit{-Iyor} to name a few. This condition relies heavily on what is `cliticlike' and gives way to wrong predictions in SA. His explanations are not based on a theoretical framework, which makes some of the explanations arbitrary. 