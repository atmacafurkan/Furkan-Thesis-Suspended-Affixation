\section{Summary}

As a summary of the literature presented in this chapter, I provide the following observations about SA:

\begin{itemize}
    \item It is a rightward bound process in the underlying morpheme order: Examples provided in \cite{kabak2007turkish}, \cite{pounder2006broken}, and \cite{guseva2017postsyntactic} show this for Turkish, German, and Mari.
    
    \item It is found in both inflectional and derivational paradigms: Examples provided in \cite{kabak2007turkish}, and \cite{yoon2017lexical} show this for Turkish and Korean.

    \item It takes place after VI and before phonological readjustments: Examples provided in  \cite{pounder2006broken}, \cite{guseva2017postsyntactic}, and \cite{erschler2018suspended} show this for German, Mari and Ossetic.
\end{itemize}

These are the observations that seem to be consistent in all the papers. However, not all the papers align in the structural analysis of SA. While the dominant account for Turkish seems to be structural sharing in nature \citep{orgun1995flat,kornfilt1996some,broadwell2008turkish,kornfilt2012revisiting}. This process is in line with \cite{ackema2004beyond,kunduraci2016morphology, bruening2018word} since in such form of language derivation an output of syntax can become an input for morphology and word formation. other accounts provided for other languages like Serbian, Mari, and Ossetic are all ellipsis analyses \citep{despic2017suspended,guseva2017postsyntactic,erschler2018suspended}. The summary of the literature for Turkish SA presents the following points to be addressed for any further study.

A study for Turkish SA needs to address the following issues:

\begin{itemize}
    \item Is SA of derivational suffixes is possible in Turkish? If so how, if not why?
    \item How  does SA respond to different conjoiners? 
    \item How a language user determines s/he is in an SA configuration?
\end{itemize}

It is the aim of this thesis to scrutinize these issues and contribute to the literature in an orderly and comprehensive manner.