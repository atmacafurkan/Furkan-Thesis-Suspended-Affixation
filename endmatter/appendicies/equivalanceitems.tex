\chapter{\MakeUppercase{experiment 3: self paced items}} \label{equivalanceitems}
\singlespacing
\section{Sentences}
\begin{myparindent}{0pt}
1\_Bence baron ve (cesur) şövalyeyi ödüllendiren kral birbirlerini/onları şatoda dinleyecek. \\
2\_Sanırım hemşire ve (zavallı) hastayı gören doktor birbirlerine/onlara ameliyatı hatırlatacak. \\
3\_Umarım muhabir ve (endişeli) görevliyi duyan bakan birbirlerini/onları toplantıda uyaracak. \\
4\_Mesela antrenör ve (utangaç) dansçıyı tanıyan sporcu birbirlerini/onları yarışmaya kaydedecek. \\
5\_Sözde adam ve (acemi) garsonu farkeden kadın birbirlerini/onları restoranda ağırlayacak. \\
6\_Öte yandan akademisyen ve (unutkan) sekreteri uyaran rektör birbirlerini/onları törene çağıracak. \\
7\_Anlaşılan kuyumcu ve (çaresiz) dönerciyi dolandıran hırsız birbirlerini/onları polise ihbar ediyor. \\
8\_Allahtan öğretmen ve (çalışkan) öğrenciyi unutan müdür birbirlerini/onları okulda görüyor. \\
9\_Belli ki yazar ve (eski) editörü arayan şair birbirlerine/onlara sokakta sesleniyor. \\
10\_Neyse ki adam ve (sakar) çırağı çağıran çilingir birbirlerine/onlara kapıyı gösteriyor. \\
11\_Anlaşılan hostes ve (genç) yolcuyu beğenen pilot birbirlerini/onları uçakta gözetliyor. \\
12\_Aslında hamal ve (kurnaz) tezgahtarı bulan sütçü birbirlerini/onları pazarda kandırıyor. \\
13\_Bu arada yarışmacı ve (şirin) sunucuyu eleştiren seyirci birbirlerini/onları salondan çıkardı. \\
14\_Demek ki şehzade ve (temkinli) veziri dinleyen padişah birbirlerini/onları savaşa uğurladı. \\
15\_Mesela subay ve (dikkatsiz) çavuşu fırçalayan general birbirlerine/onlara dikkatlice baktı. \\
16\_Ne yazık ki profesör ve (azimli) asistanı araştıran dekan birbirlerini/onları projeden vazgeçirdi. \\
17\_İyi ki veznedar ve (iyimser) cerrahı kandıran yatırımcı birbirlerini/onları satıştan caydırdı. \\
18\_Aslında büyükelçi ve (şaşkın) çevirmeni bekleyen başbakan birbirlerini/onları görüşmeye davet etti. \\
19\_Sözde müzisyen ve (alımlı) modele yaklaşan aktör birbirlerine/onlara duyuruyu okuyacak. \\
20\_Kısacası oyuncu ve (uzun) kameramana seslenen yönetmen birbirlerine/onlara sahneyi izletecek. \\
21\_Umarım yatırımcı ve (yorgun) işçiye bakan mühendis birbirlerini/onları planlarla bilgilendirecek. \\
22\_Bence fotoğrafçı ve (geçimsiz) berbere bağıran gözlükçü birbirleriyle/onlarla kahvede karşılaşacak. \\
23\_Belli ki asker ve (kaygısız) bekçiye kızan komutan birbirlerine/onlara kışlayı gösteriyor. \\
24\_Yani kasap ve (çevik) elemana güvenen pazarcı birbirlerine/onlara dükkanı emanet ediyor. \\
25\_Öte yandan marangoz ve (konuşkan) ustabaşına takılan mimar birbirlerinden/onlardan siparişi alıyor. \\
26\_Neyse ki çiçekçi ve (güleç) rehbere danışan turist birbirlerine/onlara hediyeyi seçtiriyor. \\
27\_Allahtan matbaacı ve (özensiz) teknisyene gücenen yayımcı birbirlerini/onları atölyede gördü. \\
28\_Maalesef muavin ve (özenli) biletçiye karışan şoför birbirlerini/onları güzergahtan vazgeçirdi. \\
29\_Demek ki madenci ve (ılımlı) müfettişe ulaşan tüccar birbirlerini/onları telefonla aradı. \\
30\_Üstelik denizci ve (durgun) balıkçıya inanan kaptan birbirlerinden/onlardan rotayı öğrendi. \\
31\_Nedense avukat ve (insaflı) hakime güvenen patron birbirlerinden/onlardan güvence istedi. \\
32\_Yani manav ve (saygılı) delikanlıdan bahseden terzi birbirleriyle/onlarla dalga geçiyor. \\
33\_Nedense piyanist ve (hevesli) seyirciden utanan şarkıcı birbirlerini/onları sahneye çağırıyor. \\
34\_Üstelik kadın ve (sakin) muhtardan çekinen köylü birbirlerini/onları yemeğe buyur ediyor. \\
35\_İyi ki hizmetli ve (hırslı) yöneticiden korkan aşçı birbirlerine/onlara tarifi verdi. \\
36\_Kısacası kiracı ve (dalgın) kapıcıdan bıkan emlakçı birbirlerine/onlara daireyi gezdirdi. \\
37\_Hiç değilse futbolcu ve (taraflı) hakemden usanan direktör birbirlerini/onları yönetime şikayet etti. \\
38\_Sanırım çiftçi ve (sinirli) çobandan kaçan imam birbirleriyle/onlarla meydanda karşılaştı. \\
39\_Maalesef papaz ve (bilgili) rahibeden uzaklaşan filozof birbirlerini/onları tartışmaya zorladı. \\
40\_Hiç değilse hizmetçi ve (ihmalkâr) kiracıdan huylanan tesisatçı birbirlerine/onlara tadilatı tarif etti. \\
101\_Kilitli eve giden sevimli kadın gerisin geri döndü. \\
102\_Karmaşık sokakta yalnız kalan çaresiz adam çok korktu. \\
103\_Kitabını ve defterini unutan çocuk okula geç geldi. \\
104\_Gereksiz konuları merak eden adam fazlasıyla vakit kaybediyor. \\
105\_Sabırsız öğretmeni gören okul müdürü hemen çocuklara seslendi. \\
106\_Sıcakkanlı aşçıya bakan görevli temizliğe biraz yardım etti. \\
107\_Daireyi ve binayı temizleyen kapıcının maaşına zam yapıldı. \\
108\_Yeni ameliyat olan hastayı hemşire çok fena azarladı. \\
109\_Görüşmeye geç kalan öğrenci ödevini zamanında teslim edemedi. \\
110\_Feci kazada ve sonrasında yaralanan olmaması insanları rahatlattı. \\
111\_Emekli öğretmen karşı karşıya kaldığı garip durumu anlayamadı. \\
112\_Arabasını alan ve işlerini bitiren adam gereğinden fazla yoruldu. \\
113\_Defterleriyle kalemlerini evde unutan küçük çocuk durmadan ağladı. \\
114\_Elektrik faturası ve benzin ücretlerinin arttığı bir dönemdeyiz. \\
115\_Şişenin kapağında gördüğü acayip yazıyı dikkatlice okumaya çalıştı. \\
116\_Masasında veya arabasında dağınık bir kağıt parçası bulunamadı. \\
117\_Aşçı pişirdiği yemekleri müdüre ve hizmetliye gururla gösterdi. \\
118\_Korkan hayvanları ahıra götüren çoban az daha ölüyordu. \\
119\_Şarkının notalarını karıştıran tecrübeli piyanist ne yapacağını bilemedi. \\
120\_Maçı kaybeden takımdaki futbolcular değerlendirme yapmak için bekliyorlar. \\
121\_Pistte kalan lastikler uçağın kaza yapmasına sebep olmuş. \\
122\_Kırmızı çizgileri olmayan sporcu çalışmalarını çok sert sürdürüyor. \\
123\_Ahmetle dalga geçen çocuk sonunda hak ettiğini buldu. \\
124\_Çaresiz kalan aşçı depoda kalan son sebzeleri pişirdi. \\
125\_Dinlenmeksizin çalışan işçiler çoktandır mola vermek için sabırsızlanıyor. \\
126\_Yırtılmış gömleklerini tamir ettirmek için terziye gitmiş olmalı. \\
127\_Olay yerine gelen polisler öncelikle şüphelinin eşkalini belirledi. \\
128\_Yaya geçidinde kırmızı ışığın yanmasını beklemeyen şoförler vardı. \\
129\_Sayfalarını karıştırdığı kitabı bir kenara koyup uyumaya başladı. \\
130\_Görevliler elektrik hattını tehlikeye sokan ağaç dallarını kestiler. \\
131\_Çığ tehlikesinin yüksek olduğu yollarda karayolları önlem almalı. \\
132\_Fotokopi makinasının mürekkebini değiştirmek için yeni kartuş gerekli. \\
133\_Muavinin tarif ettiği yol üzerinde dinlenme tesisi yok. \\
134\_Hostes uçuş başlamadan önce güvenlik yönergelerinin hepsini anlattı. \\
135\_Çiçekçinin sattığı laleler çok çeşitli renk ve türlerden. \\
136\_Balıkçılar derneğinin yaptığı duyuruda kotalar protesto ediliyor. \\
137\_Gerekli izinleri alan maden şirketi kazı çalışmalarına başladı. \\
138\_Sporcuları çalıştıran antrenör takımının performansından pek memnun değil. \\
139\_Çevirmenlere iş veren şirket maaşları doğru zamanda yatırmamış. \\
140\_Boyadığı tabloları sergiye çıkaran ressam fazlasıyla gururlanıyor. \\
141\_Yaz tatilini geçirmek için gittiği tatil yerinden esmerleşerek geldi. \\
142\_Karpuzu tamamen püre haline getirdikten sonra yavaşça karışıma eklemelisin. \\
143\_Ateşi yükselen bebeği hastaneye yetiştirmek için hemen arabaya koştuk. \\
144\_Konuyla ilgili açıklama yapması beklenen bakan toplantıyı terk etti. \\
145\_Okulda ve kursta işlenen konuları dikkatlice takip etmek zorundasın. \\
146\_Projeye dahil edilen konuları not almak en öncelikli işimiz. \\
147\_Mühendisler odasının hazırladığı rapora göre inşaatın zemini düzgün yapılmamış. \\
148\_Haberlerde adı geçen dönercinin ürünlerinde birçok katkı maddesi bulunmuş. \\
149\_Son yıllardaki seller giderek daha fazla zarara sebep oluyor. \\
150\_Kaptan gemideki insanları ve kargoyu korumak için demir attı. \\
151\_Yeşil ışığın yanmasıyla hızlanan araba ve motosiklet feci çarpıştı. \\
152\_Kırık sandalyeleri tamir eden marangoz çok çalıştığını söylemekten çekinmiyor. \\
153\_Kelebeğin türünü üstündeki şekiller veya kanadının şeklinden anlamaya çalışabiliriz. \\
154\_Atık sularla kirlenen ve hırpalanan dereyi temizleme işlemleri yetersiz. \\
155\_Yatırımcılar kentsel dönüşüm kapsamında yıkılan yerleri fırsat olarak görüyor. \\
156\_Film sahnesinde yeterli rolü olmayan oyuncu senariste içten yakındı. \\
157\_Kuyumcunun getirdiği bileziklerin işlemeleri gelinin ailesi tarafından çok beğenildi. \\
158\_Fotoğrafçı tek başına çektiği tüm fotoğrafları arkadaşlarıyla internetten paylaştı. \\
159\_Hep beraber gittiğimiz piknikte oynadığımız oyunları çok net hatırlıyorum. \\
160\_Sekreterin kaybettiği dosyaları tek başına arayan profesör çok sinirlendi. \\
161\_Kitapları yerine dizmekten yorulan kütüphaneci insanlardan sessiz olmalarını istiyor. \\
162\_Köşeleri eskimiş çantasını koluna geçiren doktor acilen yola çıkıyor. \\
163\_Tezgahtarın önündeki kumaşlara bakmak isteyen müşteri sesini duyurmaya çalışıyor. \\
164\_Çırağın yanlış bağladığı kabloları düzelten teknisyen çok vakit kaybetti. \\
165\_Vezirin tavsiyelerine kulak asmayan padişah orduyla birlikte sefere çıktı. \\
166\_Hastanın ameliyatı sırasında gelişen durumdan ötürü cerrah operasyonu bitirdi. \\
167\_Dersi biten öğrenciler tatillerini geçirmek üzere ailelerinin yanına gidecek. \\
168\_Hizmetli camların temizliğini bitirdikten sonra oturma odasının temizliğine başlayacak. \\
169\_Işıkları düzgün yanmayan binanın elektrik hattında problem olduğu anlaşıldı. \\
170\_Burada yaşayan köylüler kahveyi kavurduktan sonra elleriyle saatlerce dövüyorlar. \\
171\_Turistleri dolandıran rehberleri yakalayan polis basına açıklama yapmaktan kaçındı. \\
172\_Haftasonu partiye gidecek öğrenciler yanlarında yiyecek ve içecek getirmeli. \\
173\_Sokaklarda dolaşan çocukların sağlıklı büyümesi için oyun parkları yapılmalı. \\
174\_Programın yazılı olduğu ajandayı unutan sekreter kendine çok kızdı. \\
175\_Veznedar bankaya gelen müşteriye imzanlaması gereken belgeleri usulca uzattı. \\
176\_Sokak üzerindeki olağan devriyeye takılan hırsız birden kaçmaya başladı. \\
177\_Tasarımlarını arkadaşlarına gösteren çizer giysileri hemen dikmek istiyor. \\
178\_Yola çıkmadan önce hazırlıklarını tamamlayan kaptan geminin yükünü azalttı. \\
179\_Şehirdeki toplum düzenine katkıda bulunması amacıyla halk kursları açılıyor. \\
180\_Deprem sonrası oluşan hasarların tespiti için mahalleye uzmanlar gönderildi. \\
\end{myparindent}
\section{Questions}
\begin{myparindent}{0pt}
1\_subject correct\_Baron kralı dinleyecek. \\
2\_subject correct\_Hemşire doktora ameliyatı hatırlatacak. \\
3\_subject correct\_Muhabir bakanı uyaracak. \\
4\_subject correct\_Antrenör sporcuyu yarışmaya kaydedecek. \\
5\_subject correct\_Adam kadını ağırlayacak. \\
6\_subject correct\_Akademisyen rektörü törene çağıracak. \\
7\_subject correct\_Kuyumcu hırsızı ihbar ediyor. \\
8\_subject correct\_Öğretmen müdürü görüyor. \\
9\_subject correct\_Yazar şaire sesleniyor. \\
10\_subject correct\_Adam çilingire kapıyı gösteriyor. \\
11\_subject correct\_Hostes pilotu gözetliyor. \\
12\_subject correct\_Hamal sütçüyü kandırıyor. \\
13\_subject correct\_Yarışmacı seyirciyi salondan çıkardı. \\
14\_subject correct\_Şehzade padişahı uğurladı. \\
15\_subject correct\_Subay generale baktı. \\
16\_subject correct\_Profesör dekanı vazgeçirdi. \\
17\_subject correct\_Veznedar yatırımcıyı caydırdı. \\
18\_subject correct\_Büyükelçi başbakanı görüşmeye davet etti. \\
19\_subject correct\_Müzisyen aktöre duyuruyu okuyacak. \\
20\_subject correct\_Oyuncu yönetmene sahneyi izletecek. \\
21\_object correct\_Mühendis yatırımcıya baktı. \\
22\_object correct\_Gözlükçü fotoğrafçıya bağırdı. \\
23\_object correct\_Komutan askere kızdı. \\
24\_object correct\_Pazarcı kasaba güvenmiş. \\
25\_object correct\_Mimar marangoza takılmış. \\
26\_object correct\_Turist çiçekçiye danıştı. \\
27\_object correct\_Yayımcı matbaacıya gücenmiş. \\
28\_object correct\_Şoför muavine karıştı. \\
29\_object correct\_Tüccar madenciye ulaştı. \\
30\_object correct\_Kaptan denizciye inanmış. \\
31\_object correct\_Patron avukate güvendi. \\
32\_object correct\_Terzi manavdan bahsetti. \\
33\_object correct\_Şarkıcı piyanistten utandı. \\
34\_object correct\_Köylü kadından çekiniyor. \\
35\_object correct\_Aşçı hizmetliden korkuyor. \\
36\_object correct\_Emlakçı kiracıdan bıkmış. \\
37\_object correct\_Direktör futbolcudan usanmış. \\
38\_object correct\_İmam çiftçiden kaçmış. \\
39\_object correct\_Filozof papazdan uzaklaşmış. \\
40\_object correct\_Tesisatçı hizmetçiden huylanmış. \\
101\_correct\_Kadın kilitli eve gitti. \\
102\_correct\_Adam sokakta yalnız kaldı. \\
103\_correct\_Çocuk defteri unutmuş. \\
104\_correct\_Adam vakit kaybediyor. \\
105\_correct\_Öğretmen sabırsızmış. \\
106\_correct\_Görevli temizliğe yardım etti. \\
107\_correct\_Kapıcı binayı temizlemiş. \\
108\_correct\_Hasta yeni ameliyat olmuş. \\
109\_correct\_Öğrenci görüşmeye geç kalmış. \\
110\_correct\_Yaralı olmaması insanları rahatlattı. \\
111\_correct\_Öğretmen durumu anlamadı. \\
112\_correct\_Adam işlerini bitirmiş. \\
113\_correct\_Çocuk durmadan ağladı. \\
114\_correct\_Benzin ücretleri artıyor. \\
115\_correct\_Şişenin kapağında yazı var. \\
116\_correct\_Kağıt parçası bulunamadı. \\
117\_correct\_Aşçı yemekleri gösterdi. \\
118\_correct\_Hayvanlar korkmuş. \\
119\_correct\_Piyanist notaları karıştırmış. \\
120\_correct\_Takım maçı kaybetmiş. \\
121\_incorrect\_Uçak kaza yapmamış. \\
122\_incorrect\_Sporcu çalışmalarını rahat sürdürüyor. \\
123\_incorrect\_Ahmet çocukla dalga geçmiş. \\
124\_incorrect\_Aşçı depodaki etleri pişirdi. \\
125\_incorrect\_İşçiler mola verdi. \\
126\_incorrect\_Gömlekler sağlammış. \\
127\_incorrect\_Polisler öncelikle olay yerini inceledi. \\
128\_incorrect\_Hiçbir şoför ışığı beklemedi. \\
129\_incorrect\_Defteri kenara koydu. \\
130\_incorrect\_Görevliler kabloları kesti. \\
131\_incorrect\_Belediye önlem almalı. \\
132\_incorrect\_Mürekkebi doldurmak için kartuş gerekli. \\
133\_incorrect\_Yol üzerinde dinlenme tesisi var. \\
134\_incorrect\_Hostes bazı yönergeleri anlattı. \\
135\_incorrect\_Çiçekçi gül satıyormuş. \\
136\_incorrect\_Balıkçılar yasağı protesto ediyor. \\
137\_incorrect\_Şirket keşif çalışmalarına başladı. \\
138\_incorrect\_Antrenör performanstan memnun. \\
139\_incorrect\_Şirket doktorlara iş veriyor. \\
140\_incorrect\_Ressam tabloları satmış. \\
141\_correct\_Birisi yaz tatiline gitmiş. \\
142\_correct\_Karpuzu püre halinde eklenmeli. \\
143\_correct\_Bebeğin ateşi yükselmiş. \\
144\_correct\_Bakandan açıklama bekleniyormuş. \\
145\_correct\_Konuları takip etmelisin. \\
146\_correct\_Proje konuları not edilmeli. \\
147\_correct\_Mühendisler odası rapor hazırlamış. \\
148\_correct\_Ürünlerde katkı maddesi varmış. \\
149\_correct\_Sellerin verdiği zarar artıyor. \\
150\_correct\_Kaptan demir attı. \\
151\_correct\_Araba ve motosiklet çarpıştı. \\
152\_correct\_Marangoz sandalye tamir ediyormuş. \\
153\_correct\_Kelebeğin kanatlarında şekil varmış. \\
154\_correct\_Dere temizliği yetersizmiş. \\
155\_correct\_Yıkılan yerler varmış. \\
156\_correct\_Oyuncu senariste yakınmış. \\
157\_correct\_Kuyumcu bilezik getirmiş. \\
158\_correct\_Fotoğrafçı fotoğrafları paylaşmış. \\
159\_correct\_Oynadığı oyunları hatırlıyormuş. \\
160\_correct\_Sekreter dosyaları kaybetmiş. \\
161\_incorrect\_Kütüphaneci kitap dizmemiş. \\
162\_incorrect\_Çantaların köşesi yeni. \\
163\_incorrect\_Giysiler tezgahtarın önünde. \\
164\_incorrect\_Çırak kabloyu doğru bağlamış. \\
165\_incorrect\_Padişah vezirin sözünü dinledi. \\
166\_incorrect\_Cerrah operasyona devam etti. \\
167\_incorrect\_Öğrencilerin dersi bitmemiş. \\
168\_incorrect\_Hizmetli misafir odasını temizleyecek. \\
169\_incorrect\_Binanın elektrik hattında sorun yok. \\
170\_incorrect\_Köylüler kahveyi kavurmuyor. \\
171\_incorrect\_Polis rehberi yakalayamamış. \\
172\_incorrect\_Öğrenciler yiyecek getirmemeli. \\
173\_incorrect\_Sokaklarda çocuklar dolaşmıyor. \\
174\_incorrect\_Sekreter ajandayı unutmamış. \\
175\_incorrect\_Banka müdürü belgeleri uzattı. \\
176\_incorrect\_Hırsız sakince yürüdü. \\
177\_incorrect\_Çizer giysileri başkasına diktirecekmiş. \\
178\_incorrect\_Kaptan geminin yükünü arttırdı. \\
179\_incorrect\_Halk kursları kapatılıyor. \\
180\_incorrect\_Mahalleye erzak gönderildi. \\
\end{myparindent}