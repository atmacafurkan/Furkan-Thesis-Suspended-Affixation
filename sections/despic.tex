\subsection{Serbian}

According to \cite{despic2017suspended}, Serbian does not have suspended affixation, but a certain second place clitic shows some similarities to affixes in Serbian. This clitic in turn can take place in SA-like ellipsis. While my study focuses on SA in Turkish, and this section is about SA in other languages. It can not be denied that the examples of SA in Turkish verbal domain always include a relation to the clitic copula \textit{-i/ y/ $\emptyset$} (\ref{verbalSA}).

\begin{exe}
    \ex \label{verbalSA}
    \begin{xlist}
        \ex \label{verbalSA1}
        \gll 
        \textit{Ev-e} \textit{gel-ec\'{e}k} \textit{ve} \textit{uyu-yac\'{a}k-tı-m} \\ house-{\Dat} come-{\Fut} {\And} sleep-{\Fut}={\Cop}.{\Pst}-{\First}.{\Sg} \\
        
        \ex \label{verbalSA2}
        \gll 
        \textit{Ev-e} \textit{gel-ec\'{e}k} \textit{ve} \textit{uyu-yac\'{a}k} \textit{i-di-m.} \\ house-{\Dat} come-{\Fut} {\And} sleep-{\Fut} ={\Cop}-{\Pst}-{\First}.{\Sg} \\
        \glt `I was going to come home and sleep'
    \end{xlist}
\end{exe}

While (\ref{verbalSA1}) shows, an SA of Past tense and agreement markers, a closer look reveals what is actually suspended is a copular form together with tense and agreement markers. This clitic can also have an overt phonological form \textit{i} which also allow for the same SA (\ref{verbalSA2}). The overtness capability of the clitic is not enforced though, it is even ungrammatical in some instances (\ref{nompredSA}). We can arrive at the existence of a clitic from the stress pattern, since the domain of stress is the right edge of the phonological word and not the clitic group in Turkish.

\begin{exe}
    \ex \label{nompredSA}
    \begin{xlist}
        \ex 
        \gll 
        \textit{hast\'{a}} \textit{ve} \textit{yorg\'{u}n-um} \\ sick {\And} tired-{\First}.{\Sg}[{\Prs}] \\
        \glt `I am sick and tired'
    \end{xlist}
\end{exe}

Now to turn to the example in Serbian, the special instance where SA-like process takes place involves the infinitival marker \textit{-ti} and second place future clitic \textit{\'{c}e}. The bare bones explanation for second place clitics is in a clause they occupy the linearly second place. If they are cliticized to the phonological word they are attached to, then the word can occupy the first place in the clause, which by default satisfies the clitic's constraint. If the cliticized word is not at the first place in a clause, clitic detaches from that word and occupies second place.

Here I want to draw a similarity between the infinitival marker \textit{-ti} in Serbian and infinitival marker \textit{-mAK} in Turkish. What appears is that verbs are not free forms in Serbian, just like verbs are not morphological words in Turkish. When the verb is inflected, there is no need for an infinitival marker, the inflection is performed on a truncated form of what is to the left of \textit{-ti} or \textit{-mAK}.

In Serbian some phonological processes are not triggered by clitics, for the example of second place future clitic \textit{\'{c}e}, and phonologically similar diminutive suffix \textit{\'{c}e} abiding by this generalization see (\ref{serbiance}).

\begin{exe}
    \ex \label{serbiance}
    \begin{xlist}

        \ex \gll
        \textit{Vas} \textit{\'{c}e} \textit{videti} \\ you.{\Pl}.{\Acc} ={\Aux}.3.S.{\Fut} see.{\Inf} \\
        \glt `S/he will see you.'
 
        \ex \gll 
        \textit{Pa\u{s}-\'{c}e} \\ dog-{\Dim} \\
        \glt `small dog'

    \end{xlist}
    ${}$ \hfill Adapted from \cite{despic2017suspended}
\end{exe}

In (\ref{serbiance}) both the forms \textit{vas} and \textit{pas} have the same phoneme \textit{/s/}. While a diminutive suffix causes a phonological change from \textit{/s/} to [\textesh], the second place future clitic doesn't. However, when the second place future clitic is cliticized to the bound verb form, it does trigger phonological change like a suffix would in Serbian (\ref{serbiance2}).

\begin{exe}
    \ex \label{serbiance2}
    \begin{xlist}
        \ex 
        \gll 
        \textit{*Jes=\'{c}e\u{s}} \\ eat={\Aux}.{\Second}.S.{\Fut} \\
    
        \ex
        \gll 
        \textit{Je\u{s}=\'{c}e\u{s}} \\ eat={\Aux}.{\Second}.S.{\Fut} \\
        \glt `You will eat.'
    \end{xlist}
    ${}$ \hfill Adapted from \cite{despic2017suspended}
\end{exe}

The SA-like construction in Serbian does not look like the examples of SA in Turkish. First of all it takes place in the second conjunct. An example is given in (\ref{serbianSAlike}).

\begin{exe}
    \ex \label{serbianSAlike}
    \begin{xlist}
        \ex 
        \gll 
        \textit{Oti\'{c}i} \textit{\'{c}e} \textit{i} \textit{pogleda=\'{c}e} \textit{novi} \textit{film.} \\ go.{\Inf} {\Aux}.3.S.{\Fut} {\And} see={\Aux}.3.S.{\Fut} new.{\Acc} film.{\Acc} \\
        
        \ex 
        \gll 
        \textit{*Oti\'{c}i} \textit{\'{c}e} \textit{i} \textit{pogleda} \textit{novi} \textit{film.} \\ go.{\Inf} {\Aux}.3.S.{\Fut} {\And} see new.{\Acc} film.{\Acc} \\
        
        \ex 
        \gll 
        \textit{Oti\'{c}i} \textit{\'{c}e} \textit{i} \textit{pogledati} \textit{novi} \textit{film.} \\ go.{\Inf} {\Aux}.3.S.{\Fut} {\And} see.{\Inf} new.{\Acc} film.{\Acc} \\
        \glt `He will go and see the new movie'
    \end{xlist}
    ${}$ \hfill Adapted from \cite{despic2017suspended}
\end{exe}

The very first observation that can made about the sentences in (\ref{serbianSAlike}) is that what is left after the SA-like process is not a phonological string of what comes before the clitic, since what is left is an infinitival form.


Despi\'{c} goes into an in-depth analysis to refute an idea of structural sharing where the future clitic comes atop a conjoined VP structure, since both conjuncts can have contrasting subjects. Despi\'c provides the following example (\ref{serbianconjuncts}).

\begin{exe}
    \ex \label{serbianconjuncts}
    \gll 
    Polufinalni program  \'{c}e otvoriti Juentus i {Real Madrid,} a zatvoriti ga Barselone i Bajern \\ semi\_final program {\Aux}.3.S.{\Fut} open.{\Inf} Juventus {\And} {Real Madrid} {\And} close.{\Inf} 3.{\Sg} Barcelone {\And} Bayern \\
    \glt `Juventus and Real Madrid will open the semi-final program, and Barcelona and Bayern will close it.' \\
    ${}$ \hfill Adapted from \cite{despic2017suspended}
\end{exe}

Since there can be two different subjects in (\ref{serbianconjuncts}), there is no VP-level conjunction. Despi\'c further examines TP level adverbs in conjunctions, refuting a \textit{v}P level conjunction too. However I won't go into those examples and further scrutiny of the state of affairs in Serbian here. I want to reiterate an example from Despi\'c to serve the point of what is nature of \textit{\'ce} deletion process under mismatching $\varphi$-features in (\ref{serbiannomatch}).

\begin{exe}
    \ex \label{serbiannomatch}
    \begin{xlist}
        \ex \gll 
        Ti \'{c}es do\'{c}i a ja (\'{c}u) oti\'{c}i \\ {\Second}.{\Sg} {\Aux}.{\Second}.S.{\Fut} arrive.{\Inf} {\And} {\First}.{\Sg} ({\Aux}.{\First}.S.{\Fut}) leave.{\Inf} \\
        \glt `You will come and I will leave'
    \end{xlist}
    ${}$ \hfill Adapted from \cite{despic2017suspended}
\end{exe}

It is possible to delete the second place future clitic \textit{\'ce} in Serbian under mismatching $\varphi$-features, a direct contradiction to all the suspendable affixes in Turkish verbal domain, which have clitic like properties. This shows that a morphological form having affixal properties does not hinder us from capitulating on an ellipsis analysis for SA. For example, a k-paradigm agreement marker \textit{-k} `{\First}.{\Pl}' for the past tense marker \textit{-DI} can not be suspended, which contrasts with an m-paradigm agreement marker like \textit{-(y)Iz} that can be. The difference between these two suffixes is that the first is not clitic-like in nature and the second is (\ref{k-iz}).

\begin{exe}
    \ex \label{k-iz} 
    \begin{xlist}
        \ex \gll 
        Ev-e gid-ec\'{e}k ve dinlen-ec\'{e}ğ-iz \\ house-{\Dat} go-{\Fut} {\And} rest-{\Fut}-{\First}.{\Pl} \\
        \glt `We will go home and rest'
        
        \ex \gll 
        *Ev-e git-t\'i ve dinlend\'i-k \\ house-{\Dat} go-{\Pst} {\And} rest-{\Pst}-{\First}.{\Pl} \\
        \glt Intended `We went home and rested'
    \end{xlist}
\end{exe}

However, unlike the Serbian second place future clitic \textit{\'ce}, the m-paradigm agreement markers can not be suspended under mismatching $\varphi$-features (\ref{mismatchm}).

\begin{exe}
    \ex \label{mismatchm}
    \gll 
    *Sen ev-e gid-ecek ve biz dinlen-eceğ-iz \\ {\Second}.{\Sg} house-{\Dat} go-{\Fut} {\And} {\First}.{\Pl} rest-{\Fut}-{\First}.{\Pl} \\
    \glt Intended `You will go home and we will rest.'
\end{exe}

As a summary the Serbian second place future clitic shows affix like properties but it undergoes an ellipsis process where mismatches in $\varphi$-features can be overlooked. As a contrast, some agreement suffixes in Turkish show clitic like properties yet they do not undergo SA when there is a mismatch in $\varphi$-features.

















