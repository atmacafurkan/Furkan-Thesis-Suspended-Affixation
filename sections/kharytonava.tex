\subsection{\cite{kharytonava2011morphology, kharytonava2012word,kharytonava2012taming}}

In all her papers, Kharytonava specifically inspects SA in Turkish noun compounds. For a start consider the noun compounds in (\ref{compound}).

\begin{exe}
    \ex \label{compound}
    \begin{xlist}
        \ex \gll Anne-m not defter-i-ni yıka-mış. \\
        mother-{\Fsg} note book-{\Poss}.{\Tsg}-{\Acc} wash-{\Prf}[{\Tsg}] \\
        \glt `It seems like my mother washed the notebook'
        
        \ex \gll Anne-m not defter-im-i yıka-mış \\
        mother-{\Fsg} note book-{\Poss}.{\Fsg}-{\Acc} wash-{\Prf}[{\Tsg}] \\
        \glt `It seems like my mother washed my notebook'
    \end{xlist}
\end{exe}

The default agreement marker is third person singular in Turkish when no possessor is present for the compound. The SA that Kharytonava presents comes into play in compounds with shared bases. (\ref{compoundSA}) shows an example where the shared base is \textit{doğum} `birth' and the markers on the conjoined nouns can be fully expressed (No SA) or can have two shapes of SA (partial-full).

\begin{exe}
    \ex \label{compoundSA}
    \begin{xlist}
        \ex No SA\\*
        \gll doğum yer-iniz ve tarih-iniz \\ 
        birth place-{\Spl} {\And} date-{\Spl} \\
        \glt ${}$
        
        \ex \label{compoundSAb} Full SA\\*
        \gll doğum yer ve tarih-iniz \\ 
        birth place {\And} date-{\Spl} \\
        \glt ${}$
        \ex \label{compoundSAc} Partial SA\\*
        \gll doğum yer-i ve tarih-iniz \\ 
        birth place-{\Tsg} {\And} date-{\Spl} \\
        \glt `Your birthplace and birthdate'\\*
        \hfill Adapted from \cite{kharytonava2012taming}
    \end{xlist}
\end{exe}

The possessive marker is suspended in (\ref{compoundSAb}) and there is no remnant of agreement whereas (\ref{compoundSAc}) leaves behind a possessor that is {\Tsg}. The interpretation of possessive for the second conjunct is still {\Ssg}. On the surface the existence of {\Poss}.{\Tsg} after SA for a {\Poss}.{\Ssg} seems problematic. Kharytonava addresses this not as a structural sharing analysis, she rather uses Impoverishment and Feature Geometry to explain such a configuration of SA. She indicates that features are monovalent for referring expressions in Turkish and exponent insertion is modulated by Subset Principle. Table \ref{tab:kharyfeatures} shows the feature geometry she provides for Turkish possessors with the corresponding exponents.

\begin{table}[hbt!]
    \caption{Feature Geometry of {\Poss} in Turkish}
    \centering
    \begin{tabular}{|l|l|l|}
    \hline
         \multicolumn{2}{|c|}{Features} & \multirow{2}{*}{Exponent}  \\ \cline{1-2}
         Participant & Individuation  & \\ \hline
         Speaker & $\emptyset$ & \textit{-Im} \\ \hline 
         Addressee & $\emptyset$ & \textit{-In} \\ \hline 
         Speaker & Group & \textit{-ImIz} \\ \hline 
         Addressee & Group & \textit{-InIz} \\ \hline 
         $\emptyset$ & $\emptyset$ & \textit{-(s)I(n)} \\ \hline 
         $\emptyset$ & Group & \textit{-lArI} \\ \hline 
    \end{tabular}
    \label{tab:kharyfeatures}
\end{table}

SA in noun compounds works by deletion of the features. See the feature templatic view of no SA in (\ref{compoundSA}). The feature set for \textsc{Addressee}-\textsc{Group}, by Subset Principle, is \textit{-InIz}. On this templatic view the features in the first conjunct instead of the exponent itself are deleted. This feature deletion results in the following templatic view and the exponent \textit{-(s)I(n)} is inserted after the first conjunct. \cite{kharytonava2011morphology} shows that Turkish speakers prefer the Partial SA in (\ref{compoundSA}) to the full SA. This type of analysis for SA falls under an ellipsis like analysis which has more appeal and makes better predictions about SA in noun compounds than structural sharing approaches.

\begin{itemize}
    \item $\alpha$-\textsc{Addressee}-\textsc{Group} {\And} $\beta$-{\textsc{Addressee}-\textsc{Group}}
    \item $\alpha$-$\emptyset$-$\emptyset$ {\And} $\beta$-{\textsc{Addressee}-\textsc{Group}}
\end{itemize}

Using this deletion analysis, instances like (\ref{impoverished}) can also be a deletion of the referential feature alongside the tense. {\Tsg} on verbal and nominal predicate domain is not expressed by an overt phonological exponent. The readings should have contrasted in their subject readings if this were to be the case. 

\begin{exe}
    \ex \label{impoverished}
    \begin{xlist}
        \ex \gll Ben hasta ve yorgun-du-m. \\
        {\Fsg}[{\Nom}] sick[{\Tsg}] {\And} tired-{\Pst}-{\Fsg} \\
        \glt `I was sick and tired'
        
        \ex \gll Ben ev-e gid-ecek ve gel-ecek-ti-m. \\
        {\Fsg}[{\Nom}] house-{\Dat} come-{\Fut}[{\Tsg}] {\And} come-{\Fut}-{\Pst}-{\Fsg} \\
        \glt `I was going to come home and go'
    \end{xlist}
\end{exe}

