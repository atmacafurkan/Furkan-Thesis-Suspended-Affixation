\chapter*{ABSTRACT\\ \ttitle}

This study investigates Suspended Affixation (SA) and its environment. \cite{geoffrey1967turkish} is credited with the term, but observations of SA can be found in earlier Turkish grammar books \citep{emre1945turk,gencan1966dilbilgisi}. It is a morphological ellipsis process that is observed in constructions that contain conjuncts. Overt suffixes on the rightmost conjunct in a conjunction are interpreted for the other conjuncts. I provide both empirical and theoretical observations that paint a broader picture for SA. I argue for an ellipsis analysis in SA and propose analyses for the suffixes \textit{ile/=(y)lA} and \textit{-(y)Ip}. My findings indicate that the environment of SA greatly impacts its acceptability. Conjoiner that have pragmatic implicatures can hinder SA. On the other hand SA in \textit{-(y)Ip} constructions might require changes in the information structure. In the literature on Turkish SA, \cite{kabak2007turkish} provides constraints and \cite{orgun1995flat,kornfilt2012revisiting,broadwell2008turkish} provide a lexical sharing analysis. The literature on other languages adopts an ellipsis approach \citep{guseva2017postsyntactic,erschler2018suspended}. My analysis is more in line with the ellipsis approach, yet it abides by the proposed constraints in the literature. I conducted three experiments. The first (214 participants) investigated if SA of derivational suffixes was acceptable and the second (160 participants) investigated if different amounts of SA changed processing difficulty. Both experiments investigated the effect of a conjoiner choice between \textit{ve} `and' and \textit{veya} `or'. The third experiment showed that SA can be a testing ground for sentence processing in Turkish. The experiment used an environment dependent on SA and showed effects of Reanalysis and how reanalyzed readings were accessible in further tasks. All the experiment results jointly indicated that the SA environment was more crucial than solely identifying suspendable affixes.  

\newpage

\chapter*{ÖZET\\ \ttitletr}

Bu çalışma Türkçedeki Ertelenmiş Ekleri (ErE) ve bu eklerin bulunduğu yapıları inceler. Ertelenmiş eklere ingilizce adını \cite{geoffrey1967turkish} vermiş ancak bu yapı Türkçe gramer kitaplarında daha önce gözlemlenmiştir \citep{emre1945turk,gencan1966dilbilgisi}. ErE biçimbilimsel bir eksilti işlemidir. Bir bağlaşımda sadece en sağdaki bağlaşımın taşıdığı biçimi diğer bağlaşıklar anlamsal olarak taşımaktadır. Bu çalışmada ben hem deneysel hem teorik yöntemler kullanarak ErE ve bulunduğu yapılar için analizler sunmaktayım. ErE'nin kendisi için doğrudan bir eksilti analizini savunup, ErE yapıları oluşturan \textit{ile/=(y)lA} ve \textit{-(y)Ip} ekleri için yapısal çözümler öneriyorum. Bulgularım ErE'nin bulunduğu yapıyla ilişkili olduğu ve bu yapının tabi olduğu bağlamsal değişikliklerden etkilendiğini göstermekte. Bu bağlamsal değişiklikler bağlaç seçimlerinde ErE yi olumsuz etkileyebilirken, \textit{-(y)Ip} ile oluşturulan yapılarda bilgi yapısı değişikliği ErE için önkoşul haline gelebilir. Türkçe ErE ile ilgili \cite{kabak2007turkish} birtakım önkoşullar ortaya koyar ve \cite{orgun1995flat,kornfilt2012revisiting,broadwell2008turkish} ErE için yapısal paylaşım analizi öne sürer. Diğer dillerdeki ilgili kaynaklar \citep{guseva2017postsyntactic,erschler2018suspended} ErE için doğrudan eksilti analizi sunar. Benim öne sürdüğüm analiz doğrudan eksiltiye daha yakın olmakla birlikte \cite{kabak2007turkish}'ın ErE için önerdiği koşullara uymaktadır. Bu çalışmada yürüttüğüm 3 ayrı deneyi raporluyorum. Bu deneylerden ilki (214 katılımcı) ErE'nin yapım ekleriyle uyumlu olup olmadığını ve ikincisi (160 katılımcı) farklı ErE miktarlarının dil işlemesinde güçlük çıkarıp çıkarmadığını sorgulamaktadır. Ek olarak, iki deney de isimsel ve eylemsel düzlemdeki ErE içinde \textit{ve} ile \textit{veya} arasındaki bağlaç seçiminin etkilerini sorgular. Üçüncü deney (126 katılımcı) ErE'nin Türkçedeki dil işleme çalışmaları için önemli bir yapı sunabileceğini gösterir. Tüm deneyler ve yapısal analizler ErE'nin içinde bulunduğu yapının ErE'ye dahil olabilen ekleri incelemekten daha verimli olduğunu göstermektedir.