\subsection{\citet{orgun1995flat}} \label{orgun}

Orgun argues for an analysis of SA as a structural sharing process. He provides the examples in (\ref{tebrikler}). These examples show that SA of {\Poss} is ungrammatical in a string of {\Pl-\Poss}. This is peculiar considering that SA of {\Poss} is grammatical in (\ref{possSA}). 

\begin{exe}
    \ex \label{tebrikler}
    \begin{xlist}
    \ex SA of {\Poss} \label{tebriklerug}\\*
    \gll *tebrik-ler ve teşekkür-ler-im \\ 
    congrats-{\Pl} {\And} thanks-{\Pl}-{\Poss}.{\Fsg} \\
    \glt ${}$
    
    \ex SA of {\Pl-\Poss} \label{tebriklerg}\\*
    \gll tebrik ve teşekkür-ler-im \\ 
    congrats {\And} thanks-{\Pl}-{\Poss}.{\Fsg} \\
    \glt `My congratulations and thanks'
    \end{xlist}
    
    \ex \label{possSA}
    \begin{xlist}
        \ex SA of {\Poss}\\*
        \gll Kitap ve kalem-im \\
        book {\And} pencil-{\Poss}.{\Fsg} \\
        \glt `My book and my pencil'
    \end{xlist}
\end{exe}

Orgun proposes to place the suffixes {\Pl} and {\Poss} on the same hierarchical level as in Figure \ref{fig:orgun2}. This way, he explains the ungrammatical SA of {\Poss} (\ref{tebriklerug}) and grammatical SA of {\Poss} (\ref{possSA}). The string of {\Pl-\Poss} are interpreted as hierarchically equivalent so SA cannot target only one of them. 

\begin{figure}[hbt!]
    \centering
    \begin{forest}
        [N
            [N]
            [{\Pl}]
            [{\Poss}]]
    \end{forest}
    \caption{Ternary branching analysis for SA}
    \label{fig:orgun2}
\end{figure}

He provides a three-way ambiguity of an expression like \textit{it-ler-i} `dog-{\Pl}-{\Poss}' (\ref{itleri}) for the support of ternary branching (\ref{fig:orgun2}). The three-way ambiguity results from the order of composition. All items are on the same hierarchical level, so the order of composition becomes ambiguous resulting in the different readings.
\begin{exe}
    \ex \label{itleri}
    \gll \textit{it-ler-i} \\ dog-{\Pl}-{\Poss} \\
    \glt `her/his dogs' \\* `their dog' \\* `their dogs' \\*
    \hfill Adapted from \citet{orgun1995flat}
\end{exe}

Reading in between the lines, I assume that Orgun takes {\Pl} and {\Poss} suffixes to have a different interaction than any other pair of suffixes holds, in a way that they form a complex head when they are adjacent. What he is proposing is not a ternary branching but a complex head formation. A representation of this formulation reflected on the ungrammatical SA in (\ref{tebriklerug}) is given in Figure \ref{fig:furkan1}.

\begin{figure}[hbt!]
    \centering
    \begin{tikzpicture}
    \Tree[.*N
            [.N 
                [.N 
                    [.\textit{tebrik} ]
                    [.\textit{-ler} ] ]
                [.\textit{ve} ]
                [.\textit{teşekkür} ] ]    
            [ 
                [.\textit{-ler} ]
                [.\textit{im} ] ]
        ];
    \end{tikzpicture}
    \caption{{\Pl} and {\Poss} forming a complex head in ungrammatical SA}
    \label{fig:furkan1}
\end{figure}


Forming a complex head of {\Pl-\Poss} makes the interpretation of the word \textit{tebrikler} and the suffix \textit{ler-im} `{\Pl-\Poss}' ungrammatical. The same complex head, however, does not cause a problem for the grammatical SA in (\ref{tebriklerg}) as Figure \ref{fig:furkan2} shows. Figure \ref{fig:furkan2} has equivalent conjuncts and an interpretable relation between the complex suffix \textit{-ler-im} `{\Pl-\Poss}' and the nouns \textit{tebrik} `congrats', and \textit{teşekkür} `thanks'. 

\begin{figure}[hbt!]
    \centering
    \begin{tikzpicture}
    \Tree[.N
            [.N 
                [.\textit{tebrik} ]
                [.\textit{ve} ]
                [.\textit{teşekkür} ] ]
            [ 
                [.\textit{-ler} ]
                [.\textit{-im} ] ]
    ];
    \end{tikzpicture}
    \caption{{\Pl} and {\Poss} forming a complex head in grammatical SA}
    \label{fig:furkan2}
\end{figure}

Orgun goes on to show that ternary branching is needed for some morphological configurations to satisfy the minimal phonological size ($\sigma\sigma$) constraint, citing \citet{ito1989notes}, together with \citet{orgun1992turkish}. He proposes a structural sharing analysis for SA and a ternary branching for {\Pl} and {\Poss} suffixes to capture the inseparable SA of {\Pl-\Poss}. Support for ternary branching in SA comes from somewhat unrelated phonological constraints in affixation of monosyllabic words, i.e. \textit{*do-m} [$\sigma$] `do-{\Poss}.{\Fsg}', \textit{sol-üm} [$\sigma$-$\sigma$] `sol-{\Poss}.{\Fsg}'. The ungrammatical SA in (\ref{tebrikler}) is not subject to such a constraint and the three-way ambiguity of an expression like \textit{it-ler-i} `dog-{\Pl}-{\Poss}' is not convincing enough to propose ternary branching. In finalizing the observation that Orgun makes, I provided Figures \ref{fig:furkan1} and \ref{fig:furkan2} following the discussion and the examples provided in \citet{orgun1995flat} to paint a more comprehensible picture of his analysis.
