\section{Acceptability study}

In the literature of SA in Turkish it is claimed that SA is only operational for inflectional suffixes \citep{orgun1995flat,kornfilt1996some,broadwell2008turkish, kornfilt2012revisiting} with the exception of \cite{akkucs2016suspended}. I have taken a subset of the derivational suffixes that take nominal bases and produce nominals, from a list in \cite{goksel2004turkish}. I give a list of possible SA configurations with different conjoiners in (\ref{derivationalsuffixes}).


\begin{exe}
  \begin{multicols}{2}
    \ex \label{derivationalsuffixes}
    \begin{xlist}
        \ex 
        \gll 
        düş-er-cesine \\ fall-{\Aor}-{\Casina} \\
        \glt `as if falling'
        
        \ex 
        \gll 
        yalan-cı \\ lie-{\Ci} \\ 
        \glt `liar'
        
        \ex 
        \gll 
        kahve-msi renk \\ coffee-LIKE colour \\
        \glt `like coffee colour'

        \ex 
        \gll 
        üç-üncü \\ three-ORD \\
        \glt `third'
        
\columnbreak

        \ex 
        \gll 
        sorun-lu adam \\ problem-{\Inc} man \\
        \glt `troubled man'
        
        \ex 
        \gll 
        düşman-lık \\ enemy-{\Lik} \\
        \glt `enmity'
        
        \ex 
        \gll 
        sınır-sız internet \\ limit-{\Exc} internet \\
        \glt `limitless internet'
        
        \ex 
        \gll 
        iki-şer \\ two-COU \\
        \glt `two by two'
        
    
    \end{xlist}
    \end{multicols}
\end{exe}

My purpose in this experiment is to investigate if the SA of the suffixes in (\ref{derivationalsuffixes}) is possible or not. For this purpose I have designed an acceptability study where a simple yes or no answer is provided for an expression hosting an SA of the derivational suffixes. In the following subsections I lay out the method, results, and analysis of the experiment.
% \begin{table}[hbt!]
%     \caption{List of Suspendable Derivational Suffixes with Different Conjoiners}
%     \centering
%     \begin{tabular}{|ll|}
%     \hline
%     \multicolumn{2}{|c|}{with conjoiners \textit{ve} and \textit{veya}}         \\
%     \hline 
%         Suffix                  & SA                                            \\
%     \hline 
%       \textit{-CAsInA}         & \textit{düşer ve(ya) takılırcasına}           \\
%       \textit{-CI}             & \textit{boya ve(ya) badanacı}                 \\
%       \textit{-(I)msI}         & \textit{meyve ve(ya) sebzemsi}                \\
%       \textit{-(I)ncI}         & \textit{üç ve(ya) dördüncü kemikler}          \\
%       \textit{-lI}             & \textit{çikolata ve(ya) çilekli dondurma}     \\
%       \textit{-lIK}            & \textit{dost ve(ya) düşmanlık}                \\
%       \textit{-sIz}            & \textit{çikolata ve(ya) çileksiz dondurma}    \\
%       \textit{-(ş)Ar}          & \textit{üç ve(ya) altışarlı sıra}             \\
%       \hline 
%     \end{tabular}
%     \label{tab:derivationalSA}
% \end{table}

\subsection{Method}
In this subsection I report the participants, materials, and the procedure for the experiment.
\subsubsection{Participants}

The participants were NUMBER students at Boğaziçi University who were native speakers of Turkish. In exchange for their participation they have received 1 point to their overall course score on a chosen linguistics course.

\subsubsection{Materials}

The experiment is comprised of 9 different suffixes (8 derivational 1 inflectional {\Acc}) by 2 conjoiners per suffix. For each suffix there are 3 distinct items. This way I have 54 experimental items, and additional 27+27 un/grammatical fillers. A latin square design by conjoiner type is applied, forming two lists of 27. This resulted in each participant seeing only 27 experimental items and 54 fillers. An example set of experimental items for {\Acc} and {\Casina} is given in (\ref{acceptabilityexe}).

\begin{exe}
    \ex \label{acceptabilityexe}
    \begin{xlist}
    \ex \gll \textit{Ev-e} \textit{koş-ar} \textit{ve} \textit{zıpla-r-casına} \textit{gel-di-m.} \\ house-{\Dat} run-{\Aor} {\And} jump-{\Aor}-{\Casina} come-{\Pst}.{\First}.{\Sg} \\
    \glt ${}$ \hfill Der\_AND

    \ex \gll \textit{Ev-e} \textit{koş-ar} \textit{veya} \textit{zıpla-r-casına} \textit{gel-di-m.} \\ house-{\Dat} run-{\Aor} {\Or} jump-{\Aor}-{\Casina} come-{\Pst}.{\First}.{\Sg} \\
    \glt `I came home as if running and/or jumping.' \hfill Der\_OR
 
    \ex \gll \textit{Ev-e} \textit{defter} \textit{ve} \textit{kitab-ı} \textit{getir-di-m.} \\ house-{\Dat} notebook {\And} book-{\Acc} bring-{\Pst}.{\First}.{\Sg} \\
    \glt ${}$ \hfill Inf\_AND
    
    \ex \gll \textit{Ev-e} \textit{defter} \textit{veya} \textit{kitab-ı} \textit{getir-di-m.} \\ house-{\Dat} notebook {\Or} book-{\Acc} bring-{\Pst}.{\First}.{\Sg} \\
    \glt `I brought home the book and/or the notebook.' \hfill Inf\_OR
    \end{xlist}
\end{exe}

Pre and post sentence fields are not kept constant since each suffix required a different type and place of modification or use, resulting in string-wise uneven number of words. Since the only measurement of the acceptability study is grammaticality judgements with no emphasis on reading or response times, uneven number of words across experimental items is considered not to be an issue. The experiment is formed using http://spellout.net/ibexfarm/, and carried out online. For the full list of items and fillers (1-27 and 100-154), see Appendix \ref{acceptabilityitems}.

\subsubsection{Procedure}
Participants are provided a link to the experiment prompting them to a consent page. Upon giving consent participants go through 5 practice items and then they are prompted again for the beginning of the experiment. Each item is presented as a full sentence and participants decide on whether or not the sentence they encounter is a natural/ok sentence in Turkish or not. They profess their decision by pushing "Q" key for "yes" and "P" key for "no" on the keyboard. Only the participant choice regarding the grammaticality judgement is used for analysis. After the experiment is done, participants are redirected to a separate page to fill their student information to be relayed to the course's professor for the extra credit. This is kept separate of the experiment results, protecting participant anonymity.

\subsection{Results}

The results are recorded onto a csv file and imported to R \citep{team2013r}. The results are then aggregated and includes subject, item number, response and response time. Categories for suffix and conjoiner type and category for experimental item or filler is added. This resulted in NUMBER of data points. Participants with grammatical accuracy lower than 70\% are excluded from the data. This resulted in the loss of NUMBER data points.


