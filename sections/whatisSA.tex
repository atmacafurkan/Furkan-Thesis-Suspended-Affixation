
Suspended affixation (SA henceforth) is a morphological phenomenon where only one of the conjuncts carries some affixal parts that are shared with the other conjuncts. An abstract representation is given in (\ref{suspended}).
\begin{exe}
\ex \label{suspended}
\begin{xlist}
\ex \label{suspended1} A conjoiner B-$\alpha$
\ex A-$\alpha$ conjoiner B-$\alpha$
\end{xlist}
\end{exe}
For now consider both of the expressions as equal in their denotation. Although it is not often given as examples, SA is also possible with more than two conjuncts (\ref{suspended2}).
\begin{exe}
\ex \label{suspended2}
\begin{xlist}
\ex A conjoiner B conjoiner C-$\alpha$
\ex A-$\alpha$ conjoiner B-$\alpha$ conjoiner C-$\alpha$
\end{xlist}
\end{exe}
SA usually appears in languages as a backwards process, where the linearly rightmost conjunct bears the shared morphemes. However limited examples from Caucasian languages can be found \citep{erschler2012suspended, erschler2009possession} where affixal parts that occupy the left edge of the word are shared. An example for SA of {\All} from Adyghe (Northwestern Caucasian) is given in (\ref{suspendreverse}).
\begin{exe}
\ex \label{suspendreverse}
    \begin{xlist}
        \ex $\alpha-$A conjoiner B
        \ex $\alpha-$A conjoiner $\alpha-$B
        \ex \gll 
        \textit{s-j\textschwa-p\textctc a\textctc e-re} \textit{\textteshlig'ale-re} \textit{zezaox} \\ {\First}{\Sg}-{\All}-girl-{\And} boy-{\And} fight.each.other \\
        \glt `My son and daughter are fighting.'
    \end{xlist}
\hfill Adapted from \cite{erschler2012suspended}
\end{exe}

In the following paragraphs I lay out the examples and configurations of Turkish SA in two parts. First is the nominal domain and the second is the verbal domain. Most examples of SA in the nominal domain are made with the Case, Possessive, and Plural suffixes (Table \ref{tab:nominalSA}).

\begin{table}[hbt!]
    \caption{Suspendable Turkish Suffixes in Nominals}
    \centering
    \begin{tabular}{|ll|lllll}
    \hline 
        \multicolumn{2}{|c|}{Case} & \multicolumn{4}{|c|}{Possessive} & \multicolumn{1}{c|}{Plural} \\ \hline
        {\Acc} & \textit{-(y)I} &  \multicolumn{1}{l}{${}$} & 1$^{st}$ & 2$^{nd}$ & \multicolumn{1}{l|}{{\Third}$^{rd}$} & \multicolumn{1}{l|}{\textit{-lAr}} \\ \hline 
        
        {\Dat} & \textit{-(y)A} & {\Sg} & \textit{-(I)m} & \textit{-(I)n} &  \multicolumn{1}{l|}{\textit{-(s)I}} & ${}$ \\ \cline{1-6} 
        
        {\Loc} & \textit{-DA} & {\Pl} & \textit{-(I)mIz} & \textit{-(I)nIz} &  \multicolumn{1}{l|}{\textit{-lArI\tablefootnote{the use of third person plural is ambiguous between a singular noun with third person plural agreement and a plural noun with third person plural agreement.}}} & ${}$ \\ \cline{1-6}
        
        {\Abl} & \textit{-DAn} & ${}$ & ${}$ & ${}$ & ${}$ & ${}$ \\ \cline{1-2}
        
        {\Gen} & \textit{-(n)In} & ${}$ & ${}$ & ${}$ & ${}$ & ${}$ \\ \cline{1-2}
    \end{tabular}
    \label{tab:nominalSA}
\end{table}

All of the suffixes in Table \ref{tab:nominalSA} can be interpreted as $\varphi$-features and thereby inflectional, but some claim that derivational suffixes can also be suspended which is going to be addressed in the following sections. I will not give every possible suffix combinations here but some examples for them are given in (\ref{nomex}).

\begin{exe}
    \ex \label{nomex} 
    \begin{xlist}
        \ex 
        \gll 
        \textit{Hoca} \textit{ve} \textit{ders-ten} \textit{kork-uyor-um.} \\ instructor {\And} course-{\Abl} scared\_of-{\Prog}-{\First}{\Sg} \\
        \glt `I am scared of the instructor and the course.'
        
        \ex
        \gll 
        \textit{Hoca} \textit{ve} \textit{ders-im-den} \textit{kork-uyor-um.} \\ instructor {\And} course-{\First}{\Sg}.{\Poss}-{\Abl} scared\_of-{\Prog}-{\First}{\Sg} \\
        \glt `I am scared of my instructror and my course.' 
        
        \ex 
        \gll 
        \textit{Hoca} \textit{ve} \textit{ders-ler-im-den} \textit{kork-uyor-um.} \\ instructor {\And} course.{\First}{\Sg}.{\Poss}-{\Abl} scared\_of-{\Prog}-{\First}{\Sg} \\
        \glt `I am scared of my instructors and my courses.'
    \end{xlist}
\end{exe}
The sentences in (\ref{nomex}) show examples of full SA in a string of {\Pl}-{\Poss}-{\Case} in the first conjunct. An example for partial SA is given in (\ref{nomex2}).

\begin{exe}
    \ex \label{nomex2}
    \gll
    \textit{Hoca-lar-ım} \textit{ve} \textit{ders-ler-im-den} \textit{kork-uyor-um.} \\ instructor-{\Pl}-{\First}{\Sg}.{\Poss} {\And} course-{\Pl}-{\First}{\Sg}.{\Poss} scared\_of-{\Prog}-{\First}{\Sg} \\
    \glt `I am scared of my instructors and my courses.'
\end{exe}
The sentence in (\ref{nomex2}) shows the suspension of only {\Case} in the first conjunct in a string of {\Pl}-{\Poss}-{\Case}. Although it is theoretically possible (and some examples are widely acceptable), some deem an example like (\ref{nomex3}) ungrammatical where we have a suspension of {\Poss}-{\Case} in a string of {\Pl}-{\Poss}-{\Case}.

\begin{exe}
    \ex \label{nomex3}
    \gll 
    \textit{Hoca-lar} \textit{ve} \textit{ders-ler-im-den} \textit{kork-uyor-um.} \\ instructor-{\Pl} {\And} course-{\Pl}-{\First}{\Sg}.{\Poss} scared\_of-{\Prog}-{\First}{\Sg} \\
    \glt `?I am scared of my instructors and my courses.'
\end{exe}

Additionally SA need not only conjoiners like \textit{ve}, an example with nominal negator \textit{değil} is given in (\ref{nomex}).

\begin{exe}
    \ex \label{nomex4}
    \begin{xlist}
        \ex
        \gll 
        \textit{Ders} \textit{değil} \textit{hoca-lar-dan} \textit{kork-uyor-um.} \\ course {\Neg} instructor-{\Pl}-{\Abl} scared\_of-{\Prog}-{\First}{\Sg} \\

        \ex
        \gll 
        \textit{Ders-ler} \textit{değil} \textit{hoca-lar-dan} \textit{kork-uyor-um.} \\ course-{\Pl} {\Neg} instructor-{\Pl}-{\Abl} scared\_of-{\Prog}-{\First}{\Sg} \\
        \glt `I am not scared of the courses but the instructors.'
    
    \end{xlist}

\end{exe}

SA in the verbal domain has two shapes two it, one is the SA of agreement after TAM I markers, or SA of TAM II and agreement marker. See Table (\ref{tab:markers}) for a list.

\begin{table}[hbt!]
    \caption{TAM I, II, and Agreement Markers}
    \centering
    \begin{tabular}{|ll|lllll}
    \hline 
         \multicolumn{2}{|c|}{TAM I} & \multicolumn{2}{c|}{TAM II} &  \multicolumn{3}{c|}{Agreement} \\ \hline
        Progressive & \textit{-Iyor} & Past & \textit{-(y)DI} & \multicolumn{1}{|l}{${}$} & \multicolumn{1}{c}{{\Sg}} & \multicolumn{1}{c|}{{\Pl}} \\ \hline
        
        Aorist & \textit{-Ir} & Evidential & \textit{-(y)mIş} & \multicolumn{1}{|l}{1$^{st}$} & \textit{-(I)m} & \multicolumn{1}{l|}{\textit{-k}, or \textit{-(I)z}} \\ \cline{1-7}

        Future & \textit{-y)AcAK} & Conditional & \textit{-(y)sA} & \multicolumn{1}{|l}{2$^{nd}$} & \textit{-(sI)n} & \multicolumn{1}{l|}{\textit{-(sI)nIz}} \\ \cline{1-7}
        
        Necessitive & \textit{-mAlI} & ${}$ & ${}$ & \multicolumn{1}{|l}{{\Third}$^{rd}$} & - & \multicolumn{1}{l|}{\textit{-lAr}} \\ \cline{1-2} \cline{5-7}
        
        Perfect/Evidential & \textit{-mIş} & ${}$ & ${}$ & ${}$ & ${}$ & ${}$ \\ \cline{1-2}
        
        Conditional & \textit{-sA} & ${}$ & ${}$ & ${}$ & ${}$ & ${}$\\ \cline{1-2}
        
        Past & \textit{-DI} & ${}$ & ${}$ & ${}$ & ${}$ & ${}$ \\ \cline{1-2}
    \end{tabular}
    \\
    ${}$ \hfill Adapted from \cite{goksel2001auxiliary}
    \label{tab:markers}
\end{table}

Again I give some possible examples in (\ref{verbex}).

\begin{exe}
    \ex \label{verbex}
    \begin{xlist}
        \ex 
        \gll 
        \textit{Ev-e} \textit{gid-ecek} \textit{ve} \textit{dinlen-eceğ-im} \\ house-{\Dat} go-{\Fut} {\And} rest-{\Fut}-{\First}{\Sg} \\
        \glt `I will go home and rest.'
        
        \ex 
        \gll 
        \textit{Ev-e} \textit{gid-ecek} \textit{ve} \textit{dinlen-ecek-miş-im.} \\ house-{\Dat} go-{\Fut} {\And} rest-{\Fut}-{\Cop}.{\Evi}-{\First}{\Sg} \\
        \glt  `I was supposed to go home and rest.'
        
        \ex 
        \gll 
        \textit{Ev-e} \textit{gid-iyor} \textit{ve} \textit{dolan-ıyor-muş-um-dur.} \\ house-{\Dat} go-{\Fut} {\And} stroll-{\Prog}-{\Cop}.{\Evi}-{\First}{\Sg}-{\Prob} \\
        \glt `I might have been going home and strolling.'
    \end{xlist}
\end{exe}
The sentences in (\ref{verbex}) show the suspension of only Agreement, suspension of TAM II and Agreement, and suspension of TAM II, Agreement, and Probability marker \textit{-DIr}. The important observation made by the given examples is that SA is a right-bound process, meaning that the suspension of only $\alpha$ in a string of $\alpha-\beta$ is not allowed. It is either suspension $\beta$, or $\alpha-\beta$.

Additionally there are two configurations one in compounding and one in serial verb constructions in Turkish that could be considered as SA. The example in (\ref{compound-SA}) shows SA of compound/agreement marker ({\Third}{\Sg}.{\Poss} in glosses) on an inner compound. This marker's status as being phonologically overt or covert is optional in some cases. This optionality with interpretability can be seen as suspension of the compound marker \textit{-sI(n)}.

\begin{exe}
    \ex \label{compound-SA}
    \gll
    \textit{Beykoz} \textit{koru-(su)} \textit{mesire} \textit{alan-ı} \\ B grove-({\Third}{\Sg}.{\Poss}) picnic area-{\Third}{\Sg}.{\Poss} \\
    \glt `Beykoz grove's picnic area'
 \end{exe}
 
The other example of SA that comes from serial verb constructions in Turkish is achieved with the suffix \textit{-(y)Ip} (P(redicate)C(oncatonator) in glosses) as in (\ref{ipSA}).

\begin{exe}
    \ex \label{ipSA}
    \gll 
    \textit{Ev-e} \textit{gel-ip} \textit{uyu-du-m.} \\ House-{\Dat} come-{\Pc} sleep-{{\Pst}}-{\First}{\Sg} \\
    \glt `I came home and slept.'
\end{exe}

In (\ref{ipSA}), unlike the examples given so far, the insertion of the suspended affixes is not allowed. As a result of all the examples of configurations where there is seemingly SA, it seems like there is no one way of going after SA. Some of the instances of SA can be regarded as an ellipsis like process (\ref{nomex}, \ref{nomex2}, \ref{nomex3}, \ref{nomex4}, \ref{verbex}) where the insertion of the elided parts is allowed, but in some it is a structural sharing of some elements (\ref{ipSA}) where the insertion of the elided parts is not allowed.

Whether or not all derivational suffixes can be suspended is debatable but instances of some suffixes like \textit{-lI}, \textit{-sIz}, \textit{-lIK}, and \textit{-CI} ({\Inc}, {\Exc}, {\Lik}, and {\Ci} in glosses respectively) allowing SA can be found (\ref{derivationalSA}).

\begin{exe}
    \ex \label{derivationalSA}
    \begin{xlist}

        \ex
        \gll 
        \textit{Çilek} \textit{ve} \textit{çikolata-lı} \textit{dondurma} \\ strawberry {\And} chocolate-{\Inc} ice\_cream \\
        \glt `Ice cream with chocolate and strawberry'
        
        \ex
        \gll
        \textit{Şeker} \textit{ve} \textit{yağ-sız} \textit{yiyecek-ler} \\ sugar {\And} fat-{\Exc} food-{\Pl} \\
        \glt `Sugar and fat free foods'
        
        \ex 
        \gll
        \textit{Bahar} \textit{ve} \textit{yaz-lık} \textit{ceket} \\ spring {\And} summer-{\Lik} jacket \\
        \glt `Spring and summer jacket'
        
        \ex 
        \gll 
        \textit{futbol} \textit{ve} \textit{basket-çi} \\ football {\And} basketball-{\Ci} \\ 
        \glt `Football and basketball player'
    \end{xlist}
\end{exe}

The examples in (\ref{derivationalSA}) are only some of the instances of derivational SA that comes to mind. 
