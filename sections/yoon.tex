\subsection{Korean}

Another language that hosts similar phenomena like SA is Korean. Korean could be considered to be typologically closer to Turkish than the other languages German, Mari, and Serbian. \citet{yoon2005conjunction}, and \citet{yoon2017lexical} provide a good set of data and some contrasts for SA and its environment. In the following paragraphs, I give the relevant summary of the two papers.

\citet{yoon2005conjunction} present two conjunction types in Korean that differ in how their conjuncts are formed. In the first, the conjoiner suffix \textit{-kwa} ({\And} in glosses) conjoins two conjuncts, out of two only the second can be marked for {\Case}. A mirroring morphological form to this conjoiner could be the cliticized \textit{ile/=lA} in Turkish. I give an example in (\ref{koreanturkish}). The second type of conjoiner is the free form \textit{kuliko} `and', for the sake of argument it can be mirrored by \textit{ve} `and' in Turkish (\ref{koreanturkish2}).

\begin{exe}
    \ex \label{koreanturkish}
    \begin{xlist}
        \ex Korean \\* 
        \gll John-kwa Mary-ka cip-ey ka-ss-ta. \\ 
        J-{\And} M-{\Nom} home-{\Loc} go-{\Pst}-{\Decl} \\
        \glt `John and Mary went home.'\\*   
        \hfill Adapted from \citet{yoon2005conjunction}

        \ex Turkish\\*
        \gll Can=la Meryem ev-e git-ti. \\ 
        C={\And} M[{\Nom}] home-{\Dat} go-{\Pst} \\
        \glt `Can and Meryem went home.'
    \end{xlist}

    \ex \label{koreanturkish2}
    \begin{xlist}
        \ex Korean \\*
        \gll John-i kuliko Mary-ka cip-ey ka-ss-ta. \\ 
        J-{\Nom} {\And} M-{\Nom} home-{\Loc} go-{\Pst}-{\Decl} \\
        \glt John and Mary went home.\\* 
        \hfill Adapted from \citet{yoon2005conjunction}
        
        \ex Turkish \\*
        \gll Can ve Meryem ev-e git-ti-(ler). \\ 
        C {\And} M[{\Nom}] home-{\Dat} go-{\Pst}-({\Tpl}) \\
        \glt `Can and Meryem went home.'
    \end{xlist}
\end{exe}

The two different conjoiners show difference in interpretation. The reading differences lie in distributive or non-distributive readings, compatibility with collective modifiers, and compatibility with collective predicates. An example for the order of readings for both conjuncts is given in (\ref{koreanconjoiners}).

\begin{exe}
    \ex \label{koreanconjoiners}
    \begin{xlist}
        \ex \label{koreanconjoiner1} 
        \gll John-kwa Mary-ka ochen-pwul-ul pelessta. \\ 
        J-{\And} M-{\Nom} 5000-dollars-{\Acc} made \\
        
        \ex \label{koreanconjoiner2}
        \gll John-i kuliko Mary-ka ochen-pwul-ul pelessta. \\ 
        J-{\Nom} {\And} M-{\Nom} 5000-dollars-{\Acc} made \\
        \glt Reading 1: John and Mary each made \$5000. \\
        Reading 2: John and Mary together made \$5000. \\
        (\ref{koreanconjoiner1}): Reading 2\textgreater Reading 1 (\ref{koreanconjoiner2}): Reading 1\textgreater Reading 2\\*
        \hfill Adapted from \citet{yoon2005conjunction}
    \end{xlist}
\end{exe}

This preference for readings is different in both conjoiners, but it does not mean that the conjoiner \textit{-kwa} is incompatible with distributive readings. (\ref{kwadistributed}) shows a distributive reading for \textit{-kwa}. Another observation is that the conjoiner \textit{kuliko} is incompatible with collective readings (\ref{kulikocollective}).

\begin{exe}
    \ex \begin{xlist}
    \ex \label{kwadistributed}
    \gll John-kwa Mary-ka kakkak cip-ey ka-ss-ta. \\ 
    J-{\And} M-{\Nom} each home-{\Loc} go-{\Pst}-{\Decl} \\
    \glt `John and Mary each went home.'
    
    \ex \label{kulikocollective}
    \gll *?Cheli-ka kuliko Yenghi-ka chayksang-ul hamkkey mantul-ess-eyo. \\
    C-{\Nom} {\And} Y-{\Nom} desk-{\Acc} together make-{\Pst}-{\Decl} \\
    \glt Intended: `Chelswu and Yenghi made a desk together.'\\*
    \hfill Adapted from \citet{yoon2005conjunction}
    \end{xlist}
 \end{exe}

The two conjoiners differ with respect to SA. The conjoiner \textit{-kwa} triggers {\Case} SA, but the conjoiner \textit{kuliko} does not. \citet{yoon2005conjunction} show a distinction between the two conjoiners deeming \textit{-kwa} as a conjoiner for phrase levels and \textit{kuliko} as a conjoiner for clauses. These observations made so far about Korean conjoiners \textit{-kwa} and \textit{kuliko} show the importance of analyzing conjunction structure.

\citet{yoon2005conjunction} provide some data and analysis for two conjoiners in Korean, but \citet{yoon2017lexical} is focused on SA. Yoon presents derivational Korean suffixes that derive verbs or adjectives from nominal bases. These suffixes display a clear-cut difference in allowing SA. In providing SA-independent contrasts between these suffixes, Yoon presents some examples with Lexical Integrity tests of conjoined base, modifying the base, and gapping/ellipsis of the suffix. In expressing the difference between two suffix groups, Yoon uses the terms: Transparent suffix and Opaque suffix. These two terms represent a suffix's ability to be either treated as transparent and visible in morphological or syntactic derivations, or it is treated as opaque and non-compositional. (\ref{koreanlexical}) shows overt examples for the contrast between the two groups.


\begin{exe}
    \ex \label{koreanlexical}
    \begin{xlist}
    \exi{Conjoined base}
    \ex \label{koreanconjoined1}
    \gll *[Kunul-kwa kilum]-ci-n ku kos \\ 
    shade-{\And} oil-{\Der}-{\Rel} that place \\
    \glt `That plot of land, which is shaded and fertile'
    
    \ex \label{koreanconjoined2}
    \gll Ku-nun [yongkamha-n kwunin-kwa cincengha-n aykwukca]-taw-ass-ta.\\
    {\Tsg}-{\Top} courageous-{\Rel} soldier-{\And} genuine-{\Rel} patriot-{\Der}-{\Pst}-{\Decl} \\
    \glt `He really lived up to his reputation as a courageous soldier and true patriot.'
 
    \exi{Modified base}
    
    \ex \label{koreanmodified1}
    \gll Cenyek-ey-nun *[etwuw-un kunul]-ci-nun kos \\
    dusk-{\Loc}-{\Top} dark-{\Rel} shade-{\Der}-{\Rel} place\\
    \glt `A place that gets dark at dusk'
    
    \ex \label{koreanmodified2}
    \gll Ku-nun [hwullyungha-n hakca]-tap-key yenkwu-lul swi-ci anh-nunta. \\
    {\Tsg}-{\Top} outstanding-{\Rel} scholar-{\Der}-{\Comp} research-{\Acc} stop-{\Comp} {\Neg}-{\Prs} \\
    \glt `He never stops dping research, as befits his reputation as an outstanding scholar.'
    
    \exi{Gapping/Ellipsis}
    
    \ex \label{koreangapping1}
    \gll *Ku kos-un kilum-\_ kuliko i kos-un kunul-ci-ta. \\
    that place-{\Top} oil-\_ {\And} this place-{\Top} shade-{\Der}-{\Decl} \\
    \glt Intended `That place is fertile while this place is shady.'
    
    \ex \label{koreangapping2}
    \gll Cheli-nun kwunin-\_ kuliko Tongswu-nun haksayng-tap-ta. \\ 
    Cheli-{\Top} soldier {\And} Tongswu-{\Top} student-{\Der}-{\Decl} \\
    \glt `Cheli is every bit a soldier and Tongswu (every bit) a student.'\\*
    \hfill Adapted from \citet{yoon2017lexical}
    \end{xlist}
\end{exe}

(\ref{koreanlexical}) shows a clear distinction in the tests, but a suffix does not always behave the same. For example, in (\ref{tapdifferent}), the suffix \textit{-tap} behaves like \textit{-ci} in not allowing modification of base. Yoon dubs this category of suffixes as Double-duty suffix.

\begin{exe}
    \ex \label{tapdifferent}
    \gll *[Ceng-kwa alum]-taw-un sa.i \\ 
    affection-{\And} beautiful-{\Der}-{\Rel} relation \\ 
    \glt `Close and beautiful'
\end{exe}

The behaviours of suffixes in (\ref{koreanlexical}) show that derivational suffixes can have different responses to structural configurations. This is an observation that can prove useful for identifying why, if any, some derivational suffixes in Turkish can take part in SA and some cannot. Yoon, after further tests and contrasts, provides a table indicating the different category of derivations, a short version is given in Table \ref{tab:korean}.

\begin{table}[hbt!]
    \caption{Response of Different Category Suffixes in Korean to Lexical Integrity Tests}
    \centering
    \begin{tabular}{|l|l|l|l|}
    \hline 
    Suffix      & Coordination & External Modifiers & Gapping (Base) \\ \hline % & Gapping (Suffix) & Inbound Ana Island & Extraction \\ \hline 
    Opaque        & N             & N                 & N            \\ \hline % & N               & N                  & N \\ \hline 
    Transparent   & Y             & Y                 & N            \\ \hline % & Y               & Y                  & N \\ \hline 
    Double-duty   & N/Y           & N/Y               & N            \\ \hline \hline% & N/Y             & N/Y                & N/Y \\ \hline
    
    Suffix      &    Gapping (Suffix)    & Inbound Ana Island    &   Extraction \\ \hline
    Opaque      &    N                   & N                     &   N   \\ \hline 
    Transparent &    Y                   & Y                     &   N \\ \hline 
    Double-duty &    N/Y                 & N/Y                   &  N/Y \\ \hline 
    \end{tabular}\\
    ${}$ \hfill Adapted from \citet{yoon2017lexical}
    \label{tab:korean}
\end{table}

The observations of Yoon show that not all derivations are representable as one sub-syntactic and opaque process. Even the ones that have a transparent relation with syntax do not behave the same. Yoon proposes an analysis using word-internal phases, citing \citet{marantz2007phases}. The analysis boils down to these suffix categories belonging to different word phases. Opaque suffixes combine with the $\sqrt{ROOT}$ assigning the category and take place in the first phase of word derivation. Transparent suffixes combine with category assigned words and take place in the second phase of word derivation. Figure \ref{fig:devphases} illustrates both phases. 

\begin{figure}[hbt!]
    \centering
    \begin{forest}
    for tree={s sep=15mm, inner sep=0}
        [YP
            [XP, name=XP
                [X$^0$, name=X0 
                    [$\sqrt{ROOT}$]
                    [{Opaque Suffix}]]
                [{Transparent Suffix}]]
            [Y]]
    \node[above left=1em and 0.25em of X0](x1){\small 1$^{st}$phase};
    \node[right=0.5em of X0](x2){};
    \draw[overlay, thick] (x1) to[out=0, in=90] (x2);
    \node[above left=1em and 0.25em of XP](xp1){\small 2$^{nd}$phase};
    \node[right=0.5em of XP](xp2){};
    \draw[overlay, thick] (xp1) to[out=0, in=90] (xp2);
    \end{forest}
    \caption{Root internal phase in word-derivation}
    \label{fig:devphases}
\end{figure}

In Figure \ref{fig:devphases}, there is one suffix for each phase. This does not mean that an opaque suffix always culminates the first phase. According to Yoon, there could be several suffixes that could form a new Root from a base Root without category assignment as the Figure \ref{fig:firstdevphase} illustrates.

\begin{figure}[hbt!]
    \centering
    \begin{forest}
        [$\sqrt{ROOT}^3$
            [$\sqrt{ROOT}^2$
                [$\sqrt{ROOT}$]
                [{suffix}]]
            [{suffix}]]
    \end{forest}
    \caption{Derived Roots from Root bases in first word derivation phase}
    \label{fig:firstdevphase}
\end{figure}

The explanation of word formation phases captures the differences in the suffix groups of Transparent and Opaque. Opaque suffixes merge with Roots and cannot be targeted by SA, but Transparent suffixes merge with category assigned words and can be targeted by SA. This explanation can be utilized in explaining why bare verbs are not morphological words and why SA cannot take place with bare verb remnants in Turkish.