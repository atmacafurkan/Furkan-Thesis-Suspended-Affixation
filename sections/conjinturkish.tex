\section{Conjunctions in Turkish}

In this study, a conjunction refers to the whole structure of conjoined elements, for example conjunction of two nouns like \textit{kalem ve kitap} `pencil and book'. A morpheme that signals or carries out the conjunction is called the conjoiner \textit{ve} `and', and the individual elements that are conjoined are called the conjuncts \textit{kalem} `pencil', and \textit{kitap} `book'. I use these terms in the rest of my study.

There are two free form conjoiners \textit{ve} `and' and \textit{veya} `or' in Turkish. These can be used both in verbal and nominal domains to conjoin arguments and sentences. (\ref{veyave}) shows some examples for the conjoiners \textit{ve} `and' and \textit{veya} `or'.

\begin{exe}
    \ex \label{veyave}
    \begin{xlist}
        \ex \textit{ve} `and' conjoining nouns\\*
        \gll Kalem ve kitap çok pahalı. \\ 
        pencil {\And} book very expensive \\
        \glt `The pencil and the book are expensive.'
        
        \ex \textit{ve} `and' conjoining sentences\\*
        \gll Ahmet ev-e gel-di ve Mehmet on-u gör-dü. \\
        A[{\Nom}] house-{\Dat} come-{\Pst}[{\Tsg}] {\And} M[{\Nom}] him-{\Acc} see-{\Pst}[{\Tsg}] \\
        \glt `Ahmet came home, and Mehmet saw him.'
    
        \ex \textit{veya} `or' conjoining nouns\\*
        \gll Ahmet kalem veya kitap al-mak iste-m-iyor. \\ 
        A[{\Nom}] pencil {\Or} book buy-{\Nmlz} want-{\Neg}-{\Prog}[{\Tsg}] \\
        \glt `Ahmet does not want to buy a book or a pencil.'
        
        \ex \textit{veya} `or' conjoining sentences\\*
        \gll Ahmet ev-e gel-di veya Mehmet kapı-yı aç-tı. \\
        A[{\Nom}] house-{\Dat} come-{\Pst}[{\Tsg}] {\Or} M[{\Nom}] door-{\Acc} open-{\Pst}[{\Tsg}] \\
        \glt `Ahmet came home, or Mehmet opened the door.'
    \end{xlist}
\end{exe}

These conjoiners can have different functions depending on what they conjoin or in which environment they are used. A conjoiner like \textit{ve} `and' can have additive effects when used with nouns such as \textit{kalem ve kitap} `pencil and book', ordering effects when used with verbs such as \textit{koştum ve düştüm} `I ran and then fell'. A conjoiner does not necessarily need to be overt, prosodic breaks can signal conjunction like in \textit{domates, biber, patlıcan} `tomato, pepper, and eggplant'. In Turkish there are some overt prosodic operators that function like the conjoiners. These are: \textit{hem \ldots hem (de)\ldots}, \textit{ya \ldots ya (da) \ldots}, and \textit{(ya)\ldots ya da \ldots}, I give some examples in (\ref{twopartconjoiners}).

\begin{exe}
    \ex \label{twopartconjoiners}
    \begin{xlist}
    \ex \gll Ahmet hem kitab-ı hem (de) kalem-i al-dı. \\ 
    A[{\Nom}] hem book-{\Acc} hem (={\Foc}) pencil-{\Acc} take-{\Pst}[{\Tsg}] \\
    \glt `Ahmet bought both the book and the pencil'
    
    \ex \gll Ahmet ya kitab-ı ya (da) kalem-i al-dı. \\ 
    A[{\Nom}] ya book-{\Acc} ya (={\Foc}) pencil-{\Acc} take-{\Pst}[{\Tsg}] \\
    \glt `Ahmet either both the book or the pencil'
    \end{xlist}
\end{exe}

 
 
 
 