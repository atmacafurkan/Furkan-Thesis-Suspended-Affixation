\section{Analysis of the suffix \textit{-(y)Ip}} 

In this section I discuss the status of the suffix \textit{-(y)Ip} ({\Pc} in glosses). I give the structural interpretations that it should be evaluated under and the properties of the environment it forms. I argue for it to be evaluated as an environment of conjunction where SA beyond a morphological word is carried out.

\subsection{What is \textit{-(y)Ip}}

The suffix \textit{-(y)Ip} is used with verbs and only allows bare verbs, Voice, Mod$_{Abil}$, Negation, and the  suffix \textit{-(y)Iver} (I take this as Asp$_{Con}$ \cite{cinque1999adverbs} and as {\Con} in glosses) before it.  In (\ref{ipintro}), I give a set of examples for \textit{-(y)Ip}.

\begin{exe}
    \ex \label{ipintro}
    \begin{xlist}
    \ex Bare verb\\*
    \gll Ahmet koş-up düş-tü. \\ 
    A[{\Nom}] run-{\Pc} fall-{\Pst}[{\Tsg}] \\
    \glt `Ahmet ran and fell'
    
    \ex verb-{\Caus}\\*
    \gll Ahmet şişe-yi dol-dur-up temizle-di. \\
    A[{\Nom}] bottle-{\Acc} fill-{\Caus}-{\Pc} clean-{\Pst}[{\Tsg}]\\
    \glt `Ahmet filled the bottle and cleaned it.'
    
    \ex verb-{\Abil}\\*
    \gll Ahmet mantıklı düşün-ebil-ip sorun-u çöz-dü. \\ 
    A[{\Nom}] sensible think-{\Abil}-{\Pc} problem-{\Acc} solve-{\Pst}[{\Tsg}] \\
    \glt `Ahmet was able to think sensibly and solved the problem.'
    
    \ex verb-{\Neg}\\*
    \gll Ahmet ev-e gel-me-yip bekle-di. \\
    A[{\Nom}] house-{\Dat} come-{\Neg}-{\Pc} wait-{\Pst}[{\Tsg}] \\
    \glt`Ahmet did not come home and waited.'
    
    \ex verb-{\Con}\\*
    \gll Ahmet bulaşıklar-ı yıka-yıver-ip otur-du. \\
    A[{\Nom}] dishes-{\Acc} wash-{\Con}-{\Pc} sit-{\Pst}[{\Tsg}]\\
    \glt `Ahmet managed to wash the dishes and sat down.'
    
    
    \ex verb-{\Abil}-{\Neg}-{\Con}\\*
    \gll Ahmet tutun-a-ma-yıver-ip düş-tü. \\ 
    A[{\Nom}] hold-{\Abil}-{\Neg}-{\Con}-{\Pc} fall-{\Pst}[{\Tsg}]\\
    \glt`Ahmet could not manage to hold on and fell.'
    \end{xlist}
\end{exe}

There are several arguments for its structural interpretation but they mainly boil down to converb adverbial \citep{demir2014adverbial, underhill1976turkish, goksel2004turkish}, and converb conjoiner \citep{fokkens2009inflectional, johanson1995turkic, kornfilt1997turkish} analyses. In this subsection I show whether \textit{-(y)Ip} is a conjoiner or an adverbial. The sentences in (\ref{ipintro}) show that \textit{-(y)Ip} can conjoin two predicates that do not match in their Voice, Modality, and Polarity features. One contrasting behaviour of \textit{-(y)Ip} compared to other adverbial markers \textit{-(y)IncA} and \textit{-mAdAn} is given in (\ref{ipcontrast}). Under the same argument settings, \textit{-(y)Ip} is unacceptable\footnote{Contrasting subjects are grammatical with \textit{-(y)Ip} but they require changes in information structure, otherwise they are unacceptable. Exact grammatical considerations for \textit{-(y)Ip} constructions will be addressed in \S\ref{moreonyip}.} unlike \textit{-(y)IncA} and \textit{-mAdAn} ({\Pc}, {\When}, and {\Wo} in glosses respectively).

\begin{exe}
    \ex \label{ipcontrast}
    \begin{xlist}
        \ex \gll Ahmet koş-unca Mehmet düş-tü. \\ 
        A[{\Nom}] run-{\When} M[{\Nom}] fall-{\Pst}[{\Tsg}] \\
        \glt `When Ahmet ran, Mehmet fell.'
        
        \ex \gll Ahmet koş-madan Mehmet düş-tü. \\
        A[{\Nom}] run-{\Wo} M[{\Nom}] fall-{\Pst}[{\Tsg}] \\
        \glt `Mehmet fell before Ahmet ran.'
        
        \ex \gll ??Ahmet koş-up Mehmet düş-tü. \\ 
        A[{\Nom}] run-{\Pc} M[{\Nom}] fall-{\Pst}[{\Tsg}] \\
        \glt Intended `Ahmet ran and Mehmet fell.'
    \end{xlist}
\end{exe}

An objection to this observation can come from the adverbial suffix \textit{-(y)ArAK} `$\sim$ by Ving' ({\By} in glosses). It also results in the same ungrammaticality as in (\ref{ipcontester}).

\begin{exe}
    \ex \label{ipcontester}
    \begin{xlist}
    \ex \gll Ahmet koş-arak düş-tü. \\
    A[{\Nom}] run-{\By} fall-{\Pst}[{\Tsg}] \\
    \glt `Ahmet fell running'
    
    \ex \gll *Ahmet koş-arak Mehmet düş-tü. \\ 
    A[{\Nom}] run-{\By} M[{\Nom}] fall-{\Pst}[{\Tsg}] \\
    \end{xlist}
\end{exe}

\textit{-(y)Ip} deviates from \textit{-(y)ArAK} in verb-manner relation. The verb marked with \textit{-(y)ArAK} requires semantic compatibility with the main verb. If the derived reading with \textit{-(y)ArAK} is not semantically compatible as a manner for the main verb, the expression receives an odd reading. Verbs that are marked with \textit{-(y)Ip} do not require such a compatibility of manner. Manner relations are usually carried out by adverbs and adverbial clauses. In (\ref{ipandarak}), the suffix \textit{-(y)ArAK} is bound by verb-manner interpretations just like any other adverb whereas \textit{-(y)Ip} is not.

\begin{exe}
    \ex \label{ipandarak}
    \begin{xlist}
        \ex \label{ipandarak1} 
        \gll Ahmet koş-up uyu-du. \\
        A[{\Nom}] run-{\Pc} sleep-{\Pst}[{\Tsg}] \\
        \glt `Ahmet ran and slept.'
        
        \ex \label{ipandarak2} 
        \gll \%Ahmet koş-arak uyu-du. \\ 
        A[{\Nom}] run-{\By} sleep-{\Pst}[{\Tsg}] \\
        \glt `\%Ahmet slept running.'
    \end{xlist}
\end{exe}


An additional contrast of \textit{-(y)Ip} comes about in word order configurations. \textit{-(y)Ip} does not allow a word ordering under same argument settings as an adverbial suffix like \textit{-(y)ArAK} would allow. (\ref{ipandwordorder}) shows some word orderings for \textit{-(y)Ip} and \textit{-(y)ArAK}\footnote{Remember that word order changes are not free of interpretation in Turkish, they result in different information settings. See  \cite{ozturk2002turkish} for word order and change effects in Turkish.}. In these word orderings, the verb marked with \textit{-(y)Ip} and the main verb need to stay as a unit for a grammatical sentence.

\begin{exe}
\ex \label{ipandwordorder}
\begin{xlist}
    \ex \begin{xlisti}
        \ex \gll Ahmet koş-up gel-di. \\ 
        A[{\Nom}] run-{\Pc} come-{\Pst}[{\Tsg}] \\
        
        \ex \gll *koş-up Ahmet gel-di. \\ 
        run-{\Pc} A[{\Nom}] come-{\Pst}[{\Tsg}] \\
        
        \ex \gll koş-up gel-di Ahmet. \\ 
        run-{\Pc} come-{\Pst}[{\Tsg}] A[{\Nom}] \\
        \glt `Ahmet ran and came'
    \end{xlisti}
    
    \ex \begin{xlisti}
        \ex \gll Ahmet koş-arak gel-di. \\ 
        A[{\Nom}] run-{\By} come-{\Pst}[{\Tsg}] \\
        
        \ex \gll koş-arak Ahmet gel-di. \\ 
        run-{\By} A[{\Nom}] come-{\Pst}[{\Tsg}] \\
        
        \ex \gll koş-arak gel-di Ahmet. \\ 
        run-{\By} come-{\Pst}[{\Tsg}] A[{\Nom}] \\
        \glt `Ahmet came running'
        \end{xlisti}
    \end{xlist}
\end{exe}

This difference in grammaticality does not mean that \textit{-(y)Ip} has to be adjacent to the main verb, but it means that any word ordering needs to take the verb marked with \textit{-(y)Ip} and the main verb as equivalent units. If the verb marked with \textit{-(y)Ip} were to be a unit of modification for the main verb, all word order changes should have resulted in reading differences rather than ungrammaticalities. I argue that the observations made in this section distinguishes \textit{-(y)Ip} from an adverbial forming suffix. In the following section I lay out how \textit{-(y)Ip} is taken to be a conjoiner.


\subsection{What does \textit{-(y)Ip} conjoin}

In the literature where \textit{-(y)Ip} is evaluated as a conjoiner \citep{fokkens2009inflectional,johanson1995turkic,kornfilt1997turkish}, it is given the status of conjoining VPs. On first sight, the lack of any tense and agreement marker leads to evaluating \textit{-(y)Ip} as a conjoiner of VPs. Figure \ref{fig:ipconjoinerearly} illustrates this analysis.

\begin{figure}[hbt!]
    \centering
    \begin{forest}
    [VP 
        [BP 
            [VP]
            [B\\\textit{-(y)Ip}]]
        [VP]]    
    \end{forest}
    \caption{Early conjoiner analysis of \textit{-(y)Ip}}
    \label{fig:ipconjoinerearly}
\end{figure}

Conjoining only VPs might be warranted given the lack of overt inflections for the \textit{-(y)Ip} marked verb, but this analysis has couple of problems. First of which is the ability of \textit{-(y)Ip} marked verb to have inflectional suffixes of Negation, and Modality. These should immediately elevate the representation of VP to a higher structure. Not all inflectional markers are represented by overt heads, but the existence of them can be addressed cross-linguistically. The observations of \cite{cinque1999adverbs,cinque2001note} show that multiple inflectional levels for Tense, Aspect and Modality exist. These inflectional levels can have functional projections that take specific types of adverbs. These adverbs reside in the specifier position of the functional projections. For example, the two time adverbials \textit{bugün} `today' and \textit{yarın} `tomorrow' can occupy the specifier position of Tense$_{\Fut}$. If they are both used in one sentence, they form a complex adverbial that means `soon' as illustrated in (\ref{buyarin}).  

\begin{exe}
\ex \label{buyarin} 
\gll Ahmet bugün yarın kitab-ı al-ıp gel-ecek. \\ 
A[{\Nom}] today tomorrow book-{\Acc} take-{\Pc} come-{\Fut}[{\Tsg}] \\
\glt Ahmet will buy the book and come here soon.'
\end{exe}

This is easily predicted by an analysis of VP conjunction for \textit{-(y)Ip} and functional projection for Tense$_{\Fut}$ since a VP can later be marked with a single projection of tense and both adverbs occupy the same position and form a compound. The \textit{-(y)Ip} clause can have an adverb to itself that is different from the main verb. In (\ref{ipconjunctionstime}), I provide an example where \textit{-(y)Ip} marked verb and the main verb differ in their time adverbial. In (\ref{ipconjunctionstime}), performing a conjunction of VPs require only one inflectional projection of Tense$_{\Fut}$ but \textit{-(y)Ip} clause can have its own time adverbial different than the matrix clause.

\begin{exe}
\ex \label{ipconjunctionstime}
\gll Ahmet bugün kitab-ı al-ıp yarın defter-i kapla-yacak. \\
A[{\Nom}] today book-{\Acc} buy-{\Pc} tomorrow notebook-{\Acc} wrap-{\Fut}[{\Tsg}] \\
\glt `Ahmet will buy the book today and will wrap the notebook tomorrow.'
\end{exe}

Another functional projection that can be used to illustrate higher level of conjunction for \textit{-(y)Ip} comes from speech act adverbials. The two speech act adverbials \textit{dürüstçe} `frankly' and \textit{sinsice} `deviously' result in a semantically odd reading if they are used in one sentence. (\ref{dursin}) shows the resulting odd reading.

\begin{exe}
\ex \label{dursin} 
\gll \%Ahmet dürüstçe ve sinsice konuş-up davran-dı. \\ 
A[{\Nom}] frankly {\And} deviously talk-{\Pc} behave-{\Pst}[{\Tsg}] \\
\glt Intended: `Ahmet talked and behaved frankly and deviously.'
\end{exe}

Placing one of the adverbs under the \textit{-(y)Ip} marked verb and the other under the main verb does away with the odd reading in (\ref{dursin}). If the two speech adverbials were to occupy the same inflectional level, the example in (\ref{ipconjunctionsspeech}) should have also been semantically odd.

\begin{exe}
\ex \label{ipconjunctionsspeech}
\gll Ahmet dürüstçe konuş-up sinsice davran-ıyor. \\
A[{\Nom}] frankly talk-{\Pc} deviously act-{\Prog}[{\Tsg}] \\
\glt `Ahmet is talking honestly but acting deviously.'
\end{exe}

All the observations in this section shows that \textit{-(y)Ip} does not conjoin VPs but higher projections. According to \cite{cinque1999adverbs}, speech act adverbials are used with the functional projection Mood$_{speech\,act}$ that is higher than Tense and Aspect projections. I argue that \textit{-(y)Ip} is a conjoiner for full sentences. In the following section I give my analysis for \textit{-(y)Ip} conjunctions.


\subsection{Analysis of \textit{-(y)Ip}}

Analyzing \textit{-(y)Ip} as a conjunction that conjoins full sentences requires the explanation for missing and non-insertable suffixes. These suffixes range from aspect markers to person agreement markers. I propose that \textit{-(y)Ip} is a result of vocabulary insertion after SA. The specific reason for why \textit{-(y)Ip} is chosen instead of a free form conjoiner like \textit{ve} `ve' is a morphological one. In a conjunction of two sentences, suspension of suffixes on the verb is performed and a non-morphological word is left. This results in the violation of the morphological word constraint. The verb after deletion of exponents can not stand on its own. That is why a bound form conjoiner like \textit{-(y)Ip} is inserted for the conjoiner head, instead of a free form conjoiner \textit{ve} `ve'. This way I combine both conjoiner function of \textit{-(y)Ip} and the explanation for missing suffixes. I provide my analysis in Figure \ref{fig:ipanalysis}. In Turkish there aren't overt suffixes for `C' that can be suspended in \textit{-(y)Ip} constructions, that is why they are not interpreted under SA. 

\begin{figure}[hbt!]
    \centering
\begin{forest} for tree= sn edges
    [CP 
        [BP 
            [CP 
                [TP 
                    [\ldots 
                        [\ldots]
                        [X\\\cancel{-$\alpha$}, name=innerx]]
                    [T\\\cancel{-$\beta$}, name=innert]]
                [C]]
            [B\\ \textit{-(y)Ip}]]
        [CP 
            [TP 
                [\ldots 
                    [\ldots]
                    [X\\-$\alpha$, name=outerx]]
                [T\\-$\beta$, name=outert]]
            [C]]]
\node[fit= (innerx)(innert), draw, ellipse, dashed, rotate=155, scale=.75](inner){};
\node[fit= (outerx)(outert), draw, ellipse, dashed, rotate=155, scale=.75](outer){};
\node[below right= 4em of inner](innerlabel){SA};
\node[below right= 4em of outer](outerlabel){Overt suffixes};
\draw[->, thick, dashed] (outerlabel) to[out=south, in=east] node[midway, fill=white]{\textsc{recover}}(innerlabel);
\end{forest}
    \caption{Structural analysis proposal for \textit{-(y)Ip}}
    \label{fig:ipanalysis}
\end{figure}


In (\ref{ipmultiple}), I provide multiple \textit{-(y)Ip} constructions that are the results of being left with non-morphological words after SA.

\begin{exe}
\ex \label{ipmultiple}
\gll Ahmet ev-e gel-ip soyun-up uyu-muş-tur. \\
A[{\Nom}] house-{\Dat} come-{\Pc} undress-{\Pc} sleep-{\Prf}-PROB[{\Tsg}] \\
\glt `Ahmet has probably come home, undressed, and slept.'
\end{exe}

In (\ref{deriveipmultiple}) I give the order of derivation that leads to SA beyond the morphological word and the multiple instances of \textit{-(y)Ip}.

\begin{exe}
\ex \label{deriveipmultiple}
\begin{xlisti}
\ex Conjunction of n many sentences, with matching rightmost suffixes $\beta$
\ex Delete the matching nodes\\
    \begin{tikzpicture}
    \node[](leftr){};
    \node[right=0em of leftr](conj1){V$_1$\cancel{$-\beta$}};
    \node[right=0em of conj1](conj2){V$_2$\cancel{$-\beta$}};
    \node[right=0em of conj2](dots){\ldots};
    \node[right=0em of dots](conjn){V$_n-\beta$};
    \node[below=1.55em of leftr](leftcorner){};
    \draw[rounded corners=.5em, ->] (conjn.south) -- +(south:1.5em) -- node[below]{deletion}(leftcorner.east);
    \end{tikzpicture}
\ex Verbs are not morphological words, insert bound form conjoiner \textit{-(y)Ip}\\
    \begin{tikzpicture}
    \node[](leftrz){};
    \node[right=0em of leftrz](conj1){V$_1$\cancel{-$\beta$}\textit{-(y)Ip}};
    \node[right=0em of conj1](conj2){V$_2$\cancel{-$\beta$}\textit{-(y)Ip}};
    \node[right=0em of conj2](dots){\ldots};
    \node[right=0em of dots](conjn){V$_n$-$\beta$};
    \end{tikzpicture}
\end{xlisti}
\end{exe}


One possible problem for a conjoiner \textit{-(y)Ip} is its ability to co-exist with another conjoiner \textit{hem \ldots hem de \ldots} `both \ldots and \ldots' that seemingly carries out the function of the conjoiner \textit{ve} `and'. I do not take \textit{hem \ldots hem de \ldots} `both \ldots and \ldots' as a conjoiner but as focus particles that operate on the conjuncts. The counterpart \textit{ya \ldots ya da \ldots} `either \ldots or \ldots' that serves as the focus particle for the exclusive \textit{veya} `or' is ungrammatical. I give the example in (\ref{problemip}) to serve the point.

\begin{exe}
\ex \label{problemip} 
\begin{xlist}
\ex \gll hem tez yaz-ıp hem de çalış-mak ist-iyor-sun. \\ 
{\Foc} thesis write-{\Pc} {\Foc} {\Ptcp} work-{\Nmlz} want-{\Prog}-{\Ssg} \\
\glt `(You) want to both write your thesis and work.'

\ex \gll *ya tez yaz-ıp ya da çalış-mak ist-iyor-sun. \\ 
{\Foc} thesis write-{\Pc} {\Foc} {\Ptcp} work-{\Nmlz} want-{\Prog}-{\Ssg} \\

\ex \gll ya tez yaz-mak ya da çalış-mak ist-iyor-sun. \\  
{\Foc} thesis write-{\Nmlz} {\Foc} {\Ptcp} work-{\Nmlz} want-{\Prog}-{\Ssg} \\
\glt `You either want to write your thesis or you want to work.'
\end{xlist}
\end{exe}


In addition to its ability of co-existing with other conjoiner markers, \textit{-(y)Ip} can co-occur with the free form conjoiner \textit{ve} `and' as in (\ref{yipve}). These observations of \textit{-(y)Ip} co-existing with other conjoiners do not render a conjoiner analysis obsolete. There are many cases where conjoiners are used for the purposes of changing information structure in a sentence. What is important is that in both (\ref{problemip}) and (\ref{yipve}), \textit{-(y)Ip} is only grammatical in an environment where the interpretation of the sentence is equivalent to a conjunction formed by \textit{ve} `and'.

\begin{exe}
    \ex \label{yipve}
    \gll Ev-e gid-ip ve de anahtar-ı al-ma-mak tam bir aptallık. \\ 
    house-{\Dat} go-{\Pc} {\And} {\Foc} key-{\Acc} take-{\Neg}-{\Nmlz} complete a stupidity \\
    \glt `Going all the way home and not taking the key is a complete stupidity.'
\end{exe}


All the examples in the last two subsections distinguished \textit{-(y)Ip} from adverbial markers and presented it as a conjoiner of sentences. With these observations at hand, I propose that \textit{-(y)Ip} is a conjoiner that elevates the verb to a morphological word status. It does not occupy the morphological slots for the suspended suffixes, it surfaces after the suspension of affixes to satisfy the morphological word constraint. In the following subsection I discuss why SA seems to be obligatory in \textit{-(y)Ip} constructions and why the environment of \textit{-(y)Ip} is important for the discussion of SA.


\subsection{Suspended affixation and \textit{-(y)Ip}} \label{sec:SAandyip}

A morphological word is defined if the last morpheme can terminate a word independent of agreement markers \citep{kabak2007turkish}. After SA in verbs, the conjoiner \textit{`ve'} is selected if what is left is a morphological word and the conjoiner \textit{-(y)Ip} is selected if what is left after SA is not a morphological word. 

This means that insertion for the conjoiner is dependent on the morphological status of what is left after SA. I give a set of examples in (\ref{ipandsa}) that show that SA is grammatical in verbs with non-morphological word status if the conjoiner is \textit{-(y)Ip} and SA is grammatical in verbs with morphological word status if the conjoiner is \textit{ve} `and'.

\begin{exe}
\ex \label{ipandsa}
\begin{xlist}
\ex \begin{xlisti}
\ex Non-morphological word, \textit{ve} `and'\\*
\gll *kitab-ı oku ve anla-malı-ydı-m. \\ 
book-{\Acc} read {\And} understand-{\Nec}-{\Pst}-{\Fsg} \\
\glt ${}$

\ex Non-morphological word, \textit{-(y)Ip}\\*
\gll kitab-ı oku-yup anla-malı-ydı-m. \\ 
book-{\Acc} read-{\Pc} understand-{\Nec}-{\Pst}-{\Fsg} \\
\glt `I should have read and understood the book.'
\end{xlisti}

\ex \begin{xlisti}
\ex Morphological word, \textit{ve} `and'\\* 
\gll kitab-ı oku-malı ve anla-malı-ydı-m. \\ 
book-{\Acc} read-{\Nec} {\And} understand-{\Nec}-{\Pst}-{\Fsg} \\
\glt `I should have read and understood the book.'

\ex Morphological word, \textit{-(y)Ip}\\*
\gll *kitab-ı oku-malı-yıp anla-malı-ydı-m. \\ 
book-{\Acc} read-{\Nec}-{\Pc} understand-{\Nec}-{\Pst}-{\Fsg} \\
\glt ${}$
\end{xlisti}
\end{xlist}
\end{exe}

The sentences in (\ref{ipandsa}) show that the vocabulary item for the conjoiner head is selected after SA is performed. This also explains why the suffixes \textit{-mA} and \textit{-(y)Abil} can reside under the conjoiner \textit{-(y)Ip} because they do not form morphological words. I give a set of examples in (\ref{negabilip}).

\begin{exe}
\ex \label{negabilip}
\begin{xlist}
\ex {\Abil} \begin{xlisti}
\ex \gll *kitab-ı oku-yabil ve anla-yabil-miş-im. \\ 
book-{\Acc} read-{\Abil} {\And} understand-{\Abil}-{\Prf}-{\Fsg} \\
\glt ${}$

\ex \gll kitab-ı oku-yabil-ip anla-yabil-miş-im. \\ 
book-{\Acc} read-{\Abil}-{\Pc} understand-{\Abil}-{\Prf}-{\Fsg} \\
\glt `It seems like I was able to read and understand the book.'
\end{xlisti}

\ex {\Abil-\Prf} \begin{xlisti}
\ex \gll kitab-ı oku-yabil-miş ve anla-yabil-miş-im. \\ 
book-{\Acc} read-{\Abil}-{\Prf} {\And} understand-{\Abil}-{\Prf}-{\Fsg} \\
\glt `It seems like I was able to read and understand the book.'

\ex \gll *kitab-ı oku-yabil-miş-ip anla-yabil-miş-im. \\ 
book-{\Acc} read-{\Abil}-{\Prf}-{\Pc} understand-{\Abil}-{\Prf}-{\Fsg} \\
\glt ${}$
\end{xlisti}

\ex {\Neg} \begin{xlisti}
\ex \gll *kitab-ı oku-ma ve anla-ma-mış-ım. \\ 
book-{\Acc} read-{\Neg} {\And} understand-{\Neg}-{\Prf}-{\Fsg} \\
\glt ${}$

\ex \gll kitab-ı oku-ma-yıp anla-ma-mış-ım. \\ 
book-{\Acc} read-{\Neg}-{\Pc} understand-{\Neg}-{\Prf}-{\Fsg} \\
\glt `It seems like I haven't read the book and understood it.'
\end{xlisti}

\ex {\Neg-\Prf}\begin{xlisti}
\ex \gll kitab-ı oku-ma-mış ve anla-ma-mış-ım. \\ 
book-{\Acc} read-{\Neg}-{\Prf} {\And} understand-{\Neg}-{\Prf}-{\Fsg} \\
\glt `It seems like I haven't read the book and understood it.'
\ex \gll *kitab-ı oku-ma-mış-ıp anla-ma-mış-ım. \\ 
book-{\Acc} read-{\Neg}-{\Prf}-{\Pc} understand-{\Neg}-{\Prf}-{\Fsg} \\

\end{xlisti}
\end{xlist}
\end{exe}

One prediction of my analysis for SA where \textit{-(y)Ip} is present is that ambiguous cases of SA should be possible if what is left after SA is still not a morphological word. I provide such an ambiguity in (\ref{ipambiguity}). In this example, SA of {\Neg} is optional since its existence or lack of it in the first conjunct does not change the morphological word status of the conjunct.

\begin{exe}
\ex \label{ipambiguity} 
\gll Ahmet ev-e gel-ip kitab-ı oku-ma-dı. \\ 
A[{\Nom}] house-{\Dat} come-{\Pc} book-{\Acc} read-{\Neg}-{\Pst}[{\Tsg}] \\
\glt `Ahmet did not come home and did not read the book.'\\*
`Ahmet came home but did not read the book.'
\end{exe}

If SA is performed for the suffixes {\Neg}-{\Pst}[{\Tsg}] which is all the way to the bare verb itself, you get the first reading of (\ref{ipambiguity}). If SA is performed for {\Pst}[{\Tsg}], you get the second reading of (\ref{ipambiguity}). In both readings what is left after SA is not a morphological word since neither can {\Neg} nor a bare verb can terminate a word independent of agreement markers. This results in the selection of \textit{-(y)Ip} as a conjoiner. Figure \ref{fig:ambiguousip} represents the two readings of (\ref{ipambiguity}).

\begin{figure}[hbt!]
    \centering
\begin{forest} for tree= sn edges
    [\ldots 
        [BP 
            [\ldots 
                [VoiceP]
                [\ldots, name=target]]
            [B\\\textit{-(y)Ip}]]
        [\ldots 
            [\ldots 
                [VoiceP]
                [Neg, name=NEG]]
            [T, name=T, draw, circle, thick, dashed]]]   
\node[fit= (T)(NEG), draw, ellipse, thick, dashed, rotate=155, scale=.85](SA){};
\draw[->,thick,dashed] (SA) to[out=south west, in=south] node[midway, fill=white]{reading 1}(target){};
\draw[->, thick] (T) .. controls +(south east:3) and +(south:4).. node[midway, fill=white]{reading 2} (target);
\end{forest}    
    \caption{Representation of ambiguous \textit{-(y)Ip}}
    \label{fig:ambiguousip}
\end{figure}

The analysis I provided for \textit{-(y)Ip} captures its properties and what it can conjoin. It is directly related to SA and the selection for a conjoiner in the conjunction environment. It allows for ambiguous readings observed in (\ref{ipambiguity}). It shows that \textit{-(y)Ip} is a conjoiner that is affixed to non-morphological words and grants verbs the morphological word status. In the following section I address a grammaticality condition for \textit{-(y)Ip} constructions that are not directly related to SA.


\subsection{More on \textit{-(y)Ip}} \label{moreonyip}

In the previous subsections I have shown what the structural interpretation for \textit{-(y)Ip} and its relation to SA is. There is one additional property of \textit{-(y)Ip} constructions that falls out of the scope of this study, yet holds a crucial distinction for how SA is considered. I have provided (\ref{ipcontrast}) for arguing that \textit{-(y)Ip} is different from other adverbial markers. That example hosts an ungrammatical sentence formed with \textit{-(y)Ip}. To show that it is grammatical under a free form conjoiner like \textit{ve} `and' I repeat the same example in (\ref{ipvsve}) with an additional sentence.

\begin{exe}
\ex \label{ipvsve}
    \begin{xlist}
        \ex \label{ipvsveip}
        \gll *Ahmet ev-e gel-ip Mehmet uyu-du. \\ 
        A[{\Nom}] house-{\Dat} come-{\Pc} M[{\Nom}] sleep-{\Pst}[{\Tsg}]\\

        \ex \gll Ahmet ev-e gel-di ve Mehmet uyu-du. \\ 
        A[{\Nom}] house-{\Dat} come-{\Pst}[{\Tsg}] {\And} M[{\Nom}] sleep-{\Pst}[{\Tsg}] \\
        \glt `Ahmet came home and Mehmet slept.'
    \end{xlist}
\end{exe}

In my analysis, I have argued for \textit{-(y)Ip} to be considered as a conjoiner head after an SA beyond a morphological word is carried out. If this was purely the case, both sentences in (\ref{ipvsve}) should have been equally acceptable. This on the surface refutes a conjunction analysis. The remedy requires an investigation into the sentences that \textit{-(y)Ip} is acceptable with. \textit{-(y)Ip} constructions include a necessary topicalization of at least one phrase that is shared in both conjuncts. The ungrammatical (\ref{ipvsveip}) can be made grammatical by just adding a topicalized adverb that is shared by both conjuncts as in (\ref{ipvsremedy}). Some native speakers might find it difficult to interpret (\ref{ipvsremedy}). That is why I also give the example (\ref{ipvsremedy2}) that topicalizes an argument of the verb.

\begin{exe}
\ex \label{ipvsremedy}
\gll Tam o sırada Ahmet ev-e gel-ip Mehmet uyu-du. \\ 
right that time A[{\Nom}] house-{\Dat} come-{\Pc} M[{\Nom}] sleep-{\Pst}[{\Tsg}] \\
\glt`Right at that time Ahmet came home and Mehmet slept.

\ex \label{ipvsremedy2}
\gll kitab-ı Ahmet bul-up Mehmet oku-du. \\ 
book-{\Acc} A[{\Nom}] find-{\Pc} M[{\Nom}] read-{\Pst}[{\Tsg}] \\
\glt`Ahmet found the book and Mehmet read it.'
\end{exe}

This topicalization of argument is not limited to only one. Multiple arguments can be topicalized for \textit{-(y)Ip} constructions to be acceptable. I give a relatively extensive list of sentences in (\ref{ipextensive}).

\begin{exe}
\ex \label{ipextensive}
\begin{xlist}
\ex Topicalized Subject and Indirect Object\\*
\gll [Ali Deniz-e] Mehmet-i vurdur-up Naci-yi dövdür-dü. \\ 
A[{\Nom}] D-{\Dat} M-{\Acc} hit.{\Caus}-{\Pc} N-{\Acc} beat.{\Caus}-{\Pst}[{\Tsg}] \\
\glt `Ali made Deniz hit Mehmet and beat Naci.'

\ex Topicalized Subject and Object\\*
\gll [Ali Mehmet-i] Deniz-e vurdur-up Kadir-e dövdür-dü. \\ 
A[{\Nom}] M-{\Acc} D-{\Dat} hit.{\Caus}-{\Pc} K-{\Dat} beat.{\Caus}-{\Pst}[{\Tsg}] \\
\glt `Ali made Deniz hit Mehmet and made Naci beat Mehmet.'

\ex Topicalized Object and Indirect Object\\*
\gll [Mehmet-i Deniz-e] Ali vurdur-up Osman dövdür-dü. \\ 
M-{\Acc} D-{\Dat} A[{\Nom}] hit.{\Caus}-{\Pc} O[{\Nom}] beat.{\Caus}-{\Pst}[{\Tsg}] \\
\glt `Ali made Deniz hit Mehmet and Osman made Deniz beat Mehmet.'

\ex Topicalized Subject \label{iptops}\\*
\gll [Ali] Deniz-e Mehmet-i vurdur-up Kadir-e Naci-yi dövdür-dü. \\ 
A[{\Nom}] D-{\Dat} M-{\Acc} hit.{\Caus}-{\Pc} K-{\Dat} N-{\Acc} beat.{\Caus}-{\Pst}[{\Tsg}] \\
\glt `Ali made Deniz hit Mehmet and made Kadir beat Naci.'

\ex Topicalized Object\\*
\gll [Mehmet-i] Ali Deniz-e vurdur-up Kadir Naci-ye dövdür-dü. \\ 
M-{\Acc} A[{\Nom}] D-{\Dat} hit.{\Caus}-{\Pc} K[{\Nom}] N-{\Dat} beat.{\Caus}-{\Pst}[{\Tsg}] \\
\glt `Ali made Deniz hit Mehmet and Kadir made Naci beat Mehmet.' 

\ex Topicalized Indirect Object\\*
\gll [Deniz-e] Ali Mehmet-i vurdur-up Osman Naci-yi dövdür-dü. \\ 
D-{\Dat} A[{\Nom}] M-{\Acc} hit.{\Caus}-{\Pc} O[{\Nom}] N-{\Acc} beat.{\Caus}-{\Pst}[{\Tsg}] \\
\glt`Ali made Deniz hit Mehmet and Osman made Deniz beat Naci'
\end{xlist}
\end{exe}

(\ref{ipextensive}) shows that \textit{-(y)Ip} constructions require at least one gap for grammaticality even though the gaps do not correspond to the same order. Topicalizing is not the only requirement for an acceptable \textit{-(y)Ip} construction. The order of the focalized arguments also matter. I give (\ref{ipungram}) for a version of (\ref{iptops}). This property of \textit{-(y)Ip} constructions presents a key point for SA. In no other environment does SA occur so closely related to the changes in information structure. 

\begin{exe}
\ex \label{ipungram}
\gll ??[Ali] Deniz-e Mehmet-i vurdur-up Naci-yi Kadir-e dövdür-dü. \\ 
A[{\Nom}] D-{\Dat} M-{\Acc} hit.{\Caus}-{\Pc} N-{\Acc} K-{\Dat} beat.{\Caus}-{\Pst}[{\Tsg}] \\
\glt ${}$
\end{exe}



