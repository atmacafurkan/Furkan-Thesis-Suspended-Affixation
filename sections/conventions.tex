\section{Conventions utilized in the thesis}

I follow Leipzig glossing conventions \citep{comrie2008leipzig} in my language glosses\footnote{
capital letters are used to indicate possible phonological changes. K = [k] or [\textgamma], A = [\textscripta] or [e], I = [\textturnm], [i], [u], or [ü], C = [\textdyoghlig] or [\textteshlig], and D = [d] or [t]}. Concatenated morphemes are separated with a dash `-' like \textit{araba-lar} `car-{\Pl}', and non-concatenative morphemes are separated with a dot `.' like \textit{araba-m} `car-{\Poss}.{\Fsg}'. I use square brackets `[]' to indicate a morpheme with zero exponent like \textit{git-ti.} `go-{\Pst}[{\Tsg}]'. Words that hold specific relations are provided subscripts. For example, a subscript `i, j, k, \ldots' is used to indicate referents like `He$_i$ and Ahmet$_j$', a case assigning preposition or postposition can be marked by the case it assigns like `of$_{ACC}$'.


