\subsection{Ossetic}

\citet{erschler2012suspended} and \citet{erschler2018suspended} deal with SA in Ossetic. Ossetic is a language spoken in Caucasus. Ossetic displays a set of data that on the surface seems to be inconsistent when it comes to SA. For example, when a pronoun and a proper noun is conjoined, the choices of {\Case} for the both conjuncts change depending on the order of the conjuncts (\ref{ossetic}). In (\ref{ossetic1}), it seems there is no SA since the pronoun {\Ssg} is marked for {\Obl}. On the other hand, in (\ref{ossetic2}) there is SA of {\Abl} from the proper noun \textit{Alan}.

\begin{exe}
    \ex \label{ossetic} SA of {\Abl}\\*
    \begin{xlist}
        \ex \label{ossetic1} 
        \gll d\textturna w \textturna ma Alan-\textturna j tarst\textturna n. \\
        {\Ssg}.{\Obl} {\And} A-{\Abl} be.afraid.{\Pst}.{\Fsg} \\
        \glt `I am afraid of you and Alan.'
        
        \ex \label{ossetic2} 
        \gll Alan \textturna ma d\textturna w-\textturna j tarst\textturna n. \\
        A[{\Nom}] {\And} {\Ssg}-{\Abl} be.afraid.{\Pst}.{\Fsg} \\
        \glt `I am afraid of Alan and you.'\\*
        \hfill Adapted from \citet{erschler2012suspended}
    \end{xlist}
\end{exe}

\citet{erschler2012suspended} deals with SA of {\Case} in Ossetic. He provides some background into the case system of Ossetic before moving on with examples and analysis of SA. Definite animates, and personal pronouns are obligatorily marked {\Obl}, inanimate objects are marked {\Nom}, and modifiers are not case marked. All plural nouns in Ossetic lose their final [\textturna] sound when marked by vowel initial case markers. This is taken to be a phonological constraint since consonant initial case markers do not trigger the same alternation. (\ref{osseticdeletion}) shows an example of dropping [\textturna].

\begin{exe}
    \ex \label{osseticdeletion}
    \begin{xlist}
        \ex \gll {b\textturna\textchi-t\textturna} \\ horse-{\Pl}[{\Nom}] \\
        
        \ex \gll {b\textturna\textchi-t-\textschwa} \\ horse-{\Pl}-{\Obl} \\
        \glt \hfill Adapted from \citet{erschler2012suspended}
    \end{xlist}
\end{exe}

Erschler proposes some constraints, first of which is that any case marker can be suspended. This is not so much of a constraint but an observation. The examples in (\ref{osseticSA}) host SA for {\Obl}, {\Sup}, {\Abl}, and {\Loc}.

\begin{exe}
    \ex \label{osseticSA}
    \begin{xlist}
        \ex SA of {\Obl}\\* 
        \gll Soslan {\textturna ma} {Zalijn-i} {\textchi\textturna\textdzlig r\textturna} \\ 
        S {\And} Z-{\Obl} house. \\
        \glt `the house of Soslan and Zalina.'
        
        \ex SA of {\Sup}\\*
        \gll {Alan} {\textturna ma} {Soslan-b\textturna l} {is-\textturna mbaltt\textturna n}. \\ 
        A {\And} S-{\Sup} {\Prv}-meet.{\Pst}.{\Fsg} \\
        \glt `I met Alan and Soslan.'
        
        \ex SA of {\Abl}\\*
        \gll {Alan} {\textturna ma} {Soslan-b\textturna j} {tarst\textturna n}. \\ 
        A {\And} S-{\Abl} be.afraid.{\Pst}.{\Fsg} \\
        \glt `I was afraid of Alan and Soslan.'
        
        \ex SA of {\Loc}\\*
        \gll {budur} {\textturna ma} {\textinvscr\textturna d-i} {ber\textturna} {\v{c}'ewu-t\textturna} {i\v{s}-\v{s}erdtonc\textturna}. \\ 
        field {\And} forest-{\Loc} many bird-{\Pl} {\Prv}-find.{\Pst}.{\Tpl} \\
        \glt `They found many birds in the field and the forest.'\\*
        \hfill Adapted from \citet{erschler2012suspended}
    \end{xlist}
\end{exe}

The second constraint is that the first conjunct in SA should be the base of the case marker, without phonological processes like [\textturna] deletion (\ref{osseticSA2}).

\begin{exe}
    \ex \label{osseticSA2}
    \begin{xlist}
    \ex 
    \gll 
    {b\textturna\textchi-t-im\textturna} {\textturna m\textturna} {g\textturna l-t-im\textturna} \\ horse-{\Pl}-{\Com} {\And} ox-{\Pl}-{\Com} \\
    
    \ex 
    \gll 
    {*b\textturna\textchi-t} {\textturna m\textturna} {g\textturna l-t-im\textturna} \\ horse-{\Pl} {\And} ox-{\Pl}-{\Com} \\
    
    \ex 
    \gll 
    {b\textturna\textchi-ta} {\textturna m\textturna} {g\textturna l-t-im\textturna} \\ horse-{\Pl} {\And} ox-{\Pl}-{\Com} \\
    \glt `with horses and oxen'\\*
    \hfill Adapted from \citet{erschler2012suspended}
    \end{xlist}
\end{exe}

Complying with the same constraint, personal pronouns that have different bases for some of the cases need to have those bases as their remnants in the first conjunct (\ref{osseticSA3}).

\begin{exe}
    \ex \label{osseticSA3}
    \begin{xlist}
        \ex \gll 
        {d\textturna w/*du} {\textturna ma} {Alan-b\textturna l} {is-\textturna mbaltt\textturna n}. \\ {\Ssg}[{\Obl}]/*{\Ssg}[{\Nom}] {\And} A-{\Sup} {\Prv}-meet.{\Pst}.{\Fsg} \\
        \glt `I met you and Alan.'
        
        \ex \gll 
        {d\textturna w/*du} {\textturna ma} {Alan-\textturna j} {t\textturna rsun}. \\ {\Ssg}[{\Obl}]/*{\Ssg}[{\Nom}] {\And} A-{\Abl} be.afraid.{\Prs}.{\Fsg} \\
        \glt `I am afraid of you and Alan.'\\*
        \hfill Adapted from \citet{erschler2012suspended}
    \end{xlist}
\end{exe}

The third constraint for Ossetic SA is what is left after suspension should be an independent (morphological) word. The two branches of Ossetic differ in regarding a reciprocal form `each other' as an independent word. In Iron Ossetic, it is an independent word and can take part in SA whereas the Digor counterpart is not an independent word and does not take place in SA (\ref{osseticSA4}).

\begin{exe}
    \ex \label{osseticSA4}
    \begin{xlist}
        \ex Iron Ossetic\\*
        \gll {?n\textturna=d\textschwa w\textturna} {g\textturna dy-je} {k\textturna r\textturna zi} {\textturna m\textturna} {n\textturna=k$^w$\textschwa z-\textturna j} {t\textturna r\v{s}-\textschwa nc}. \\
        {\Poss}{\Fpl}=two cat-{\Obl} each.other {\And} {\Poss}{\Fpl}=dog-{\Abl} be.afraid.{\Prs}.{\Tpl} \\
        \glt `Our two cats are afraid of each other and of our dog.'
        
        \ex Digor Ossetic\\*
        \gll {*n\textturna=duw\textturna} {tiki\v{s}-i} {k\textturna r\textturna\textdyoghlig e} {\textturna ma} {n\textturna=kuj-\textturna j} {t\textturna rs-unc\textturna}. \\ {\Poss}{\Fpl}=two cat-{\Obl} each.other {\And} {\Poss}{\Fpl}=dog-{\Abl} be.afraid.{\Prs}.{\Tpl} \\
        \glt \hfill Adapted from \citet{erschler2012suspended}
    \end{xlist}
\end{exe}

The fourth constraint of Ossetic SA is that what is left after SA should not have idiosyncratic meaning. This constraint relates to the {\Third}{\Sg} pronoun form \textit{w\textschwa m} which has the meaning `there' that serves as the base for the Dative marked {\Third}{\Sg} pronoun (\ref{osseticSA5}).

\begin{exe}
    \ex \label{osseticSA5}
    \begin{xlist}
        \ex \gll {w\textschwa m} {\textturna m\textturna} {m\textturna din\textturna-j\textturna n} {didin\textdyoghlig\textschwa t\textturna} {ratta}. \\ 
        there {\And} M-{\Dat} flowers gave \\
    
    \ex \gll {w\textschwa m-\textturna n} {\textturna m\textturna} {m\textturna din\textturna-j\textturna n} {didin\textdyoghlig\textschwa t\textturna} {ratta}. \\ 
    {\Tsg}-{\Dat} {\And} M-{\Dat} flowers gave \\
    \glt `S/he gave flowers to her and Madina.'\\*
    \hfill Adapted from \citet{erschler2012suspended}
    \end{xlist}
\end{exe}

The final constraint for Ossetic SA is that when both conjuncts are pronouns no suspended affixation takes place, a point illustrated in (\ref{osseticSA6}).

\begin{exe}
    \ex \label{osseticSA6}
    \begin{xlist}
    \ex \gll 
    {m\textturna n-b\textturna l} {\textturna m\textturna} {d\textturna w-b\textturna l} {\textturna ww\textturna nduj}. \\ {\Fsg}-{\Sup} {\And} {\Ssg}-{\Sup} believe.{\Prs}.{\Third}{\Sg} \\
    \glt `S/he believes me and you.'
    
    \ex \gll 
    {*m\textturna n} {\textturna m\textturna} {d\textturna w-b\textturna l} {\textturna ww\textturna nduj}. \\ {\Fsg}[{\Obl}] {\And} {\Ssg}-{\Sup} believe.{\Prs}.{\Third}{\Sg} \\
    \glt Intended `S/he believes me and you.'\\*
    \hfill Adapted from \citet{erschler2012suspended}
    \end{xlist}
\end{exe}

Following these observations, Erschler argues that SA needs to be a phonological deletion process after vocabulary insertion instead of a structural sharing process. Erschler argues against an approach where case markers are treated as syntactic projections. This in turn makes the structural sharing argument less appealing. He provides the examples in (\ref{osseticdepict}) where the complements of adpositions cannot control depictives, but case marked arguments can.

\begin{exe}
    \ex \label{osseticdepict}
    \begin{xlist}
        \ex \gll 
        {soslan} {\textchi et\textturna g-i} {\textchi\textturna cc\textturna} {rasug-\textturna j} {\textdzlig or-uj}. \\ S[{\Nom}] X-{\Obl} with drunk-{\Abl} talk-{\Prs}.{\Third}{\Sg} \\
        \glt `Soslan$_i$ is talking to Xetag$_i$ when he$_{i/*j}$ is drunk.'
    
        \ex \gll 
        {soslan} {\textchi et\textturna g-b\textturna l} {rasug-\textturna j=d\textturna r} {\textturna ww\textturna nd-uj}. \\ S[{\Nom}] X-{\Sup} drunk-{\Abl}={\Emp} believe-{\Prs}.{\Third}{\Sg} \\
        \glt `Soslan$_i$ believes in Xetag$_i$ even when he$_{i/j}$ is drunk.'\\*
        \hfill Adapted from \citet{erschler2012suspended}
    \end{xlist}
\end{exe}

In \citet{erschler2018suspended}, he further develops the approach of ellipsis for SA. He provides the alternative question configurations in which SA can take place (\ref{osseticalternative}) to show that SA is an ellipsis process.

\begin{exe}
    \ex \label{osseticalternative}
    \begin{xlist}
        \ex \gll {s\textturna rm\textturna t(-m\textturna)} {\textturna vi} {uruzm\textturna g-m\textturna} {\textdzlig urdtaj?} \\ 
        S(-{\All}) {\Or}.{\Q} U-{\All} you.called \\ 
        \glt `Did you call Sarmat or Uruzmag?'
        
        \ex \gll {ad\textturna jmag} {k$^w$\textschwa d} {f\textturna\v{z}\textschwa nd?} {arv-\textschwa} {c'\textturna\textinvscr(-\textturna j)} {\textturna vi} {\v{s}\textschwa\textdyoghlig\textschwa t-\textturna j} {rajg$^w$\textschwa rd} \\ 
        human how appeared sky-{\Obl} blue-{\Abl} {\Or}.{\Q} clay-{\Abl} was.born \\
        \glt `How did the humans appear? Were they born from the sky blue or from clay?'\\*
        \hfill Adapted from \citet{erschler2018suspended}
    \end{xlist}
\end{exe}

I mirror the examples in (\ref{turkishalternative}) for Turkish in two ways. First, the exclusive alternative question is formed by two question clitics \textit{=mI}. Second is a disjunctive yes/no question which is formed with \textit{or} `veya'. The exclusive alternative question does not allow SA, but the disjunctive yes/no question does.

\begin{exe}
    \ex \label{turkishalternative}
    \begin{xlist}
    \ex \gll {Ali*(-yi)=mi} {Mehmet-i=mi} {ara-dı-n?} \\ 
    A-{\Acc}={\Q} M-{\Acc}={\Q} call-{\Pst}-{\Ssg} \\
    \glt `Did you call Ali, or did you call Mehmet?'
    
    \ex \gll {Ali} {veya} {Mehmet-i=mi} {ara-dı-n?} \\
    A {\Or} M-{\Acc}={\Q} call-{\Pst}-{\Ssg} \\
    \glt `Did you call Ali or Mehmet?'
    \end{xlist}
\end{exe}

Turkish exclusive alternative questions do not allow for SA unlike Ossetic. One important point needs to be made here. The question clitic \textit{=mI} in Turkish is a focusing element which draws focus to the preceding argument it is attached to. In exclusive alternative questions, the question clitic \textit{=mI} focuses the target word for SA.

Erschler moves on to pinpointing where the deletion process takes place after claiming that SA is an ellipsis process. He uses the DM framework, and argues that SA takes place after vocabulary insertion but before morpheme specific readjustments. The support for SA taking place after vocabulary insertion comes from the example in (\ref{osseticVI}) since the fragment after SA is the base for {\Sup} and not the base for {\Nom}. The support for SA taking place before morpheme specific phonological adjustments comes from the example in (\ref{osseticPA}) since the phonological assimilations of [g]\textgreater[\textdyoghlig] and [k]\textgreater[\textteshlig] dont take place in the first conjuncts under SA of {\Obl}.

\begin{exe}
    \ex \begin{xlist}
    \ex \label{osseticVI}
    \gll {d\textturna w(-b\textturna l)/*du} {\textturna ma} {m\textturna din\textturna-b\textturna l} {is\textturna mbaltt\textturna n}. \\ 
    {\Ssg}.{\Obl}-({\Sup})/{\Ssg}.{\Nom} {\And} M-{\Sup} {\Fsg}.met \\
    \glt `I met you and Madina.'
    
    \ex \label{osseticPA}
    \begin{xlisti}
    \ex \gll {park} {\textturna m\textturna} {w\textschwa n\textdyoghlig-\textschwa}. \\ 
    park {\And} street-{\Obl} \\
    \glt `in/of the street and the park.'
    
    \ex \gll {w\textschwa ng} {\textturna m\textturna} {par\textteshlig-\textschwa}. \\ 
    street {\And} park-{\Obl} \\
    \glt `in/of the park and the street.'\\*
    \hfill Adapted from \citet{erschler2018suspended}
    \end{xlisti}
    \end{xlist}
\end{exe}

Erschler argues that SA is a backward ellipsis process under identity where not all conjuncts should bear \textsc{[+{\Emp}]} feature. He cites \citet{herbeck2016controlling} in defense of positing information structure features in the lexicon for lexical items where Herbeck argues that Spanish overt pronouns have feature \textsc{[+{\Foc}]}. Overt pronouns need to be discourse configured hence the feature \textsc{[+{\Emp}]} because Ossetic is a pro-drop language like Turkish (cf. \citet{ozturk2002turkish} overt Turkish pronouns). 
