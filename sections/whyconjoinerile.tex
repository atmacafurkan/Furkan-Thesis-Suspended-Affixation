\section{Why \textit{ile/=lA} is a conjoiner}

While \textit{ile} allows for insertion of the {\Pl} and the {\Poss} suffixes before itself and allows for SA of them, it does not allow the insertion of {\Acc}. However it still distributes the case over to the first conjunct. By this observation we need to add another type of configuration for SA where we have a selective process that allows only for the insertion of some elided parts. An argument that could be made against such an approach would be to posit that \textit{ile} itself is a case marker and that's why insertion of {\Case} is not allowed before \textit{ile}. However this line of approach would have problems outside of SA to begin with. For example, a verb that assigns lexical case, \textit{kork} `to fear', and \textit{başla} can have arguments that are conjoined with \textit{ile} (\ref{ilelex}).

\begin{exe}
    \ex \label{ilelex}
    \begin{xlist}
    \ex 
    \gll 
    \textit{Kedi=yle} \textit{köpek-ten} \textit{kork-uyor-um} \\ cat={\And} dog-{\Abl} fear-{\Prog}-{\First}{\Sg} \\
    \glt `I am scared of the cat and the dog.'
    
    \ex 
    \gll 
    \textit{Sigara=yla} \textit{alkol-e} \textit{başla-dı-m.} \\ cigarette={\And} alcohol-{\Dat} start-{\Pst}-{\First}{\Sg} \\
    \glt `I started to use cigarette and alcohol.'
    \end{xlist}
\end{exe}

If we were to assume that \textit{ile} itself is a case marker, than the lexical case assigning \textit{kork} `to fear', and \textit{başla} `to start' would select an argument that are not case marked to their selection of {\Abl} and {\Dat}. A further claim could be to assume that \textit{ile} itself is not a conjoiner and is just a case marker that is assigned by a zero conjoiner as in Figure \ref{fig:zeroconjoiner}.

\begin{figure}[hbt!]
    \centering
    \begin{forest}
    for tree={inner sep=0}
    [DP 
        [BP 
            [DP 
                [NP]
                [D, name=D]]
            [B, name=conj]]
        [DP 
            [NP]
            [D]]]
    \draw[semithick, dashed, ->] (conj) to[out=south, in=east] node[right, fill=white]{\textit{ile}, $u$Case}(D);
    \end{forest}
    \caption{Considering \textit{ile} as a case assigned by a zero conjoiner head}
    \label{fig:zeroconjoiner}
\end{figure}

This way the only case that is assigned by the lexical case assigning verbs fall to the second conjunct. However contemplating that \textit{ile} is a case suffix that is assigned by a zero conjoiner results in problems for SA (\ref{ileSA}).
\begin{exe}
    \ex \label{ileSA}
    \begin{xlist}
        \ex 
        \gll 
        \textit{Kalem=le} \textit{kitab-ım-ı} \textit{bul-du-m.} \\ pencil={\And} book-{\First}{\Sg}.{\Poss}-{\Acc} find-{\Pst}-{\First}{\Sg} \\
        \glt `I found my pencil and my book.'
    
        \ex
        \gll 
        \textit{Kalem=le} \textit{kitap-lar-ım-ı} \textit{bul-du-m.} \\ pencil={\And} book-{\Pl}-{\First}{\Sg}.{\Poss} find-{\Pst}-{\First}{\Sg} \\
        \glt `I found my pencils and my books'
    \end{xlist}
\end{exe}

The examples in (\ref{ileSA}) would violate the rightward bound nature of SA, since {\Poss} and {\Pl} suffixes before \textit{ile} would be subject to suspension but not \textit{ile} itself. 

% Such configurations would have the templatic views in (\ref{ileSAtemplatic}, morphemes inside [] are suspended)

% \begin{exe}
%     \ex \label{ileSAtemplatic} 
%     \begin{xlist}
%         \ex {\Np}-[{\Poss}]-{\Case} conjoiner {\Np}-{\Poss}-{\Case}
%         \ex {\Np}-[{\Pl}-{\Poss}]-{\Case} conjoiner {\Np}-{\Pl}-{\Poss}-{\Case}
%     \end{xlist}
% \end{exe}

Stemming from the observations we made in (\ref{ilelex}) and (\ref{ileSA}), \textit{ile} itself is a conjoiner head and not a case suffix. Additionally, the phrase that \textit{ile} conjoins is not case markable, yet it can be marked for number and possession. The final representation for \textit{ile} as a conjoiner, following \cite{ozturk2016possessive} in positing a small "n" head responsible for number and agreement marking, is given in Figure \ref{fig:ile}.

\begin{figure}[hbt!]
    \centering
    \begin{forest}
    for tree={inner sep=0}
    [nP 
        [BP 
            [nP 
                [NP]
                [n]]
            [B\\\textit{ile/=lA}]]
        [nP 
            [NP]
            [n]]]
    \end{forest}
    \caption{Representation of \textit{ile} as a conjoiner}
    \label{fig:ile}
\end{figure}
