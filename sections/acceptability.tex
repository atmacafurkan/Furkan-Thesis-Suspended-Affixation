\section{Experiment 1; Acceptability}

In the literature of SA in Turkish it is claimed that SA is only operational for inflectional suffixes \citep{orgun1995flat,kornfilt1996some,broadwell2008turkish, kornfilt2012revisiting} with the only few exceptions in \cite{kabak2007turkish} and \cite{akkucs2016suspended} arguing for it to be possible in some derivational suffixes with some constraints, namely the suffixes \textit{-CI} and \textit{-lIK}. To be able to argue for or against any claim I have set out to design a simple acceptability study. I have selected only the derivational suffixes that take nominal bases and produce nominals, from a list provided in \cite{goksel2004turkish}. I take the morphological word observation of \cite{kabak2007turkish} and do not investigate SA of derivational suffixes that do not take nominals as a base or produce verbals. In choosing which derivational suffixes to test for SA, the important condition is to look for ambiguity. If the sentence is still grammatical without a reconstruction of the suffix grammaticality judgements of yes and no is not going to be informative. Taking this constraint into account, I give a list of possible SA configurations with different conjoiners in Table \ref{tab:derivationalSA}.

\begin{exe}
    \ex \label{derivationalsuffixes}
    \begin{xlist}
    \begin{multicols}{2}
       
        \ex 
        \gll 
        düş-er-cesine \\ fall-{\Aor}-AS.IF \\
        \glt `as if falling'
        
        \ex 
        \gll 
        yalan-cı \\ lie-AN \\ 
        \glt `liar'
        
        \ex 
        \gll 
        kahve-msi renk \\ coffee-LIKE colour \\
        \glt `like coffee colour'

        \ex 
        \gll 
        üç-üncü \\ three-ORD \\
        \glt `third'
\columnbreak
        \ex 
        \gll 
        sorun-lu adam \\ problem-INC man \\
        \glt `troubled man'
        
        \ex 
        \gll 
        düşman-lık \\ enemy-TH \\
        \glt `enmity'
        
        \ex 
        \gll 
        sınır-sız internet \\ limit-EXC internet \\
        \glt `limitless internet'
        
        \ex 
        \gll 
        iki-şer \\ two-COU \\
        \glt `two by two'
        
    \end{multicols}
    \end{xlist}
\end{exe}


\begin{table}[hbt!]
    \caption{List of Suspendable Derivational Suffixes with Different Conjoiners}
    \centering
    \begin{tabular}{|ll|}
    \hline
    \multicolumn{2}{|c|}{with conjoiners \textit{ve} and \textit{veya}}         \\
    \hline 
        Suffix                  & SA                                            \\
    \hline 
       \textit{-CAsInA}         & \textit{düşer ve(ya) takılırcasına}           \\
       \textit{-CI}             & \textit{boya ve(ya) badanacı}                 \\
       \textit{-(I)msI}         & \textit{meyve ve(ya) sebzemsi}                \\
       \textit{-(I)ncI}         & \textit{üç ve(ya) dördüncü kemikler}          \\
       \textit{-lI}             & \textit{çikolata ve(ya) çilekli dondurma}     \\
       \textit{-lIK}            & \textit{dost ve(ya) düşmanlık}                \\
       \textit{-sIz}            & \textit{çikolata ve(ya) çileksiz dondurma}    \\
       \textit{-(ş)Ar}          & \textit{üç ve(ya) altışarlı sıra}             \\
       \hline 
    \end{tabular}
    \label{tab:derivationalSA}
\end{table}

\subsection{Method}
\subsubsection{Participants}

The participants were NUMBER students at Boğaziçi University who were native speakers of Turkish. In exchange for their participation they have received 1 point to their overall course score on a chosen linguistics course.

\subsubsection{Materials}

The experiment is comprised of 9 different suffixes (8 derivational 1 inflectional ACC) by 2 conjoiners per suffix. For each suffix 3 distinct items are produced. A total of 54 experimental items, and additional 27+27 un/grammatical fillers make up the experiment. A latin square design by conjoiner type is applied. Resulting in each participant seeing only 27 experimental items and 54 fillers.  A templatic view of how the experimental items look is given in (\ref{accepttemplate}).

\begin{exe}
    \ex \label{accepttemplate}
    PreSentence N1-[Suffix] CONJOINER N2-Suffix PostSentence
\end{exe}

pre and post sentence fields are not kept constant since each suffix required a different type and place of modification or use, resulting in string-wise uneven number of words. Since the only measurement of the acceptability study is grammaticality judgements with no emphasis on reading or response times, uneven number of words across experimental items is considered not to be an issue. The experiment is formed using http://spellout.net/ibexfarm/, and carried out online. 

\subsubsection{Procedure}
Participants are provided a link to the experiment prompting them to a consent page. Upon giving consent participants go through 5 practice items and then they are prompted again to the beginning of the experiment. Each item is presented as a full sentence and participants decide on whether or not the sentence they encounter is a natural/ok sentence in Turkish or not. They profess their decision by pushing "Q" key for "yes" and "P" key for "no". Only the participant choice regarding their grammaticality judgement is recorded for analysis.



