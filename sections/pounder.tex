\subsection{German}

In \cite{pounder2006broken}, Pounder presents some example configurations in German for ellipsis like morphological phenomena. These phenomena, called morphological brachylogy in the paper, include SA, conjunction reduction, and shared bases in German. The paper puts a higher emphasis on a diachronic difference in SA of suffixes. Pounder claims that these ellipsis like processes can be employed in many levels of grammar, the inflectional paradigm, word-formations, and compounding to name a few. While the paper itself provides and lays out a nice presentation of data, this summary revolves around brachylogy of affixes that I refer to as SA for consistency.

Before moving on with examples of SA in German, I reiterate one of Pounder's examples where a conjunction that has two prefixed verb conjuncts undergoes an ellipsis like process. In this example the two conjuncts are prefixed verbs, both of which share the same base. See (\ref{pounderex1}) for an instance of shared base for prefixes in a conjunction. 

\begin{exe}
    \ex \label{pounderex1}
    \begin{xlist}
        \ex \label{pounderex1a} 
        \gll 
        \textit{werde...} \textit{nicht} \textit{re-,} \textit{sondern} \textit{ent-sozialisier-t} \\ be... {\Neg} {\Pref}- but\_rather {\Pref}-socialize-{\Part} \\
        \glt `be... not socialized but rather desocialized'
    
        \ex \label{pounderex1b}
        \gll 
        \textit{nicht} \textit{re-sozialisier-t} \textit{sondern} \textit{ent-sozialisier-t} \\ {\Neg} {\Pref}-socialize-{\Part} but {\Pref}-socialize-{\Part} \\
        \glt `not resocialised but rather desocialized'
    \end{xlist}
\hfill Adapted from \cite{pounder2006broken}
\end{exe}

In the paper Pounder dubs what is left after the elision of the morphological part as `fragment' whereas what is elided or reconstructed is called `recuperand', and the form that the language user infers the recuperand from is called `target'. For example in (\ref{pounderex1a}) the fragment is the prefix \textit{re-}, the recuperand is \textit{sozialisiert}, and the target is \textit{sozialisiert}. In this instance recuperand and target might be the same but in some constructions the phonological forms might differ. The important thing is from the target you decide on the morphological element that you will reconstruct as the recuperand with the fragment. The observation that we can draw from the paper about this construction is that the fragment needs to be a phonological word, and Pounder cites \cite{smith2000word},  for this constraint. Although (\ref{pounderex1}) is not SA, it is important to point out that the fragment or the remnant after the elision like process goes by the phonological word instead of morphological. 

Now to come to the examples of SA in German, I reiterate yet another example from Pounder in (\ref{pounderSA}). I also provide a sort of mirroring example from Turkish.

\begin{exe}
    \ex \label{pounderSA}
    \begin{xlist}
        \ex \label{germanderSA}
        \gll 
        \textit{freund-} \textit{oder} \textit{feind-schaft-lich-e} \textit{Beziehungen} \\ friend- {\Or} enemy-{\Der}-{\Der}-{\Pl} relations \\
        \glt `with relations of friendship or enmity'
        
        \hfill Adapted from \cite{pounder2006broken}
        \ex \label{turkishderSA} 
        \gll 
        \textit{dost} \textit{veya} \textit{düşman-lığ-ı} \textit{bitir-en} \textit{ilişki-ler} \\ friend {\Or} enemy-{\Der} end-{\Fp} relation-{\Pl} \\
        \glt `the relations that end friendship or enmity'
    \end{xlist}
\end{exe}

The expression in (\ref{germanderSA}) shows an instance of SA for the suffixes \textit{-schaft} and \textit{-lich}. The first one is a derivational suffix and the second one is inflectional. I tried to mirror a similar configuration in (\ref{turkishderSA}) where there is SA of a derivational suffix \textit{-lIK} and an inflectional accusative \textit{-(y)I}. However Pounder reports that this process in German has a phonological constraint, namely a suffix beginning with a vowel can not take part in SA (\ref{germannoSA}).

\begin{exe}
    \ex \label{germannoSA}
    \gll 
    \textit{*die} \textit{Provenz-al-} \textit{und} \textit{Roman-isch-en} \textit{Dichter} \\ the.{\Pl} Provence-{\Der} {\And} romance-{\Der}-{\Pl} poets \\
    \glt Intended `the Provençal and Romantic poets'
    
    \hfill Adapted from \cite{pounder2006broken}
\end{exe}

In (\ref{germannoSA}) the suffix \textit{isch} begins with a vowel. Pounder cites \cite{booij1985coordination} in reporting that the vowel initial suffix leads to a mismatch between the phonological and morphological word. The paper, however, shows a historical contrast in the contemporal ungrammaticality of (\ref{germannoSA}) where SA exists in written form. Pounder claims that German standardization is behind the ungrammaticality of (\ref{germannoSA}) and provides some examples from 17$^{th}$ and 18$^{th}$ century German (\ref{germansimilarSA}).

\begin{exe}
    \ex \label{germansimilarSA}
    \begin{xlist}
        \ex 
        \gll
        \textit{Absicht-} \textit{und} \textit{Regl-en} \\ intention- {\And} rule-{\Pl} \\
        \glt `Intensions and rules'
        
        \ex \label{germanbaseSA}
        \gll 
        \textit{Geberd-} \textit{und} \textit{Bewegung-en} \\ gesture- {\And} movement-{\Pl} \\
        \glt `Gestures and movements'
        
        \ex 
        \gll
        \textit{bey} \textit{dorf-} \textit{und} \textit{stet-en} \\ by village- {\And} town-{\Pl}.{\Dat} \\
        \glt `In villages and towns'
    \end{xlist}
    \hfill Adapted from \cite{pounder2006broken}
\end{exe}

There is a very important point to make in these examples, particularly in (\ref{germanbaseSA}). Pounder notes, normally the singular \textit{Geberd-}undergoes umlaut and becomes \textit{Geb\"{a}rde} in pluralization, but in the suspended version no umlaut does not take place. (\ref{germansimilarSA}) shows that SA takes place before such phonological operations like umlaut.

Now consider the example in (\ref{turkishbaseSA}) where the first person singular pronoun undergoes base modification when used with Dative.

\begin{exe}
    \ex \label{turkishbaseSA}
    \begin{xlist}
        \ex
        \gll
        \textit{*Ban} \textit{ve} \textit{Ahmet-e} \textit{bak-tı.} \\ {\First}{\Sg} {\And} Ahmet-{\Dat} look-{\Pst}[{\Third}{\Sg}] \\
    
        \ex 
        \gll
        \textit{*Ben} \textit{ve} \textit{Ahmet-e} \textit{bak-tı.} \\ {\First}{\Sg} {\And} Ahmet-{\Dat} look-{\Pst}[{\Third}{\Sg}] \\
        
        \ex \gll 
        \textit{Ban-a} \textit{ve} \textit{Ahmet-e} \textit{bak-tı.} \\ {\First}{\Sg}-{\Dat} {\And} Ahmet-{\Dat} look-{\Pst}\\
        \glt `(S/he) looked at me and Ahmet'
    \end{xlist}
\end{exe}

In the German example (\ref{germanbaseSA}) the reconstruction of the fragment and the recuperand seems to be employed at a more abstract level than phonology since the umlaut, a base modification, in the first conjunct need not be carried out. In the Turkish example (\ref{turkishbaseSA}), we see that the reconstruction of the fragment and the recuperand can not override an expected base modification in the fragment, or even further we see that SA can not be carried out at all with base modified fragments.

The importance of \cite{pounder2006broken} is, unlike the literature in Turkish, Pounder goes on to interrogate the formulation of conjunction where SA takes place. The formulation of conjunction might define the nature of SA. If one takes the position and says conjunction is performed only at a clausal level, than SA is an ellipsis process. On the other hand, if one takes the position that conjunction can be performed at any phrase level XP then SA can be represented as a structural sharing of the suspended affixes by the conjuncts.
































