% glossings to be used in the paper
\makeglossaries

\newleipzig{aor}{aor}{aorist}
\newleipzig{and}{and}{and}
\newleipzig{case}{case}{case}
\newleipzig{evi}{ev}{evidential}
\newleipzig{prob}{prob}{probability}
\newleipzig{nec}{nec}{necessitive}
\newleipzig{pref}{pref}{prefix}
\newleipzig{fp}{fp}{free participle}
\newleipzig{iness}{iness}{inessive}
\newleipzig{ill}{ill}{illative}
\newleipzig{pron}{pron}{pronoun}
\newleipzig{dim}{dim}{diminutive}
\newleipzig{conj}{conj}{conjoiner}
\newleipzig{sup}{sup}{suppletive}
\newleipzig{prv}{prv}{preverb}
\newleipzig{emp}{emp}{emphasis}
\newleipzig{or}{or}{or}
\newleipzig{abil}{abil}{ability}
\newleipzig{con}{con}{conative}
\newleipzig{adp}{adp}{adposition}
\newleipzig{lcase}{lcase}{local case}
\newleipzig{scase}{scase}{structural case}
\newleipzig{pc}{pc}{predicate concatenator}
\newleipzig{np}{np}{noun phrase}
\newleipzig{when}{when}{derivational \textit{-(y)IncA}}
\newleipzig{wo}{wo}{derivational \textit{-mAdAn}}
\newleipzig{by}{by}{derivational \textit{-(y)ArAK}}
\newleipzig{part}{part}{partitive}
\newleipzig{der}{der}{derivational suffix}
\newleipzig{infl}{infl}{inflectional suffix}
\newleipzig{cat}{cat}{lexical category}
\newleipzig{noun}{noun}{noun}
\newleipzig{num}{num}{number}
\newleipzig{lex}{lex}{lexical item}
\newleipzig{bp}{bp}{boolean phrase}
\newleipzig{lf}{lf}{logical form}
\newleipzig{pp}{pp}{past participle}