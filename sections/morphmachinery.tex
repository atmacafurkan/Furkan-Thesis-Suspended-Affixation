\section{Morphological machinery}

The dictionary description for `morphology' as it is used in other fields refers to the shape or form of an object. In the case of language, this usually boils down to the words and their identifiable lexical and functional parts. In this study, I regard a functional head as a morpheme and not the identifiable or concatenative forms. This means that an expression like \textit{fell} consists of two morphemes: one lexical \textit{fall} and one functional {\Pst}. In this study, some tenets of Distributed Morphology (DM) \citep{halle1993distributed, halle1994some} are used. Namely the notions of abstract morphemes, vocabulary items, late insertion and readjustment rules as adapted from \citet{embick2005status}. I give some explanations and examples in the following subsections.


\subsection{Abstract morpheme}

Abstract morphemes are composed exclusively of non-phonetic features, such as {\Pst} or {\Pl}. These morphemes occupy the morphological structure together with lexical morphemes. Further processes of vocabulary insertion and readjustment rules form the phonological exponents. This means that an expression like \textit{kitap-lar-ım} `my books' has the morphemes of `book-{\Pl}-{\Poss}' as is its glossing. 

\subsection{Vocabulary item}

Vocabulary items pair a morphosyntactic context with a phonological exponent where the exponent is a sequence of phonetic feature complexes that can be addressed for phonological changes. A morpheme can have different vocabulary items. A straightforward example is the vocabulary items for {\Pl} in English. The general vocabulary item for {\Pl} is `-s', but  different vocabulary items like `-ren, -en, $\emptyset$, \ldots' can be inserted depending on the lexical morpheme that {\Pl} is attached to.

\subsection{Late insertion}

Late insertion, or vocabulary insertion is the introduction of phonological exponents to the morphological or syntactic structure after they are formed. For the purposes of this study, it means that phonological representations follow after all the structural representations are at place instead of working in tandem. There is a Subset Principle \citep{halle2000distributed} that regulates late insertion for pairing a collection of features to a vocabulary item. Subset principle states that the smallest vocabulary item that has most of the features as its subset is inserted as a phonological exponent. 

\subsection{Readjustment rules}

Readjustment rules are phonological rules which affect changes in each morphosyntactic context and typically include lists of lexical and abstract morphemes that undergo or trigger these changes. These rules represent the phonological processes such as assimilation and vowel harmony.


\subsection{Example}

In an expression like \textit{çök-tü-m.} (sit.down-{\Pst}-{\Fsg}) `I sat down' in Turkish, there are 1 lexical and 3 abstract morphemes. The vocabulary items for the lexical morpheme and the abstract morphemes are: \textit{çök} as lexical morpheme, \textit{-di} as {\Pst}, and \textit{-m} as {\Fsg}. The items are then inserted and form \textit{çök-di-m}, phonological readjustments begin after this point. The assimilation and vowel harmony takes place, then the expression becomes \textit{çök-tü-m.} `I sat down'. 

DM hosts many other notions and operations like Root, Fusion, Fission, and Impoverishment \citep{halle2000distributed,bonet1991morphology, embick2015morpheme} to explain and capture several morphological phenomena. I do not particularly use all tenets of DM, which regards morphology as either a part of syntax or a continuum of it. Some aspects of DM, and the theory in general, are recently criticized by \citet{spencer2019manufacturing}. My focus is mainly on DM being a realizational approach to morphology. Meaning that phonological exponents and morphological structure do not follow a one to one match and morphology comes before phonological exponents. I adopt some tenets of DM for my analyses and arguments. I do not use the compositional properties that DM assigns to morphology as an integrated or a continuous module to syntax. \citet{ackema2007morphology} best exemplifies my application of late insertion and vocabulary items while still holding morphology as a separate module in language.