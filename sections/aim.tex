\section{Aim of the thesis}

The aim of this thesis is to explore the constraints affecting Suspended Affixation (henceforth SA) and how SA can relate to sentence processing. SA is the elision of affixes from conjuncts in a conjunction environment. In (\ref{convention}) the suffix {\Acc} is only overt on the second conjunct but it is interpreted for the first conjunct too.

\begin{exe}
\ex \label{convention} SA of {\Acc}\\*
\gll Kitap ve kalem-i al-dı. \\ book {\And} pencil-{\Acc} take-{\Pst}[{\Tsg}] \\
\glt `S/he took the book and the pencil.'
\end{exe}

In discussing SA and its constraints, I provide analyses by drawing inferences from both empirical and theoretical devices. These analyses include empirical investigations into the division of literature in treating derivational suffixes, the processing cost of suspension, and observations that contradict some theoretical constraints proposed for SA. I also bring the effect of the environment, conjunctions, into the discussion of Turkish SA.

There can be many points and facets in the discussion of SA. I cover and provide arguments for the points in (\ref{questionsintro}). These are about what the literature disagrees on, misses to address, or proposes.

\begin{exe}
\ex \label{questionsintro}
\begin{xlisti}
    \ex What are the analyses for SA in Turkish and in other languages?
    \ex What kind of morphemes can be targeted by SA?
    \ex Does the conjoiner have an effect in performing SA?
    \ex What is the processing cost of SA?
    \ex How can SA be used for sentence processing?
    \ex Is SA beyond a morphological word possible?  
\end{xlisti}
\end{exe}