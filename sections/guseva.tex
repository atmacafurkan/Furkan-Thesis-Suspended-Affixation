\subsection{Mari}

Mari is an Eastern Uralic language that has a rather interesting set of data when it comes to SA. \cite{guseva2017postsyntactic} (GW henceforth) provides some examples and analysis for SA in Mari. In Mari the order of the morphemes in the nominal domain show a relatively free order. The morphemes in question are {\Pl}, {\Poss}, Structural and Local cases. Some possible orders of these morphemes are given in Table \ref{tab:mariorder}.

\begin{table}[hbt!]
    \caption{Mari Nominal Domain Morpheme Order}
    \centering
    \begin{tabular}{|ll|}
    \hline 
        {\Pl} \textgreater {\Poss} & \textit{pasu-vlak-na}  \\
        {\Poss} \textgreater {\Pl} & \textit{pasu-na-vlak} \\ \hline
        {\Pl} \textgreater {\Lcase} & \textit{pasu-vlak-e\u{s}te} \\
        {\Pl} \textgreater {\Scase} & \textit{pasu-vlak-em} \\ \hline
        {\Lcase} \textgreater {\Poss} & \textit{pasu-\u{s}te-na} \\
        {\Poss} \textgreater {\Scase} & \textit{pasu-na-m} \\ \hline
        {\Pl} \textgreater {\Lcase} \textgreater {\Poss} & \textit{pasu-vlak-e\u{s}te-na} \\
        {\Poss} \textgreater {\Pl} \textgreater {\Lcase} & \textit{?pasu-na-vlak-e\u{s}te} \\ \hline
        {\Pl} \textgreater {\Poss} \textgreater {\Scase} & \textit{pasu-vlak-na-m} \\
        {\Poss} \textgreater {\Pl} \textgreater {\Scase} & \textit{pasu-na-vlak-em} \\ \hline 
        \multicolumn{2}{|l|}{\textit{pasu} `garden', \textit{-vlak} {\Pl}, \textit{-na} {\Poss}.{\First}.{\Pl}, \textit{-(e)\u{s}te} {\Iness}, \textit{-(e)m} {\Acc}} \\
        \hline 
    \end{tabular}
    \label{tab:mariorder} \\
    ${}$ \hfill Adapted from \cite{guseva2017postsyntactic}
\end{table}

The examples in Table \ref{tab:mariorder} show an optional positioning for the {\Poss} marker. The {\Poss} either occupies the left or the right edge of the morphemes, where the right edge can only build up to the {\Scase}. In a sense {\Scase} is a barrier that {\Poss} can not alternate to the right of. The important examples come into play in SA constructions. See (\ref{mariSA1}) for two of them.

\begin{exe}
    \ex \label{mariSA1}
    \begin{xlist}
        \ex
        \gll
        \textit{Üder} \textit{mej-en} \textit{u\u{s}e-m} \textit{den} \textit{tej-en} \textit{süm-e\u{s}te-t.} \\ girl {\First}.{\Sg}-{\Gen} mind-{\Poss}.{\First}.{\Sg} {\And} {\Second}.{\Sg}-{\Gen} heart-{\Iness}-{\Poss}.{\Second}.{\Sg} \\
        \glt `The girl is in my mind and in your heart'
        
        \ex 
        \gll 
        \textit{Pjötr} \textit{kart-em} \textit{mej-en} \textit{perde\u{z}-em} \textit{den} \textit{omsa-\u{s}ke-\u{z}e} \textit{pi\u{z}ekta} \\ Peter map-{\Acc} {\First}.{\Sg}-{\Pron}-{\Gen} door-{\Poss}.{\First}.{\Sg} {\And} wall-{\Ill}-{\Poss}.{\Third}.{\Sg} pin.{\Third}.{\Sg}.{\Prs} \\
        \glt `Peter pins maps to my door and his wall'
    \end{xlist}
    ${}$ \hfill Adapted from \cite{guseva2017postsyntactic}
\end{exe}

In (\ref{mariSA1}) the templatic view of SA looks like the following:
\begin{itemize}
    \item N1-[{\Lcase}]-{\Poss} conjoiner N2-{\Lcase}-{\Poss}
\end{itemize}

The observations up to (\ref{mariSA1}) were that SA is a rightward bound process. However (\ref{mariSA1}) contradicts this observation. This non-rightward SA is not limited to {\Poss} and {\Lcase}, another example is given in (\ref{mariSA2}).
\begin{exe}
    \ex \label{mariSA2}
    \gll 
    \textit{A-vlak} \textit{tud-en} \textit{sad-\u{s}e} \textit{den} \textit{memn-an} \textit{pasu-vlak-e\u{s}te-na} \textit{mod-et} \\ child-{\Pl} {\Third}.{\Sg}-{\Gen} garden-{\Poss}.{\Third}.{\Sg} {\And} {\First}.{\Pl}-{\Gen} field-{\Pl}-{\Iness}-{\Poss}.{\First}.{\Pl} play-{\Third}.{\Pl}.{\Prs} \\
    \glt `The children are playing in his garden and in our fields'  \\
    ${}$ \hfill Adapted from \cite{guseva2017postsyntactic}
\end{exe}

Again the templatic view of SA in (\ref{mariSA2}) looks like:
\begin{itemize}
    \item N1-[{\Pl}-{\Iness}]-{\Poss} conjoiner N2-{\Pl}-{\Iness}-{\Poss}
\end{itemize}

The two examples of (\ref{mariSA1}), and (\ref{mariSA2}) might point us to drop the observation that SA is a rightward bound process. Before changing the observation, however, it is important to see that in Mari, there are two linearizations of {\Poss}, {\Pl} and {\Lcase}. If we were to take the surface form and say {\Lcase}-{\Poss}, and {\Pl}-{\Lcase}-{\Poss} orderings allow a non-rightward bound SA, we would be overlooking the other possible forms in the language which are {\Poss}-{\Lcase}, and {\Poss}-{\Pl}-{\Lcase}. In the same paper there is a rather convincing observation that shows the ellipsis like nature of SA in Mari, adapted here as (\ref{mariSA3}).

\begin{exe}
 \ex \label{mariSA3}
    \begin{xlist}
        \ex \label{mariSA3a}
        \gll 
        \textit{Pörjeng} \textit{memnam} \textit{da} \textit{nunem} \textit{u\u{z}-e\u{s}} \\ man.{\Nom} us.{\Acc} {\And} them.{\Acc} see-{\Third}.{\Sg}.{\Prs} \\
        
        \ex \label{mariSA3c}
        \gll 
        \textit{*Pörjeng} \textit{me} \textit{da} \textit{nunem} \textit{u\u{z}-e\u{s}} \\ man.{\Nom} us.{\Acc} {\And} them.{\Acc} see-{\Third}.{\Sg}.{\Prs} \\
        
        \ex \label{mariSA3b} 
        \gll
        \textit{Pörjeng} \textit{memna} \textit{da} \textit{nunem} \textit{u\u{z}-e\u{s}} \\ man.{\Nom} us {\And} them.{\Acc} see-{\Third}.{\Sg}.{\Prs} \\
        \glt `The man sees us and them'
    \end{xlist}
    ${}$ \hfill Adapted from \cite{guseva2017postsyntactic}
\end{exe}

The first person plural pronoun is \textit{me} in Mari, and in the accusative form the stem for the accusative case changes from \textit{me} to \textit{memna}. While SA is possible with the plural stem \textit{memna} it is not possible with \textit{me}. A similar base or stem change in Turkish also happens when first and second person singular pronouns (\textit{ben, sen}) are used with the dative case \textit{-(y)A} (\textit{bana, sana}). Turkish does not seem to be allowing SA in such instances (\ref{mariturkish}), with or without base or stem change.

\begin{exe}
    \ex \label{mariturkish}
    \begin{xlist}
        \ex 
        \gll 
        \textit{Ban-a} \textit{ve} \textit{san-a} \textit{kitab-ı} \textit{bul-du.} \\ {\First}.{\Sg}-{\Dat} {\And} {\Second}.{\Sg}-{\Dat} book-{\Acc} buy-{\Pst}[{\Third}.{\Sg}] \\
        \glt `S/he bought the book for me and you'
        
        \ex 
        \gll 
        \textit{*Ben-} \textit{ve} \textit{san-a} \textit{kitab-ı} \textit{bul-du.} \\ {\First}.{\Sg}-{\Dat} {\And} {\Second}.{\Sg}-{\Dat} book-{\Acc} buy-{\Pst}[{\Third}.{\Sg}] \\
        
        \ex 
        \gll 
        \textit{*Ban-} \textit{ve} \textit{san-a} \textit{kitab-ı} \textit{bul-du.} \\ {\First}.{\Sg}-{\Dat} {\And} {\Second}.{\Sg}-{\Dat} book-{\Acc} buy-{\Pst}[{\Third}.{\Sg}] \\
    \end{xlist}
\end{exe}

GW go on to analyze SA in Mari with proposed projections for {\Poss}, {\Pl}, and Case suffixes with NumP, DP, and KP (a similar proposal that we ended up refuting in Section \ref{kornfilt}). Following \cite{merchant2015ineffable} they propose an underlying order like (\ref{mariordering}). Onto this order a process of D-lowering takes place and the new ordering looks like (\ref{mariorderinglowered}). It is at the order of (\ref{mariorderinglowered}) that SA marks morphemes as a rightward process for zero exponance (shown with a subscript 0) as in (\ref{mariSAmark}). Later a D-metathesis is performed and the ordering for vocabulary insertion looks like (\ref{mariSAform}). This way we end up with the SA example of (\ref{mariSA2}), partly repeated here for the reader's convenience.
\begin{exe}
    \ex \begin{xlist}
    \ex \label{mariordering}
    [[[ NP ] Num ]$_{NumP}$ D ]$_{DP}$ K ]$_{KP}$ \hfill Underlying Order
    
     \ex \label{mariorderinglowered}
    [[[ NP ] D Num ]$_{NumP}$ t$_D$ ]$_{DP}$ K ]$_{KP}$ \hfill D-Lowering
    
    \ex \label{mariSAmark}
    [[[ NP ] D Num_0 ]$_{NumP}$ t$_D$ ]$_{DP}$ K_0 ]$_{KP}$ \hfill SA marking
    
    \ex \label{mariSAform}
    [[[ NP ] D K_0 Num_0 ]$_{NumP}$ t$_D$ ]$_{DP}$ t$_K$ ]$_{KP}$ \hfill D-Metathesis
    \end{xlist}
     ${}$ \hfill Adapted from \cite{guseva2017postsyntactic}
\end{exe}
\begin{exe}
\small 
    \exi{(\ref{mariSA2}')}
    \gll 
    \textit{tud-en} \textit{sad-\u{s}e} \textit{den} \textit{memn-an} \textit{pasu-vlak-e\u{s}te-na} \\ {\Third}.{\Sg}-{\Gen} garden-{\Poss}.{\Third}.{\Sg} {\And} {\First}.{\Pl}-{\Gen} field-{\Pl}-{\Iness}-{\Poss}.{\First}.{\Pl}\\ 
    \glt `\ldots in his garden and in our fields'
\end{exe}



The two important observations to be made from \cite{guseva2017postsyntactic} is how to point the place that SA takes place. First, the Mari examples  (\ref{mariSA1}), and (\ref{mariSA2}) clearly show that SA is not performed at the surface form. This observation is vital to distinguish SA from Backward Ellipsis in Turkish because Backward Ellipsis takes the surface form into account. Second, (\ref{mariSA3}) shows that SA does not operate morphemes on a derivational level before their phonological representations are in place, yet (\ref{mariturkish}) show that even taking those representations into account does not result in SA either. 