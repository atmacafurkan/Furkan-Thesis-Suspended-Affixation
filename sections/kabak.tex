\subsection{\cite{kabak2007turkish}} \label{kabaksummary}
Among the papers discussing SA in Turkish, Kabak's paper seems to be the most extensive in terms of providing how SA can take shape in both verbal and nominal domains. The paper provides some conditions for SA. The analysis of Kabak relies on the definition of a morphological word. He defines a morphological word if the final morpheme can terminate a word independently of agreement markers. He claims that any inflectional morpheme can be suspended as long as the remainder is a morphological word. Kabak proposes the following:
\begin{itemize}
    \item Terminal suffix: \textit{A suffix that is allowed to appear at the end of a word, where further affixation is not obligatory.}
\end{itemize}

He claims that only terminal suffixes can be suspended. He posits that bare verbs are not morphological words in Turkish. He provides Table \ref{tab:terminalmorphemes} for Verbal terminal suffixes. If any suspension attempt is made with these morphemes, it is only permitted under the condition that what is left is a morphological word. 
\begin{table}[hbt!]
\caption{Verbal Terminal Morphemes}
    \centering
    \begin{tabular}{|ll|}
    \hline 
                                {(i) Agreement markers} &  \\ \hline
    \multirow{5}{20em}{(ii) Aspect/ Modality markers}  & {\Aor} \textit{-(I)r/(A)r} \\ 
                                                        & {\Prog} \textit{-Iyor} \\
                                                        & {\Fut} \textit{-(y)AcAK} \\
                                                        & {\Evi} \textit{-mIş} \\
                                                        & {\Nec} \textit{-mAlI} \\ \hline
    \multirow{2}{20em}{(iii) Converb markers}           & \textit{-(y)IncA} \\
                                                        & \textit{-(y)Ip} \\
    \hline                                                         
    \end{tabular}
    \label{tab:terminalmorphemes}
    \begin{flushright}
    Adapted from \cite{kabak2007turkish}
    \end{flushright}
\end{table}

Kabak classifies clitics like \textit{=mI} `={\Q}', and \textit{=DA} `{\Foc}' as non-terminal morphemes but recognizes their ability to end an expression in Turkish (\ref{clitics}).

\begin{exe}
    \ex \label{clitics}

    \begin{xlist}
        \ex \gll koş-tu-n mu? \\ 
        run-{{\Pst}}-{\Second}{\Sg} ={\Q} \\
        \glt `Did you run?'
  
        \ex \gll ağla-mış-sın da \\ cry-{\Evi}-{\Second}{\Sg} ={\Foc} \\ 
        \glt `It looks like you have cried also.'\\*
        \hfill Adapted from \cite{kabak2007turkish}
    \end{xlist}
\end{exe}

Kabak argues against \cite{kornfilt1996some}'s formulation for SA (\ref{kornfiltsa}) with two points. According to \cite{kornfilt1996some}'s analysis only the copular forms and further inflectional morphemes can be suspended in the verbal domain.
\begin{exe}
    \ex \label{kornfiltsa}
    [V$_{Participle}$ conjunction V$_{Participle}$] + V$_{Copula}$ + Inflectional Morphemes
\end{exe}

First, some forms that can be the complements of the copula are not participles and do not always give way to grammatical instances of SA. Although \cite{kornfilt1996some} does not define \textit{-DI} as forming a participle, it is still able to be a complement to a copula \textit{i}. That is why SA in (\ref{kornfiltpart}) should be grammatical because what is left is a complement to a copula.

\begin{exe}
    \ex \label{kornfiltpart}
    \begin{xlist}
        \ex \gll *O yaz Finike-ye git-ti ve deniz-e gir-di-y-di-k. \\ 
        that summer Finike-{\Dat} go-{{\Pst}} {\And} sea-{\Dat} enter-{{\Pst}}-{\Cop}-{{\Pst}}-{\Fpl} \\
        \glt Intended `That summer (we) went to Finike and went swimming'
        
        \ex \gll *Ev-imiz-i sat-sa ve dükkan al-sa-y-dı-k. \\ 
        house-{\Poss}.{\Fpl}-{\Acc} sell-{\Cond} {\And} shop buy-{\Cond}-{\Cop}-{{\Pst}}-{\Fpl} \\
        \glt `(We) wish (we) have sold our house and bought a shop'\\*
        \hfill Adapted from \cite{kabak2007turkish}
    \end{xlist}
\end{exe}

Second, the participle that is formed by the suffix \textit{-mAlI} behaves different than other participles formed by \textit{-mIş} and \textit{-(y)AcAK}. Participles formed by \textit{-mAlI} can't modify nouns (\ref{kabakmali}). It should be noted that not all participle forms can modify nouns (e.g. formed with \textit{-Iyor}), and despite a lack of modifying capability, \textit{-mAlI} formed participle acts as predicted by \cite{kornfilt1996some} in an SA configuration (\ref{malikornfilt}).

\begin{exe}
\ex \begin{xlist}
    \ex \label{kabakmali}
    \gll *çalış-malı adam \\ work-{\Nec} man \\
    
    \ex \label{malikornfilt} 
    \gll ev-e git-meli ve uyu-malı-yım. \\ 
    home-{\Dat} go-{\Nec} {\And} sleep-{\Nec}-{\Fsg} \\
    \glt `I need to go home and sleep.'
\end{xlist}
\end{exe}

Another point Kabak provides with the example (\ref{plurposs}), is the suspension of {\Poss} when used together with {\Pl}. This contradicts the observations of \cite{orgun1995flat}.

\begin{exe}
    \ex \label{plurposs} SA of {\Poss} \\ 
    \gll Asker-ler ve komutan-lar-ımız. \\ 
    soldier-{\Pl} {\And} commander-{\Pl}-{\Fpl}.{\Poss} \\
    \glt `Our soldiers and commanders'\\*
    \hfill Adapted from \cite{kabak2007turkish}
\end{exe}

Kabak points out the effects of some phonological changes in what is left after SA, specifically {\Fsg} and {\Ssg} pronouns that go under base modification (\textit{ben} {\textgreater} \textit{ban}, \textit{sen} {\textgreater} \textit{san}) when affixed with {\Dat} as shown in (\ref{kabakphonology}).

\begin{exe}
\ex \label{kabakphonology}
\begin{xlist}
\ex \gll Kargo-lar Ahmet ve Mehmet-e gel-di. \\ 
shipment-{\Pl} A {\And} M-{\Dat} come-{\Pst}[{\Tsg}] \\
\glt `The shipments arrived for Ahmet and Mehmet'

\ex \gll *Kargo-lar ben ve san-a gel-di. \\ 
shipment-{\Pl} {\Fsg} {\And} {\Ssg}-{\Dat} come-{\Pst}[{\Tsg}] \\

\ex \gll *Kargo-lar ban ve san-a gel-di. \\ 
shipment-{\Pl} {\Fsg} {\And} {\Ssg}-{\Dat} come-{\Pst}[{\Tsg}] \\

\ex \gll 
Kargolar ban-a ve san-a gel-di. \\ 
shipment-{\Pl} {\Fsg}-{\Dat} {\And} {\Ssg}-{\Dat} come-{\Pst}[{\Tsg}] \\
\glt `The shipments arrived for me and you'
\end{xlist}
\end{exe}


For the SA in the verbal domain, Kabak provides an approach that is rather interesting. He makes an observation from \cite{good2005morphosyntax}, in the spirit of \cite{erdal2000clitics}, about the agreement paradigms in Turkish. In his citing, Kabak says that the z-paradigm of agreement markers contains cliticized forms of words, and the k-paradigm of agreement markers has lexical suffixes. Kabak realizes the shortcomings of this approach and notes there are constructions in which the k-paradigm SA is applicable and other conditions where the z-paradigm SA is not applicable. The k-paradigm is not suspended on its own, but it is suspendable if the tense marker it is attached to is in a TAM II position as in (\ref{tam2}).

\begin{exe}
\ex \label{tam2} 
\gll Ev-e git-miş ve uyu-muş-tu-k. \\ 
home-{\Dat} go-{\Evi} {\And} sleep-{\Evi}-{\Pst}-{\Fpl} \\
\glt `There was the time we went home and slept.'
\end{exe}

As a last summary Kabak gives the following points for SA in the verbal domain:
\begin{exe}
\sn \begin{xlisti}
    \ex the ability of a verbal morpheme to terminate a word is related to its ability to stand without an agreement marker
    \ex SA is only applicable if what is left after suspension is a morphological word, and both the conjuncts end with terminal morphemes
    \ex Conjuncts with cliticlike endings are interpreted as 3$^{rd}$ person singular, causing agreement mismatches in SA
    \ex Nonfinal conjunct's terminal suffix must be overt\\*
    \hfill Adapted from \cite{kabak2007turkish}
    \end{xlisti}
\end{exe}

Kabak recognizes that in SA what is relevant is the size of what is left after suspension. The `cliticlike' condition on his third point is not clear-cut, and can be extended to other suffixes which have 3$^{rd}$ person singular suffixes which allow SA, that can seemingly end a word without copula ( \textit{-mIş}, \textit{-(y)AcAK}, and \textit{-Iyor}, to name a few). This condition relies heavily on what is `cliticlike'. The paramount observation that \cite{kabak2007turkish} provides is the relation between a successful SA and what is left as a morphological word. \footnote{A note of Kabak's informs the reader about the grammaticality judgments that come from 4 native speakers including himself. They all use, as he mentions, the `İstanbul' variety of Turkish. Some refer to `Istanbul' variety as `standard' Turkish. I oppose both the terms since no extensive or comprehensive study is provided to define what constitutes a `standard' or `Istanbul' variety of Turkish. I take Kabak's statement as his care for not including some regional changes, for example, in agreement paradigms like those later provided in \cite{saug2013verbal} for Denizli Dialect, which hosts some observations for the sole unsespendability of the k-paradigm agreement markers.} Kabak evaluates the examples of SA with derivational suffixes as natural coordination of nouns in the lexicon \citep{walchli2005co}, and does not regard such examples as SA.

