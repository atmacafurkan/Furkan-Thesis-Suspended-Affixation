\section{Suspended affixation and \textit{ile/=(y)lA}} \label{sec:ile}

In this section, I present the clitic \textit{ile/=(y)lA} in Turkish that serves several functions. My aim is to show how the conjoiner function of this clitic relates to SA. I argue that \textit{ile/=(y)lA} is morphologically the conjoiner head but its phonological size includes the place where {\Case} is encoded. According to \cite{goksel2004turkish} \textit{ile/=(y)lA} can be used as a case marker and a conjoiner as in (\ref{ilefunctions}). Being a clitic, \textit{ile/=lA} is outside the phonological word and thereby unstressable.

\begin{exe}
    \ex \label{ilefunctions} 
    \begin{xlist}
    \ex Instrumental \label{ileins}\\*
    \gll Şişe-yi çakmak ile aç-tı.\\ 
    bottle-{\Acc} lighter {\Ins} open-{\Pst}[{\Tsg}] \\
    \glt `S/he opened the bottle with a lighter.'
    
    \ex Comitative \label{ilecomit}\\*
    \gll Ahmet ev-e Mehmet ile (birlikte) gel-di. \\
    A[{\Nom}] house-{\Dat} M {\Com} (together) come-{\Pst}[{\Tsg}] \\
    \glt `Ahmet came home (together) with Mehmet'
    
    \ex Conjoiner \label{ileconj}\\*
    \gll kitap ile kalem çok pahalı. \\
    book {\And} pencil very expensive \\
    \glt `The book and the pencil is very expensive.'
    \end{xlist}
\end{exe}

The first function of \textit{ile/=(y)lA} in (\ref{ileins}) is like a semantic case \cite{woolford2006lexical} that seemingly does not have a case assigner. The second function of \textit{ile/=(y)lA} in (\ref{ilecomit}) is like a semantic case that can have an overt or covert case assigner, a postposition, \textit{birlikte} `together'. The third function of \textit{ile/=(y)lA} in (\ref{ileconj}) is a conjoiner. I am only interested in the conjoiner function of the clitic \textit{ile/=(y)lA} ({\And}, and ={\And} in glosses). I give an example of SA with \textit{ile/=(y)lA} in (\ref{ileandSA}).

\begin{exe}
    \ex \label{ileandSA}
    \begin{xlist}
    \ex \gll Ahmet kitap=la kalem-ler-i al-dı. \\ 
    A[{\Nom}] book={\And} pencil-{\Pl}-{\Acc} take-{\Pst}[{\Tsg}] \\
    \glt `Ahmet took the books and the pencils.'\\*
    `Ahmet took the book and the pencils.'
    \end{xlist}
\end{exe}


\subsection{SA in \textit{ile/=lA} constructions}

SA of {\Pl}, or {\Poss} in the environment of the clitic \textit{ile/=(y)lA} is ambiguous like it is in a conjunction formed with \textit{ve} `and'. The clitic \textit{ile/=(y)lA} allows for insertion of {\Pl} and {\Poss} suffixes between itself and the noun it is attached to. It does not allow the insertion of {\Case} but it allows SA of them, as shown in (\ref{ileprops}).

\begin{exe}
\ex \label{ileprops}
    \begin{xlist}
    \ex \gll *kitap-lar-ım-ı=ylA defter-ler-i al-dı-m. \\ 
    book-{\Pl}-{\Poss}.{\Fsg}-{\Acc}={\And} notebook-{\Pl}-{\Acc} take-{\Pst}-{\Fsg} \\

    \ex \gll kitap-lar-ım=lA defter-ler-i al-dı-m. \\ 
    book-{\Pl}-{\Poss}.{\Fsg}={\And} notebook-{\Pl}-{\Acc} take-{\Pst}-{\Fsg} \\
    \glt `I took my books and the notebooks.'
\end{xlist}
\end{exe}

As a general constraint, SA takes place for the rightmost terminal nodes. If the rightmost terminal node does not match the suspended affixes, SA does not take place. In (\ref{ileSA}) all the second conjuncts have the rightmost {\Pl-\Acc}. 

\begin{exe}
    \ex \label{ileSA}
    \begin{xlist}
        \ex \label{ilesa1}\gll Kalem=le kitap-lar-ı al. \\ 
        pencil={\Case} book-{\Pl}-{\Acc} take.{\Imp} \\
        \glt `Take the pencils and the books'
        
        \ex \label{ilesa2}\gll Kalem ve kitap-lar-ı al. \\ 
        pencil {\And} book-{\Pl}-{\Acc} take.{\Imp} \\
        \glt `Take the pencils and the books'
    \end{xlist}
\end{exe}

If the clitic \textit{ile/=lA} in (\ref{ilesa1}) were to be a case marker, it would mismatch with {\Acc}. This should have stopped SA of {\Pl}. This is not the case and both sentences in (\ref{ileSA}) are examples of SA. There is no SA environment in Turkish that violates the rightward-bound process of deletion, and positing \textit{ile/=(y)lA} as an exception is not needed if an explanation that captures both the SA capability and inability of {\Case} insertion can be given. The examples in (\ref{ileSA}) would violate the rightward-bound nature of SA since {\Poss} and {\Pl} suffixes before \textit{ile/=(y)lA} would be subject to suspension but not \textit{ile/=(y)lA} itself. I argue that, in its conjoiner function, \textit{ile/=(y)lA} itself is a conjoiner head and not a case mark assigned by a zero conjoiner head.


\subsection{What does \textit{ile/=lA} conjoin?}

The phrase that \textit{ile/=(y)lA} conjoins is not marked for {\Case}, yet it can be marked for number and possession. The first approach to conjoiner \textit{ile/=(y)lA} can use the insertable and non-insertable suffixes to determine the size of a conjunct for \textit{ile/=(y)lA}. In \S \ref{SAanalysisproposal} I proposed to place both number and agreement suffixes on the small `n' head. If {\Case} can not be inserted before \textit{ile/=(y)lA} but number and possession can be, then it might be conjoining nPs. In Figure \ref{fig:ile}, I give a representation for \textit{ile/=(y)lA} conjoining nPs.

\begin{figure}[hbt!]
    \centering
    \begin{forest}
    for tree={inner sep=0}
    [nP 
        [BP 
            [nP 
                [NP]
                [n]]
            [B\\\textit{ile/=(y)lA}]]
        [nP 
            [NP]
            [n]]]
    \end{forest}
    \caption{Representation of \textit{ile/=(y)lA} as a conjoiner of nPs}
    \label{fig:ile}
\end{figure}

This analysis argues for a small conjunction of two inflectional levels before a DP layer and after the lexical item. One issue with this analysis comes about when the conjuncts are modified with a modifier that requires a DP layer. In Turkish, there is a suffix \textit{-ki} that is attached to {\Loc} marked nouns and it either derives an adjectival modifier or a pronominal. I give the examples in (\ref{kiexample}) to show the difference of adjectival \textit{-ki} than a normal adjectival modifier.

\begin{exe}
\ex \label{kiexample}
    \begin{xlist}
    \ex \label{kiexample1}
    \gll Ahmet küçük kitap bul-a-ma-dı. \\
    A[{\Nom}] small book find-{\Abil}-{\Neg}-{\Pst}[{\Tsg}] \\ 
    \glt `Ahmet bought some small book'

    \ex \label{kiexample2}
    \gll *Ahmet araba-da-ki kitap bul-a-ma-dı. \\ 
    A[{\Nom}] car-{\Loc}-ki book find-{\Abil}-{\Neg}-{\Pst}[{\Tsg}] \\
    
    \ex \label{kiexample3}
    \gll Ahmet araba-da-ki kitab-ı bul-a-ma-dı. \\ 
    A[{\Nom}] car-{\Loc}-ki book-{\Acc} find-{\Abil}-{\Neg}-{\Pst}[{\Tsg}] \\
\glt `Ahmet took the book in the car.'
\end{xlist}
\end{exe}

The nouns that are modified with an adjective can be non-referential as in (\ref{kiexample1}), but nouns that are modified with \textit{ki} derived modifiers can not be non-referential (\ref{kiexample2}). This shows that \textit{ki} modifiers require a position where the noun is already referential, and according to \cite{ozturk2002turkish} the DP layer is the place where referentiality is encoded. If the clitic \textit{ile/=(y)lA} were to be analyzed as in Figure \ref{fig:ile}, \textit{ki} derived modifiers should have rendered (\ref{ilecomplex}) ungrammatical.

\begin{exe}
\ex \label{ilecomplex} 
\gll Ahmet masa-da-ki kitap=la vazo-da-ki çiçeğ-i getir-di. \\
A[{\Nom}] table-{\Loc}-ki book={\And} vase-{\Loc}-ki flower-{\Acc} bring-{\Pst}[{\Tsg}] \\
\glt`Ahmet brought the book on the table and the flower in the vase.'
\end{exe}

The structural interpretation of \textit{ile/=lA} now has a the following problem. The use of \textit{ile/=(y)lA} in (\ref{ileSA}) and (\ref{ilecomplex}) are uses of the clitic as a conjoiner morpheme and it does not allow {\Case} insertion. The \textit{ki} derived modifiers require a DP layer so positing a conjunction of nPs is not feasible. Additional support for DP level conjunction in \textit{ile/=(y)lA} comes from the nominalized sentences in Turkish. \textit{ile/=(y)lA} can conjoin two nominalized sentences as in (\ref{ilesentence}).

\begin{exe}
\ex \label{ilesentence} 
\gll Ben-im ev-e gel-me-m=le sen-in uyan-ma-n aynı an-da ol-ma-dı. \\ 
{\Fsg}-{\Gen} house-{\Dat} come-{\Nmlz}-{\Fsg}={\And} {\Ssg}-{\Gen} wake\_up-{\Nmlz}-{\Ssg} same moment-{\Loc} happen-{\Neg}-{\Pst}[{\Tsg}] \\
\glt`Me coming home and you waking up did not happen at the same time.'
\end{exe}

Following from all the observations, I argue that the inability to insert overt case markers before \textit{ile/=lA} does not stem from the lack of a DP layer or whether \textit{ile/=(y)lA} functions as {\Case}. It is rather based on the vocabulary insertion. I propose to consider \textit{ile/=(y)lA} as a conjoiner like \textit{ve} `and' that can conjoin DP level nouns, and its phonological size includes the DP head and the BP head. I provide Figure \ref{fig:ilefinal} for a final representation of my proposal ({\Acc} is just a placeholder, any other {\Case} is applicable). In this representation I show the phonological insertion for the morphemes. The DP head is still morphologically encoded with {\Case}, but its vocabulary insertion is overwritten by the clitic \textit{ile/=(y)lA} that serves as the BP head morphologically and syntactically.

\begin{figure}[hbt!]
    \centering
    \begin{forest} for tree= {sn edges, s sep=15mm}
    [DP 
        [BP 
            [DP
                [nP, s sep =10mm 
                    [NP]
                    [n\\{\Pl-\Poss}]]
                [D, name=D]]
            [B, name=B]]
        [DP]]
    \node[right=0em of D](case){[\Acc]};
    \node[fit= (D)(case), draw, ellipse, dashed, scale=0.8](phon){};
    \node[below=0em of phon](item){\textit{-(y)I}};
    \node[fit=(D)(item)(phon)(B), draw, thick, ellipse, dashed, scale=0.8, rotate=145](clitic){};
    \node[below right=6em of clitic](item2){\textit{ile/=(y)lA}};
    \end{forest}
    \caption{Conjoiner \textit{ile/=(y)lA} phonologically occupying conjoiner head and {\Case}}
    \label{fig:ilefinal}
\end{figure}

This analysis is against an approach that uses subset principle where a vocabulary item is inserted for a place that it contains the morphemes for. Figure \ref{fig:ilefinal} places the vocabulary item for the clitic \textit{ile/=lA} on `D' and `B' with granting it the morphological realization of only `B'. This is solved by a procedure of Impoverishment \citep{bonet1991morphology}, where a specific vocabulary item does not contain all the morphemes it is inserted for. This means that \textit{ile/=lA} always triggers an operation of impoverishment at vocabulary insertion, it morphologically represents `B' but occupies the phonological space for both `D' and `B'. A position class morphology \citep{inkelas1993nimboran,stump1993position} in this case might not be helpful, since the conjoiner head is not an inflection on the noun but a clitic that needs a phonological host.