\subsection{Ossetic}

\cite{erschler2012suspended} and \cite{erschler2018suspended} deals specifically with SA in Ossetic. Ossetic is a language spoken in Northern Georgia and bordering Russia. Ossetic displays a set of data that on the surface seems to be inconsistent when it comes to SA. For example, when a pronominal and a pronoun is conjoined, the choices of case for the both conjuncts seem to change depending on the order of the conjuncts (\ref{ossetic}).

\begin{exe}
    \ex \label{ossetic}
    \begin{xlist}
        \ex \label{ossetic1}
        \gll 
        \textit{d\textturna w} \textit{\textturna ma} \textit{Alan-\textturna j} \textit{tarst\textturna n} \\
        {\Second}{\Sg}.{\Obl} {\And} Alan-{\Abl} be.afraid.{\Pst}.{\First}{\Sg} \\
        \glt `I am afraid of you and Alan'
        
        \ex \label{ossetic2}
        \gll 
        \textit{Alan} \textit{\textturna ma} \textit{d\textturna w-\textturna j} \textit{tarst\textturna n} \\
        Alan[{\Nom}] {\And} {\Second}{\Sg}-{\Abl} be.afraid.{\Pst}.{\First}{\Sg} \\
        \glt `I am afraid of Alan and you'
    \end{xlist}
    ${}$ \hfill Adapted from \cite{erschler2012suspended}
\end{exe}

While in (\ref{ossetic1}) it seems that there is no SA, but in (\ref{ossetic2}) there seems to be suspension of {\Abl} from the proper noun \textit{Alan}. Erschler assumes that SA is an ellipsis process that takes place after vocabulary insertion at PF, where the morphemes are present for semantic calculation but phonologically null. 

\cite{erschler2012suspended} deals specifically with SA of case in Ossetic. Before moving on with examples and analysis of SA, Erschler provides some background into the case system of Ossetic. Definite animates, and personal pronouns are obligatorily marked {\Obl}, inanimate objects are marked {\Nom}, and modifiers are not case marked. All plural nouns in Ossetic loose their final [\textturna] sound when marked by vowel initial case markers. This is taken to be a phonological constraint, since consonant initial case markers do not trigger the same alternation (\ref{osseticdeletion}).

\begin{exe}
    \ex \label{osseticdeletion}
    \begin{xlist}
        \ex \gll
        \textit{b\textturna\textchi-t\textturna} \\ horse-{\Pl}[{\Nom}] \\
        
        \ex \gll 
        \textit{b\textturna\textchi-t-\textschwa} \\ horse-{\Pl}-{\Obl} \\
    \end{xlist}
    ${}$ \hfill Adapted from \cite{erschler2012suspended}
\end{exe}

With these in mind, Erschler proposes some constraints, first of which is that anty case marker may be suspended. This is not so much of a constraint but an observation, with the support of the examples in (\ref{osseticSA}).

\begin{exe}
    \ex \label{osseticSA}
    \begin{xlist}
        \ex \gll 
        \textit{Soslan} \textit{\textturna ma} \textit{Zalijn-i} \textit{\textchi\textturna\textdzlig r\textturna} \\ Soslan[{\Nom}] {\And} Zalina-{\Obl} house \\
        \glt `the house of Soslan and Zalina'
        
        \ex \gll 
        \textit{Alan} \textit{\textturna ma} \textit{Soslan-b\textturna l} \textit{is-\textturna mbaltt\textturna n} \\ Alan[{\Nom}] {\And} Soslan-{\Sup} {\Prv}-meet.{\Pst}.{\First}{\Sg} \\
        \glt `I met Alan and Soslan'
        
        \ex \gll 
        \textit{Alan} \textit{\textturna ma} \textit{Soslan-b\textturna j} \textit{tarst\textturna n} \\ Alan[{\Nom}] {\And} Soslan-{\Abl} be.afraid.{\Pst}.{\First}{\Sg} \\
        \glt `I was afraid of Alan and Soslan'
        
        \ex \gll 
        \textit{budur} \textit{\textturna ma} \textit{\textinvscr\textturna d-i} \textit{ber\textturna} \textit{\v{c}'ewu-t\textturna} \textit{i\v{s}-\v{s}erdtonc\textturna} \\ field[{\Nom}] {\And} forest-{\Loc} many bird-{\Pl} {\Prv}-find.{\Pst}.{\Third}{\Pl} \\
        \glt `They found many birds in the field and the forest'
    \end{xlist}
    ${}$ \hfill Adapted from \cite{erschler2012suspended}
\end{exe}

The second constraint is that the first conjunct in SA should be the base of the case marker, without phonological processes like [\textturna] deletion (\ref{osseticSA2}).

\begin{exe}
    \ex \label{osseticSA2}
    \begin{xlist}
    \ex 
    \gll 
    \textit{b\textturna\textchi-t-im\textturna} \textit{\textturna m\textturna\footnotemark} \textit{g\textturna l-t-im\textturna} \\ horse-{\Pl}-{\Com} {\And} ox-{\Pl}-{\Com} \\
    
    \ex 
    \gll 
    \textit{*b\textturna\textchi-t} \textit{\textturna m\textturna} \textit{g\textturna l-t-im\textturna} \\ horse-{\Pl} {\And} ox-{\Pl}-{\Com} \\
    
    \ex 
    \gll 
    \textit{b\textturna\textchi-ta} \textit{\textturna m\textturna} \textit{g\textturna l-t-im\textturna} \\ horse-{\Pl} {\And} ox-{\Pl}-{\Com} \\
    \glt `with horses and oxen'
    \end{xlist}
    ${}$ \hfill Adapted from \cite{erschler2012suspended}
\end{exe}

\footnotetext{
\cite{erschler2012suspended}'s examples consist of two branches of Ossetic, Iron and Digor. For the sake of summarizing, I do not specify which examples are which and reiterate respective examples freely to serve the points made. Due to that, the reader may observe some differences in lexical items.
}

Complying with the same constraint, personal pronouns that have different bases for some of the cases need to have those bases as their remnants in the first conjunct (\ref{osseticSA3}).

\begin{exe}
    \ex \label{osseticSA3}
    \begin{xlist}
        \ex \gll 
        \textit{d\textturna w/*du} \textit{\textturna ma} \textit{alan-b\textturna l} \textit{is-\textturna mbaltt\textturna n} \\ {\Second}{\Sg}[{\Obl}]/*{\Second}{\Sg}[{\Nom}] {\And} Alan-{\Sup} {\Prv}-meet.{\Pst}.{\First}{\Sg} \\
        \glt `I met you and Alan'
        
        \ex \gll 
        \textit{d\textturna w/*du} \textit{\textturna ma} \textit{alan-\textturna j} \textit{t\textturna rsun} \\ {\Second}{\Sg}[{\Obl}]/*{\Second}{\Sg}[{\Nom}] {\And} Alan-{\Abl} be.afraid.{\Prs}.{\First}{\Sg} \\
        \glt `I am afraid of you and Alan'
    \end{xlist}
    ${}$ \hfill Adapted from \cite{erschler2012suspended}
\end{exe}

The third constraint for Ossetic SA is what is left after suspension should be an independent (morphological) word. The two branches of Ossetic differ in regarding a reciprocal form `each other' as an independent word. In Iron Ossetic it is an independent word and can take part in SA whereas the Digor counterpart is not an independent word and does not take place in SA (\ref{osseticSA4}).

\begin{exe}
    \ex \label{osseticSA4}
    \begin{xlist}
        \ex 
        \gll 
        \textit{*n\textturna=duw\textturna} \textit{tiki\v{s}-i} \textit{k\textturna r\textturna\textdyoghlig e} \textit{\textturna ma} \textit{n\textturna=kuj-\textturna j} \textit{t\textturna rs-unc\textturna} \\ poss.{\First}.{\Pl}=two cat-{\Obl} each.other {\And} poss.{\First}.{\Pl}=dog-{\Abl} be.afraid.{\Prs}.{\Third}{\Pl} \\
    
        \ex 
        \gll 
        \textit{?n\textturna=d\textschwa w\textturna} \textit{g\textturna dy-je} \textit{k\textturna r\textturna zi} \textit{\textturna m\textturna} \textit{n\textturna=k^w\textschwa z-\textturna j} \textit{t\textturna r\v{s}-\textschwa nc} \\ poss.{\First}.{\Pl}=two cat-{\Obl} each.other {\And} poss.{\First}.{\Pl}=dog-{\Abl} be.afraid.{\Prs}.{\Third}{\Pl} \\
        \glt `Our two cats are afraid of each other and of our dog'
    \end{xlist}
\end{exe}

The fourth constraint of Ossetic SA is that what is left after SA should not have idiosyncratic meaning. This constraint relates to the {\Third}{\Sg} pronoun form \textit{w\textschwa m} which has the meaning `there' that serves as the base for the Dative marked {\Third}{\Sg} pronoun (\ref{osseticSA5}).

\begin{exe}
    \ex \label{osseticSA5}
    \begin{xlist}
        \ex 
    \gll 
    \textit{w\textschwa m} \textit{\textturna m\textturna} \textit{m\textturna din\textturna-j\textturna n} \textit{didin\textdyoghlig\textschwa t\textturna} \textit{ratta} \\ there {\And} Madina-{\Dat} flowers gave \\
    
    \ex 
    \gll 
    \textit{w\textschwa m-\textturna n} \textit{\textturna m\textturna} \textit{m\textturna din\textturna-j\textturna n} \textit{didin\textdyoghlig\textschwa t\textturna} \textit{ratta} \\ s/he-{\Dat} {\And} Madina-{\Dat} flowers gave \\
    \glt `S/he gave flowers to her and Madina'
    \end{xlist}
    ${}$ \hfill Adapted from \cite{erschler2012suspended}
\end{exe}

The final constraint for Ossetic SA is that when both conjuncts are pronouns no suspended affixation takes place, a point illustrated in (\ref{osseticSA6}).

\begin{exe}
    \ex \label{osseticSA6}
    \begin{xlist}
    \ex \gll 
    \textit{m\textturna n-b\textturna l} \textit{\textturna m\textturna} \textit{d\textturna w-b\textturna l} \textit{\textturna ww\textturna nduj} \\ {\First}{\Sg}-{\Sup} {\And} {\Second}{\Sg}-{\Sup} believe.{\Prs}.{\Third}{\Sg} \\
    \glt `S/he believes me and you'
    
    \ex \gll 
    \textit{*m\textturna n} \textit{\textturna m\textturna} \textit{d\textturna w-b\textturna l} \textit{\textturna ww\textturna nduj} \\ {\First}{\Sg}[{\Obl}] {\And} {\Second}{\Sg}-{\Sup} believe.{\Prs}.{\Third}{\Sg} \\
    \glt Intended `S/he believes me and you'
    \end{xlist}
    ${}$ \hfill Adapted from \cite{erschler2012suspended}
\end{exe}

Following these observations Erschler argues that SA needs to be a phonological deletion process after vocabulary insertion instead of a structural sharing process. Since come of the remnants would be nons-words that only have meaning when considered adjacent with the suspended affix. Meaning that remnants might not be well-formed words. Additionally, Erschler argues against an approach where case markers are treated as syntactic projections. Which in turn makes the structural sharing argument less appealing. He provides the following examples in (\ref{osseticdepict}) where complements of adpositions can not control depictives, but case marked arguments can.

\begin{exe}
    \ex \label{osseticdepict}
    \begin{xlist}
        \ex \gll 
        \textit{soslan} \textit{\textchi et\textturna g-i} \textit{\textchi\textturna cc\textturna} \textit{rasug-\textturna j} \textit{\textdzlig or-uj} \\ Soslan[{\Nom}] Xetag-{\Obl} with drunk-{\Abl} talk-{\Prs}.{\Third}{\Sg} \\
        \glt `Soslan_i is talking to Xetag_i when he_{i/*j} is drunk.'
    
        \ex \gll 
        \textit{soslan} \textit{\textchi et\textturna g-b\textturna l} \textit{rasug-\textturna j=d\textturna r} \textit{\textturna ww\textturna nd-uj} \\ Soslan[{\Nom}] Xetag-{\Sup} drunk-{\Abl}={\Emp} believe-{\Prs}.{\Third}{\Sg} \\
        \glt `Soslan_i believes in Xetag_i even when he_{i/j} is drunk'
    \end{xlist}
\end{exe}

In \cite{erschler2018suspended}, Erschler further develops the approach of ellipsis for SA. To show that SA is an ellipsis process Erschler provides the alternative question configurations in which SA can take place (\ref{osseticalternative}).

\begin{exe}
    \ex \label{osseticalternative}
    \begin{xlist}
        \ex \gll 
        \textit{s\textturna rm\textturna t(-m\textturna)} \textit{\textturna vi} \textit{uruzm\textturna g-m\textturna} \textit{\textdzlig urdtaj?} \\ Sarmat(-{\All}) {\Or}.Q Uruzmag-{\All} you.called \\ 
        \glt `Did you call Sarmat or Uruzmag ?'
        
        \ex \gll 
        \textit{ad\textturna jmag} \textit{k^w\textschwa d} \textit{f\textturna\v{z}\textschwa nd?} \textit{arv-\textschwa} \textit{c'\textturna\textinvscr(-\textturna j)} \textit{\textturna vi} \textit{\v{s}\textschwa\textdyoghlig\textschwa t-\textturna j} \textit{rajg^w\textschwa rd} \\ human how appeared sky-{\Obl} blue-{\Abl} {\Or}.Q clay-{\Abl} was.born \\
        \glt `How did the humans appear? Were they born from the sky blue or from clay?'
    \end{xlist}
\end{exe}

To draw a comparison to the examples in Ossetic, I try to mirror the examples in (\ref{turkishalternative}) in two ways, one which is the true alternative question made by two question clitics \textit{=mI} after each alternative, the other is a disjunctive yes/no question. 

\begin{exe}
    \ex \label{turkishalternative}
    \begin{xlist}
    \ex
    \gll 
    \textit{Ali*(-yi)=mi} \textit{Mehmet-i=mi} \textit{ara-dı-n?} \\ Ali-{\Acc}=Q Mehmet-{\Acc}=Q call-{\Pst}-{\Second}{\Sg} \\
    \glt `Did you call Ali or did you call Mehmet?'
    
    \ex 
    \gll 
    \textit{Ali} \textit{veya} \textit{Mehmet-i=mi} \textit{ara-dı-n?} \\ Ali {\Or} Mehmet-{\Acc}=Q call-{\Pst}-{\Second}{\Sg} \\
    \glt `Did you call Ali or Mehmet?'
    \end{xlist}
\end{exe}

The examples in (\ref{turkishalternative}) show that in Turkish alternative questions does not allow for SA unlike Ossetic. However, disjunctive SA lets such a configuration. One important point needs to be made here though, the question clitic \textit{=mI} in Turkish is a focusing element which draws focus to the preceding argument it is attached to. That's why a deletion process is not allowed, since focused elements can not be elided (CITE). This in a way both supports a deletion analysis, since question types contrast in focus and SA properties, and also refutes SA with an analysis of alternative questions on a par with Ossetic.

After claiming that SA is an ellipsis process, Erschler moves onto pinpointing where this deletion process takes place. He uses the DM framework, and argues that SA takes place after VI but before morpheme specific readjustments. For examples the support for SA taking place after VI comes from the example in (\ref{osseticVI}), and the support for SA taking place before morpheme specific phonological adjustments comes from the example in (\ref{osseticPA}).

\begin{exe}

    \ex \begin{xlist}
    \ex \label{osseticVI}
    \gll 
    \textit{d\textturna w(-b\textturna l)/*du} \textit{\textturna ma} \textit{m\textturna din\textturna-b\textturna l} \textit{is\textturna mbaltt\textturna n} \\ {\Second}{\Sg}.{\Obl}-({\Sup})/{\Second}{\Sg}.{\Nom} {\And} Madina-{\Sup} {\First}{\Sg}.met \\
    \glt `I met you and Madina'
    
    \ex \label{osseticPA}
    \begin{xlisti}
    \ex
    \gll 
    \textit{park} \textit{\textturna m\textturna} \textit{w\textschwa n\textdyoghlig-\textschwa} \\ park {\And} street-{\Obl} \\
    \glt `in/of the street and the park'
    
    \ex 
    \gll 
    \textit{w\textschwa ng} \textit{\textturna m\textturna} \textit{par\textteshlig-\textschwa} \\ street {\And} park-{\Obl} \\
    \glt `in/of the park and the street'
    \end{xlisti}
    \end{xlist}
\end{exe}

With these examples, Erschler concludes that SA takes place after VI and before PA. In formulating the structural constraints for SA, Erschler posits that it is a backward ellipsis process under identity where not all conjuncts should bear \textsc{[+{\Emp}]} feature. He cites \cite{herbeck2016controlling} in defense of positing information structure features in the lexicon for lexical items where Herbeck argues that Spanish overt pronouns have feature \textsc{[+{\Foc}]}. As Ossetic is a pro-drop language like Turkish, cf. \cite{ozturk2002turkish} overt Turkish pronouns, overt pronouns need to be discourse configured hence the feature \textsc{[+{\Emp}]}. 

% \begin{exe}
%     \ex 
%     \begin{xlist}
%     \ex \gll 
%     \textit{ma\textchi=ta} \textit{m\textturna n-m\textturna} \textit{g\textturna \v{s}g\textturna} \textit{k\textturna r\textturna zi} \textit{\textchi or\v{z}} \textit{\v{z}on\textturna m} \\ {\First}.{\Pl}=CTR {\First}{\Sg}[{\Obl}]-all according each.other well know.{\Prs}{\First}.{\Pl} \\ 
%     \glt `But we, in my opinion, know each other well'
    
%     \ex \gll 
%     \textit{w\textturna d} \textit{k\textturna r\textturna \textdyoghlig e-*(j)} \textit{\textchi^w\textturna zd\textturna r} \textit{ba-l\textturna d\textturna rd\v{z}inan} \\ then each.other-{\Obl} better {\Prv}-understand.FUT.{\First}.{\Pl} \\
%     \glt `Then we will better understand each other.'
    
%     \end{xlist}
% \end{exe}