\section{Why \textit{-(y)Ip} is a conjoiner}

Since it is only used with verbs and in a very limited manner allowing only voice, Mod$_{Abil}$ and Negation under it, considering \textit{-(y)Ip} as a conjoiner instead of an adverbial clause is quite cumbersome. To be able to show that it is a conjoiner I present some examples where \textit{-(y)Ip}(PC in glosses) behaves different than other adverbial suffixes in Turkish.

For a simple example of \textit{-(y)Ip} consider the example in (\ref{ipintro}).

\begin{exe}
    \ex \label{ipintro}
    \gll 
    \textit{Ahmet} \textit{koş-up} \textit{düş-tü.} \\ Ahmet[{\Nom}] run-PC fall-{\Pst}[{\Third}.{\Sg}] \\
    \glt `Ahmet ran and fell'
\end{exe}

The example in (\ref{ipintro}) shows that \textit{-(y)Ip} is able to conjoin two predicates that do not match in their Voice, unergative and unaccusative, yet has a non-contrasting subject. The first contrasting behaviour of \textit{-(y)Ip} is, unlike adverbial suffixes like \textit{-IncA} `$\sim$when' (WHEN in glosses ), and \textit{-mAdAn} (WO in glosses) `$\sim$without Ving', that adverbial suffixes allow for contrasting subjects, but \textit{-(y)Ip} does not (\ref{ipcontrast}).

\begin{exe}
    \ex \label{ipcontrast}
    \begin{xlist}
        \ex \gll 
        \textit{Ahmet} \textit{koş-unca} \textit{Mehmet} \textit{düş-tü.} \\ Ahmet[{\Nom}] run-WHEN Mehmet[{\Nom}] fall-{\Pst}[{\Third}.{\Sg}] \\
        \glt `When Ahmet ran, Mehmet fell'
        
        \ex \gll 
        \textit{Ahmet} \textit{koş-madan} \textit{Mehmet} \textit{düş-tü} \\ Ahmet[{\Nom}] run-WO Mehmet[{\Nom}] fall-{\Pst}[{\Third}.{\Sg}] \\
        \glt `Mehmet fell without Ahmet running'
        
        \ex \gll 
        \textit{*Ahmet} \textit{koş-up} \textit{Mehmet} \textit{düş-tü} \\ Ahmet[{\Nom}] run-PC Mehmet[{\Nom}] fall-{\Pst}[{\Third}.{\Sg}] \\
        \glt Intended `Ahmet ran and Mehmet fell'
        
    \end{xlist}
\end{exe}

An objection to this observation can come from the fact that there is an adverbial suffix that also hinders contrasting subject use which is \textit{-(y)ArAK} `$\sim$ by Ving' (BY in glosses) as in (\ref{ipcontester}).

\begin{exe}
    \ex \label{ipcontester}
    \begin{xlist}
    \ex
    \gll 
    \textit{Ahmet} \textit{koş-arak} \textit{düş-tü} \\ Ahmet[{\Nom}] run-BY fall-{\Pst}[{\Third}.{\Sg}] \\
    \glt `Ahmet fell running'
    
    \ex \gll 
    \textit{*Ahmet} \textit{koş-arak} \textit{Mehmet} \textit{düş-tü} \\ Ahmet[{\Nom}] run-BY Mehmet[{\Nom}] fall-{\Pst}[{\Third}.{\Sg}] \\
    \end{xlist}
\end{exe}

However \textit{-(y)Ip} deviates from \textit{-(y)ArAK} in verb-manner relation and it deviates from all the adverbial suffixes including \textit{-(y)ArAK} in word ordering restrictions. When it comes to the verb-manner selection see the examples in (\ref{ipandarak}).

\begin{exe}
    \ex \label{ipandarak}
    \begin{xlist}
        \ex \label{ipandarak1} \gll 
        \textit{Ahmet} \textit{uyu-yup} \textit{kalk-tı.} \\ Ahmet[{\Nom}] sleep-PC get\_up-{\Pst}[{\Third}.{\Sg}] \\
        \glt `Ahmet slept and woke up.'
        
        \ex \label{ipandarak2} \gll 
        {\%\textit{Ahmet}} \textit{uyu-yarak} \textit{kalk-tı.} \\ Ahmet[{\Nom}] sleep-BY get\_up-{\Pst}[{\Third}.{\Sg}] \\
    \end{xlist}
\end{exe}

It is clear that in (\ref{ipandarak2}) the adverbial suffix \textit{-(y)ArAK} is bound by verb-manner interpretations just like any other adverb whereas it is not the case for \textit{-Ip} as shown in (\ref{ipandarak2}). 

The second contrast of \textit{-(y)Ip} is the fact that it disallows some word orders. While it allows for the subject to precede it or be post verbal, it does not allow insertion of subject in between the suffixed verb and the main verb unlike the adverbial suffixes (\ref{ipandwordorder})\footnote{Remember that word order changes are not free of interpretation in Turkish, they result in different information settings. See CITE for word order and change effects in Turkish}. 

\begin{exe}
    \ex \label{ipandwordorder}
    \begin{xlist}
        \ex \begin{xlisti}
            \ex \gll 
            \textit{Ahmet} \textit{koş-up} \textit{gel-di.} \\ Ahmet[{\Nom}] run-PC come-{\Pst}[{\Third}.{\Sg}] \\
            \ex \gll 
            \textit{*koş-up} \textit{Ahmet} \textit{gel-di.} \\ run-PC Ahmet[{\Nom}] come-{\Pst}[{\Third}.{\Sg}] \\
            \ex \gll 
            \textit{koş-up} \textit{gel-di} \textit{Ahmet.}\\ run-PC come-{\Pst}[{\Third}.{\Sg}] Ahmet[{\Nom}] \\
            \glt `Ahmet ran and came'
        \end{xlisti}
        
        \ex \begin{xlisti}
            \ex \gll 
            \textit{Ahmet} \textit{koş-arak} \textit{gel-di.} \\ Ahmet[{\Nom}] run-BY come-{\Pst}[{\Third}.{\Sg}] \\
            \ex \gll 
            \textit{koş-arak} \textit{Ahmet} \textit{gel-di.} \\ run-BY Ahmet[{\Nom}] come-{\Pst}[{\Third}.{\Sg}] \\
            \ex \gll 
            \textit{koş-arak} \textit{gel-di} \textit{Ahmet.} \\ run-BY come-{\Pst}[{\Third}.{\Sg}] Ahmet[{\Nom}] \\
            \glt `Ahmet came running'
        \end{xlisti}
    \end{xlist}
\end{exe}

This behaviour of \textit{-(y)Ip} not allowing a word order with an intervening subject mirrors the binding properties of possessive constructions in Turkish (\ref{ipandbinding}).

\begin{exe}
    \ex \label{ipandbinding}
    \begin{xlist}
        \ex \gll 
        \textit{Ahmet-in_i} \textit{kuzen-i_i} \textit{ve} \textit{arkadaş-ı_i} \\ Ahmet-{\Gen} cousin-{\Poss}.{\Third}.{\Sg} {\And} friend-{\Poss}.{\Third}.{\Sg} \\
        \glt `Ahmet's cousin and friend'
        \ex \gll 
        \textit{kuzen-i_{*_{i/j}}} \textit{ve} \textit{Ahmet-in_i} \textit{arkadaş-ı_i} \\ cousin-{\Poss}.{\Third}.{\Sg} {\And} Ahmet-{\Gen} friend-{\Poss}.{\Third}.{\Sg} \\
        \glt `Someone's cousin and Ahmet's friend'
    \end{xlist}
\end{exe}

Stemming from the observations made in (\ref{ipcontrast}), (\ref{ipandarak}), and (\ref{ipandwordorder}) I propose that \textit{-(y)Ip} is a VoiceP level conjoiner. However suggesting as such comes with some drawbacks. When negation comes into play in the conjunction, things become problematic for a conjoiner \textit{-(y)Ip}. There are two problems, one of which is a problem of ambiguity when the secınd conjunct is marked with negation and this problem can be explained more or less with minimal assumptions. However, the second problem arises when the first conjunct is marked with negation and the second one is not. The solution to the second problem involves making an assumption about negation in Turkish being somewhat derivational.

The example for the first example comes about in examples like (\ref{refneg}).
\begin{exe}
    \ex \label{refneg}
    \gll 
    \textit{Ahmet} \textit{yürü-yüp} \textit{düş-me-di.} \\ Ahmet[{\Nom}] walk-PC fall-{\Neg}-{\Pst}[{\Third}.{\Sg}] \\
    \glt Reading 1: `Ahmet did not walk and did not fall' \\ Reading 2: `Ahmet walked but did not fall'
\end{exe}

The remedy for Reading 2 of (\ref{refneg}) is a movement analysis of \&P to FocP, thereby we observe a movement-trace effect in interpretation (CITE) as in Figure \ref{fig:refneg}

\begin{figure}[hbt!]
    \centering
    \begin{forest}
    for tree={inner sep=0} 
[FocP, s sep=20mm 
    [BP, name= specF, tikz={\node[draw,dashed, fit to=tree]{};}
        [VoiceP 
            [PRO_i]
            [Voice' 
                [VP\\\textit{yürü}]
                [Voice, name=botox]]]
        [B\\\textit{-yüp}]]
    [Foc' 
        [TP 
            [DP\\\textit{Ahmet_i}]
            [T' 
                [NegP
                    [VoiceP 
                        [\sout{BP}, name=conjP]
                        [VoiceP 
                            [PRO_i, name=specVoi]
                            [Voice' 
                                [VP\\\textit{koş}]
                                [Voice]]]]
                    [Neg\\\textit{-ma}]]
                [T\\\textit{-dı}]]]
        [Foc]]]
        \node[below=1em of botox](x){};
        \draw[->] (conjP) to[out=west, in=south] node[left,fill=white]{at LF}(x);
\end{forest}
     \caption{Focus raising of first conjunct in \textit{-(y)Ip} conjunction}
    \label{fig:refneg}
\end{figure}

This way we end up with a surface form that is {\Neg}\textgreater BP and an LF movement form that is BP\textgreater {\Neg}. The problem arises when only the first conjunct is marked with negation and the second one is not, with an interpretation that reflects the configuration. Conjunction analysis in such a configuration would end up in hierarchically unequivalent conjuncts. Because the first conjunct would be NegP and the second would be VoiceP. However an assumption about the Negation itself can change this hierarchy mismatch. If we are to assume that negation as a projection comes right after VP or VoiceP, we would be disregarding some of its properties. First of all, negation takes scope of higher projections like T, even if it is situated below it in the structure. As in (\ref{neg}).

\begin{exe}
    \ex \label{neg}
    \begin{xlist}
        \ex \label{neg1}\gll 
        \textit{Ahmet} \textit{ev-e} \textit{g\'el-me-di.} \\ Ahmet[{\Nom}] house-{\Dat} come-{\Neg}-{\Pst}[{\Third}.{\Sg}] \\
        \glt `Ahmet did not come home.'
        
        \ex \label{neg2}\gll 
        \textit{Ahmet} \textit{ev-e} \textit{g\'el-me-yebil-ir.} \\ Ahmet[{\Nom}] house-{\Dat} come-{\Neg}-PROB-AOR[{\Third}.{\Sg}] \\
        \glt `Ahmet might not come home.'
        
        \ex \label{neg3}\gll 
        \textit{Ahmet} \textit{ev-e} \textit{gel-\'e-me-di.} \\ Ahmet[{\Nom}] house-{\Dat} come-{\Abil}-{\Neg}-{\Pst}[{\Third}.{\Sg}] \\
        \glt `Ahmet could not come home'
    \end{xlist}
\end{exe}

Situating negation with the visible order and mirroring from English projection structure would have wrong predictions. While it hierarchically aligns with English in (\ref{neg1}), and \ref{neg2} it is not the case in (\ref{neg3}). Another difference from English is that negation causes stress change in the verb. It shifts the stress immediately to the left of it. This is a behaviour of clitics in Turkish. Since we have seen in Serbian second place clitics where a clitic chose a place in the linear word order, we can posit a similar rule for negation that is situated at a higher level but selects the VoiceP to be attached to. It should also be noted that in Turkish nominal clauses the negator is not \textit{=mA} but \textit{değil}, sometimes referred to as sentential negator. While \textit{değil} is compatible with both verbal and nominal clauses \textit{=mA} is not (\ref{madegil}).

\begin{exe}
    \ex \label{madegil}
    \begin{xlist}
        \ex \gll 
        \textit{Ahmet} \textit{hasta} \textit{(değil)} \\ Ahmet[{\Nom}] sick[{\Third}.{\Sg}] {\Neg} \\
        \glt `Ahmet is sick (is not sick)'
        
        \ex \gll 
        \textit{Ahmet} \textit{ev-e} \textit{gel-di} \textit{değil} \\ Ahmet[{\Nom}] house-{\Dat} come-{\Pst}[{\Third}.{\Sg}] {\Neg} \\
        \glt `Ahmet came home (it is not the case that he did)'
        
        \ex \gll 
        \textit{Ahmet} \textit{hasta} \textit{(*ma)} \\ Ahmet[{\Nom}] sick[{\Third}.{\Sg}] {\Neg} \\
        \glt `Ahmet is sick (*is not sick)'
        
        \ex \gll 
        \textit{Ahmet} \textit{ev-e} \textit{gel-(me)-di.} \\ Ahmet[{\Nom}] house-{\Dat} come-{\Neg}-{\Pst}[{\Third}.{\Sg}] \\
        \glt `Ahmet did (not) come home.'
    \end{xlist}
\end{exe}

The observations we made in (\ref{neg}) and (\ref{madegil}) shows the heavy clitic nature of the negation, and it operating only on verbal predicates. That is why, instead of positing a NegP that is atop VoiceP, I make use of Pol(arity)P atop TP. This way we can better explain some scope relations Turkish presents in (\ref{negscope}).

\begin{exe}
    \ex \label{negscope}
    \begin{xlist}
        \ex \gll 
        \textit{Herkes} \textit{ev-e} \textit{git-me-di.} \\ everybody[{\Nom}] house-{\Dat} go-{\Neg}-{\Pst}[{\Third}.{\Sg}] \\
        \glt `Not everyone went home. (some did)' \hfill {\Neg}\textgreater$\forall$
    \end{xlist}
\end{exe}

However there are some people, usually from Muğla and adjacent provinces in Turkey, that get $\forall$\textgreater {\Neg} reading in (\ref{negscope}). We can explain this interpretation as an interpretation resulting from \textit{=mA}'s phonological placement. Since its placement gives the order of T\textgreater {\Neg}\textgreater V, since the quantifier would move to SpecTP for Case and EPP. A projection that's higher than T with a clitic analysis for \textit{=mA} is better able to explain the scope readings, stress change and selective use. 

After the detour of negation in Turkish and its nature, I turn to the implications of this for \textit{-Ip} as a conjoiner. We simply have a single PolP above TP that has to distribute two polarity settings for two verbal predicates, and that is why we can have contrastive {\Neg} + AFF, AFF + {\Neg} readings, doing away with a focus movement. The finalized representation is given in Figure \ref{fig:ipfinal}.

\begin{figure}[hbt!]
    \centering
    \begin{forest}
    [TP, s sep=15mm
        [DP_i, name=specT]
        [T', s sep=15mm
            [VoiceP 
                [BP 
                    [VoiceP 
                        [PRO_i]
                        [Voice' 
                            [VP]
                            [Voice]]]
                    [B]]
                [VoiceP 
                    [PRO_i, name=DP]
                    [Voice' 
                        [VP]
                        [Voice]]]]
            [T]]]
    \end{forest}
    \caption{Updated representation of \textit{-Ip} as a predicate conjoiner}
    \label{fig:ipfinal}
\end{figure}

In this case, the head Pol can select to either individually tag polarities of each VoiceP or a single one to the conjoined VoiceP. Of course separate tagging of the predicates would be costly, which would be reflected in grammaticality judgments of language users. In this analysis of negation and \textit{-Ip} the only assumption is that Pol head can disperse more than one polarity in conjoined phrases. Which I take as the most minimal assuptions given the observations about \textit{-Ip} and negation clitic \textit{=mA} we made so far.

Further notes on PolP making use of clitics instead of Turkish having a {\Neg} suffix can come from questions in Turkish. The morpheme \textit{=mI} that is used in questions is also a clitic in Turkish, which also displays selective behaviour of where to be situated as in (\ref{cliticmi}).

\begin{exe}
    \ex \label{cliticmi}
    \begin{xlist}
        \ex \gll
        \textit{Ahmet} \textit{gel-di} \textit{=mi?} \\ Ahmet[{\Nom}] come-{\Pst}[{\Third}.{\Sg}] ={\Q} \\
        \glt `Did Ahmet come?'
        
        \ex \gll
        \textit{Ahmet} \textit{mi} \textit{gel-di?} \\ Ahmet[{\Nom}] ={\Q} come-{\Pst}[{\Third}.{\Sg}] \\
        \glt `Was it Ahmet that came'
        
        \ex \gll 
        \textit{Ahmet} \textit{kitab-ı} \textit{=mı} \textit{al-dı?} \\ Ahmet[{\Nom}] book-{\Acc} ={\Q} take-{\Pst}[{\Third}.{\Sg}] \\
        \glt `Was it the book that Ahmet took'
        
        \ex \gll 
        \textit{Ahmet} \textit{hızlı} \textit{mı} \textit{koş-tu?} \\ Ahmet[{\Nom}] fast run-{\Pst}[{\Third}.{\Sg}] \\
        \glt `Was it fast how Ahmet ran?'
    \end{xlist}
\end{exe}
This behaviour shows that on the purposes of questioning parts of a sentence the clitic \textit{=mI} can move freely next to the questioned phrase in the clause. Even though the sentence overall is still a question instead hosting an embedded question. This function of \textit{=mI} is adressed  

% This way I entertain the structure in Figure \ref{fig:neg}.

% \begin{figure}[hbt!]
%     \centering
%     \begin{tikzpicture}
%         \Tree[.FocP
%                 [.XP ]
%                 [.Foc' 
%                     [.NegP 
%                         [.TP 
%                             [.\node(voi){VoiceP}; 
%                                 [.DP ]
%                                 [.Voice' 
%                                     [.VP 
%                                         [.DP ]
%                                         [.V ] ]
%                                     [.Voice ] ] ]
%                             [.T ] ]
%                         [.\node(neg){Neg\\\textit{=mA}}; ] ]
%                     [.Foc ] ]
%         ];
%     \end{tikzpicture}
%     \caption{Placing Negation at a higher hierarchical level}
%     \label{fig:neg}
% \end{figure}




