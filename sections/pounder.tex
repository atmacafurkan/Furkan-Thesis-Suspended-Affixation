\subsection{German}

In \cite{pounder2006broken}, Pounder presents some example configurations in German for ellipsis-like morphological phenomena. These phenomena, called morphological brachylogy in the paper, include SA, conjunction reduction, and shared bases in German. The paper puts a higher emphasis on a diachronic difference in SA of suffixes. Pounder claims that these ellipsis-like processes can be employed in many levels of grammar, the inflectional paradigm, word-formations, and compounding to name a few. While the paper itself provides and lays out a nice presentation of data, this summary revolves around brachylogy of affixes that I refer to as SA for consistency.

I reiterate one of Pounder's examples before moving on with examples of SA in German. In the example (\ref{pounderex1}), the two conjuncts are prefixed verbs, both of which share the same base. The shared base is a verb and the prefixes are conjoined in interpretation. A dash `-' is used to indicate that there is a missing piece in the word. 

\begin{exe}
    \ex \label{pounderex1}
    \begin{xlist}
        \ex \label{pounderex1a} 
        \gll 
        \textit{werde...} \textit{nicht} \textit{re-,} \textit{sondern} \textit{ent-sozialisier-t} \\ be... {\Neg} {\Pref}- but\_rather {\Pref}-socialize-{\Part} \\
        \glt `be... not socialized but rather desocialized'
    
        \ex \label{pounderex1b}
        \gll 
        \textit{nicht} \textit{re-sozialisier-t} \textit{sondern} \textit{ent-sozialisier-t} \\ {\Neg} {\Pref}-socialize-{\Part} but {\Pref}-socialize-{\Part} \\
        \glt `not resocialised but rather desocialized'\\*
        \hfill Adapted from \cite{pounder2006broken}
    \end{xlist}
\end{exe}

Pounder dubs what is left after the elision of the morphological part as `fragment' whereas what is elided or reconstructed is called `recuperand', and the form that the language user infers the recuperand from is called `target'. For example, in (\ref{pounderex1a}) the fragment is the prefix \textit{re-}, the recuperand is \textit{sozialisiert}, and the target is \textit{sozialisiert}. (\ref{recuperand}) shows an example from Turkish. In this example the fragment is a noun \textit{kitap} `book', the recuperand is {\Acc}, and the target is \textit{kalem-i} (pencil-{\Acc}) `the pencil'.

\begin{exe}
    \ex \label{recuperand} 
    \gll kitap ve kalem-i al-dı-m. \\
    book {\And} pencil-{\Acc} take-{\Pst}-{\Fsg} \\ 
    \glt `I took the book and the pencil'
\end{exe}

I reiterate another example from Pounder in (\ref{pounderSA}) for the example of SA and I provide a mirroring example from Turkish.

\begin{exe}
    \ex \label{pounderSA}
    \begin{xlist}
        \ex \label{germanderSA}
        \gll freund- oder feind-schaft-lich-e Beziehungen \\ 
        friend- {\Or} enemy-{\Der}-{\Der}-{\Pl} relations \\
        \glt `with relations of friendship or enmity' \\*
        \hfill Adapted from \cite{pounder2006broken}
        
        \ex \label{turkishderSA} 
        \gll dost veya düşman-lığ-ı bitir-en ilişki-ler \\ 
        friend {\Or} enemy-{\Der} end-{\Fp} relation-{\Pl} \\
        \glt `the relations that end friendship or enmity'
    \end{xlist}
\end{exe}

The expression in (\ref{germanderSA}) shows an instance of SA for the suffixes \textit{-schaft} and \textit{-lich}, both suffixes are derivational. I gave a similar configuration in (\ref{turkishderSA}) where there is SA of a derivational suffix \textit{-lIK} and {\Acc}. Pounder reports that this process in German has a phonological constraint and cites \cite{smith2000word}. 
(\ref{germannoSA}) shows a suffix that changes the make-up of a phonological word and it cannot be suspended.

\begin{exe}
    \ex \label{germannoSA}
    \gll *die Provenz-al- und Roman-isch-en Dichter \\ 
    the.{\Pl} Provence-{\Der} {\And} romance-{\Der}-{\Pl} poets \\
    \glt Intended `the Provençal and Romantic poets'\\*
    \hfill Adapted from \cite{pounder2006broken}
\end{exe}

In (\ref{germannoSA}) the suffix \textit{isch} begins with a vowel. Pounder cites \cite{booij1985coordination} in reporting that the vowel initial suffix leads to a mismatch between the phonological and morphological word. The paper, however, shows a historical contrast in the contemporal ungrammaticality of (\ref{germannoSA}) where SA exists in written form. Pounder claims that German standardization is behind the ungrammaticality of (\ref{germannoSA}) and provides some examples from 17$^{th}$ and 18$^{th}$ century German (\ref{germansimilarSA}).

\begin{exe}
    \ex \label{germansimilarSA}
    \begin{xlist}
        \ex \gll Absicht- und Regl-en \\ 
        intention- {\And} rule-{\Pl} \\
        \glt `Intensions and rules'
        
        \ex \label{germanbaseSA}
        \gll Geberd- und Bewegung-en \\ 
        gesture- {\And} movement-{\Pl} \\
        \glt `Gestures and movements'
        
        \ex \gll bey dorf- und stet-en \\ 
        by village- {\And} town-{\Pl}.{\Dat} \\
        \glt `In villages and towns'\\*
        \hfill Adapted from \cite{pounder2006broken}
    \end{xlist}
\end{exe}

There is an important point to make in (\ref{germanbaseSA}). Pounder notes that the fragment \textit{Geberd-} is not the base modified counter part of \textit{Geb\"{a}rden}. In the suspended version no umlaut takes place. This shows that SA takes place before a phonological operation like umlaut.
The example (\ref{turkishbaseSA}) shows an example of base modification in Turkish. {\Fsg} pronoun goes under base modification from \textit{ben} to \textit{ban} when it is marked for {\Dat}. SA is not felicitous with both bases.

\begin{exe}
    \ex \label{turkishbaseSA}
    \begin{xlist}
        \ex \gll *Ban ve Ahmet-e bak-tı. \\ 
        {\Fsg} {\And} Ahmet-{\Dat} look-{\Pst}[{\Third}{\Sg}] \\
    
        \ex \gll *Ben ve Ahmet-e bak-tı. \\
        {\Fsg} {\And} Ahmet-{\Dat} look-{\Pst}[{\Third}{\Sg}] \\
        
        \ex \gll Ban-a ve Ahmet-e bak-tı. \\ 
        {\Fsg}-{\Dat} {\And} Ahmet-{\Dat} look-{\Pst}\\
        \glt `(S/he) looked at me and Ahmet'
    \end{xlist}
\end{exe}

In the German example (\ref{germanbaseSA}) the reconstruction of the fragment and the recuperand is at a more abstract level than phonology since there is no umlaut in the first conjunct. In the Turkish example (\ref{turkishbaseSA}), the reconstruction of the fragment and the recuperand cannot override an expected base modification in the fragment, or even further SA cannot be carried out at all with base modified fragments. \cite{pounder2006broken} goes on to interrogate the formulation of conjunction where SA takes place unlike the literature in Turkish.































