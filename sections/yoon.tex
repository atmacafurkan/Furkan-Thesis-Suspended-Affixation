\subsection{Korean}

Another language that hosts similar phenomena like SA is Korean. Korean could be considered to be typologically closer to Turkish than the other languages German, Mari, and Serbian we have seen so far. On SA-like structures in Korean, \cite{yoon2005conjunction}, and \cite{yoon2017lexical} provide a good set of data and some contrasts. In the following paragraphs I try to give the relevant summary of the two papers.

\cite{yoon2005conjunction} presents two conjunction types in Korean, that differ in how their conjuncts are formed. In the first, the conjoiner suffix \textit{-kwa} ({\Conj} in glosses) conjoins two conjuncts, out of two only the second can be marked for case. A mirrorring morphological form to this conjoiner could be the cliticized \textit{ile} in Turkish. I give an example in (\ref{koreanturkish}).

\begin{exe}
    \ex \label{koreanturkish}
    \begin{xlist}
        \ex
        \gll 
        \textit{John-kwa} \textit{Mary-ka} \textit{cip-ey} \textit{ka-ss-ta} \\ John-{\Conj} Mary-{\Nom} home-{\Loc} go-{\Pst}-{\Decl} \\
        ${}$ \hfill Adapted from \cite{yoon2005conjunction}

        \ex 
        \gll 
        \textit{John=la} \textit{Mary} \textit{ev-e} \textit{git-ti-(ler)} \\ John={\And} Mary[{\Nom}] home-{\Dat} go-{\Pst}-(3.{\Pl}) \\
        \glt `John and Mary went home'
    \end{xlist}
    ${}$ \hfill Adapted from \cite{yoon2005conjunction}
\end{exe}

The second type of conjoiner is the free form \textit{kuliko}, again for the sake of the argument can be mirrored by \textit{ve} in Turkish (\ref{koreanturkish2}).

\begin{exe}
    \ex \label{koreanturkish2}
    \begin{xlist}
        \ex 
        \gll 
        \textit{John-i} \textit{kuliko} \textit{Mary-ka} \textit{cip-ey} \textit{ka-ss-ta} \\ John-{\Nom} {\And} Mary-{\Nom} home-{\Loc} go-{\Pst}-{\Decl} \\
        ${}$ \hfill Adapted from \cite{yoon2005conjunction}
        
        \ex 
        \gll 
        \textit{John} \textit{ve} \textit{Mary} \textit{ev-e} \textit{git-ti-(ler)} \\ John[?{\Nom}] {\And} Mary[{\Nom}] home-{\Dat} go-{\Pst}-(3.{\Pl}) \\
        \glt `John and Mary went home'
    \end{xlist}
    ${}$ \hfill Adapted from \cite{yoon2005conjunction}
\end{exe}

The two different conjoiners show differences in interpretations. The reading differences lie in distributive or non-distributive readings, compatibility with collective modifiers, and compatibility with collective predicates. An example for the order of readings for both conjuncts is given in (\ref{koreanconjoiners}).

\begin{exe}
    \ex \label{koreanconjoiners}
    \begin{xlist}
        \ex \label{koreanconjoiner1} 
        \gll 
        \textit{John-kwa} \textit{Mary-ka} \textit{ochen-pwul-ul} \textit{pelessta} \\ John-{\Conj} Mary-{\Nom} 5000-dollars-{\Acc} made \\
        
        \ex \label{koreanconjoiner2}
        \gll 
        \textit{John-i} \textit{kuliko} \textit{Mary-ka} \textit{ochen-pwul-ul} \textit{pelessta} \\ John-{\Nom} {\And} Mary-{\Nom} 5000-dollars-{\Acc} made \\
        \glt Reading 1: John and Mary each made \$5000 \\
        Reading 2: John and Mary together made \$5000 \\
        (\ref{koreanconjoiner1}): Reading 2\textgreater Reading 1 (\ref{koreanconjoiner2}): Reading 1\textgreater Reading 2
    \end{xlist}
    ${}$ \hfill Adapted from \cite{yoon2005conjunction}
\end{exe}

While this preference for readings are different in both conjoiners, it does not mean that the conjoiner \textit{-kwa} is incompatible with distributive readings. (\ref{kwadistributed}) shows a distributive reading for \textit{-kwa}. Another observation, that Yoon et al. makes is that the conjoiner \textit{kuliko} is incompatible with collective readings (\ref{kulikocollective}).

\begin{exe}
    \ex \begin{xlist}
    \ex \label{kwadistributed}
    \gll 
    \textit{John-kwa} \textit{Mary-ka} \textit{kakkak} \textit{cip-ey} \textit{ka-ss-ta} \\ John-{\Conj} Mary-{\Nom} each home-{\Loc} go-{\Pst}-{\Decl} \\
    \glt `John and Mary each went home'
    
    \ex \label{kulikocollective}
    \gll 
    \textit{*?Cheli-ka} \textit{kuliko} \textit{Yenghi-ka} \textit{chayksang-ul} \textit{hamkkey} \textit{mantul-ess-eyo} \\
    Cheli-{\Nom} {\And} Yenghi-{\Nom} desk-{\Acc} together make-{\Pst}-{\Decl} \\
    \glt Intended: `Chelswu and Yenghi made a desk together'
    \end{xlist}
    ${}$ \hfill Adapted from \cite{yoon2005conjunction}
 \end{exe}

The relation that the two conjoiners hold with respect to SA is that the conjoiner \textit{-kwa} seems to be triggering an instance of case SA, and the conjoiner \textit{kuliko} does not. Yoon et al. shows a distinction between the two conjoiners deeming \textit{-kwa} as a conjoiner for phrase levels and \textit{kuliko} as a conjoiner for clauses. These observations made so far about Korean conjoiners \textit{-kwa} and \textit{kuliko} show the importance of analyzing conjunction structure. From that analysis a point for SA as structural sharing or ellipsis process can be drawn. 

While \cite{yoon2005conjunction} provides some data and analysis for two conjoiners in Korean, \cite{yoon2017lexical} is focused on SA directly. Yoon presents derivational Korean suffixes that derive verbs or adjectives from nominal bases. These suffixes display a clear cut difference in allowing SA. In providing SA-independent contrasts between these suffixes, Yoon presents some examples with Lexical Integrity tests of conjoined base, modifying the base, and gapping/ellipsis of the suffix. In expressing the difference between two suffix groups, Yoon uses the following terms Transparent suffix, and Opaque suffix. These two terms represent a suffix's ability to be either treated as transparent and visible in morphological or syntactic derivations, or treated as opaque and non-compositional. Overt examples showing the contrast of these two suffix groups are given in (\ref{koreanlexical})


\begin{exe}
    \ex \label{koreanlexical}
    \begin{xlist}
    \exi{Conjoined base}
    \ex \label{koreanconjoined1}
    \gll
    \textit{*[Kunul-kwa} \textit{kilum]-ci-n} \textit{ku} \textit{kos} \\ shade-{\Conj} oil-CHARACTERIZED-REL that place \\
    \glt `That plot of land, which is shaded and fertile'
    
    \ex \label{koreanconjoined2}
    \gll 
    \textit{Ku-nun} \textit{[yongkamha-n} \textit{kwunin-kwa} \textit{cincengha-n} \textit{aykwukca]-taw-ass-ta}\\
    3.{\Sg}-{\Top} courageous-REL soldier-{\Conj} genuine-REL patriot-BE.LIKE-{\Pst}-{\Decl} \\
    \glt `He really lived up to his reputation as a courageous soldier and true patriot'
 
    \exi{Modified base}
    
    \ex \label{koreanmodified1}
    \gll 
    \textit{Cenyek-ey-nun} \textit{*[etwuw-un} \textit{kunul]-ci-nun} \textit{kos} \\ dusk-{\Loc}-{\Top} dark-REL shade-CHARACTERIZED.BY-REL place\\
    \glt `A place that gets dark at dusk'
    
    \ex \label{koreanmodified2}
    \gll 
    \textit{Ku-nun} \textit{[hwullyungha-n} \textit{hakca]-tap-key} \textit{yenkwu-lul} \textit{swi-ci} \textit{anh-nunta} \\ 3.{\Sg}-{\Top} outstanding-REL scholar-BE.LIKE-COMP research-{\Acc} stop-COMP NEG-PRS \\
    \glt `He never stops dping research, as befits his reputation as an outstanding scholar'
    
    \exi{Gapping/Ellipsis}
    
    \ex \label{koreangapping1}
    \gll 
    \textit{*Ku} \textit{kos-un}  \textit{kilum-\_} \textit{kuliko} \textit{i} \textit{kos-un} \textit{kunul-ci-ta} \\ that place-{\Top} oil-\_ {\And} this place-{\Top} shade-CHARACTERIZED.BY-{\Decl} \\
    \glt Intended `That place is fertile while this place is shady'
    
    \ex \label{koreangapping2}
    \gll 
    \textit{Cheli-nun} \textit{kwunin-\_} \textit{kuliko} \textit{Tongswu-nun} \textit{haksayng-tap-ta.} \\ Cheli-{\Top} soldier {\And} Tongswu-{\Top} student-BE.LIKE-{\Decl} \\
    \glt `Cheli is every bit a soldier and Tongswu (every bit) a student.'
    \end{xlist}
    ${}$ \hfill Adapted from \cite{yoon2017lexical}
\end{exe}

While (\ref{koreanlexical}) shows a clear distinction in the tests, a suffix does not always behave the same. For example in (\ref{tapdifferent}), the suffix \textit{-tap} behaves like \textit{-ci} in not allowing modification of base . Yoon dubs this category of suffixes as Double-duty suffix.
\begin{exe}
    \ex \label{tapdifferent}
    \gll 
    \textit{*[Ceng-kwa} \textit{alum]-taw-un} \textit{sa.i} \\ affection-{\Conj} beautiful-BE.LIKE-REL relation \\ 
    \glt `Close and beautiful'
\end{exe}

The behaviours of suffixes in (\ref{koreanlexical}) show that derivational suffixes can have different responses to structural configurations. An observation that can prove useful for identifying why, if any, some derivational suffixes in Turkish can take part in SA and some not. Yoon, after further tests and contrasts, provides a table indicating the different category of derivations, a short version of it is given in Table \ref{tab:korean}.

\begin{table}[hbt!]
    \caption{Response of Different Category Suffixes in Korean to Lexical Integrity Tests}
    \centering
    \begin{tabular}{|p{1.8cm}|p{2.1cm}|p{1.7cm}|p{1.6cm}|p{1.6cm}|p{2cm}|p{1.8cm}|}
    \hline 
                        & Coordination & External Modifiers & Gapping (Base) & Gapping (Suffix) & Inbound Ana Island & Extraction \\ \hline 
    Opaque Suffix       & N             & N                 & N             & N                 & N   & N \\ \hline 
    Transparent Suffix  & Y             & Y                 & N             & Y                 & Y   & N \\ \hline 
    Double-duty Suffix  & N/Y           & N/Y               & N             & N/Y               & N/Y & N/Y \\ \hline 
    \end{tabular}
    \label{tab:korean}
\end{table}

The important observations that we can draw from Yoon is that not all derivations are equally representable as one sub-syntactic and opaque process. Also, even the ones that usually have transparent relations with syntax and syntactic operations do not always behave the same. 

In putting these groups of suffixes into a theoretical framework, Yoon makes use of Word-internal phases, citing \cite{marantz2007phases} within DM. The explanation Yoon provides boils down to these suffix categories belonging to different word derivation bases. Opaque suffixes combine with the $\sqrt{ROOT}$ assigning the category and take place in the first phase of word derivation. Transparent suffixes combine with category assigned words and take place in the second phase of word derivation. Both phases are shown in Figure \ref{fig:devphases}.

\begin{figure}[hbt!]
    \centering
    \begin{forest}
    for tree={s sep=15mm, inner sep=0}
        [YP
            [XP, name=XP
                [X^0, name=X0 
                    [$\sqrt{ROOT}$]
                    [{Opaque Suffix}]]
                [{Transparent Suffix}]]
            [Y]]
    \node[above left=1em and 0.25em of X0](x1){\small 1^{st}phase};
    \node[right=0.5em of X0](x2){};
    \draw[overlay, thick] (x1) to[out=0, in=90] (x2);
    \node[above left=1em and 0.25em of XP](xp1){\small 2^{nd}phase};
    \node[right=0.5em of XP](xp2){};
    \draw[overlay, thick] (xp1) to[out=0, in=90] (xp2);
    \end{forest}
    \caption{Root internal phase in word-derivation}
    \label{fig:devphases}
\end{figure}

Figure \ref{fig:devphases} however, does not mean that an opaque suffix always culminates the first phase. According to Yoon, there could be several suffixes that could form a new Root from a base Root without category assignment as in Figure (\ref{fig:firstdevphase}).

\begin{figure}[hbt!]
    \centering
    \begin{forest}
        [$\sqrt{ROOT}^3$
            [$\sqrt{ROOT}^2$
                [$\sqrt{ROOT}$]
                [{suffix}]]
            [{suffix}]]
    \end{forest}
    \caption{Derived Roots from Root bases in first word derivation phase}
    \label{fig:firstdevphase}
\end{figure}


The explanation that Yoon provides, and as the Figure \ref{fig:devphases} shows is that Opaque suffixes combine with category-less $\sqrt{ROOT}$s. That's why, even though the operation itself is similar to syntactic merge, the internal structure of these suffixes are not visible to syntactic operations. On the other hand, transparent suffixes combine with bases that are morphological words with syntactic categories and that's why they are visible to operations like base modification, conjoined base and SA. This explanation can be utilized in explaining why bare verbs are not morphological words and why SA can not take place with bare verb remnants in Turkish.