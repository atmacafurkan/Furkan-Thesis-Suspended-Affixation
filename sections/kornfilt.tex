\subsection{\cite{kornfilt2012revisiting}} \label{kornfilt}
In her 2012 paper, Kornfilt reiterates points in \cite{kornfilt1996some}. Mainly that SA is a syntactic operation much like gapping or ellipsis, that can only target syntactic categories, and she gives her account of RNR (Right Node Raising) to account for SA. She claims that a suffix can be suspended only if it has a syntactic projection. In this way she predicts to posit functional heads like Plural (PlurP), Case (KP), and Possession (PossP) since all three can have SA distinctly ((\ref{heads}), Figure \ref{fig:kornfilt}).
\begin{exe}
    \ex \label{heads}
    \begin{xlist}
        \ex \label{heads1}
        \gll 
        \textit{Kitap} \textit{ve} \textit{defter-ler} \\ book {\And} notebook-{\Pl} \\
        \glt Reading1: `Books and notebooks' \\ Reading2: `A book and notebooks'
        
        \ex \label{heads2}
        \gll 
        \textit{Kitap} \textit{ve} \textit{defter-i} \textit{al-dı-m.} \\ book {\And} notebook-{\Acc} buy-{\Pst}-{\First}{\Sg} \\
        \glt `I bought the book and the notebook.'
        
        \ex
        \gll
        \textit{Kitap} \textit{ve} \textit{defter-im} \textit{nerede?} \\ book {\And} notebook-{\First}{\Sg}.{\Poss} where \\
        \glt Reading1: `Where is the book and my notebook?' \\ Reading2: `Where is my book and notebook?'
    \end{xlist}
\end{exe}

\begin{figure}[hbt!]
    \centering
    \begin{forest}
        [ConjP, s sep= 30mm 
            [Conj' 
                [XP_1 
                    [YP]
                    [X_1, name=x1]]
                [Conj]
                [XP_2 
                    [YP]
                    [X_2, name=x2]]]
            [X, name=x3]]
\draw[rounded corners=1em, ->] (x1.south) -- ++(south:2em) -| (x3.south);
\draw[rounded corners=1em, ->] (x2.south) -- ++(south:1em) -| (x3.south);
    \end{forest}
    \caption{RNR proposal for SA}
    \label{fig:kornfilt}
\end{figure}

This analysis is also the same analysis that Kornfilt provides for backwards ellipsis for a sentence like (\ref{backwardellipsis}) as in Figure \ref{fig:backwardellipsis}.

\begin{exe}
    \ex \label{backwardellipsis}
    \gll
    \textit{Ahmet} \textit{al-dı} \textit{ve} \textit{Mehmet} \textit{sat-tı} \textit{kitab-ı.} \\ Ahmet[{\Nom}] buy-{\Pst}[{\Third}{\Sg}] {\And} Mehmet[{\Nom}] sell-{\Pst}[{\Third}{\Sg}] book-{\Acc} \\
    \glt `Ahmet bought and Mehmet sold the book.'
\end{exe}

\begin{figure}[hbt!]
    \centering
    \begin{forest}
        [ConjP, s sep=30mm 
            [Conj' 
                [TP 
                    [DP\\\textit{Ahmet}_i]
                    [T'
                        [VoiceP 
                            [\sout{DP}_i]
                            [Voice' 
                                [VP 
                                    [\sout{DP}, name=tk]
                                    [V\\\textit{al}]]
                                [Voice]]]
                        [T\\\textit{-dı}]]]
                [Conj\\\textit{ve}]
                [TP 
                    [DP\\\textit{Mehmet}_j]
                    [T' 
                        [VoiceP 
                            [\sout{DP}_j] 
                            [Voice' 
                                [VP 
                                    [\sout{DP}, name=tl]
                                    [V\\\textit{sat}]]
                                [Voice]]]
                        [T\\\textit{-tı}]]]]
            [DP\\\textit{kitab-ı}, name=DP ]]
        \draw[rounded corners=1em, ->] (tk.south) -- ++(south:2.5em) -| (DP.south);
        \draw[rounded corners=1em, ->] (tl.south) -- ++(south:1.5em) -| (DP.south);
    \end{forest}
    \caption{RNR analysis for Backward Ellipsis}
    \label{fig:backwardellipsis}
\end{figure}

This way Kornfilt regards SA as another ellipsis process operating on projection heads instead of phrases.