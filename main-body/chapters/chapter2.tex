\chapter{\MakeUppercase{literature survey}} \label{litsurvey}


\cite{geoffrey1967turkish} is credited with the term SA and first observations of it in Turkish. A closer look in the literature written in Turkish reveals that the phenomenon is observed and even addressed before \citep{emre1945turk,gencan1966dilbilgisi}.

\citet[par 219]{emre1945turk} observes that copular forms of \textit{-DI} and \textit{-mIş} can be suspended together with the agreement markers. He does not recognize the sole suspendability of the agreement markers. One difference in his observation is that not performing suspended affixation is interpreted as an emphasis on the conjuncts. This is a point which has not been raised since in the literature of SA, the effect of SA in interpretation. He does not provide examples of SA in the nominal domain. \citet[par 382]{gencan1966dilbilgisi} gives similar examples of SA in the verbal domain, however he includes examples of SA in the nominal domain and almost all possible suffixes are represented. He claims that performing SA creates discrepancies in the sentence so avoiding them makes the contrasts in a sentence more apparent.  

Categorized as an agglutinating language with many inflectional and derivational functions represented mostly by distinct morphemes, Turkish has many affixes that can be suspended, both in nominal and verbal domains. Some articles that exclusively examine SA in Turkish are: \cite{orgun1995flat}, \cite{kabak2007turkish}, \cite{broadwell2008turkish}, \cite{kornfilt2012revisiting}, \cite{kharytonava2012word,kharytonava2012taming}, and \cite{akkucs2016suspended}.

Some other articles investigating SA in other languages are: \cite{erschler2012suspended}, and \cite{erschler2018suspended} for Ossetic, \cite{yoon2017lexical} for Korean, \cite{despic2017suspended} for Serbian, \cite{guseva2017postsyntactic} for Mari, and \cite{pounder2006broken} for German. These articles range from giving the relative data and its limitations to the structural accounts and predictions for SA. In this chapter I first summarise the literature regarding SA in Turkish, later I summarise the literature regarding SA in other languages. I finish the chapter by providing some accounts for conjunctions.


\section{Suspended affixation in Turkish}

\subsection{\cite{orgun1995flat}} \label{orgun}

Orgun provides an analysis of SA as a structural sharing process. The so called `suspended' affixes actually take a complex base as shown in Figure \ref{fig:orgun}.

\begin{figure}[hbt!]
    \centering
    \begin{tikzpicture}
    \Tree[.X
            [.X
                [.X ]
                [.Conj ]
                [.X ]]
            [.suffix ]
    ];
    \end{tikzpicture}
    \caption{Structural sharing analysis of \cite{orgun1995flat} for SA}
    \label{fig:orgun}
\end{figure}

With this analysis in mind, consider (\ref{tebrikler}). 
\begin{exe}
    \ex \label{tebrikler}
    \begin{xlist}
    \ex 
    \gll 
    \textit{tebrik-ler-im} \textit{ve} \textit{teşekkür-ler-im} \\ congrats-{\Pl}-{\First}{\Sg}.{\Poss} {\And} thanks-{\Pl}-{\First}{\Sg}.{\Poss} \\
    
    \ex \label{tebriklerug}
    \gll
    \textit{*tebrik-ler} \textit{ve} \textit{teşekkür-ler-im} \\ congrats-{\Pl} {\And} thanks-{\Pl}-{\First}{\Sg}.{\Poss} \\
    
    \ex \label{tebriklerg}
    \gll 
    \textit{tebrik} \textit{ve} \textit{teşekkür-ler-im} \\ congrats {\And} thanks-{\Pl}-{\First}{\Sg}.{\Poss} \\
    \glt `My congratulations and thanks'
    \end{xlist}
\end{exe}

In a string of NP-{\Pl}-{\Poss}, the suspension of only {\Poss} does not seem to be valid. So far we have seen that SA is a rightward bound process, and suspension of only {\Poss} does not violate this constraint. We have also seen that in a string of NP-{\Poss}, suspension of only {\Poss} is valid (\ref{possSA}).

\begin{exe}
    \ex \label{possSA}
    \begin{xlist}
        \ex 
        \gll 
        \textit{Kitab-ım} \textit{ve} \textit{kalem-im} \\ book-{\Poss}.1{\Sg} {\And} pencil-{\First}{\Sg}.{\Poss} \\
        
        \ex 
        \gll 
        \textit{Kitap} \textit{ve} \textit{kalem-im} \\ book {\And} pencil-{\First}{\Sg}.{\Poss} \\
        \glt `My book and my pencil'
    \end{xlist}
\end{exe}

Stemming from the examples in (\ref{tebrikler}), Orgun proposes that the suffixes {\Pl} and {\Poss} need to be placed on the same hierarchical level as a ternary branching (Figure \ref{fig:orgun2}). Since both suffixes are separately able to have a conjoined base, but when both come together suspension of only {\Poss} becomes impossible.

\begin{figure}[hbt!]
    \centering
    \begin{forest}
        [N
            [N]
            [{\Pl}]
            [{\Poss}]]
    \end{forest}
    \caption{Ternary branching analysis of \cite{orgun1995flat}}
    \label{fig:orgun2}
\end{figure}


Orgun provides a three-way ambiguity of an expression like \textit{it-ler-i} NP-{\Pl}-{\Poss} in Turkish for the support of ternary branching (\ref{itleri}). 
\begin{exe}
    \ex \label{itleri}
    \gll \textit{it-ler-i} \\ dog-{\Pl}-{\Poss} \\
    \glt `her/his dogs' \\ `their dog' \\ `their dogs' \\
    ${}$ \hfill Adapted from \cite{orgun1995flat}
\end{exe}

Reading in between the lines, I assume that Orgun takes {\Pl} and {\Poss} suffixes to form a strange relation such that they have a different interaction than any other suffix holds. In such a way that they may form, whenever adjacent, a complex head. In a sense what he is actually proposing is not a ternary branching but a complex head formation. A representation of this formulation reflected on the ungrammatical SA in (\ref{tebriklerug}) is given in Figure \ref{fig:furkan1}.

\begin{figure}[hbt!]
    \centering
    \begin{tikzpicture}
    \Tree[.*N
            [.N 
                [.N 
                    [.\textit{tebrik} ]
                    [.\textit{-ler} ] ]
                [.\textit{ve} ]
                [.\textit{teşekkür} ] ]    
            [ 
                [.\textit{-ler} ]
                [.\textit{im} ] ]
        ];
    \end{tikzpicture}
    \caption{{\Pl} and {\Poss} forming a complex head in ungrammatical SA}
    \label{fig:furkan1}
\end{figure}

There are possible problems resulting from the formulation of a complex head in Figure \ref{fig:furkan1}, such as unequivalent conjuncts for the conjunction, and the uninterpretable relation of the complex head \textit{-ler-im} with the noun \textit{tebrik-ler}. The same complex head, however, does not cause a problem for the grammatical SA in (\ref{tebriklerg}) as shown in Figure \ref{fig:furkan2}.

\begin{figure}[hbt!]
    \centering
    \begin{tikzpicture}
    \Tree[.N
            [.N 
                [.\textit{tebrik} ]
                [.\textit{ve} ]
                [.\textit{teşekkür} ] ]
            [ 
                [.\textit{-ler} ]
                [.\textit{-im} ] ]
    ];
    \end{tikzpicture}
    \caption{{\Pl} and {\Poss} forming a complex head in grammatical SA}
    \label{fig:furkan2}
\end{figure}

Figure \ref{fig:furkan2} has equivalent conjuncts and an interpretable relation between the complex suffix \textit{-ler-im} and the nouns \textit{tebrik} `congrats', and \textit{teşekkür} `thanks'. In fact the same three way ambiguity is achievable with this way for an expression like \textit{kedi ve köpek-ler-i} (\ref{tebrikler2}).

\begin{exe}
    \ex \label{tebrikler2}
    \gll 
    \textit{kedi} \textit{ve} \textit{köpek-ler-i} \\ congrats {\And} thanks-{\Pl}-{\Poss} \\
    \glt `his/ her cats and dogs' \\ `their cat and dog' \\ `their cats and dogs'
\end{exe}



Orgun goes on to show that ternary branching is needed for some morphological configurations to satisfy the minimal phonological size $\sigma\sigma$ constraint, citing \cite{ito1989notes}, together with \cite{orgun1992turkish}. While capturing this inseparable {\Pl} and {\Poss} suspension, the paper proposes a structural sharing analysis for SA and a ternary branching for {\Pl} and {\Poss} suffixes. Support for ternary branching in SA comes from somewhat unrelated phonological constraints in affixation of monosyllabic words, i.e. \textit{*do-m} [$\sigma$] `do-{\First}{\Sg}.{\Poss}', \textit{sol-üm} [$\sigma$-$\sigma$] `sol-{\First}{\Sg}.{\Poss}'. The ungrammatical SA in (\ref{tebrikler}) is not subject to such a constraint and the three way ambiguity of an expression like \textit{it-ler-i} `dog-{\Pl}-{\Poss}' is not convincing enough to propose ternary branching. In finalizing the observation that Orgun makes, I provided Figures \ref{fig:furkan1} and \ref{fig:furkan2} following the discussion and the examples provided in \cite{orgun1995flat} to paint a more comprehensible picture of his analysis.

% Orgun provides an analysis for SA of the Plural marker \textit{-lAr} and the Possessive marker. The analysis is a flat branching of the markers. See (\ref{flatbranching}) for a representation.
% \begin{exe}
%     \ex \label{flatbranching}
%     \begin{xlist}
%         \ex 
%         \gll 
%         \textit{tebrik-ler-im} \\ congrats-{\Pl}-{\First}{\Sg}.{\Poss} \\
%         \glt `My congratulations'
%         \begin{multicols}{2}
%         \sn \begin{tikzpicture}
%         \Tree[.*N
%                 [.N 
%                     [.tebrik\\congrats ]
%                     [.ler\\{\Pl} ] ] 
%                 [.im\\{\First}{\Sg}.{\Poss} ] 
%         ];
%         \end{tikzpicture}
        
%         \sn \begin{tikzpicture}
%         \Tree[.N 
%                 [.tebrik\\congrats ]
%                 [.ler\\{\Pl} ]
%                 [.im\\{\First}{\Sg}.{\Poss} ]
%         ];
%         \end{tikzpicture}
%         \end{multicols}

%     \end{xlist}
% \hfill Adapted from \cite{orgun1995flat}
% \end{exe}
% The analysis to SA, according to Orgun is that the plural and possessive markers form ternary branching, and should be suspended together. This is how he explains the ungrammaticality of (\ref{tebrikler}).
% \begin{exe}
%     \ex \label{tebrikler}
%     \gll
%     \textit{*tebrik-ler} \textit{ve} \textit{teşekkür-ler-im} \\ congrats-{\Pl} and thanks-{\Pl}-{\First}{\Sg}.{\Poss} \\
%     \glt Intended: `My congratulations and thanks'
% \end{exe}
% Since both of the affixes are of the same level, they need to be suspended together and that results in grammaticality (\ref{tesekkur}).
% \begin{exe}
%     \ex 
%     \gll 
%     \textit{tebrik} \textit{ve} \textit{teşekkür-ler-im} \\ congrats and thanks-{\Pl}-{\First}.{\Sg} \\ 
%     \glt `My congratulations and thanks'
% \end{exe}
% Orgun bases his argument on the three way ambiguity of an expression like \textit{it-ler-i} provided in (\ref{itleri}). By this analysis Orgun rejects the assumption of readings following from hierarchy. Since hierarchical structures should have allowed (\ref{tebrikler}) to be grammatical.


% The problem with this approach is that the only standing observations for Orgun's proposal are a three way ambiguity of (\ref{itleri}), and the difficulty of interpretation in the suspension of only Possessive when combined with non-suspended Plural. A solution to the three way ambiguity might be resolved by other machineries in morphology, it is not novel for Turkish to seemingly not have an overt exponent where we expect one yet the function of the exponent seems to be carried out somehow, e.g. non-overt Accusative, and missing compound markers in multi-level compounds. It is however important in identifying the issue of increased difficulty or even ungrammaticality in the suspension of Possessive when combined with non-suspended Plural.  

% This paper only deals with SA of plural and possessive markers in the nominal domain. The assumptions of dismissing a hierarchical representation for different readings are at best debatable. The analysis only puts forward the observation that suspension of only the possessive marker when used with a plural is not permitted. I see this as the only remark worth pursuing in the paper.

\subsection{\cite{kabak2007turkish}}
Among the papers discussing SA in Turkish, Kabak's paper seems to be the most extensive in terms of providing how SA can take shape in both verbal and nominal domains. The paper provides the conditions in which we can expect SA. The analysis of Kabak relies on the definition of a morphological word. Kabak claims that any inflectional morpheme can be suspended as long as the remainder is a morphological word. Kabak proposes the following:
\begin{itemize}
    \item Terminal suffix: \textit{A suffix that is allowed to appear at the end of a word, where further affixation is not obligatory.}
\end{itemize}
He claims that only terminal suffixes can be suspended. He posits that bare verbs are not morphological words in Turkish. He provides Table (\ref{tab:terminalmorphemes}) for Verbal terminal suffixes.
\begin{table}
\caption{Verbal Terminal Morphemes}
    \centering
    \begin{tabular}{|ll|}
    \hline 
                                    {(i) Agreement markers} &  \\ \hline
         \multirow{5}{20em}{(ii) Aspect/ Modality markers}  & {\Aor} \textit{-(I)r/(A)r} \\ 
                                                            & {\Prog} \textit{-Iyor} \\
                                                            & {\Fut} \textit{-(y)AcAK} \\
                                                            & {\Evi} \textit{-mIş} \\
                                                            & {\Nec} \textit{-mAlI} \\ \hline
        \multirow{2}{20em}{(iii) Converb markers}           & \textit{-(y)IncA} \\
                                                            & \textit{-(y)Ip} \\
    \hline                                                         
    \end{tabular}
    \label{tab:terminalmorphemes}
\begin{flushright}
    Adapted from \cite{kabak2007turkish}
\end{flushright}
\end{table}

If any suspension attempt is made with these morphemes, it is only permitted under the condition that what is left is a morphological word. Kabak classifies such clitics \textit{=mI}, and \textit{=DA} as non-terminal but also recognizes their ability to end an expression in Turkish (\ref{clitics}).

\begin{exe}
    \ex \label{clitics}
    \begin{xlist}
    \begin{multicols}{2}
        \ex
        \gll
        \textit{koş-tu-n} \textit{mu?} \\ run-{{\Pst}}-{\Second}{\Sg} =Q \\
        \glt `Did you run?'
        
        \ex 
        \gll
        \textit{ağla-mış-sın} \textit{da} \\ cry-{\Evi}-{\Second}{\Sg} =TOO \\ 
        \glt `It looks like you have cried also.'
    \end{multicols}
    \end{xlist}
    \hfill Adapted from \cite{kabak2007turkish}
\end{exe}
Kabak argues against \cite{kornfilt1996some}'s formulation for SA (\ref{kornfiltsa}) with two examples. According to \cite{kornfilt1996some}'s analysis only the copular forms and further inflectional morphemes can be suspended.
\begin{exe}
    \ex \label{kornfiltsa}
    [V$_{Participle}$ conjunction V$_{Participle}$] + V$_{Copula}$ + Inflectional Morphemes
\end{exe}

First, some forms that can be complements of copula are not participles and do not always give way to grammatical instances of SA. Although \cite{kornfilt1996some} does not define \textit{-DI} as a participle, it is still able to be a complement to a copular \textit{i}. Hence an expected SA in (\ref{kornfiltpart}) where the suspension is applied to the copula and not the participle.

\begin{exe}
    \ex \label{kornfiltpart}
    \begin{xlist}
        \ex
        \gll
        \textit{*O} \textit{yaz} \textit{Finike-ye} \textit{git-ti} \textit{ve} \textit{deniz-e} \textit{gir-di-y-di-k.} \\ that summer Finike-{\Dat} go-{{\Pst}} {\And} sea-{\Dat} enter-{{\Pst}}-{\Cop}-{{\Pst}}-{\First}{\Pl} \\
        
        \ex 
        \gll 
        \textit{*Ev-im-iz-i} \textit{sat-sa} \textit{ve} \textit{bir} \textit{dükkan} \textit{al-sa-y-dı-k.} \\ house-1-{\Pl}-{\Acc} sell-{\Cond} {\And} a shop buy-{\Cond}-{\Cop}-{{\Pst}}-{\First}{\Pl} \\
    \end{xlist}
\hfill Adapted from \cite{kabak2007turkish}
\end{exe}

Second, at least one suffix deemed participle, namely the necessitative marker \textit{-mAlI}, does not behave like a participle that can modify NPs unlike other participles like \textit{-mIş} and \textit{-(y)AcAK} (\ref{kabakmali}). It should be noted however, contra Kabak, not all participle forms can modify nouns for example \textit{-Iyor}, and despite a lack of modifying capability, \textit{-mAlI} acts as predicted by \cite{kornfilt1996some}.
\begin{exe}
    \ex \label{kabakmali}
    \gll 
    \textit{*çalış-malı} \textit{adam} \\ work-{\Nec} man \\
\end{exe}
% Kornfilt deems any inflection that is a complement of copula as participle, and Kabak points to this generalization being faulty in accurately predicting instances of SA in verbal domain.

Another point Kabak provides, contradicting Orgun, is the suspension of possessive when coming after a plural marker (\ref{plurposs}).

\begin{exe}
    \ex \label{plurposs}
    \gll
    \textit{Asker-ler} \textit{ve} \textit{komutan-lar-ımız.} \\ soldier-{\Pl} {\And} commander-{\Pl}-{\First}{\Pl}.{\Poss} \\
    \glt `Our soldiers and commanders'
    \hfill Adapted from \cite{kabak2007turkish}
\end{exe}

Kabak provides an approach that is rather interesting. He makes an observation from \cite{good2005morphosyntax}, in the spirit of \cite{erdal2000clitics}, about the agreement paradigms in Turkish. In his citing, Kabak says that z-paradigm of agreement markers contain cliticized forms of words, and k-paradigm of agreement has lexical suffixes, thereby explaining the ungrammaticalities in (\ref{kornfiltpart}). Kabak also realizes the shortcomings of this approach and notes there are constructions in which k-paradigm SA is applicable and other conditions where z-paradigm SA is not applicable. As a last summary of Kabak's observations, he gives the following points in verbal domain SA:
\begin{exe}
\sn \begin{xlisti}
    \ex the ability of a verbal morpheme to terminate a word is related to its ability to stand without an agreement marker
    \ex SA is only applicable if what is left after suspension is a morphological word, and both the conjuncts end with terminal morphemes
    \ex Conjuncts with cliticlike endings are interpreted as 3$^{rd}$ person singular, causing agreement mismatches in SA
    \ex Nonfinal conjunct's terminal suffix must be overt
    \end{xlisti}
    \hfill Adapted from \cite{kabak2007turkish}
\end{exe}

Kabak recognizes that in SA what is relevant is actually the size of what is left after SA. Not on the lower end of the size, but also at the higher end. However the `cliticlike' condition on his third point is not clear-cut, and can be extended to other suffixes which have 3$^{rd}$ person singular suffixes which allow SA, which seemingly can end a word without copula. \textit{-mIş}, \textit{-(y)AcAK}, and \textit{-Iyor} to name a few. This condition relies heavily on what is `cliticlike' and gives way to wrong predictions in SA. His explanations are not based on a theoretical framework, which makes some of the explanations arbitrary. 

\subsection{\cite{broadwell2008turkish}}
Broadwell provides a representation for SA using the tools of Lexical Functional Grammar (LFG henceforth). In this approach the two identical phrases form a new phrase in conjunction that has the same structural properties of its parts. After this point, the suspended affixes are added. The phonological exponent of the right edge conjunct and the suspended affixes are \textit{coinstantiated} as one word. Figure \ref{fig:lexicalshare} illustrates the structural representation for the SA of {\Pl-\Poss} in (\ref{tebrikler3}).

\begin{exe}
    \ex \label{tebrikler3}
    \gll 
    tebrik ve teşekkür-ler-im \\ congrats and thanks-{\Pl}-{\First}{\Sg} \\ 
    \glt `My congratulations and thanks'
\end{exe}
    
\begin{figure}[hbt!]
    \centering
\begin{tikzpicture}
    \Tree[.PossP
            [.PlurP 
                [.NP 
                    [.NP\\tebrik ]
                    [.Conj\\ve ]
                    [.\node(NP2){NP}; ] ]
                [.\node(plr){Plur}; ] ]
            [.\node(PS){Poss}; ]
]
\node[below right= 1em of NP2, draw](co){teşekkür-ler-im};
\draw[thick, ->] (NP2) -- (co);
\draw[thick, ->] (plr) -- (co);
\draw[thick, ->] (PS) -- (co);
\end{tikzpicture}
    \caption{Lexical sharing analysis of {\Pl} and {\Poss} in SA}
    \label{fig:lexicalshare}
\end{figure}

Broadwell claims that this way of representation for SA saves us from three things:
\begin{itemize}
    \item  interpreting affixes that can suspend as clitics
    \item positing conjunction in the lexicon
    \item having special annotation for the rightmost conjunct
\end{itemize}

An important point which Broadwell makes is that Turkish is relatively productive in SA, but it also makes distinctions that cannot be addressed with a purely lexical approach. It might be posited that SA is only permitted with affixes that can attach to conjoined phrases. This analysis however does not explain why the suspension of {\Poss} is ungrammatical in a string of {\Pl-\Poss} and does not explain how to categorize suffixes that can have conjoined bases, missing the morphological word requirement of SA in the verbal domain.


\subsection{\cite{kornfilt2012revisiting}} \label{kornfilt}
In her 2012 paper, Kornfilt reiterates points in \cite{kornfilt1996some}. Mainly that SA is a syntactic operation much like gapping or ellipsis, that can only target syntactic categories, and she gives her account of RNR (Right Node Raising) to account for SA. She claims that a suffix can be suspended only if it has a syntactic projection. In this way she predicts to posit functional heads like Plural (PlurP), Case (KP), and Possession (PossP) since all three can have SA distinctly ((\ref{heads}), Figure \ref{fig:kornfilt}).
\begin{exe}
    \ex \label{heads}
    \begin{xlist}
        \ex \label{heads1}
        \gll 
        \textit{Kitap} \textit{ve} \textit{defter-ler} \\ book {\And} notebook-{\Pl} \\
        \glt Reading1: `Books and notebooks' \\ Reading2: `A book and notebooks'
        
        \ex \label{heads2}
        \gll 
        \textit{Kitap} \textit{ve} \textit{defter-i} \textit{al-dı-m.} \\ book {\And} notebook-{\Acc} buy-{\Pst}-{\First}{\Sg} \\
        \glt `I bought the book and the notebook.'
        
        \ex
        \gll
        \textit{Kitap} \textit{ve} \textit{defter-im} \textit{nerede?} \\ book {\And} notebook-{\First}{\Sg}.{\Poss} where \\
        \glt Reading1: `Where is the book and my notebook?' \\ Reading2: `Where is my book and notebook?'
    \end{xlist}
\end{exe}

\begin{figure}[hbt!]
    \centering
    \begin{forest}
        [ConjP, s sep= 30mm 
            [Conj' 
                [XP_1 
                    [YP]
                    [X_1, name=x1]]
                [Conj]
                [XP_2 
                    [YP]
                    [X_2, name=x2]]]
            [X, name=x3]]
\draw[rounded corners=1em, ->] (x1.south) -- ++(south:2em) -| (x3.south);
\draw[rounded corners=1em, ->] (x2.south) -- ++(south:1em) -| (x3.south);
    \end{forest}
    \caption{RNR proposal for SA}
    \label{fig:kornfilt}
\end{figure}

This analysis is also the same analysis that Kornfilt provides for backwards ellipsis for a sentence like (\ref{backwardellipsis}) as in Figure \ref{fig:backwardellipsis}.

\begin{exe}
    \ex \label{backwardellipsis}
    \gll
    \textit{Ahmet} \textit{al-dı} \textit{ve} \textit{Mehmet} \textit{sat-tı} \textit{kitab-ı.} \\ Ahmet[{\Nom}] buy-{\Pst}[{\Third}{\Sg}] {\And} Mehmet[{\Nom}] sell-{\Pst}[{\Third}{\Sg}] book-{\Acc} \\
    \glt `Ahmet bought and Mehmet sold the book.'
\end{exe}

\begin{figure}[hbt!]
    \centering
    \begin{forest}
        [ConjP, s sep=30mm 
            [Conj' 
                [TP 
                    [DP\\\textit{Ahmet}_i]
                    [T'
                        [VoiceP 
                            [\sout{DP}_i]
                            [Voice' 
                                [VP 
                                    [\sout{DP}, name=tk]
                                    [V\\\textit{al}]]
                                [Voice]]]
                        [T\\\textit{-dı}]]]
                [Conj\\\textit{ve}]
                [TP 
                    [DP\\\textit{Mehmet}_j]
                    [T' 
                        [VoiceP 
                            [\sout{DP}_j] 
                            [Voice' 
                                [VP 
                                    [\sout{DP}, name=tl]
                                    [V\\\textit{sat}]]
                                [Voice]]]
                        [T\\\textit{-tı}]]]]
            [DP\\\textit{kitab-ı}, name=DP ]]
        \draw[rounded corners=1em, ->] (tk.south) -- ++(south:2.5em) -| (DP.south);
        \draw[rounded corners=1em, ->] (tl.south) -- ++(south:1.5em) -| (DP.south);
    \end{forest}
    \caption{RNR analysis for Backward Ellipsis}
    \label{fig:backwardellipsis}
\end{figure}

This way Kornfilt regards SA as another ellipsis process operating on projection heads instead of phrases.

\subsection{\cite{kharytonava2011morphology, kharytonava2012word,kharytonava2012taming}}

In all her papers, Kharytonava specifically inspects SA in Turkish noun compounds. For a start consider the noun compounds in (\ref{compound}).

\begin{exe}
    \ex \label{compound}
    \begin{xlist}
        \ex \gll Anne-m not defter-i-ni yıka-mış. \\
        mother-{\Fsg} note book-{\Poss}.{\Tsg}-{\Acc} wash-{\Prf}[{\Tsg}] \\
        \glt `It seems like my mother washed the notebook'
        
        \ex \gll Anne-m not defter-im-i yıka-mış \\
        mother-{\Fsg} note book-{\Poss}.{\Fsg}-{\Acc} wash-{\Prf}[{\Tsg}] \\
        \glt `It seems like my mother washed my notebook'
    \end{xlist}
\end{exe}

The default agreement marker is third person singular in Turkish when no possessor is present for the compound. The SA that Kharytonava presents comes into play in compounds with shared bases. (\ref{compoundSA}) shows an example where the shared base is \textit{doğum} `birth' and the markers on the conjoined nouns can be fully expressed (No SA) or can have two shapes of SA (partial-full).

\begin{exe}
    \ex \label{compoundSA}
    \begin{xlist}
        \ex No SA\\*
        \gll doğum yer-iniz ve tarih-iniz \\ 
        birth place-{\Spl} {\And} date-{\Spl} \\
        \glt ${}$
        
        \ex \label{compoundSAb} Full SA\\*
        \gll doğum yer ve tarih-iniz \\ 
        birth place {\And} date-{\Spl} \\
        \glt ${}$
        \ex \label{compoundSAc} Partial SA\\*
        \gll doğum yer-i ve tarih-iniz \\ 
        birth place-{\Tsg} {\And} date-{\Spl} \\
        \glt `Your birthplace and birthdate'\\*
        \hfill Adapted from \cite{kharytonava2012taming}
    \end{xlist}
\end{exe}

The possessive marker is suspended in (\ref{compoundSAb}) and there is no remnant of agreement whereas (\ref{compoundSAc}) leaves behind a possessor that is {\Tsg}. The interpretation of possessive for the second conjunct is still {\Ssg}. On the surface the existence of {\Poss}.{\Tsg} after SA for a {\Poss}.{\Ssg} seems problematic. Kharytonava addresses this not as a structural sharing analysis, she rather uses Impoverishment and Feature Geometry to explain such a configuration of SA. She indicates that features are monovalent for referring expressions in Turkish and exponent insertion is modulated by Subset Principle. Table \ref{tab:kharyfeatures} shows the feature geometry she provides for Turkish possessors with the corresponding exponents.

\begin{table}[hbt!]
    \caption{Feature Geometry of {\Poss} in Turkish}
    \centering
    \begin{tabular}{|l|l|l|}
    \hline
         \multicolumn{2}{|c|}{Features} & \multirow{2}{*}{Exponent}  \\ \cline{1-2}
         Participant & Individuation  & \\ \hline
         Speaker & $\emptyset$ & \textit{-Im} \\ \hline 
         Addressee & $\emptyset$ & \textit{-In} \\ \hline 
         Speaker & Group & \textit{-ImIz} \\ \hline 
         Addressee & Group & \textit{-InIz} \\ \hline 
         $\emptyset$ & $\emptyset$ & \textit{-(s)I(n)} \\ \hline 
         $\emptyset$ & Group & \textit{-lArI} \\ \hline 
    \end{tabular}
    \label{tab:kharyfeatures}
\end{table}

SA in noun compounds works by deletion of the features. See the feature templatic view of no SA in (\ref{compoundSA}). The feature set for \textsc{Addressee}-\textsc{Group}, by Subset Principle, is \textit{-InIz}. On this templatic view the features in the first conjunct instead of the exponent itself are deleted. This feature deletion results in the following templatic view and the exponent \textit{-(s)I(n)} is inserted after the first conjunct. \cite{kharytonava2011morphology} shows that Turkish speakers prefer the Partial SA in (\ref{compoundSA}) to the full SA. This type of analysis for SA falls under an ellipsis like analysis which has more appeal and makes better predictions about SA in noun compounds than structural sharing approaches.

\begin{itemize}
    \item $\alpha$-\textsc{Addressee}-\textsc{Group} {\And} $\beta$-{\textsc{Addressee}-\textsc{Group}}
    \item $\alpha$-$\emptyset$-$\emptyset$ {\And} $\beta$-{\textsc{Addressee}-\textsc{Group}}
\end{itemize}

Using this deletion analysis, instances like (\ref{impoverished}) can also be a deletion of the referential feature alongside the tense. {\Tsg} on verbal and nominal predicate domain is not expressed by an overt phonological exponent. The readings should have contrasted in their subject readings if this were to be the case. 

\begin{exe}
    \ex \label{impoverished}
    \begin{xlist}
        \ex \gll Ben hasta ve yorgun-du-m. \\
        {\Fsg}[{\Nom}] sick[{\Tsg}] {\And} tired-{\Pst}-{\Fsg} \\
        \glt `I was sick and tired'
        
        \ex \gll Ben ev-e gid-ecek ve gel-ecek-ti-m. \\
        {\Fsg}[{\Nom}] house-{\Dat} come-{\Fut}[{\Tsg}] {\And} come-{\Fut}-{\Pst}-{\Fsg} \\
        \glt `I was going to come home and go'
    \end{xlist}
\end{exe}



\subsection{\cite{akkucs2016suspended}}

Akkuş provides some examples for SA in derivational suffixes. He argues that the existence of such examples is not numerous but not that rare. Her provides some examples like (\ref{akkusder}).

\begin{exe}
\ex \label{akkusder}
\begin{xlist}
    \ex \gll \ldots yedi ve yirmi-nci bölüm-ler \ldots \\ 
    \ldots seven {\And} twenty-{\Der} episode-{\Pl} \ldots \\
    \glt `\ldots seventh and twentieth episodes \ldots'
    
    \ex \gll \ldots beş lira ve on dolar-lık banknot-lar \ldots \\
    \ldots five lira {\And} ten dollar-{\Der} banknote-{\Pl} \ldots \\
    \glt `[five lira and ten dollar] worth banknotes'
    
    \ex \gll \ldots Deprem ve Afet-zede An-ma Yürü-yüş-ü \ldots \\ \ldots earthquake {\And} disaster-{\Der} remember-{\Nmlz} walk-{\Nmlz}-{\Acc} \ldots \\ 
    \glt `[Earthquake and Disaster] Victims Remembrance March'
    
    \ex \gll \ldots dost ve arkadaş-ça bir hava \ldots \\ \ldots fellow {\And} friend-{\Der} {\Det} air \ldots \\
    \glt Lit:`a [friend and fellow]-like armosphere' \\ Mean: `a friendly and amiable atmosphere'\\*
    \hfill Adapted from \cite{akkucs2016suspended}
\end{xlist}
\end{exe}

Akkuş argues that a natural coordination explanation \citep{walchli2005co} provided in \cite{kabak2007turkish} falls short of explaining instances of derivational SA. Akkuş reiterates examples from \cite{ackema2004beyond, lieber2006lexical} and points to two options for explaining derivational SA. First is what is provided in \cite{lieber2006lexical}, which suggests that morphology has access to the output of syntax. Second is what is provided in \cite{ackema2004beyond}, which suggests three modules in language, namely syntax, semantics, and phonology, that can have interactions with one another placing morphology within syntax. Both the approaches would allow for morphological elements to have complex bases for derivation or inflection.


\section{Interim summary of the literature}

The literature of SA for Turkish provides some valuable observations, that make it easier to navigate the problems and workings of SA in Turkish. They feature useful data, approaches like LFG, syntactic movements like RNR, and comprehensive coverage of morphological constraints in SA. In the following paragraphs, I summarise the points made in the literature about SA, and in what ways it can be improved. Then I put a finger on unaddressed issues.

\cite{orgun1995flat} puts forward an anomalous behaviour in the suspension of {\Poss} in a string of {\Pl-\Poss}. Orgun's solution is to hierarchically align the two for handling the problem of inseparable suspension of {\Poss}.

The observations of \cite{kabak2007turkish} indicate that the morphological size of what is left after suspension is crucial for a successful SA. Bare verbs are not considered as morphological words even though they are phonological words and get stress under negation \textit{-mA}. The observation of morphological word constraint in SA is quite important since some similar phenomenon of a backwards process in, for example, German only requires the remnant after suspension to be a phonological word \citep{smith2000word, pounder2006broken,kenesei2007semiwords}. Kabak's paper shows that SA might be possible with some derivational suffixes, yet he strongly suggests that the base for the derivational suffix is a compound like noun that uses a conjoiner for its parts.

\cite{broadwell2008turkish} entertains a different mode of operation for the analysis of SA. Rather than the suspended suffix originating in both conjuncts, the conjoined phrase is only merged with a single projection of the `suspended' suffix. Later, as a tool of LFG, the rightward elements coinstantiate as a single word of multiple exponents, appearing as though only the second conjunct has the suffix whereas structurally it is shared and the two conjuncts are at the same level of representation.

According to \cite{kornfilt1996some, kornfilt2012revisiting}, SA is a syntactic operation of RNR, and suspendable suffixes are projections in syntax. This defines a line for the capability of SA for derivational and inflectional suffixes. Her analysis does not explain why SA of {\Case} is not ambiguous, but the SA of {\Pl} or {\Poss} is. The importance of Kornfilt's proposal is the observation of the productivity of SA in the inflectional paradigm. This places an analysis of SA more in the structural side that should have access to syntactic inputs.

\cite{kharytonava2011morphology,kharytonava2012word,kharytonava2012taming} deviate from all the others in dealing with SA because they deal with a peculiar SA observed in noun compounds. The preference studies that Kharytonava have carried out suggest that partial SA conditions are preferred more than complete morpheme deletions. Unfortunately, the reporting of the studies are not very clear. Only percentages in terms of participant preferences are provided. Furthermore, some arbitrary schemes for grouping subjects by choice frequency are used to draw inferences from the responses being interpreted as grammatical or not. 

\cite{akkucs2016suspended} points to instances of derivational SA and argues that they need an explanation contra Kabak's view of natural coordination \citep{walchli2005co}. He argues for a revised understanding of what Lexical Integrity Hypothesis in the sense of either \cite{ackema2004beyond} or \cite{lieber2006lexical} in explaining instances of derivational SA. Akkuş's paper is the only paper that argues for a structural interpretation of SA in derivational suffixes. 

As a conclusion, the current literature for Turkish SA provides possible solutions and analyses for SA. The literature does not have a good standing when it comes to answering in what level of language derivation SA takes place. It is not pinpointed well enough to argue for or against any analysis, be it ellipsis or structural sharing. There is no exposure of different conjoiners and what they bring to SA. An analysis of should take the SA environment into consideration for a better understanding of the constraints that govern SA.


\section{Suspended affixation in other languages}

The focus and effort of this study is limited to SA in Turkish, but it is beneficial to take observations from other languages where similar SA-like phenomena exist. In the following subsections I provide summaries of such articles in a chronological order. \cite{pounder2006broken} shows examples of SA and conjunction reduction from German, \cite{guseva2017postsyntactic} shows examples of SA from Mari, \cite{despic2017suspended} shows an example of a certain Serbian clitic that mimics SA-like behaviour, \cite{yoon2017lexical} shows examples from Korean, and \cite{erschler2012suspended,erschler2018suspended} show examples from Ossetic.

\subsection{German}

In \cite{pounder2006broken}, Pounder presents some example configurations in German for ellipsis like morphological phenomena. These phenomena, called morphological brachylogy in the paper, include SA, conjunction reduction, and shared bases in German. The paper puts a higher emphasis on a diachronic difference in SA of suffixes. Pounder claims that these ellipsis like processes can be employed in many levels of grammar, the inflectional paradigm, word-formations, and compounding to name a few. While the paper itself provides and lays out a nice presentation of data, this summary revolves around brachylogy of affixes that I refer to as SA for consistency.

Before moving on with examples of SA in German, I reiterate one of Pounder's examples where a conjunction that has two prefixed verb conjuncts undergoes an ellipsis like process. In this example the two conjuncts are prefixed verbs, both of which share the same base. See (\ref{pounderex1}) for an instance of shared base for prefixes in a conjunction. 

\begin{exe}
    \ex \label{pounderex1}
    \begin{xlist}
        \ex \label{pounderex1a} 
        \gll 
        \textit{werde...} \textit{nicht} \textit{re-,} \textit{sondern} \textit{ent-sozialisier-t} \\ be... {\Neg} {\Pref}- but\_rather {\Pref}-socialize-{\Part} \\
        \glt `be... not socialized but rather desocialized'
    
        \ex \label{pounderex1b}
        \gll 
        \textit{nicht} \textit{re-sozialisier-t} \textit{sondern} \textit{ent-sozialisier-t} \\ {\Neg} {\Pref}-socialize-{\Part} but {\Pref}-socialize-{\Part} \\
        \glt `not resocialised but rather desocialized'
    \end{xlist}
\hfill Adapted from \cite{pounder2006broken}
\end{exe}

In the paper Pounder dubs what is left after the elision of the morphological part as `fragment' whereas what is elided or reconstructed is called `recuperand', and the form that the language user infers the recuperand from is called `target'. For example in (\ref{pounderex1a}) the fragment is the prefix \textit{re-}, the recuperand is \textit{sozialisiert}, and the target is \textit{sozialisiert}. In this instance recuperand and target might be the same but in some constructions the phonological forms might differ. The important thing is from the target you decide on the morphological element that you will reconstruct as the recuperand with the fragment. The observation that we can draw from the paper about this construction is that the fragment needs to be a phonological word, and Pounder cites \cite{smith2000word},  for this constraint. Although (\ref{pounderex1}) is not SA, it is important to point out that the fragment or the remnant after the elision like process goes by the phonological word instead of morphological. 

Now to come to the examples of SA in German, I reiterate yet another example from Pounder in (\ref{pounderSA}). I also provide a sort of mirroring example from Turkish.

\begin{exe}
    \ex \label{pounderSA}
    \begin{xlist}
        \ex \label{germanderSA}
        \gll 
        \textit{freund-} \textit{oder} \textit{feind-schaft-lich-e} \textit{Beziehungen} \\ friend- {\Or} enemy-{\Der}-{\Der}-{\Pl} relations \\
        \glt `with relations of friendship or enmity'
        
        \hfill Adapted from \cite{pounder2006broken}
        \ex \label{turkishderSA} 
        \gll 
        \textit{dost} \textit{veya} \textit{düşman-lığ-ı} \textit{bitir-en} \textit{ilişki-ler} \\ friend {\Or} enemy-{\Der} end-{\Fp} relation-{\Pl} \\
        \glt `the relations that end friendship or enmity'
    \end{xlist}
\end{exe}

The expression in (\ref{germanderSA}) shows an instance of SA for the suffixes \textit{-schaft} and \textit{-lich}. The first one is a derivational suffix and the second one is inflectional. I tried to mirror a similar configuration in (\ref{turkishderSA}) where there is SA of a derivational suffix \textit{-lIK} and an inflectional accusative \textit{-(y)I}. However Pounder reports that this process in German has a phonological constraint, namely a suffix beginning with a vowel can not take part in SA (\ref{germannoSA}).

\begin{exe}
    \ex \label{germannoSA}
    \gll 
    \textit{*die} \textit{Provenz-al-} \textit{und} \textit{Roman-isch-en} \textit{Dichter} \\ the.{\Pl} Provence-{\Der} {\And} romance-{\Der}-{\Pl} poets \\
    \glt Intended `the Provençal and Romantic poets'
    
    \hfill Adapted from \cite{pounder2006broken}
\end{exe}

In (\ref{germannoSA}) the suffix \textit{isch} begins with a vowel. Pounder cites \cite{booij1985coordination} in reporting that the vowel initial suffix leads to a mismatch between the phonological and morphological word. The paper, however, shows a historical contrast in the contemporal ungrammaticality of (\ref{germannoSA}) where SA exists in written form. Pounder claims that German standardization is behind the ungrammaticality of (\ref{germannoSA}) and provides some examples from 17$^{th}$ and 18$^{th}$ century German (\ref{germansimilarSA}).

\begin{exe}
    \ex \label{germansimilarSA}
    \begin{xlist}
        \ex 
        \gll
        \textit{Absicht-} \textit{und} \textit{Regl-en} \\ intention- {\And} rule-{\Pl} \\
        \glt `Intensions and rules'
        
        \ex \label{germanbaseSA}
        \gll 
        \textit{Geberd-} \textit{und} \textit{Bewegung-en} \\ gesture- {\And} movement-{\Pl} \\
        \glt `Gestures and movements'
        
        \ex 
        \gll
        \textit{bey} \textit{dorf-} \textit{und} \textit{stet-en} \\ by village- {\And} town-{\Pl}.{\Dat} \\
        \glt `In villages and towns'
    \end{xlist}
    \hfill Adapted from \cite{pounder2006broken}
\end{exe}

There is a very important point to make in these examples, particularly in (\ref{germanbaseSA}). Pounder notes, normally the singular \textit{Geberd-}undergoes umlaut and becomes \textit{Geb\"{a}rde} in pluralization, but in the suspended version no umlaut does not take place. (\ref{germansimilarSA}) shows that SA takes place before such phonological operations like umlaut.

Now consider the example in (\ref{turkishbaseSA}) where the first person singular pronoun undergoes base modification when used with Dative.

\begin{exe}
    \ex \label{turkishbaseSA}
    \begin{xlist}
        \ex
        \gll
        \textit{*Ban} \textit{ve} \textit{Ahmet-e} \textit{bak-tı.} \\ {\First}{\Sg} {\And} Ahmet-{\Dat} look-{\Pst}[{\Third}{\Sg}] \\
    
        \ex 
        \gll
        \textit{*Ben} \textit{ve} \textit{Ahmet-e} \textit{bak-tı.} \\ {\First}{\Sg} {\And} Ahmet-{\Dat} look-{\Pst}[{\Third}{\Sg}] \\
        
        \ex \gll 
        \textit{Ban-a} \textit{ve} \textit{Ahmet-e} \textit{bak-tı.} \\ {\First}{\Sg}-{\Dat} {\And} Ahmet-{\Dat} look-{\Pst}\\
        \glt `(S/he) looked at me and Ahmet'
    \end{xlist}
\end{exe}

In the German example (\ref{germanbaseSA}) the reconstruction of the fragment and the recuperand seems to be employed at a more abstract level than phonology since the umlaut, a base modification, in the first conjunct need not be carried out. In the Turkish example (\ref{turkishbaseSA}), we see that the reconstruction of the fragment and the recuperand can not override an expected base modification in the fragment, or even further we see that SA can not be carried out at all with base modified fragments.

The importance of \cite{pounder2006broken} is, unlike the literature in Turkish, Pounder goes on to interrogate the formulation of conjunction where SA takes place. The formulation of conjunction might define the nature of SA. If one takes the position and says conjunction is performed only at a clausal level, than SA is an ellipsis process. On the other hand, if one takes the position that conjunction can be performed at any phrase level XP then SA can be represented as a structural sharing of the suspended affixes by the conjuncts.


































\subsection{Mari}

Mari is an Eastern Uralic language that has a rather interesting set of data when it comes to SA. \cite{guseva2017postsyntactic} (GW henceforth) provide some examples and analysis for SA in Mari. In (\ref{mariSA}) I give examples of SA from Mari. Previous observations of SA have shown that it is a rightward-bound process, but the examples in (\ref{mariSA}) show SA that is not rightward-bound.

\begin{exe}
    \ex \label{mariSA}
    \begin{xlist}
        \ex SA of {\Iness}\\*
        \gll Üder mej-en u\u{s}e-m den tej-en süm-e\u{s}te-t. \\ 
        girl {\Fsg}-{\Gen} mind-{\Poss}.{\Fsg} {\And} {\Ssg}-{\Gen} heart-{\Iness}-{\Poss}.{\Ssg} \\
        \glt `The girl is in my mind and in your heart'
        
        \ex SA of {\Ill}\\*
        \gll Pjötr kart-em mej-en perde\u{z}-em den omsa-\u{s}ke-\u{z}e pi\u{z}ekta \\ 
        Peter map-{\Acc} {\Fsg}-{\Gen} door-{\Poss}.{\Fsg} {\And} wall-{\Ill}-{\Poss}.{\Tsg} pin.{\Tsg}.{\Prs}\\
        \glt `Peter pins maps to my door and his wall'
   
        \ex SA of {\Pl}-{\Iness}\\*
        \gll A-vlak tud-en sad-\u{s}e den memn-an pasu-vlak-e\u{s}te-na mod-et \\ 
        child-{\Pl} {\Tsg}-{\Gen} garden-{\Poss}.{\Tsg} {\And} {\Fpl}-{\Gen} field-{\Pl}-{\Iness}-{\Poss}.{\Fpl} play-{\Tpl}.{\Prs} \\
        \glt `The children are playing in his garden and in our fields'\\*
        \hfill Adapted from \cite{guseva2017postsyntactic}
    \end{xlist}     
\end{exe}

For a clear illustration of the SA examples in (\ref{mariSA}) I give the abstract representation of SA for each one of the examples in (\ref{maritemplate}).

\begin{exe}
    \ex \label{maritemplate}
        \begin{xlist}
        \ex N1-[{\Iness}]-{\Poss} conjoiner N2-{\Iness}-{\Poss}
        \ex N1-[{\Ill}]-{\Poss} conjoiner N2-{\Ill}-{\Poss}
        \ex N1-[{\Pl}-{\Iness}]-{\Poss} conjoiner N2-{\Pl}-{\Iness}-{\Poss}
    \end{xlist}
\end{exe}

This peculiar SA should not be taken as an evidence against its rightward-bound nature. In Mari, the order of the morphemes in the nominal domain show a relatively free order. The morphemes in question are {\Pl}, {\Poss}, Structural and Local cases ({\Scase} and {\Lcase} in glosses respectively). Table \ref{tab:mariorder} shows some possible orders of these morphemes. There is an optional positioning for the {\Poss} marker. The {\Poss} either occupies the left or the right edge of the morphemes, where the right edge can only build up to the {\Scase}. It is a barrier that {\Poss} cannot alternate to the right of.

\begin{table}[hbt!]
    \caption{Mari Nominal Domain Morpheme Order}
    \centering
    \begin{tabular}{|ll|}
    \hline 
        {\Pl} \textgreater {\Poss} & \textit{pasu-vlak-na}  \\
        {\Poss} \textgreater {\Pl} & \textit{pasu-na-vlak} \\ \hline
        {\Pl} \textgreater {\Lcase} & \textit{pasu-vlak-e\u{s}te} \\
        {\Pl} \textgreater {\Scase} & \textit{pasu-vlak-em} \\ \hline
        {\Lcase} \textgreater {\Poss} & \textit{pasu-\u{s}te-na} \\
        {\Poss} \textgreater {\Scase} & \textit{pasu-na-m} \\ \hline
        {\Pl} \textgreater {\Lcase} \textgreater {\Poss} & \textit{pasu-vlak-e\u{s}te-na} \\
        {\Poss} \textgreater {\Pl} \textgreater {\Lcase} & \textit{?pasu-na-vlak-e\u{s}te} \\ \hline
        {\Pl} \textgreater {\Poss} \textgreater {\Scase} & \textit{pasu-vlak-na-m} \\
        {\Poss} \textgreater {\Pl} \textgreater {\Scase} & \textit{pasu-na-vlak-em} \\ \hline 
        \multicolumn{2}{|l|}{\textit{pasu} `garden', \textit{-vlak} {\Pl}, \textit{-na} {\Poss}.{\Fpl}, \textit{-(e)\u{s}te} {\Iness}, \textit{-(e)m} {\Acc}} \\
        \hline 
    \end{tabular}
    \label{tab:mariorder} \\
    ${}$ \hfill Adapted from \cite{guseva2017postsyntactic}
\end{table}

In Mari, there are two linearizations of {\Poss}, {\Pl} and {\Lcase}. SA with the surface orderings of {\Lcase}-{\Poss} and {\Pl}-{\Lcase}-{\Poss} goes against the rightward-bound constraint, but this observation overlooks the other possible orders of {\Poss}-{\Lcase} and {\Poss}-{\Pl}-{\Lcase}. This ambiguous ordering of morphemes is the clue to understanding in what level of derivation SA takes place. This is the point that GW show with an example, adapted here as (\ref{mariSA3}).

\begin{exe}
 \ex \label{mariSA3}
    \begin{xlist}
        \ex \label{mariSA3a}
        \gll 
        \textit{Pörjeng} \textit{memnam} \textit{da} \textit{nunem} \textit{u\u{z}-e\u{s}} \\ man.{\Nom} us.{\Acc} {\And} them.{\Acc} see-{\Tsg}.{\Prs} \\
        
        \ex \label{mariSA3c}
        \gll 
        \textit{*Pörjeng} \textit{me} \textit{da} \textit{nunem} \textit{u\u{z}-e\u{s}} \\ man.{\Nom} us.{\Acc} {\And} them.{\Acc} see-{\Tsg}.{\Prs} \\
        
        \ex \label{mariSA3b} 
        \gll
        \textit{Pörjeng} \textit{memna} \textit{da} \textit{nunem} \textit{u\u{z}-e\u{s}} \\ man.{\Nom} us {\And} them.{\Acc} see-{\Tsg}.{\Prs} \\
        \glt `The man sees us and them'\\*
        \hfill Adapted from \cite{guseva2017postsyntactic}
    \end{xlist}
\end{exe}

The {\Fpl} pronoun is \textit{me} in Mari, and the stem for {\Acc} changes from \textit{me} to \textit{memna}. SA is not possible with \textit{me}, but it is possible with the plural stem \textit{memna}. A similar base or stem change in Turkish also happens when {\Fsg} and {\Ssg} pronouns are used with {\Dat} (\textit{ben \textgreater bana, sen \textgreater sana}). Turkish does not allow SA in such instances (\ref{mariturkish}), with or without base or stem change.

\begin{exe}
    \ex \label{mariturkish}
    \begin{xlist}
        \ex SA with unchanged base\\*
        \gll *Ben ve san-a kitab-ı bul-du. \\ 
        {\Fsg} {\And}  {\Ssg}-{\Dat} book-{\Acc} buy-{\Pst}[{\Tsg}] \\
        \glt ${}$
        
        \ex SA with base change\\*
        \gll *Ban ve san-a kitab-ı bul-du. \\ 
        {\Fsg} {\And}  {\Ssg}-{\Dat} book-{\Acc} buy-{\Pst}[{\Tsg}] \\
        \glt ${}$
        
        \ex No SA\\*
        \gll Ban-a ve san-a kitab-ı bul-du. \\ 
        {\Fsg}-{\Dat} {\And}  {\Ssg}-{\Dat} book-{\Acc} buy-{\Pst}[{\Tsg}] \\
        \glt `S/he bought the book for me and you'
    \end{xlist}
\end{exe}

GW go on to analyze SA in Mari with proposed projections for {\Poss}, {\Pl}, and {\Case} as NumP, DP, and KP. Following \cite{merchant2015ineffable} they propose an underlying order like (\ref{mariordering}). Onto this order a process of D-lowering takes place and the new ordering looks like (\ref{mariorderinglowered}). It is at the order of (\ref{mariorderinglowered}) that SA marks morphemes for zero exponance (shown with a subscript 0) as in (\ref{mariSAmark}). Later a D-metathesis is performed and the ordering for vocabulary insertion looks like (\ref{mariSAform}). This is how the suffix orderings in (\ref{mariSA}) are achieved, an example is partly repeated here.
\begin{exe}
    \ex \begin{xlist}
    \ex \label{mariordering}
    [[[ NP ] Num ]$_{NumP}$D]$_{DP}$ K ]$_{KP}$ \hfill Underlying Order
    
     \ex \label{mariorderinglowered}
    [[[ NP ] D Num ]$_{NumP}$ t$_D$ ]$_{DP}$ K ]$_{KP}$ \hfill D-Lowering
    
    \ex \label{mariSAmark}
    [[[ NP ] D Num$_0$ ]$_{NumP}$ t$_D$ ]$_{DP}$ K$_0$]$_{KP}$ \hfill SA marking
    
    \ex \label{mariSAform}
    [[[ NP ] D K$_0$ Num$_0$ ]$_{NumP}$ t$_D$ ]$_{DP}$ t$_K$ ]$_{KP}$ \hfill D-Metathesis
    \end{xlist}
     ${}$ \hfill Adapted from \cite{guseva2017postsyntactic}

    \exi{(\ref{mariSA}')} SA of {\Pl-\Iness}\\*
    \gll tud-en sad-\u{s}e den memn-an pasu-vlak-e\u{s}te-na \\ 
    {\Tsg}-{\Gen} garden-{\Poss}.{\Tsg} {\And} {\Fpl}-{\Gen} field-{\Pl}-{\Iness}-{\Poss}.{\Fpl}\\ 
    \glt `\ldots in his garden and in our fields'
\end{exe}

There are important observations to be made in \cite{guseva2017postsyntactic}. First, the examples in (\ref{mariSA}) show that SA is not performed at the surface form. This observation is vital to distinguish SA from Backward Ellipsis in Turkish because Backward Ellipsis takes the surface form into account. Second, (\ref{mariSA3}) shows that SA does not operate morphemes on a derivational level before their phonological representations are in place, yet (\ref{mariturkish}) show that even taking those representations into account does not result in a successful SA in Turkish. 

\subsection{Serbian}

According to \cite{despic2017suspended}, Serbian does not have suspended affixation, but a certain second place clitic shows some similarities to affixes in Serbian. This clitic in turn can take place in SA-like ellipsis. While my study focuses on SA in Turkish, and this section is about SA in other languages. It can not be denied that the examples of SA in Turkish verbal domain always include a relation to the clitic copula \textit{-i/ y/ $\emptyset$} (\ref{verbalSA}).

\begin{exe}
    \ex \label{verbalSA}
    \begin{xlist}
        \ex \label{verbalSA1}
        \gll 
        \textit{Ev-e} \textit{gel-ec\'{e}k} \textit{ve} \textit{uyu-yac\'{a}k-tı-m} \\ house-{\Dat} come-{\Fut} {\And} sleep-{\Fut}={\Cop}.{\Pst}-{\First}{\Sg} \\
        
        \ex \label{verbalSA2}
        \gll 
        \textit{Ev-e} \textit{gel-ec\'{e}k} \textit{ve} \textit{uyu-yac\'{a}k} \textit{i-di-m.} \\ house-{\Dat} come-{\Fut} {\And} sleep-{\Fut} ={\Cop}-{\Pst}-{\First}{\Sg} \\
        \glt `I was going to come home and sleep'
    \end{xlist}
\end{exe}

While (\ref{verbalSA1}) shows, an SA of Past tense and agreement markers, a closer look reveals what is actually suspended is a copular form together with tense and agreement markers. This clitic can also have an overt phonological form \textit{i} which also allow for the same SA (\ref{verbalSA2}). The overtness capability of the clitic is not enforced though, it is even ungrammatical in some instances (\ref{nompredSA}). We can arrive at the existence of a clitic from the stress pattern, since the domain of stress is the right edge of the phonological word and not the clitic group in Turkish.

\begin{exe}
    \ex \label{nompredSA}
    \begin{xlist}
        \ex 
        \gll 
        \textit{hast\'{a}} \textit{ve} \textit{yorg\'{u}n-um} \\ sick {\And} tired-{\First}{\Sg}[{\Prs}] \\
        \glt `I am sick and tired'
    \end{xlist}
\end{exe}

Now to turn to the example in Serbian, the special instance where SA-like process takes place involves the infinitival marker \textit{-ti} and second place future clitic \textit{\'{c}e}. The bare bones explanation for second place clitics is in a clause they occupy the linearly second place. If they are cliticized to the phonological word they are attached to, then the word can occupy the first place in the clause, which by default satisfies the clitic's constraint. If the cliticized word is not at the first place in a clause, clitic detaches from that word and occupies second place.

Here I want to draw a similarity between the infinitival marker \textit{-ti} in Serbian and infinitival marker \textit{-mAK} in Turkish. What appears is that verbs are not free forms in Serbian, just like verbs are not morphological words in Turkish. When the verb is inflected, there is no need for an infinitival marker, the inflection is performed on a truncated form of what is to the left of \textit{-ti} or \textit{-mAK}.

In Serbian some phonological processes are not triggered by clitics, for the example of second place future clitic \textit{\'{c}e}, and phonologically similar diminutive suffix \textit{\'{c}e} abiding by this generalization see (\ref{serbiance}).

\begin{exe}
    \ex \label{serbiance}
    \begin{xlist}

        \ex \gll
        \textit{Vas} \textit{\'{c}e} \textit{videti} \\ you.{\Pl}.{\Acc} ={\Aux}.{\Third}{\Sg}.{\Fut} see.{\Inf} \\
        \glt `S/he will see you.'
 
        \ex \gll 
        \textit{Pa\u{s}-\'{c}e} \\ dog-{\Dim} \\
        \glt `small dog'

    \end{xlist}
    ${}$ \hfill Adapted from \cite{despic2017suspended}
\end{exe}

In (\ref{serbiance}) both the forms \textit{vas} and \textit{pas} have the same phoneme \textit{/s/}. While a diminutive suffix causes a phonological change from \textit{/s/} to [\textesh], the second place future clitic doesn't. However, when the second place future clitic is cliticized to the bound verb form, it does trigger phonological change like a suffix would in Serbian (\ref{serbiance2}).

\begin{exe}
    \ex \label{serbiance2}
    \begin{xlist}
        \ex 
        \gll 
        \textit{*Jes=\'{c}e\u{s}} \\ eat={\Aux}.{\Second}{\Sg}.{\Fut} \\
    
        \ex
        \gll 
        \textit{Je\u{s}=\'{c}e\u{s}} \\ eat={\Aux}.{\Second}{\Sg}.{\Fut} \\
        \glt `You will eat.'
    \end{xlist}
    ${}$ \hfill Adapted from \cite{despic2017suspended}
\end{exe}

The SA-like construction in Serbian does not look like the examples of SA in Turkish. First of all it takes place in the second conjunct. An example is given in (\ref{serbianSAlike}).

\begin{exe}
    \ex \label{serbianSAlike}
    \begin{xlist}
        \ex 
        \gll 
        \textit{Oti\'{c}i} \textit{\'{c}e} \textit{i} \textit{pogleda=\'{c}e} \textit{novi} \textit{film.} \\ go.{\Inf} {\Aux}.{\Third}{\Sg}.{\Fut} {\And} see={\Aux}.{\Third}{\Sg}.{\Fut} new.{\Acc} film.{\Acc} \\
        
        \ex 
        \gll 
        \textit{*Oti\'{c}i} \textit{\'{c}e} \textit{i} \textit{pogleda} \textit{novi} \textit{film.} \\ go.{\Inf} {\Aux}.{\Third}{\Sg}.{\Fut} {\And} see new.{\Acc} film.{\Acc} \\
        
        \ex 
        \gll 
        \textit{Oti\'{c}i} \textit{\'{c}e} \textit{i} \textit{pogledati} \textit{novi} \textit{film.} \\ go.{\Inf} {\Aux}.{\Third}{\Sg}.{\Fut} {\And} see.{\Inf} new.{\Acc} film.{\Acc} \\
        \glt `He will go and see the new movie'
    \end{xlist}
    ${}$ \hfill Adapted from \cite{despic2017suspended}
\end{exe}

The very first observation that can made about the sentences in (\ref{serbianSAlike}) is that what is left after the SA-like process is not a phonological string of what comes before the clitic, since what is left is an infinitival form.


Despi\'{c} goes into an in-depth analysis to refute an idea of structural sharing where the future clitic comes atop a conjoined VP structure, since both conjuncts can have contrasting subjects. Despi\'c provides the following example (\ref{serbianconjuncts}).

\begin{exe}
    \ex \label{serbianconjuncts}
    \gll 
    Polufinalni program  \'{c}e otvoriti Juentus i {Real Madrid,} a zatvoriti ga Barselone i Bajern \\ semi\_final program {\Aux}.{\Third}{\Sg}.{\Fut} open.{\Inf} Juventus {\And} {Real Madrid} {\And} close.{\Inf} 3.{\Sg} Barcelone {\And} Bayern \\
    \glt `Juventus and Real Madrid will open the semi-final program, and Barcelona and Bayern will close it.' \\
    ${}$ \hfill Adapted from \cite{despic2017suspended}
\end{exe}

Since there can be two different subjects in (\ref{serbianconjuncts}), there is no VP-level conjunction. Despi\'c further examines TP level adverbs in conjunctions, refuting a \textit{v}P level conjunction too. However I won't go into those examples and further scrutiny of the state of affairs in Serbian here. I want to reiterate an example from Despi\'c to serve the point of what is nature of \textit{\'ce} deletion process under mismatching $\varphi$-features in (\ref{serbiannomatch}).

\begin{exe}
    \ex \label{serbiannomatch}
    \begin{xlist}
        \ex \gll 
        Ti \'{c}es do\'{c}i a ja (\'{c}u) oti\'{c}i \\ {\Second}.{\Sg} {\Aux}.{\Second}{\Sg}.{\Fut} arrive.{\Inf} {\And} {\First}{\Sg} ({\Aux}.{\First}{\Sg}.{\Fut}) leave.{\Inf} \\
        \glt `You will come and I will leave'
    \end{xlist}
    ${}$ \hfill Adapted from \cite{despic2017suspended}
\end{exe}

It is possible to delete the second place future clitic \textit{\'ce} in Serbian under mismatching $\varphi$-features, a direct contradiction to all the suspendable affixes in Turkish verbal domain, which have clitic like properties. This shows that a morphological form having affixal properties does not hinder us from capitulating on an ellipsis analysis for SA. For example, a k-paradigm agreement marker \textit{-k} `{\First}.{\Pl}' for the past tense marker \textit{-DI} can not be suspended, which contrasts with an m-paradigm agreement marker like \textit{-(y)Iz} that can be. The difference between these two suffixes is that the first is not clitic-like in nature and the second is (\ref{k-iz}).

\begin{exe}
    \ex \label{k-iz} 
    \begin{xlist}
        \ex \gll 
        Ev-e gid-ec\'{e}k ve dinlen-ec\'{e}ğ-iz \\ house-{\Dat} go-{\Fut} {\And} rest-{\Fut}-{\First}.{\Pl} \\
        \glt `We will go home and rest'
        
        \ex \gll 
        *Ev-e git-t\'i ve dinlend\'i-k \\ house-{\Dat} go-{\Pst} {\And} rest-{\Pst}-{\First}.{\Pl} \\
        \glt Intended `We went home and rested'
    \end{xlist}
\end{exe}

However, unlike the Serbian second place future clitic \textit{\'ce}, the m-paradigm agreement markers can not be suspended under mismatching $\varphi$-features (\ref{mismatchm}).

\begin{exe}
    \ex \label{mismatchm}
    \gll 
    *Sen ev-e gid-ecek ve biz dinlen-eceğ-iz \\ {\Second}.{\Sg} house-{\Dat} go-{\Fut} {\And} {\First}.{\Pl} rest-{\Fut}-{\First}.{\Pl} \\
    \glt Intended `You will go home and we will rest.'
\end{exe}

As a summary the Serbian second place future clitic shows affix like properties but it undergoes an ellipsis process where mismatches in $\varphi$-features can be overlooked. As a contrast, some agreement suffixes in Turkish show clitic like properties yet they do not undergo SA when there is a mismatch in $\varphi$-features.



















\subsection{Korean}

Another language that hosts similar phenomena like SA is Korean. Korean could be considered to be typologically closer to Turkish than the other languages German, Mari, and Serbian we have seen so far. On SA-like structures in Korean, \cite{yoon2005conjunction}, and \cite{yoon2017lexical} provide a good set of data and some contrasts. In the following paragraphs I try to give the relevant summary of the two papers.

\cite{yoon2005conjunction} presents two conjunction types in Korean, that differ in how their conjuncts are formed. In the first, the conjoiner suffix \textit{-kwa} ({\Conj} in glosses) conjoins two conjuncts, out of two only the second can be marked for case. A mirrorring morphological form to this conjoiner could be the cliticized \textit{ile} in Turkish. I give an example in (\ref{koreanturkish}).

\begin{exe}
    \ex \label{koreanturkish}
    \begin{xlist}
        \ex
        \gll 
        \textit{John-kwa} \textit{Mary-ka} \textit{cip-ey} \textit{ka-ss-ta} \\ John-{\Conj} Mary-{\Nom} home-{\Loc} go-{\Pst}-{\Decl} \\
        ${}$ \hfill Adapted from \cite{yoon2005conjunction}

        \ex 
        \gll 
        \textit{John=la} \textit{Mary} \textit{ev-e} \textit{git-ti-(ler)} \\ John={\And} Mary[{\Nom}] home-{\Dat} go-{\Pst}-({\Third}{\Pl}) \\
        \glt `John and Mary went home'
    \end{xlist}
    ${}$ \hfill Adapted from \cite{yoon2005conjunction}
\end{exe}

The second type of conjoiner is the free form \textit{kuliko}, again for the sake of the argument can be mirrored by \textit{ve} in Turkish (\ref{koreanturkish2}).

\begin{exe}
    \ex \label{koreanturkish2}
    \begin{xlist}
        \ex 
        \gll 
        \textit{John-i} \textit{kuliko} \textit{Mary-ka} \textit{cip-ey} \textit{ka-ss-ta} \\ John-{\Nom} {\And} Mary-{\Nom} home-{\Loc} go-{\Pst}-{\Decl} \\
        ${}$ \hfill Adapted from \cite{yoon2005conjunction}
        
        \ex 
        \gll 
        \textit{John} \textit{ve} \textit{Mary} \textit{ev-e} \textit{git-ti-(ler)} \\ John[?{\Nom}] {\And} Mary[{\Nom}] home-{\Dat} go-{\Pst}-({\Third}{\Pl}) \\
        \glt `John and Mary went home'
    \end{xlist}
    ${}$ \hfill Adapted from \cite{yoon2005conjunction}
\end{exe}

The two different conjoiners show differences in interpretations. The reading differences lie in distributive or non-distributive readings, compatibility with collective modifiers, and compatibility with collective predicates. An example for the order of readings for both conjuncts is given in (\ref{koreanconjoiners}).

\begin{exe}
    \ex \label{koreanconjoiners}
    \begin{xlist}
        \ex \label{koreanconjoiner1} 
        \gll 
        \textit{John-kwa} \textit{Mary-ka} \textit{ochen-pwul-ul} \textit{pelessta} \\ John-{\Conj} Mary-{\Nom} 5000-dollars-{\Acc} made \\
        
        \ex \label{koreanconjoiner2}
        \gll 
        \textit{John-i} \textit{kuliko} \textit{Mary-ka} \textit{ochen-pwul-ul} \textit{pelessta} \\ John-{\Nom} {\And} Mary-{\Nom} 5000-dollars-{\Acc} made \\
        \glt Reading 1: John and Mary each made \$5000 \\
        Reading 2: John and Mary together made \$5000 \\
        (\ref{koreanconjoiner1}): Reading 2\textgreater Reading 1 (\ref{koreanconjoiner2}): Reading 1\textgreater Reading 2
    \end{xlist}
    ${}$ \hfill Adapted from \cite{yoon2005conjunction}
\end{exe}

While this preference for readings are different in both conjoiners, it does not mean that the conjoiner \textit{-kwa} is incompatible with distributive readings. (\ref{kwadistributed}) shows a distributive reading for \textit{-kwa}. Another observation, that Yoon et al. makes is that the conjoiner \textit{kuliko} is incompatible with collective readings (\ref{kulikocollective}).

\begin{exe}
    \ex \begin{xlist}
    \ex \label{kwadistributed}
    \gll 
    \textit{John-kwa} \textit{Mary-ka} \textit{kakkak} \textit{cip-ey} \textit{ka-ss-ta} \\ John-{\Conj} Mary-{\Nom} each home-{\Loc} go-{\Pst}-{\Decl} \\
    \glt `John and Mary each went home'
    
    \ex \label{kulikocollective}
    \gll 
    \textit{*?Cheli-ka} \textit{kuliko} \textit{Yenghi-ka} \textit{chayksang-ul} \textit{hamkkey} \textit{mantul-ess-eyo} \\
    Cheli-{\Nom} {\And} Yenghi-{\Nom} desk-{\Acc} together make-{\Pst}-{\Decl} \\
    \glt Intended: `Chelswu and Yenghi made a desk together'
    \end{xlist}
    ${}$ \hfill Adapted from \cite{yoon2005conjunction}
 \end{exe}

The relation that the two conjoiners hold with respect to SA is that the conjoiner \textit{-kwa} seems to be triggering an instance of case SA, and the conjoiner \textit{kuliko} does not. Yoon et al. shows a distinction between the two conjoiners deeming \textit{-kwa} as a conjoiner for phrase levels and \textit{kuliko} as a conjoiner for clauses. These observations made so far about Korean conjoiners \textit{-kwa} and \textit{kuliko} show the importance of analyzing conjunction structure. From that analysis a point for SA as structural sharing or ellipsis process can be drawn. 

While \cite{yoon2005conjunction} provides some data and analysis for two conjoiners in Korean, \cite{yoon2017lexical} is focused on SA directly. Yoon presents derivational Korean suffixes that derive verbs or adjectives from nominal bases. These suffixes display a clear cut difference in allowing SA. In providing SA-independent contrasts between these suffixes, Yoon presents some examples with Lexical Integrity tests of conjoined base, modifying the base, and gapping/ellipsis of the suffix. In expressing the difference between two suffix groups, Yoon uses the following terms Transparent suffix, and Opaque suffix. These two terms represent a suffix's ability to be either treated as transparent and visible in morphological or syntactic derivations, or treated as opaque and non-compositional. Overt examples showing the contrast of these two suffix groups are given in (\ref{koreanlexical})


\begin{exe}
    \ex \label{koreanlexical}
    \begin{xlist}
    \exi{Conjoined base}
    \ex \label{koreanconjoined1}
    \gll
    \textit{*[Kunul-kwa} \textit{kilum]-ci-n} \textit{ku} \textit{kos} \\ shade-{\Conj} oil-CHARACTERIZED-REL that place \\
    \glt `That plot of land, which is shaded and fertile'
    
    \ex \label{koreanconjoined2}
    \gll 
    \textit{Ku-nun} \textit{[yongkamha-n} \textit{kwunin-kwa} \textit{cincengha-n} \textit{aykwukca]-taw-ass-ta}\\
    3.{\Sg}-{\Top} courageous-REL soldier-{\Conj} genuine-REL patriot-BE.LIKE-{\Pst}-{\Decl} \\
    \glt `He really lived up to his reputation as a courageous soldier and true patriot'
 
    \exi{Modified base}
    
    \ex \label{koreanmodified1}
    \gll 
    \textit{Cenyek-ey-nun} \textit{*[etwuw-un} \textit{kunul]-ci-nun} \textit{kos} \\ dusk-{\Loc}-{\Top} dark-REL shade-CHARACTERIZED.BY-REL place\\
    \glt `A place that gets dark at dusk'
    
    \ex \label{koreanmodified2}
    \gll 
    \textit{Ku-nun} \textit{[hwullyungha-n} \textit{hakca]-tap-key} \textit{yenkwu-lul} \textit{swi-ci} \textit{anh-nunta} \\ 3.{\Sg}-{\Top} outstanding-REL scholar-BE.LIKE-COMP research-{\Acc} stop-COMP NEG-PRS \\
    \glt `He never stops dping research, as befits his reputation as an outstanding scholar'
    
    \exi{Gapping/Ellipsis}
    
    \ex \label{koreangapping1}
    \gll 
    \textit{*Ku} \textit{kos-un}  \textit{kilum-\_} \textit{kuliko} \textit{i} \textit{kos-un} \textit{kunul-ci-ta} \\ that place-{\Top} oil-\_ {\And} this place-{\Top} shade-CHARACTERIZED.BY-{\Decl} \\
    \glt Intended `That place is fertile while this place is shady'
    
    \ex \label{koreangapping2}
    \gll 
    \textit{Cheli-nun} \textit{kwunin-\_} \textit{kuliko} \textit{Tongswu-nun} \textit{haksayng-tap-ta.} \\ Cheli-{\Top} soldier {\And} Tongswu-{\Top} student-BE.LIKE-{\Decl} \\
    \glt `Cheli is every bit a soldier and Tongswu (every bit) a student.'
    \end{xlist}
    ${}$ \hfill Adapted from \cite{yoon2017lexical}
\end{exe}

While (\ref{koreanlexical}) shows a clear distinction in the tests, a suffix does not always behave the same. For example in (\ref{tapdifferent}), the suffix \textit{-tap} behaves like \textit{-ci} in not allowing modification of base . Yoon dubs this category of suffixes as Double-duty suffix.
\begin{exe}
    \ex \label{tapdifferent}
    \gll 
    \textit{*[Ceng-kwa} \textit{alum]-taw-un} \textit{sa.i} \\ affection-{\Conj} beautiful-BE.LIKE-REL relation \\ 
    \glt `Close and beautiful'
\end{exe}

The behaviours of suffixes in (\ref{koreanlexical}) show that derivational suffixes can have different responses to structural configurations. An observation that can prove useful for identifying why, if any, some derivational suffixes in Turkish can take part in SA and some not. Yoon, after further tests and contrasts, provides a table indicating the different category of derivations, a short version of it is given in Table \ref{tab:korean}.

\begin{table}[hbt!]
    \caption{Response of Different Category Suffixes in Korean to Lexical Integrity Tests}
    \centering
    \begin{tabular}{|p{1.8cm}|p{2.1cm}|p{1.7cm}|p{1.6cm}|p{1.6cm}|p{2cm}|p{1.8cm}|}
    \hline 
                        & Coordination & External Modifiers & Gapping (Base) & Gapping (Suffix) & Inbound Ana Island & Extraction \\ \hline 
    Opaque Suffix       & N             & N                 & N             & N                 & N   & N \\ \hline 
    Transparent Suffix  & Y             & Y                 & N             & Y                 & Y   & N \\ \hline 
    Double-duty Suffix  & N/Y           & N/Y               & N             & N/Y               & N/Y & N/Y \\ \hline 
    \end{tabular}
    \label{tab:korean}
\end{table}

The important observations that we can draw from Yoon is that not all derivations are equally representable as one sub-syntactic and opaque process. Also, even the ones that usually have transparent relations with syntax and syntactic operations do not always behave the same. 

In putting these groups of suffixes into a theoretical framework, Yoon makes use of Word-internal phases, citing \cite{marantz2007phases} within DM. The explanation Yoon provides boils down to these suffix categories belonging to different word derivation bases. Opaque suffixes combine with the $\sqrt{ROOT}$ assigning the category and take place in the first phase of word derivation. Transparent suffixes combine with category assigned words and take place in the second phase of word derivation. Both phases are shown in Figure \ref{fig:devphases}.

\begin{figure}[hbt!]
    \centering
    \begin{forest}
    for tree={s sep=15mm, inner sep=0}
        [YP
            [XP, name=XP
                [X^0, name=X0 
                    [$\sqrt{ROOT}$]
                    [{Opaque Suffix}]]
                [{Transparent Suffix}]]
            [Y]]
    \node[above left=1em and 0.25em of X0](x1){\small 1^{st}phase};
    \node[right=0.5em of X0](x2){};
    \draw[overlay, thick] (x1) to[out=0, in=90] (x2);
    \node[above left=1em and 0.25em of XP](xp1){\small 2^{nd}phase};
    \node[right=0.5em of XP](xp2){};
    \draw[overlay, thick] (xp1) to[out=0, in=90] (xp2);
    \end{forest}
    \caption{Root internal phase in word-derivation}
    \label{fig:devphases}
\end{figure}

Figure \ref{fig:devphases} however, does not mean that an opaque suffix always culminates the first phase. According to Yoon, there could be several suffixes that could form a new Root from a base Root without category assignment as in Figure (\ref{fig:firstdevphase}).

\begin{figure}[hbt!]
    \centering
    \begin{forest}
        [$\sqrt{ROOT}^3$
            [$\sqrt{ROOT}^2$
                [$\sqrt{ROOT}$]
                [{suffix}]]
            [{suffix}]]
    \end{forest}
    \caption{Derived Roots from Root bases in first word derivation phase}
    \label{fig:firstdevphase}
\end{figure}


The explanation that Yoon provides, and as the Figure \ref{fig:devphases} shows is that Opaque suffixes combine with category-less $\sqrt{ROOT}$s. That's why, even though the operation itself is similar to syntactic merge, the internal structure of these suffixes are not visible to syntactic operations. On the other hand, transparent suffixes combine with bases that are morphological words with syntactic categories and that's why they are visible to operations like base modification, conjoined base and SA. This explanation can be utilized in explaining why bare verbs are not morphological words and why SA can not take place with bare verb remnants in Turkish.

\subsection{Ossetic}

\citet{erschler2012suspended} and \citet{erschler2018suspended} deal with SA in Ossetic. Ossetic is a language spoken in Caucasus. Ossetic displays a set of data that on the surface seems to be inconsistent when it comes to SA. For example, when a pronoun and a proper noun is conjoined, the choices of {\Case} for the both conjuncts change depending on the order of the conjuncts (\ref{ossetic}). In (\ref{ossetic1}), it seems there is no SA since the pronoun {\Ssg} is marked for {\Obl}. On the other hand, in (\ref{ossetic2}) there is SA of {\Abl} from the proper noun \textit{Alan}.

\begin{exe}
    \ex \label{ossetic} SA of {\Abl}\\*
    \begin{xlist}
        \ex \label{ossetic1} 
        \gll d\textturna w \textturna ma Alan-\textturna j tarst\textturna n. \\
        {\Ssg}.{\Obl} {\And} A-{\Abl} be.afraid.{\Pst}.{\Fsg} \\
        \glt `I am afraid of you and Alan.'
        
        \ex \label{ossetic2} 
        \gll Alan \textturna ma d\textturna w-\textturna j tarst\textturna n. \\
        A[{\Nom}] {\And} {\Ssg}-{\Abl} be.afraid.{\Pst}.{\Fsg} \\
        \glt `I am afraid of Alan and you.'\\*
        \hfill Adapted from \citet{erschler2012suspended}
    \end{xlist}
\end{exe}

\citet{erschler2012suspended} deals with SA of {\Case} in Ossetic. He provides some background into the case system of Ossetic before moving on with examples and analysis of SA. Definite animates, and personal pronouns are obligatorily marked {\Obl}, inanimate objects are marked {\Nom}, and modifiers are not case marked. All plural nouns in Ossetic lose their final [\textturna] sound when marked by vowel initial case markers. This is taken to be a phonological constraint since consonant initial case markers do not trigger the same alternation. (\ref{osseticdeletion}) shows an example of dropping [\textturna].

\begin{exe}
    \ex \label{osseticdeletion}
    \begin{xlist}
        \ex \gll {b\textturna\textchi-t\textturna} \\ horse-{\Pl}[{\Nom}] \\
        
        \ex \gll {b\textturna\textchi-t-\textschwa} \\ horse-{\Pl}-{\Obl} \\
        \glt \hfill Adapted from \citet{erschler2012suspended}
    \end{xlist}
\end{exe}

Erschler proposes some constraints, first of which is that any case marker can be suspended. This is not so much of a constraint but an observation. The examples in (\ref{osseticSA}) host SA for {\Obl}, {\Sup}, {\Abl}, and {\Loc}.

\begin{exe}
    \ex \label{osseticSA}
    \begin{xlist}
        \ex SA of {\Obl}\\* 
        \gll Soslan {\textturna ma} {Zalijn-i} {\textchi\textturna\textdzlig r\textturna} \\ 
        S {\And} Z-{\Obl} house. \\
        \glt `the house of Soslan and Zalina.'
        
        \ex SA of {\Sup}\\*
        \gll {Alan} {\textturna ma} {Soslan-b\textturna l} {is-\textturna mbaltt\textturna n}. \\ 
        A {\And} S-{\Sup} {\Prv}-meet.{\Pst}.{\Fsg} \\
        \glt `I met Alan and Soslan.'
        
        \ex SA of {\Abl}\\*
        \gll {Alan} {\textturna ma} {Soslan-b\textturna j} {tarst\textturna n}. \\ 
        A {\And} S-{\Abl} be.afraid.{\Pst}.{\Fsg} \\
        \glt `I was afraid of Alan and Soslan.'
        
        \ex SA of {\Loc}\\*
        \gll {budur} {\textturna ma} {\textinvscr\textturna d-i} {ber\textturna} {\v{c}'ewu-t\textturna} {i\v{s}-\v{s}erdtonc\textturna}. \\ 
        field {\And} forest-{\Loc} many bird-{\Pl} {\Prv}-find.{\Pst}.{\Tpl} \\
        \glt `They found many birds in the field and the forest.'\\*
        \hfill Adapted from \citet{erschler2012suspended}
    \end{xlist}
\end{exe}

The second constraint is that the first conjunct in SA should be the base of the case marker, without phonological processes like [\textturna] deletion (\ref{osseticSA2}).

\begin{exe}
    \ex \label{osseticSA2}
    \begin{xlist}
    \ex 
    \gll 
    {b\textturna\textchi-t-im\textturna} {\textturna m\textturna} {g\textturna l-t-im\textturna} \\ horse-{\Pl}-{\Com} {\And} ox-{\Pl}-{\Com} \\
    
    \ex 
    \gll 
    {*b\textturna\textchi-t} {\textturna m\textturna} {g\textturna l-t-im\textturna} \\ horse-{\Pl} {\And} ox-{\Pl}-{\Com} \\
    
    \ex 
    \gll 
    {b\textturna\textchi-ta} {\textturna m\textturna} {g\textturna l-t-im\textturna} \\ horse-{\Pl} {\And} ox-{\Pl}-{\Com} \\
    \glt `with horses and oxen'\\*
    \hfill Adapted from \citet{erschler2012suspended}
    \end{xlist}
\end{exe}

Complying with the same constraint, personal pronouns that have different bases for some of the cases need to have those bases as their remnants in the first conjunct (\ref{osseticSA3}).

\begin{exe}
    \ex \label{osseticSA3}
    \begin{xlist}
        \ex \gll 
        {d\textturna w/*du} {\textturna ma} {Alan-b\textturna l} {is-\textturna mbaltt\textturna n}. \\ {\Ssg}[{\Obl}]/*{\Ssg}[{\Nom}] {\And} A-{\Sup} {\Prv}-meet.{\Pst}.{\Fsg} \\
        \glt `I met you and Alan.'
        
        \ex \gll 
        {d\textturna w/*du} {\textturna ma} {Alan-\textturna j} {t\textturna rsun}. \\ {\Ssg}[{\Obl}]/*{\Ssg}[{\Nom}] {\And} A-{\Abl} be.afraid.{\Prs}.{\Fsg} \\
        \glt `I am afraid of you and Alan.'\\*
        \hfill Adapted from \citet{erschler2012suspended}
    \end{xlist}
\end{exe}

The third constraint for Ossetic SA is what is left after suspension should be an independent (morphological) word. The two branches of Ossetic differ in regarding a reciprocal form `each other' as an independent word. In Iron Ossetic, it is an independent word and can take part in SA whereas the Digor counterpart is not an independent word and does not take place in SA (\ref{osseticSA4}).

\begin{exe}
    \ex \label{osseticSA4}
    \begin{xlist}
        \ex Iron Ossetic\\*
        \gll {?n\textturna=d\textschwa w\textturna} {g\textturna dy-je} {k\textturna r\textturna zi} {\textturna m\textturna} {n\textturna=k$^w$\textschwa z-\textturna j} {t\textturna r\v{s}-\textschwa nc}. \\
        {\Poss}{\Fpl}=two cat-{\Obl} each.other {\And} {\Poss}{\Fpl}=dog-{\Abl} be.afraid.{\Prs}.{\Tpl} \\
        \glt `Our two cats are afraid of each other and of our dog.'
        
        \ex Digor Ossetic\\*
        \gll {*n\textturna=duw\textturna} {tiki\v{s}-i} {k\textturna r\textturna\textdyoghlig e} {\textturna ma} {n\textturna=kuj-\textturna j} {t\textturna rs-unc\textturna}. \\ {\Poss}{\Fpl}=two cat-{\Obl} each.other {\And} {\Poss}{\Fpl}=dog-{\Abl} be.afraid.{\Prs}.{\Tpl} \\
        \glt \hfill Adapted from \citet{erschler2012suspended}
    \end{xlist}
\end{exe}

The fourth constraint of Ossetic SA is that what is left after SA should not have idiosyncratic meaning. This constraint relates to the {\Third}{\Sg} pronoun form \textit{w\textschwa m} which has the meaning `there' that serves as the base for the Dative marked {\Third}{\Sg} pronoun (\ref{osseticSA5}).

\begin{exe}
    \ex \label{osseticSA5}
    \begin{xlist}
        \ex \gll {w\textschwa m} {\textturna m\textturna} {m\textturna din\textturna-j\textturna n} {didin\textdyoghlig\textschwa t\textturna} {ratta}. \\ 
        there {\And} M-{\Dat} flowers gave \\
    
    \ex \gll {w\textschwa m-\textturna n} {\textturna m\textturna} {m\textturna din\textturna-j\textturna n} {didin\textdyoghlig\textschwa t\textturna} {ratta}. \\ 
    {\Tsg}-{\Dat} {\And} M-{\Dat} flowers gave \\
    \glt `S/he gave flowers to her and Madina.'\\*
    \hfill Adapted from \citet{erschler2012suspended}
    \end{xlist}
\end{exe}

The final constraint for Ossetic SA is that when both conjuncts are pronouns no suspended affixation takes place, a point illustrated in (\ref{osseticSA6}).

\begin{exe}
    \ex \label{osseticSA6}
    \begin{xlist}
    \ex \gll 
    {m\textturna n-b\textturna l} {\textturna m\textturna} {d\textturna w-b\textturna l} {\textturna ww\textturna nduj}. \\ {\Fsg}-{\Sup} {\And} {\Ssg}-{\Sup} believe.{\Prs}.{\Third}{\Sg} \\
    \glt `S/he believes me and you.'
    
    \ex \gll 
    {*m\textturna n} {\textturna m\textturna} {d\textturna w-b\textturna l} {\textturna ww\textturna nduj}. \\ {\Fsg}[{\Obl}] {\And} {\Ssg}-{\Sup} believe.{\Prs}.{\Third}{\Sg} \\
    \glt Intended `S/he believes me and you.'\\*
    \hfill Adapted from \citet{erschler2012suspended}
    \end{xlist}
\end{exe}

Following these observations, Erschler argues that SA needs to be a phonological deletion process after vocabulary insertion instead of a structural sharing process. Erschler argues against an approach where case markers are treated as syntactic projections. This in turn makes the structural sharing argument less appealing. He provides the examples in (\ref{osseticdepict}) where the complements of adpositions cannot control depictives, but case marked arguments can.

\begin{exe}
    \ex \label{osseticdepict}
    \begin{xlist}
        \ex \gll 
        {soslan} {\textchi et\textturna g-i} {\textchi\textturna cc\textturna} {rasug-\textturna j} {\textdzlig or-uj}. \\ S[{\Nom}] X-{\Obl} with drunk-{\Abl} talk-{\Prs}.{\Third}{\Sg} \\
        \glt `Soslan$_i$ is talking to Xetag$_i$ when he$_{i/*j}$ is drunk.'
    
        \ex \gll 
        {soslan} {\textchi et\textturna g-b\textturna l} {rasug-\textturna j=d\textturna r} {\textturna ww\textturna nd-uj}. \\ S[{\Nom}] X-{\Sup} drunk-{\Abl}={\Emp} believe-{\Prs}.{\Third}{\Sg} \\
        \glt `Soslan$_i$ believes in Xetag$_i$ even when he$_{i/j}$ is drunk.'\\*
        \hfill Adapted from \citet{erschler2012suspended}
    \end{xlist}
\end{exe}

In \citet{erschler2018suspended}, he further develops the approach of ellipsis for SA. He provides the alternative question configurations in which SA can take place (\ref{osseticalternative}) to show that SA is an ellipsis process.

\begin{exe}
    \ex \label{osseticalternative}
    \begin{xlist}
        \ex \gll {s\textturna rm\textturna t(-m\textturna)} {\textturna vi} {uruzm\textturna g-m\textturna} {\textdzlig urdtaj?} \\ 
        S(-{\All}) {\Or}.{\Q} U-{\All} you.called \\ 
        \glt `Did you call Sarmat or Uruzmag?'
        
        \ex \gll {ad\textturna jmag} {k$^w$\textschwa d} {f\textturna\v{z}\textschwa nd?} {arv-\textschwa} {c'\textturna\textinvscr(-\textturna j)} {\textturna vi} {\v{s}\textschwa\textdyoghlig\textschwa t-\textturna j} {rajg$^w$\textschwa rd} \\ 
        human how appeared sky-{\Obl} blue-{\Abl} {\Or}.{\Q} clay-{\Abl} was.born \\
        \glt `How did the humans appear? Were they born from the sky blue or from clay?'\\*
        \hfill Adapted from \citet{erschler2018suspended}
    \end{xlist}
\end{exe}

I mirror the examples in (\ref{turkishalternative}) for Turkish in two ways. First, the exclusive alternative question is formed by two question clitics \textit{=mI}. Second is a disjunctive yes/no question which is formed with \textit{or} `veya'. The exclusive alternative question does not allow SA, but the disjunctive yes/no question does.

\begin{exe}
    \ex \label{turkishalternative}
    \begin{xlist}
    \ex \gll {Ali*(-yi)=mi} {Mehmet-i=mi} {ara-dı-n?} \\ 
    A-{\Acc}={\Q} M-{\Acc}={\Q} call-{\Pst}-{\Ssg} \\
    \glt `Did you call Ali, or did you call Mehmet?'
    
    \ex \gll {Ali} {veya} {Mehmet-i=mi} {ara-dı-n?} \\
    A {\Or} M-{\Acc}={\Q} call-{\Pst}-{\Ssg} \\
    \glt `Did you call Ali or Mehmet?'
    \end{xlist}
\end{exe}

Turkish exclusive alternative questions do not allow for SA unlike Ossetic. One important point needs to be made here. The question clitic \textit{=mI} in Turkish is a focusing element which draws focus to the preceding argument it is attached to. In exclusive alternative questions, the question clitic \textit{=mI} focuses the target word for SA.

Erschler moves on to pinpointing where the deletion process takes place after claiming that SA is an ellipsis process. He uses the DM framework, and argues that SA takes place after vocabulary insertion but before morpheme specific readjustments. The support for SA taking place after vocabulary insertion comes from the example in (\ref{osseticVI}) since the fragment after SA is the base for {\Sup} and not the base for {\Nom}. The support for SA taking place before morpheme specific phonological adjustments comes from the example in (\ref{osseticPA}) since the phonological assimilations of [g]\textgreater[\textdyoghlig] and [k]\textgreater[\textteshlig] dont take place in the first conjuncts under SA of {\Obl}.

\begin{exe}
    \ex \begin{xlist}
    \ex \label{osseticVI}
    \gll {d\textturna w(-b\textturna l)/*du} {\textturna ma} {m\textturna din\textturna-b\textturna l} {is\textturna mbaltt\textturna n}. \\ 
    {\Ssg}.{\Obl}-({\Sup})/{\Ssg}.{\Nom} {\And} M-{\Sup} {\Fsg}.met \\
    \glt `I met you and Madina.'
    
    \ex \label{osseticPA}
    \begin{xlisti}
    \ex \gll {park} {\textturna m\textturna} {w\textschwa n\textdyoghlig-\textschwa}. \\ 
    park {\And} street-{\Obl} \\
    \glt `in/of the street and the park.'
    
    \ex \gll {w\textschwa ng} {\textturna m\textturna} {par\textteshlig-\textschwa}. \\ 
    street {\And} park-{\Obl} \\
    \glt `in/of the park and the street.'\\*
    \hfill Adapted from \citet{erschler2018suspended}
    \end{xlisti}
    \end{xlist}
\end{exe}

Erschler argues that SA is a backward ellipsis process under identity where not all conjuncts should bear \textsc{[+{\Emp}]} feature. He cites \citet{herbeck2016controlling} in defense of positing information structure features in the lexicon for lexical items where Herbeck argues that Spanish overt pronouns have feature \textsc{[+{\Foc}]}. Overt pronouns need to be discourse configured hence the feature \textsc{[+{\Emp}]} because Ossetic is a pro-drop language like Turkish (cf. \citet{ozturk2002turkish} overt Turkish pronouns). 


\section{Summary}

As a summary of the literature presented in this chapter, I provide the following observations about SA:

\begin{itemize}
    \item It is a rightward-bound process in the underlying morpheme order: Examples provided in \citet{kabak2007turkish}, \citet{pounder2006broken}, and \citet{guseva2017postsyntactic} show this for Turkish, German, and Mari.
    
    \item It is found both in inflectional and derivational paradigms: Examples provided in \citet{akkucs2016suspended}, and \citet{yoon2017lexical} show this for Turkish and Korean.

    \item It takes place after vocabulary insertion and before phonological readjustments: Examples provided in  \citet{pounder2006broken}, \citet{guseva2017postsyntactic}, and \citet{erschler2018suspended} show this for German, Mari and Ossetic.
\end{itemize}

These are the observations that seem to be consistent in all the articles. However, not all the articles align in the structural analysis of SA. The dominant account for Turkish seems to be structural sharing in nature \citep{orgun1995flat,kornfilt1996some,broadwell2008turkish,kornfilt2012revisiting}. This account is in line with \citet{ackema2004beyond,kunduraci2016morphology}, and \citet{bruening2018word} since an output of syntax can become an input for morphology and word formation in such form of language derivation. The accounts provided for other languages like Serbian, Mari, and Ossetic are all ellipsis analyses \citep{despic2017suspended,guseva2017postsyntactic,erschler2018suspended}. The summary of the literature for Turkish SA presents the following points to be addressed for any further study. It is the aim of this thesis to scrutinize these issues and contribute to the literature in an orderly and comprehensive manner.

\begin{itemize}
    \item Is SA of derivational suffixes possible in Turkish? If so how, if not why?
    \item What empirical studies can be used to determine the processing cost of SA?
    \item How does SA interact with sentence processing?
\end{itemize}



The environment of SA is conjunction. I give what conjunction analysis I follow and what the constraints are in forming conjunctions in the following section.

\section{Conjunction}

The functional cue or signal for such conjunction usually have a conjoiner like \textit{veya} `or' and \textit{ve} `and'. These structures are not necessarily additive, and depending on the parts they are putting together, the relations that the parts hold to one another can change. A conjoiner like \textit{ve} `and' can have additive properties when it conjoins nouns, but an ordering one when it conjoins sentences. (\ref{conjoinerexample}) shows an example for each.

\begin{exe}
\ex \label{conjoinerexample}
\begin{xlist}
\ex \gll Ahmet kalem ve kitap al-dı. \\
A[{\Nom}] pencil {\And} book buy-{\Pst}[{\Tsg}] \\
\glt `Ahmet bought some pencils and books.'

\ex \gll Ahmet ev-e git-ti ve bulaşığ-ı yıka-dı. \\
A[{\Nom}] house-{\Dat} go-{\Pst}[{\Tsg}] {\And} dishes-{\Acc} wash-{\Pst}[{\Tsg}] \\
\glt `Ahmet went home and washed the dishes.'
\end{xlist}
\end{exe}

The structural representation of conjunctions can prove a bit difficult when other language processes are considered. One interesting behaviour of conjunctions is that the extraction of a conjunct from the conjunction is not felicitous. This is commonly known as Coordinate Structure Constraint \citep{ross1967constraints}. (\ref{cscturkish}) illustrates this constraint in Turkish.

\begin{exe}
\ex \label{cscturkish} 
\gll *Ahmet ne ve kitap al-mış? \\ 
A[{\Nom}] what {\And} book buy-{\Pst}[{\Tsg}] \\
\glt `*Ahmet bought what and book?'
\end{exe}

In addition to this behaviour, conjunctions are not always carried out by overt conjoiners. Some instances of conjunctions can be signalled by small prosodic breaks. I give an example of this in (\ref{noovertconjoiner}) where commas indicate prosodic breaks.

\begin{exe}
\ex \label{noovertconjoiner} 
\begin{xlist}
\ex \gll Ahmet pazar-dan domates, biber, patlıcan al-dı. \\ 
A[{\Nom}] market-{\Abl} tomato pepper aubergine buy-{\Pst}[{\Tsg}] \\
\glt `Ahmet bought tomatoes, peppers, and aubergines from the market.'

\ex \gll Ahmet pazar-a git-ti, domates al-dı. \\ 
A[{\Nom}] market-{\Dat} go-{\Pst}[{\Tsg}] tomato buy-{\Pst}[{\Tsg}] \\
\glt `Ahmet went to the market, and bought tomatoes'

\end{xlist}
\end{exe}

Constraints like CSC and the possibility of conjoining more than two elements with or without conjoiners made conjunctions receive a ternary branching analysis. This analysis regards all the conjuncts as elements of the same hierarchical level. Figure \ref{fig:conjunction} shows a simple example for conjunction of three conjuncts.

\begin{figure}[hbt!]
    \centering
    \begin{forest}
    [XP 
        [XP]
        [XP]
        [XP]]
    \end{forest}
    \caption{Early conjunction analysis}
    \label{fig:conjunction}
\end{figure}

This analysis however is problematic when binding principles \citep{chomsky1993lectures,haegeman1994introduction} are considered. More specifically Principle B which states that a pronoun must be free in its binding domain. For a simple consideration of what constitutes a binding domain I use the c-command relation. (\ref{violateB}) shows Principle B in Turkish. In this example, the proper noun \textit{Ahmet} c-commands the pronoun `o(n)' {\Tsg}. This means that the pronoun cannot be co-referential with the proper noun since it is in the binding domain of the pronoun.

\begin{exe}
\ex \label{violateB} 
\gll Ahmet$_i$ on$_{*i/j}$-un arkadaş-ın-ı sev-iyor.\\ 
A[{\Nom}] {\Tsg}-{\Gen} friend-{\Poss}.{\Tsg}-{\Acc} like-{\Prog} \\
\glt Ahmet$_i$ likes his$_{*i/j}$ friend.
\end{exe}

An analysis like Figure \ref{fig:conjunction} predicts all conjuncts to c-command one another. This means that no conjunct should be able to bind a pronoun within the conjunction. (\ref{conjunctionB}) shows an example that goes against such a prediction. In this example the pronoun \textit{o} {\Tsg} can be co-referential with a proper noun \textit{Ahmet} even if they are in a conjunction.


\begin{exe}
\ex \label{conjunctionB} 
\gll Ahmet$_i$ ve on$_{i/j}$-un arkadaş-lar-ı \\ 
A {\And} {\Tsg}-{\Gen} friend-{\Pl}-{\Tsg} \\
\glt `Ahmet$_i$ and his$_{i/j}$ friends'
\end{exe}

Co-referentiality in (\ref{conjunctionB}) would have been infelicitous if the pronoun \textit{Ahmet} were to c-command the other conjunct. This means that a ternary branching analysis that treats all conjuncts belonging to the same hierarchical level is problematic.

There are at least three different ways that a binary representation of conjunctions can be achieved. These are \cite{munn1993topics}'s adjoined Boolean Phrase (BP) analysis, \cite{johannessen1998coordination}'s Co(njunction/ordination) Phrase (\&P) analysis, and lastly \cite{te2005deriving}'s pure merge analysis. I briefly explore these analyses in the next subsections. 


\subsection{BP analysis} \label{bpanalysis}
\cite{munn1993topics} revisits and revises the observations made in \cite{munn1987coordinate} for an asymmetric structural interpretation for conjunctions. He proposes that conjoiners form a boolean phrase, and work on the basis of semantics. The conjoiner takes an argument, makes a boolean phrase (BP), and takes another semantically equivalent argument to form a complete conjunction. The resulting structure bears the syntactic category of the last argument. Figure \ref{fig:booleanphrase} illustrates a basic representation of the analysis.

\begin{figure}[hbt!]
    \centering
    \begin{forest}
    [XP$_{\textless \sigma, \tau\textgreater}$
        [XP$_{\textless \sigma, \tau\textgreater}$]
        [BP 
            [B]
            [XP$_{\textless \sigma, \tau\textgreater}$ /YP$_{\textless \sigma, \tau\textgreater}$ ]]]
    \end{forest}
    \caption{Boolean phrase analysis of conjunction}
    \label{fig:booleanphrase}
\end{figure}

The analysis Munn provides is head initial, and it works on the semantic denotation of the conjuncts. The only requirement for a conjunction is the semantic equivalence. The example (\ref{unevenconjuncts}) shows conjunction of two different syntactic categories in Turkish. The first conjunct is an adverb phrase and the other is a post-positional phrase.

\begin{exe}
\ex \label{unevenconjuncts} 
\begin{xlist}
\ex \gll Ahmet dikkatlice ve azim-le çalış-ıyor. \\ A[{\Nom}] carefully {\And} tenacity-{\Ins} work-{\Prog}[{\Tsg}] \\
\glt `Ahmet is working carefully and with tenacity.'
\end{xlist}
\end{exe}

Changing the headedness of the analysis can fit it into Turkish and predict the correct c-command relations for (\ref{conjunctionB}). Figure \ref{fig:turkishbp} illustrates an abstract representation of BP and conjunction.

\begin{figure}[hbt!]
    \centering
    \begin{forest}
    [XP$_{\textless \sigma, \tau\textgreater}$ 
        [BP 
            [XP$_{\textless \sigma, \tau\textgreater}$/ YP$_{\textless \sigma, \tau\textgreater}$]
            [B\\conjoiner]]
        [XP$_{\textless \sigma, \tau\textgreater}$]]
    \end{forest}
    \caption{Structural representation of BP for Turkish}
    \label{fig:turkishbp}
\end{figure}


\subsection{\&P analysis}
\cite{johannessen1998coordination} proposes asymmetric conjunction analysis following the irregularities that conjunctions display in several languages.\footnote{the title of her work is `Coordination', and the explanations are provided with that naming. For the sake of cohesiveness I replace the `Coordination' with `Conjunction'} She categorizes conjunctions into unbalanced and balanced conjunctions where a balanced conjunction has, order wise, reversible conjuncts with no cost of grammaticality or form but an unbalanced conjunction does not have reversible conjuncts without a cost of change in the conjuncts or grammaticality. The unbalanced conjunctions can have different types. One of those types that Johannessen dubs `assigning type unbalanced conjunction' is the base argument for the peculiarities of conjunctions. 

In the assigning type conjunctions, one of the conjuncts determine the syntactic relations that the conjunction and other processes hold, such as agreement on the verb. An example for person agreement from Czech (\ref{johanczech}) and and example of gender agreement from Latin (\ref{johanlatin}) are provided in Johannessen where one of the conjuncts determine the agreement. In (\ref{johanczech}), the verb holds person agreement with the first conjunct. In (\ref{johanlatin}), the verb holds gender agreement with the second conjunct.

\begin{exe}
\ex \begin{xlist}
    \ex Czech \label{johanczech}\\*
    \gll Půjdu tam [j\`{a} a ty]. \\ 
    will.go.{\Fsg} there {\Fsg} {\And} {\Ssg} \\
    \glt `You and I will go there.'
    
    \ex Latin \label{johanlatin}\\*
    \gll [Populi provinciaeque] liberatae sunt. \\ 
    people.{\M}.{\Pl} province.{\F}.{\Pl}.{\And} liberated.{\F}.{\Pl} are \\
    \glt `The people and the provinces are liberated.' \\*
    \hfill as cited in \cite{johannessen1998coordination} 
\end{xlist}
\end{exe}

Johannessen goes onto presenting more conjunctions of this type to show the conjunction should receive its own syntactic category so that the kind of constructions like assigning unbalanced conjunctions can be accounted for. Figure \ref{fig:johancop} illustrates the structural representation she proposes.

\begin{figure}[hbt!]
    \centering
    \begin{forest}
    [\&P 
        [XP]
        [\&' 
            [\&]
            [XP]]]
    \end{forest}
    \caption{Conjunction phrase analysis}
    \label{fig:johancop}
\end{figure}

In this analysis the conjoiner is a functional head that takes two arguments and projects a conjunction phrase. The headedness of the structure follows from the language and in the case of Turkish, the first conjunct is the first argument of the conjoiner and the second conjunct is the second argument. The final conjunction phrase carries the syntactic label of the second conjunct, if syntactic processes that require lexical categories are concerned.

One shortcoming of Johannessen's analysis is that she uses examples of SA from languages like Eastern Mari, Old Uighur, and Turkish to argue for unbalanced conjunctions. I repeat some examples provided by \cite{johannessen1998coordination} for unbalanced conjunctions in (\ref{johansa}). These examples fall into examples of SA. This is not a concern for her analysis, but I mention it here for its relevance to my study.

\begin{exe}
\ex \label{johansa}
\begin{xlist}
    \ex Eastern Mari, SA of {\Pl}\\* 
    \gll [Rveze den yd\textschwa rvlak] mod\textschwa t \\ 
    boy {\And} girl.{\Pl} play.{\Tpl} \\
    \glt `The boy(s) and the girls are playing.'  
    
    \ex Old Uighur, SA of {\Acc}\\*
    \gll [Jala\textipa{\ng}uq-lar tynly\textgamma-lar-y\textgamma] \\ 
    man-{\Pl} animal-{\Pl}-{\Acc} \\
    \glt `the men and the creatures'
    
    \ex Turkish, SA of {\Pl} and {\Acc}\\*
    \gll Elma veya armut-lar-ı ye-di-niz mi? \\ 
    apple {\Or} pear-{\Pl}-{\Acc} eat-{\Pst}-{\Spl} ={\Q} \\
    \glt `Did you eat the apples or the pears?'\\*
    \hfill Adapted from \cite{johannessen1998coordination}
\end{xlist}

\end{exe}

\subsection{Pure merge}

\cite{te2005deriving} provides some theory internal objections to both the analysis of \cite{munn1993topics} and the analysis of \cite{johannessen1998coordination}. These include the assumptions that both the analyses hold with respect to the conjunct positions. The analysis of Munn suggests that the boolean phrase, which has the conjoiner and one conjunct, is adjoined to the other conjunct. The analysis of Johannessen suggests that the conjoiner projects to a conjunction phrase where one of the conjuncts is the complement and the other conjunct is placed on the specifier position of the conjunction phrase. Te Velde argues that the specifier adjunct positions should be subject to movement in theory. Movement out of a conjunct on the other hand is not permitted \citep{ross1967constraints}.  

Te Velde argues for an analysis that regards a conjoiner as a defective syntactic category with no phrase projection akin to BP or \&P. He claims that conjunction is carried out at the base positions with `Pure Merge' as he cites \cite{chomsky1999derivation}. The conjoiner signals a process of conjunction that triggers certain constraints that are set for a conjunction. These include the copying and checking over the syntactic and semantic features, where the features differ in their influence over the well-formedness of the conjunction. This solves a theory internal problem in terms of the place status of conjuncts. Base generation removes the analyses of adjunction or specifier positions. 

Te Velde provides an example from German where two prepositions are conjoined and used with a single noun. In (\ref{teveldeconj}), the preposition \textit{in} `in' assigns {\Dat} and \textit{um} `around' assigns {\Acc}. The noun \textit{Stadt} is used with an accusative article \textit{die} instead of a dative \textit{der}. Te Velde argues that there is no independent evidence to argue for an ellipsis analysis to account for (\ref{teveldeconj}) as in (\ref{teveldeconjalter}). 

\begin{exe}
\ex \begin{xlist}
    \ex \label{teveldeconj} \gll 
   Wir kaufen heute in$_{\Dat}$ und um$_{\Acc}$ die Stadt ein\\ 
   we buy today in {\And} around the.{\Acc} city in \\
   \glt `We're going shopping in and around the city.'\\
   
   \ex \label{teveldeconjalter}
   \textit{Wir kaufen heute in \sout{der Stadt} und um die Stadt ein}\\*
   \hfill \cite{te2005deriving}
   \end{xlist}
\end{exe}

\begin{figure}[hbt!]
    \centering
    \begin{forest}
    [PP 
        [P 
            [P\\\textit{in}]
            [P\rlap{ $\Uparrow$Merge}
                [\&\\\textit{und}]
                [P\\\textit{um}]]]
        [DP\\\textit{die Stadt}]]
    \end{forest}
    \caption{Base generated conjunction}
    \label{fig:tevelde}
\end{figure}

I have provided three analyses of conjunctions in this section. All of them have a hierarchical representation. \cite{munn1993topics} provides an adjunction analysis of BP where BP consists of one conjunct and a conjoiner. BP is later adjoined to the other conjunct. \cite{johannessen1998coordination} provides a full conjunction phrase analysis where one of the conjuncts is the complement and the other is the specifier of \&P which is headed by a conjoiner. \cite{te2005deriving} provides a pure merge analysis where one of the conjuncts is merged with the other at base position. In this study I follow the analysis of \cite{munn1993topics}. The analysis of Johannessen places one of the conjuncts on a specifier position which should be open to movements as Te Velde argues. Te Velde further argues against an adjunction analysis of Munn but he recognizes that adjunction and merge do not have clear distinctions to argue against. Te Velde's arguments mostly revolve around arguing against a conjoiner that could check or assign case, or a specifier position for conjunctions. I recognize that Te Velde's analysis can prove useful as a general interpretation of conjunction but none of the examples he provides are adjusted for a head final and an agglutinative language like Turkish. One of the examples Te Velde provides right after (\ref{teveldeconj}) is (\ref{teveldeconj2}). He provides the structural representation in Figure \ref{fig:tevelde2} for the analysis of (\ref{teveldeconj2}).

\begin{exe}
    \ex \label{teveldeconj2} 
    \gll Fritz dankt und begrü{\ss}t den Herrn \\
    F thanks {\And} greets the.{\Acc} gentleman \\
    \glt `Fritz thanks and greets the gentleman' \\*
    \hfill \cite{te2005deriving}
\end{exe}

\begin{figure}[hbt!]
    \centering
    \begin{forest}
    [TP 
        [YP]
        [T' 
            [T]
            [T'\rlap{ $\Uparrow$} 
                [\&]
                [T' 
                    [T]
                    [\ldots]]]]]
    \end{forest}
    \caption{Te Velde tense conjunction}
    \label{fig:tevelde2}
\end{figure}

I give a sentence with argument structure of (\ref{teveldeconj2}) in (\ref{teveldeagainst}). The same structural analysis Te Velde provides cannot be carried out for Turkish. The functional head for tense is suffixed to the verb. A base merge of a partial construction to the head projection of tense as in Figure \ref{fig:tevelde2} is not possible.

\begin{exe}
    \ex \label{teveldeagainst} 
    \gll Ahmet adam-ı gör-dü ve çağır-dı. \\ 
    A[{\Nom}] man-{\Acc} see-{\Pst}[{\Tsg}] {\And} call-{\Pst}[{\Tsg}] \\
    \glt `Ahmet saw and called the man.'
\end{exe}

Accounting for the sentences like (\ref{teveldeagainst}) requires a whole other exploration of the mechanisms of conjunction that Te Velde provides. Not all are related to this study. That is why I only use the semantic equivalence condition for a successful conjunction of phrases and adopt \cite{munn1993topics}'s analysis in treating conjunctions.