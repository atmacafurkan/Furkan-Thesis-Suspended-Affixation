\section{Interim summary of Literature}

The literature of SA for Turkish provides some valuable observations, that make it easier to navigate the problems and workings of SA in Turkish. They feature useful data, approaches like LFG, syntactic movements like RNR, and comprehensive coverage of morphological constraints in SA which are immensely useful. I try to summarise what the benefits that the literature about Turkish SA provides are, and in what ways it can be improved. I also try to put a finger on what is not addressed in the following paragraphs.  

First, the observation of \cite{orgun1995flat} puts forward an anomalous behaviour in the suspension of possessive when used with plural, these are also the only two suffixes that can be overtly expressed under the conjoiner \textit{ile}. Orgun's solution is to hierarchically align the two for handling the problem of inseparable suspension of possessive when with plural.

Second, the observations of \cite{kabak2007turkish} indicates that the morphological size of what is left after suspension is crucial for a successful SA. Also bare verbs do not seem to be considered as morphological words in Turkish, even though they are phonological words under negation \textit{-mA} since they get to be the stress domain. The observation of morphological word constraint in SA is quite important since some similar phenomenon of a backwards process in, for example, German only requires the remnant after suspension to be a phonological word \citep{smith2000word, pounder2006broken,kenesei2007semiwords}. Besides this contribution, Kabak's paper also shows that SA might be possible with some derivational suffixes, yet he strongly suggests that the base for the derivational suffix is a compound like noun that uses a conjoiner for its parts. This idea about the SA of derivational suffixes is challenged later by \cite{akkucs2016suspended} to a certain extent.

Third, is that of \cite{broadwell2008turkish}. This approach entertains a different mode of operation for its analysis. Rather than the suspended suffix originating in both of the conjuncts, the conjoined phrase is only merged with a single projection of the `suspended' suffix. Later, as a tool of LFG, the rightward elements coinstantiate as a single word of multiple exponents, appearing as though only the second conjunct has the suffix whereas structurally it is shared and the two conjuncts are at the same level of representation.

Forth approach is provided by \cite{kornfilt1996some, kornfilt2012revisiting}. In Kornfilt's proposal suspension of affixes is just a syntactic operation of RNR, and suspendable suffixes are projections in syntax. Thus defining a line for the capability of SA for derivational and inflectional suffixes. While it is plausible to have movement based analysis for SA, some assumptions of Kornfilt, and the possible predictions they make do not hold true in examples of Case SA in Turkish. The analysis that Kornfilt provides does not explain why SA of Case does not result in ambiguous readings while the SA of plural and possessive does. The most that comes out of Kornfilt's proposal is that SA needs to be placed on a field that has almost full access to syntax, since it seems to be more productive in the inflectional paradigm of suffixes than it is in the derivational paradigm.

Second to last, come the observations of \cite{kharytonava2012word,kharytonava2012taming}

Last is that of \cite{akkucs2016suspended}

As a conclusion the current literature for Turkish SA provides possible solutions and analyses for SA. The literature does not have a good standing when it comes to answering what level of derivation SA takes place.  It is not pinpointed well to argue for or against any analysis that treat SA either as a ellipsis or sharing of syntactic elements, there is no exposure of different conjoiners and what they bring to SA. On a more scalar view Turkish might be employing conjoiners that are suffixal like \textit{-Ip}, and conjoiners that are clitics like \textit{ile}, and also other free forms like \textit{veya, ya da, hem ... hem, etc.}. These conjoiners differ in their ability and places of conjunction. For SA to be dealt with properly, scrutiny of conjoiners and their relation to their conjuncts is crucial. I think my efforts should be comprised to investigate those aspects of SA, and fill in the gaps for some observations provided in the literature.  