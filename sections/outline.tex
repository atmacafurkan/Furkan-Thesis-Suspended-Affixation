\section{Outline of the thesis}

In Chapter 2, I present the current theoretical considerations of SA in Turkish followed by SA or related phenomena in other languages. I then present different accounts for how conjunctions are represented. In Chapter 3, I present 2 exploratory experiments with 214 and 160 participants. Both try to answer the following questions respectively:

\begin{itemize}
    \item Is SA reserved for the inflectional paradigm?
    \item What is the processing cost of SA by the number of affixes?
    \item What is the effect of a conjoiner \textit{veya} `or'?
\end{itemize}

In the pursuit of answering these questions I provide the results of the two experiments I have conducted.
The first is an acceptability study and the second is a self paced reading study. In Chapter 4, I present an experiment with 132 participants that tries to answer the following question:
\begin{itemize}
    \item How does SA interact with sentence processing?
\end{itemize}
In the pursuit of answering this question, I present what process is assumed to take place in SA by the structural analyses provided for it in \citet{broadwell2008turkish,kornfilt2012revisiting,guseva2017postsyntactic}, and \citet{erschler2018suspended}. Instead of trying to justify which analysis best represents what SA is, I take on the prediction that all the analyses would make and the consequences of which for sentence processing. I present a language environment where the effects can be investigated and how the results can be interpreted. In Chapter 5, I present some analyses for SA by drawing inferences from the experiment results and the theoretical outlines. I present analyses for the suffixes \textit{ile/=lA} and \textit{-(y)Ip}. In Chapter 6, I give my conclusions and the further points that can be pursued for the study of SA. I provide what were my expectations and the workflow I had during the process of writing this thesis. 